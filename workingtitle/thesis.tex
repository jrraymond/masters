\documentclass[11pt, ma]{westhesis}
\usepackage[utf8]{inputenc}
\usepackage{natbib}
\usepackage{amsmath}
\usepackage{amsfonts}
\usepackage{amssymb}
\usepackage{bussproofs}
\usepackage{listings}
%\usepackage{ntheorem}
\usepackage{float}
\usepackage{upgreek}
\usepackage{stmaryrd}
\usepackage[font=small]{caption}
\usepackage{float}

% ND comments.
\usepackage[normalem]{ulem}
\setlength{\marginparwidth}{1.25in}
\reversemarginpar
\newcommand{\ndComment}[2][\relax]{\ifx#1\relax\else\uline{#1}\fi\marginpar{\renewcommand{\baselinestretch}{1.0}\tiny ND: {#2}}}
\newcommand{\ndTypo}[1]{\ndComment[#1]{$\longrightarrow$}}
\newcommand{\ndDel}[1]{\sout{#1}\ndComment{$\longrightarrow$}}

\newcommand{\LP}{\langle}
\newcommand{\RP}{\rangle}
\newcommand{\LB}{\llbracket}
\newcommand{\RB}{\rrbracket}
\newcommand{\quadthree}{\qquad\quad}
\newcommand{\quadfour}{\quadthree\quad}
\newcommand{\quadfive}{\quadfour\quad}
\newcommand{\quadsix}{\quadfive\quad}
\newcommand{\quadseven}{\quadsix\quad}
\newcommand{\quadeight}{\quadseven\quad}
\newcommand{\quadten}{\quadfive\quadfive}

\newcommand{\llangle}{\LP\!\LP}
\newcommand{\rrangle}{\RP\!\RP}
%\newcommand{\llambda}{\lambda\!\!\lambda}
%\newcommand{\pplus}{\raisebox{0.25ex}{+}\!\!\!\!\!+}

\makeatletter
\newcommand{\fs@singlerule}{%
	\fs@plain%
	\def\@fs@mid{\vskip1ex\hrule\vskip1ex}
}
\makeatother
\floatstyle{singlerule}
\restylefloat{figure}
\newfloat{program}{t}{pgm}
\floatname{program}{Program}

\makeatletter
\def\old@comma{,}
\catcode`\,=13
\def,{%
  \ifmmode%
    \old@comma\discretionary{}{}{}%
  \else%
    \old@comma%
  \fi%
}
\makeatother

\allowdisplaybreaks


\department{Mathematics and Computer Science}
\submitdate{TDS}
\advisor{Norman Danner}

\title{Extracting Cost Recurrences from Sequential and Parallel Functional Programs}
\author{Justin Raymond}
\submitdate{\today}

\theoremstyle{plain}
\newtheorem{theorem}{Theorem}[chapter]
\newtheorem{corollary}{Corollary}[theorem]
\newtheorem{lemma}[theorem]{Lemma}

\theoremstyle{definition}
\newtheorem{defn}{Definition}[chapter]
%\newenvironment{definition}[1][Definition]{\begin{trivlist}
%\item[\hskip \labelsep {\bfseries #1}]}{\end{trivlist}}

\newcommand{\T}[1]{\texttt{#1}}
\lstset{mathescape=true,basicstyle=\ttfamily}

\newcommand\numberthis{\addtocounter{equation}{1}\tag{\theequation}}
\begin{document}
%
\begin{abstract}
Complexity analysis aims to predict the resources, most often time and space,
which a program requires.  We build on previous work by \cite{Danner2013} and
\cite{Danner2015} which formalizes the extraction of recurrences for evaluation
cost from higher order functional programs. Source language programs are
translated into a complexity language. The translation of a program is a
pair of a cost, a bound on the cost of evaluating the program to a value, and a
potential, the cost of future use of the value. We use the formalization to
analyze the time complexity of higher order functional programs. We also
demonstrate the flexibility of the method by extending it to parallel cost
semantics. In parallel cost semantics, costs are cost graphs, which express
dependencies between subcomputations in the program. We prove by logical
relations that the extracted recurrences are an upper bound on the evaluation
cost of the original program. We also give examples of the analysis of higher
order functional programs under the parallel evaluation semantics. We also
prove the recurrence for the potential of a program does not depend on the cost
of the program.
\end{abstract}
%
\begin{acknowledgements}
  Thank you to my adviser Norman Danner for having the patience to put up with
  me the past year. Without him, this thesis would not have made it past the
  title page. Thanks also to Jim Lipton, Dan Licata, and Danny Krizanc who,
  along with Norman Danner, taught me everything I know about Computer Science.
  \par
  Thank you to my readers: Norman Danner, Dan Licata, and Saleh Aliyari.
\end{acknowledgements}
%

\frontmatter
\maketitle
\makededication
\makeack
\makeabstract        %TODO

\tableofcontents

\mainmatter
\chapter{Introduction}

\section{Complexity Analysis}
\paragraph{}
The efficiency of programs is categorized by how the resource usage of a program increases with the input size in the limit.
This is often called the asymptotic efficiency or complexity of a program.
Asymptotic efficiency abstracts away the details of efficiency, allowing programs to be compared without knowledge of specific hardware architecture or the size and shape of the programs input (\citet{Cormen2001}).
However, traditional complexity analysis is first-order; the asymptotic efficiency of a program is only be expressed in terms of its input.
For example the analysis of the function \T{map:(a -> b) -> [b] -> [a]}, which applies a function to every element in a list, traditionally ignores the cost of applying it's first argument.
Consequently it shows the asymptotic efficiency of \T{map} is linear in the length of the list.
When mapping constant cost functions, such as fixed width integer addition, over a list, first order analysis is sufficient for predicting the cost of running a program.
However when mapping a nontrivial function over a list, first order complexity analysis may not accurately describe the cost.
The cost of mapping the nontrivial function over a small list of large elements will be larger than mapping the function over a larger list of smaller elements.
Higher order analysis allows us to say the complexity of \T{map} is a product of the length of the list and the cost of the function. 

\paragraph{}
This thesis will build on work by Danner, in which the complexity of an expression is composed of a cost and a potential.

\section{Previous Work} \paragraph{}
\citet{Danner2007}, building on the work of others, introduced the idea that the complexity of an expression consists of a cost, representing an upper bound on the time it takes to evaluate the expression, and a potential, representing the cost of future uses of the expression.
The notion of a potential is key because it allows the analysis of higher-order expressions. 
The complexity of a higher order function such as \T{map} depends on the potential of its argument function.
They developed a type system for ATR, a call-by-value version of PCF, that consists of a part restricting the sizes of values of expressions and a part restricting the cost of evaluating a expression.
Programs written in ATR are constrained by the type system as to run in less than type-2 polynomial time.
\citet{Danner2009} extended this work to express more forms of recursion, in particular those required by insertion sort and selection sort.

\paragraph{}
\citet{Danner2013} utilized the notion of thinking of the complexity of an expression as a pair of a cost and a potential to statically analyze the complexity of a higher-order functional language with structural list recursion.
The expressions in the higher-order functional language with structural list recursion, referred to as the source language, are mapped to expressions in a complexity language by a translation function.
The translated expression describes an upper bound on the complexity of the original programs.

\paragraph{}
\citet{Danner2015} built on this work to formalize the extraction of recurrences from a higher-order functional language with structural recursion on arbitrary inductive data types.
Arbitrary inductive data types are handled semantically using programmer-specified sizes of data types.
Sizes must be programmer-specified because the structure of a data type does not always determine the interpretation of the size of a data type.
Also, there exist different reasonable interpretations of size, and some may be preferable to others depending on what is being analyzed.
For example, if the size of a list is interpreted as the length of the list, then the complexity \T{map$\langle$f,xs$\rangle$} will be order length of \T{xs}.
If the size of a list is interpreted as a pair of the length and its largest element, then the complexity will depend on the length of \T{xs}, the potential of \T{f}, and the largest element.


\section{Contribution}

\paragraph{}
This thesis contains a catalog of example of the extraction of recurrences from from functional programs using the approach by \citet{Danner2015}. The examples are compare this approach with other methods, such as those by \citet{Avanzini2015} and \citet{HoffHof2010}, highlighting the strengths and weakness of this approach compared with others. The examples include fold, reversing a list and parametric insertion sort.

\paragraph{}
This thesis also demonstrates the recurrence for the potential does not depend on the recurrence for the cost. Consequently we can extract the recurrence for the potential and analyze it independently.

\paragraph{}
This thesis also extends the \citet{Danner2015} approach to parallel programs. We alter the cost sematnics and the complexity translation to produce costs which are not integers, but graphs which reflect the dependencies between expressions in the program. The work and span of the program can be extracted from the cost graph, which indicate the cost of the program when run in parallel.

\chapter{Higher Order Complexity Analysis}

Programs are written in the source language. Then the program is translated
to a complexity language. The semantic interpretation of the complexity
language program may be used to analyse the complexity of the original
program.

\section{Source Language}
The source language is the simply typed lambda calculus with \T{Unit},
products, suspensions, user-defined inductive datatypes and a recursion
construct. Valid signatures, types, and constructor arguments are given in
Figure \ref{fig:source_lang_sigs_types_constructors}. The types, expressions,
and typing judgments of the source language are given in Figure
\ref{fig:source_lang_syntax_types}.  Evaluation is call-by-value and the rules
for evaluation are given in Figure \ref{fig:source_lang_oper_sem}.

\T{unit} is a singleton type with only one inhabitant, the value
$\LP\RP$, also called unit.

Product types are a compound types consisting of an ordered pair of types.
Products are introduced using $\LP e_0, e_1 \RP$ Since evaluation is
call-by-value, products are strict.  So both expressions of in product must be
evaluated before the product may be destructured.  Products are eliminated
using \T{split}.

A suspension is an unevaluated computation.  A suspension has type \T{susp}
$\tau$ where $\tau$ is the type of the suspended computation.  Suspensions are
introduced using the \T{delay}$(e)$ operator.  Suspensions are eliminated using
the \T{force}$(e)$ operator, which evaluates the suspended computation.


A program using datatypes must have a top-level signature $\psi$ consisting of
datatype declarations of the form
%
\[ \T{datatype} \delta = C^\delta_0 \T{of} \phi_{C_0}[\delta] \ |\ ...\ |\ C^\delta_{n-1} \T{of} \phi_{C_{n-1}}[\delta] \]
%
Each datatype may only refer to datatypes declared earlier in the signature.
This prevents general recursive datatypes.  The argument to each constructor is
given by a strictly positive functor $\phi$, which is one of $t$, $\tau$,
$\phi_0 \times \phi_1$, and $\tau \rightarrow \phi$.  The identity functor $t$
represents recursive occurrence of the datatype.  The constant functor $\tau$
represents a non-recursive type.  The product functor $\phi_0 \times \phi_1$
represents a pair of arguments.  The constant exponential $\tau \rightarrow \phi$
represents a function type.  The introduction forms for datatypes are the
constructors.  The elimination form for a datatype is the \T{rec} construct.


The \T{rec} construct allows for structural recursion.  \T{rec} is given an
argument to recurse on and a sequence of statements corresponding to each
constructor for the datatype of the first argument.  The first argument to
\T{rec} is evaluated to a value, and then depending on the outermost
constructor of the value, \T{rec} evaluates to the appropriate branch.


\T{map} is used to lift functions from $\sigma \rightarrow \tau$ to
$\phi[\sigma] \rightarrow \phi[\tau]$. \T{map} is restricted to syntactic
values and is used in the operational semantics to insert recursive calls in
their places. For example, if recursing on a value that does not contain a
recursive occurrence of a datatype, such as a boolean or a tree leaf, then
\T{map} does not insert a recursive call anywhere.


\begin{figure}[H]
  \caption{Source language syntax and types}
  \label{fig:source_lang_syntax_types}
  Types
  \begin{align*}
    \tau &::= \T{unit}\ |\ \tau \times \tau\ |\ \tau \rightarrow \tau\ |\ \T{susp}\ \tau\ |\ \delta \\
    \phi &::= t\ |\ \tau\ |\ \phi \times \phi\ |\ \tau \rightarrow \phi \\
    \T{datatype}\ \delta &= C^\delta_0 \T{of} \phi_{C_0}[\delta]\ |\ ...\ |\ C^\delta_{n-1} \T{of} \phi_{C_{n-1}}[\delta]
  \end{align*}

  Expressions
  \begin{align*}
    v &::= x\ |\ \LP\RP\ |\ \LP v, v \RP\ |\ \lambda x.e\ |\ \T{delay}(e)\ |\ C\ v \\
    e &::= x\ |\ \LP\RP\ |\ \LP e, e \RP\ |\ \T{split}(e, x.x.e)\ |\ \lambda x.e\ |\ e\ e \\
      &\quad\ |\ \T{delay}(e)\ |\ \T{force}(e)\ |\ C^\delta\ e\ |\ \T{rec}^\delta(e, \overline{C \mapsto x.e_C}) \\
      &\quad\ |\ \T{map}^\phi(x.v, v)\ |\ \T{let}(e, x.e) \\
    n &::= 0\ |\ 1\ |\ n + n
  \end{align*}

  Typing Judgments

  \AxiomC{}
  \UnaryInfC{$\gamma, x : \sigma \vdash x : \sigma$}
  \DisplayProof
  \AxiomC{}
  \UnaryInfC{$\gamma \vdash \LP \RP : \T{unit}$}
  \DisplayProof

  \bigskip

  \AxiomC{$\gamma \vdash e_0 : \tau_0$}
  \AxiomC{$\gamma \vdash e_1 : \tau_1$}
  \BinaryInfC{$\LP e_0, e_1 \RP : \tau_0 \times \tau_1$}
  \DisplayProof
  \AxiomC{$\gamma \vdash e_0 : \tau_0 \times \tau_1$}
  \AxiomC{$\gamma, x_0 : \tau_0, x_1 : \tau_1 \vdash e_1 : \tau$}
  \BinaryInfC{$\gamma \vdash \T{split}(e_0, x_0.x_1.e_1) : \tau$}
  \DisplayProof

  \bigskip

  \AxiomC{$\gamma, x : \sigma \vdash e : \tau$}
  \UnaryInfC{$\gamma \vdash \lambda x.e : \sigma \rightarrow \tau$}
  \DisplayProof
  \AxiomC{$\gamma \vdash e_0 : \sigma \rightarrow \tau$}
  \AxiomC{$\gamma \vdash e_1 : \sigma$}
  \BinaryInfC{$\gamma \vdash e_0\ e_1 : \tau$}
  \DisplayProof

  \bigskip

  \AxiomC{$\gamma \vdash e : \tau$}
  \UnaryInfC{$\gamma \vdash \T{delay}(e) : \T{susp}\ \tau$}
  \DisplayProof
  \AxiomC{$\gamma \vdash e : \T{susp}\ \tau$}
  \UnaryInfC{$\gamma \vdash \T{force}(e) : \tau$}
  \DisplayProof

  \bigskip

  \AxiomC{$\gamma \vdash e : \phi_C[\delta]$}
  \UnaryInfC{$\gamma \vdash C^\delta\ e : \delta$}
  \DisplayProof
  \AxiomC{$\gamma \vdash e : \delta$}
  \AxiomC{$\forall C . \gamma, x : \phi_C[\delta \times \T{susp}\ \tau] \vdash e_C : \tau$}
  \BinaryInfC{$\gamma \vdash \T{rec}^\delta(e, \overline{C \mapsto x.e_C}) : \tau$}
  \DisplayProof

  \bigskip

  \AxiomC{$\gamma, x : \tau_0 \vdash v_1 : \tau_1$}
  \AxiomC{$\gamma \vdash v_0 : \phi[\tau_0]$}
  \BinaryInfC{$\T{map}^\phi(x.v_1, v_0) : \phi[\tau_1]$}
  \DisplayProof
  \AxiomC{$\gamma \vdash e_0 : \sigma$}
  \AxiomC{$\gamma, x : \sigma \vdash e_1 : \tau$}
  \BinaryInfC{$\T{let}(e_0, x.e_1) : \tau$}
  \DisplayProof
\end{figure}


\begin{figure}[H]
  \caption{Source language valid signatures, types, and constructor arguments}
  \label{fig:source_lang_sigs_types_constructors}

  Signatures: $\psi$ \T{sig}

  \bigskip

  \AxiomC{}
  \UnaryInfC{$\LP\RP$ \T{sig}}
  \DisplayProof
  \quadfive
  \AxiomC{$\delta \notin \forall\text{C}(\psi \vdash \phi_C \T{ok})$}
  \UnaryInfC{$\psi, \T{ datatype } \delta = \overline{C \T{ of } \phi_C[\delta]} \T{ sig}$}
  \DisplayProof

  \bigskip \bigskip

  Types : $\psi \vdash \tau\ \T{type}$

  \bigskip

  \AxiomC{}
  \UnaryInfC{$\psi \vdash \T{unit type}$}
  \DisplayProof
  \quad
  \AxiomC{$\psi \vdash \tau_0\ \T{type}$}
  \AxiomC{$\psi \vdash \tau_1\ \T{type}$}
  \BinaryInfC{$\psi \vdash \tau_0 \times \tau_1\ \T{type}$}
  \DisplayProof

  \bigskip

  \AxiomC{$\psi \vdash \tau_0\ \T{type}$}
  \AxiomC{$\psi \vdash \tau_1\ \T{type}$}
  \BinaryInfC{$\psi \vdash \tau_0 \rightarrow \tau_1\ \T{type}$}
  \DisplayProof
  \quad
  \AxiomC{$\psi \vdash \tau\ \T{type}$}
  \UnaryInfC{$\psi \vdash \T{susp}\ \tau\ \T{type}$}
  \DisplayProof
  \quad
  \AxiomC{$\delta \in \psi$}
  \UnaryInfC{$\psi \vdash \delta\ \T{type}$}
  \DisplayProof

  \bigskip

  Constructor arguments: $\psi \vdash \phi \T{ ok}$

  \bigskip

  \AxiomC{}
  \UnaryInfC{$\psi \vdash t\ \T{ ok}$}
  \DisplayProof
  \quadseven
  \AxiomC{$\psi \vdash \tau \T{ type}$}
  \UnaryInfC{$\psi \vdash \tau \T{ ok}$}
  \DisplayProof

  \bigskip

  \AxiomC{$\psi \vdash \phi_0\ \T{ok}$}
  \AxiomC{$\psi \vdash \phi_1\ \T{ok}$}
  \BinaryInfC{$\psi \vdash \phi_0 \times \phi_1\ \T{ok}$}
  \DisplayProof
  \quadfive
  \AxiomC{$\psi \vdash \tau\ \T{type}$}
  \AxiomC{$\psi \vdash \phi\ \T{ok}$}
  \BinaryInfC{$\psi \vdash \tau \rightarrow \phi\ \T{ok}$}
  \DisplayProof
\end{figure}


\begin{figure}[H]
  \caption{Source language operational semantics}
  \label{fig:source_lang_oper_sem}

  \bigskip

  \AxiomC{$e_0 \downarrow^{n_0} v_0$}
  \AxiomC{$e_1 \downarrow^{n_1} v_1$}
  \BinaryInfC{$\LP e_0, e_1 \RP \downarrow^{n_0 + n_1} \LP v_0, v_1 \RP$}
  \DisplayProof
  \quad
  \AxiomC{$e_0 \downarrow^{n_0} \LP v_0, v_1 \RP$}
  \AxiomC{$e_1[v_0/x_0, v_1/x_1] \downarrow^{n_1} v$}
  \BinaryInfC{$\T{split}(e_0, x_0.x_1.e_1) \downarrow^{n_0 + n_1} v$}
  \DisplayProof

  \bigskip

  \AxiomC{$e_0 \downarrow^{n_0} \lambda x.e_0'$}
  \AxiomC{$e_1 \downarrow^{n_1} v_1$}
  \AxiomC{$e_0'[v_1/x] \downarrow^n v$}
  \TrinaryInfC{$e_0\ e_1 \downarrow^{1 + n_0 + n_1 + n} v$}
  \DisplayProof
  \quad
  \AxiomC{}
  \UnaryInfC{$\T{delay}(e) \downarrow^0 \T{delay}(e)$}
  \DisplayProof

  \bigskip

  \AxiomC{$e \downarrow^{n_0} \T{delay}(e_0)$}
  \AxiomC{$e_0 \downarrow^{n_1} v$}
  \BinaryInfC{$\T{force}(e) \downarrow^{n_0 + n_1} v$}
  \DisplayProof
  \quad
  \AxiomC{$e \downarrow^n v$}
  \UnaryInfC{$C e \downarrow^n C v$}
  \DisplayProof

  \bigskip

  \AxiomC{$e \downarrow^{n_0} C v_0$}
  \AxiomC{$\T{map}^{\phi_C}(y.\LP y, \T{delay}(rec(y, \overline{C \mapsto x.e_C}))\RP, v_0) \downarrow^{n_1} v_1$}
  \AxiomC{$e_C[v_1/x] \downarrow^{n_2} v$}
  \TrinaryInfC{$rec(e, \overline{C \mapsto x.e_C}) \downarrow^{1 + n_0 + n_1 + n_2} v$}
  \DisplayProof

  \bigskip

  \AxiomC{}
  \UnaryInfC{$\T{map}^t(x.v, v_0) \downarrow^0 v[v_0/x]$}
  \DisplayProof
  \quad
  \AxiomC{}
  \UnaryInfC{$\T{map}^\tau(x.v, v_0) \downarrow^0 v_0$}
  \DisplayProof

  \bigskip

  \AxiomC{$\T{map}^{\phi_0}(x.v, v_0) \downarrow^{n_0} v_0'$}
  \AxiomC{$\T{map}^{\phi_1}(x.v, v_1) \downarrow^{n_1} v_1'$}
  \BinaryInfC{$\T{map}^{\phi_0 \times \phi_1}(x.v, \LP v_0, v_1 \RP) \downarrow^{n_0 + n_1} \LP v_0', v_1'\RP$}
  \DisplayProof

  \AxiomC{}
  \UnaryInfC{$\T{map}^{\tau \to \phi}(x.v, \lambda y.e) \downarrow^0 \lambda y.\T{let}(e, z.\T{map}^\phi(x.v, z))$}
  \DisplayProof
  \quad
  \AxiomC{$e_0 \downarrow^{n_0} v_0$}
  \AxiomC{$e_1[v_0/x] \downarrow^{n_1} v$}
  \BinaryInfC{$\T{let}(e_0, x.e_1) \downarrow^{n_0 + n_1} v$}
  \DisplayProof
\end{figure}

\section{Complexity Language}

The types, expressions, and typing judgments of the complexity language are given in Figure \ref{fig:complexity_lang}.
The complexity language is similar to the source language with a few exceptions.

Suspensions are no longer present.

Tuples are destructured using projections instead of \T{split}.

\begin{figure}[H]
  \caption{Complexity language types, expressions, and typing judgments}
  \label{fig:complexity_lang}

  Types
  \begin{align*}
    T &::= \textbf{C} \ |\ \T{unit} \ |\ \Delta \ |\ T \times T \ |\ T \rightarrow T \\
    \Phi &::= t \ |\ T \ |\ \Phi \times \Phi \ |\ T \rightarrow \Phi \\
    \textbf{C} &::= 0\ |\ 1\ |\ 2\ |\ ... \\
    \T{datatype}\Delta &= C^\Delta_0 \T{of} \Phi_{C_0}[\Delta] \ |\ ... \ |\ C^\Delta_{n-1} \T{of} \Phi_{C_{n-1}}[\Delta]
  \end{align*}

  Expressions
  \begin{align*}
    E &::= x | 0 | 1 | E + E | \LP\RP | \LP E,E \RP | \\
      &\quad \pi_0 E | \pi_1 E | \lambda x.E | E\ E | C^\delta\ E | \text{rec}^\Delta(E, \overline{C \mapsto x.E_C})
  \end{align*}

  Typing Judgments

  \bigskip

  \AxiomC{}
  \UnaryInfC{$\Gamma, x : T \vdash x : T$}
  \DisplayProof
  \quad
  \AxiomC{}
  \UnaryInfC{$\Gamma \vdash 0 : \textbf{C}$}
  \DisplayProof
  \quad
  \AxiomC{}
  \UnaryInfC{$\Gamma \vdash 1 : \textbf{C}$}
  \DisplayProof
  \quad
  \AxiomC{}
  \UnaryInfC{$\Gamma \vdash \LP\RP : \textbf{unit}$}
  \DisplayProof

  \bigskip

  \AxiomC{$\Gamma \vdash E_0 : \textbf{C}$}
  \AxiomC{$\Gamma \vdash E_1 : \textbf{C}$}
  \BinaryInfC{$\Gamma \vdash E_0 + E_1 : \textbf{C}$}
  \DisplayProof
  \quad
  \AxiomC{$\Gamma \vdash E_0 : T_0$}
  \AxiomC{$\Gamma \vdash E_1 : T_1$}
  \BinaryInfC{$\Gamma \vdash \LP E_0, E_1 \RP : T_0 \times T_1$}
  \DisplayProof

  \bigskip

  \AxiomC{$\Gamma \vdash E : T_0 \times T_1$}
  \UnaryInfC{$\Gamma \vdash \pi_i E : T_i$}
  \DisplayProof
  \quad
  \AxiomC{$\Gamma, x : T_0 \vdash E : T_1$}
  \UnaryInfC{$\Gamma \vdash \lambda x.E : T_0 \rightarrow T_1$}
  \DisplayProof

  \bigskip

  \AxiomC{$\Gamma \vdash E_0 : T_0 \rightarrow T_1$}
  \AxiomC{$\Gamma \vdash E_1 : T_0$}
  \BinaryInfC{$\Gamma \vdash E_0\ E_1 : T_1$}
  \DisplayProof
  \quad
  \AxiomC{$\Gamma \vdash E : \Phi_C[\Delta]$}
  \UnaryInfC{$\Gamma \vdash C^\Delta E : \Delta$}
  \DisplayProof

  \bigskip

  \AxiomC{$\Gamma \vdash E : \Delta$}
  \AxiomC{$\forall C . \Gamma, x : \Phi_C[\Delta \times T] \vdash E_C : T$}
  \BinaryInfC{$\Gamma \vdash \text{rec}^\Delta(E, \overline{C \mapsto x.E_C}) : T$}
  \DisplayProof

\end{figure}

The translation from the source language to the complexity language is given in
Figure \ref{fig:complexity_translation_types} and Figure
\ref{fig:complexity_translation_expressions}.
%
\begin{figure}[H]
  \caption{Translation from source language to complexity language types.}
  \label{fig:complexity_translation_types}
  %
  \begin{align*}
    \|\tau\| &= \textbf{C} \times \llangle \tau \rrangle \\
    \llangle\T{unit}\rrangle &= \T{unit} \\
    \llangle \sigma \times \tau \rrangle &= \llangle \sigma \rrangle \times \llangle \tau \rrangle \\
    \llangle \sigma \rightarrow \tau \rrangle &= \llangle \sigma \rrangle \rightarrow \|\tau\| \\
    \llangle \T{susp}\ \tau \rrangle &= \|\tau\| \\
    \llangle \delta \rrangle &= \delta \\
  \end{align*}
  %
  \begin{align*}
    \|\phi\| &= \textbf{C} \times \llangle \phi \rrangle \\
    \llangle t \rrangle &= t \\
    \llangle \tau \rrangle &= \llangle \tau \rrangle \\
    \llangle \phi_0 \times \phi_1 \rrangle &= \llangle \phi_0 \rrangle \times \llangle phi_1 \rrangle \\
    \llangle \tau \rightarrow \phi \rrangle &= \llangle \phi \rrangle \rightarrow \|\phi\| \\
  \end{align*}
  \begin{align*}
    \llangle \psi \rrangle &= \text{ for each }\delta \in \psi, \delta = C_0^\delta \T{ of } \llangle \phi_{C_0}\rrangle[\delta], . . . , C_{n-1}^\delta \T{ of } \llangle \phi_{n-1}\rrangle[\delta] \\
  \end{align*}
  %
\end{figure}
%
\begin{figure}[H]
  \caption{Translation from source language to complexity language expressions.}
  \label{fig:complexity_translation_expressions}
  \begin{align*}
    \|x\| &= \LP 0, x \RP \\
    \|\LP\RP\| &= \LP 0, \LP\RP\RP \\
    \|\LP e_0,e_1 \RP\| &= \LP \|e_0\|_c + \|e_1\|_c, \LP \|e_0\|_p, \|e_1\|_p\RP\RP \\
    \|\T{split}(e_0, x_0.x_1.e_1)\| &= \|e_0\|_c +_c \|e_1\|[\pi_0\|e_0\|_p/x_0, \pi_1\|e_0\|_p/x_1] \\
    \|\lambda x.e\| &= \LP 0, \lambda x.\|e\| \RP \\
    \|e_0\ e_1\| &= (1 + \|e_0\|_c + \|e_1\|_c) +_c \|e_0\|_p \|e_1\|_p \\
    \|\T{delay}(e)\| &= \LP 0, \|e\|\RP \\
    \|\T{force}(e)\| &= \|e\|_c +_c \|e\|_p \\
    \|\text{C}^\delta_i\ e\| &= \LP \|e\|_c, \text{C}^\delta_i \|e\|_p \RP \\
    \|\T{rec}^\delta(e, \overline{C \mapsto x.e_C})\| &= \|e\|_c +_c \T{rec}^\delta(\|e\|_p, \overline{C \mapsto x.1 +_c \|e_C\|}) \\
    \|\T{map}^\phi(x.v_0, v_1)\| &= \LP 0, \T{map}^{\llangle \phi \rrangle}(x.\|v_0\|_p, \|v_1\|_p)\RP \\
    \|\T{let}(e_0, x.e_1)\rrangle &= \|e_0\|_c +_c \|e_1\|[\|e_0\|_p/x]
  \end{align*}
  %
\end{figure}

% ---------------------------------------- FAST REVERSE ----------------------------------------
\chapter{Fast Reverse}

Fast reverse is an implementation reverse in linear time complexity. 
A naive implementation of reverse appends the head of the list to recursively
reversed tail of  the list. Fast reverse instead uses an abstraction to delay
the consing. As this is the first example, we will walk through the translation
and interpretation in gory detail. In following examples we will relegate the
walk-through of the translation to the appendices, where the reader can peruse
them, perhaps over a glass of carbenet sauvignon, as a relaxing end to a
stressful day.

The definition of the list datatype holds no suprises.
\[ \T{datatype list} = \T{Nil of unit}\ |\ \T{Cons of int} \times \T{list} \]

The implementation of fast reverse is not obvious. We write a function \T{rev}
that applies an auxilary function to an empty list to produce the result.  The
specification of reverse is \T{rev [$x_0,\dots,x_{n-1}$] =
[$x_{n-1},\dots,x_0$]}. The specification of the auxilary function
\T{rec(xs,$\dots$)} is \T{rec($[x_0,\dots,x_{n-1}],\dots$)
[$y_0,\dots,y_{m-1}$] = [$x_{n-1},\dots,x_0,y_0,\dots,y_{m-1}$]}.

\begin{lstlisting}
rev xs = $\lambda$xs.rec(xs,
               Nil $\mapsto$ $\lambda$a.a,
               Cons$\mapsto$b.split(b,x.c.split(c,xs'.r.
                        $\lambda$a.force(r) Cons$\langle$x,a$\rangle$))) Nil
\end{lstlisting}

Notice that the implementation of \T{rev} would be much cleaner if we where
able to pattern match on cases of the \T{rec}. Below is \T{rev} written with
this syntactic sugar.

\begin{lstlisting}
rev = $\lambda$xs.rec(xs, Nil $\mapsto \lambda$a.a,
              Cons$\mapsto\langle$y$\langle$ys,r$\rangle\rangle$.$\lambda$b.force(r) Cons$\langle$x,b$\rangle$) Nil
\end{lstlisting}

Each recursive call creates an abstraction that applies the recursive call on
the tail of the list to the list created by consing the head of the list onto
the abstraction argument. The recursive calls builds nested abstractions as
deep as the length of the list which is collapsed by application of the
outermost abastraction to \T{Nil}. Below we show the evaluation of \T{rev}
applied to a small list of just two elements.

\begin{lstlisting}
rev (Cons$\langle$0,Cons$\langle$1, Nil$\rangle\rangle$) $\to$ 
  rec(Cons$\langle$0,Cons$\langle$1,Nil$\rangle\rangle$,
      Nil $\mapsto\lambda$a.a
      Cons$\mapsto$b.split(b,x.c.split(c,xs'.r.
               $\lambda$a.force(r) Cons$\langle$x,a$\rangle$))) Nil
  $\to^* (\lambda$a0.($\lambda$a1.($\lambda$a2.a2) Cons$\langle$1,a1$\rangle$) Cons$\langle$0,a0$\rangle$) Nil
  $\to_\beta$ ($\lambda$a1.($\lambda$a2.a2) Cons$\langle$1,a1$\rangle$) Cons$\langle$0,Nil$\rangle$
  $\to_\beta$ ($\lambda$a2.a2) Cons$\langle$1,Cons$\langle$0,Nil$\rangle\rangle$
  $\to_\beta$ Cons$\langle$1,Cons$\langle$0,Nil$\rangle\rangle$
\end{lstlisting}

% -------------------- BEGIN REV TRANSLATION SPLIT --------------------   
We will walk through the translation from the source language to the complexity
language.
%
\begin{flalign*}
  \|\T{rev}\| &= \|\lambda xs.\T{rec}(xs, \T{Nil}\mapsto\lambda\T{a.a,} \\
              &\quad \T{Cons}\mapsto b.\T{split}(b,x.c.\T{split}(c,xs'.r.\lambda a.\T{force}(r)\ \T{Cons}\langle\T{x,a}\rangle)))\ \T{Nil}\| \\
\end{flalign*}
%
% BEGIN ABSTRACTION
%
First we apply the rule for translating an abstraction. The rule is
$\|\lambda x. e\| = \langle 0, \lambda x. \|e\|\rangle$.
%
\begin{flalign*}
  \|\T{rev}\| &= \|\lambda xs.\T{rec}(xs, \T{Nil} \mapsto\lambda a.a, \\
              &\quadthree \T{Cons}\mapsto b.\T{split}(b,x.c.\T{split}(c,xs'.r.\lambda a.\T{force}(r) \T{Cons}\langle x,a\rangle)))\ \T{Nil}\| \\
              &\quad = \langle 0, \lambda xs.\|\T{rec}(xs, \T{Nil}\mapsto\lambda a.a, \\
              &\quadthree \T{Cons}\mapsto b.\T{split}(b,x.c.\T{split}(c,xs'.r.\lambda a.\T{force}(r) \T{Cons}\langle x,a\rangle)))\ \T{Nil}\|\rangle \\
\end{flalign*}
%
The next translation is an application. The rule for translating an application is
$\|e_0\ e_1\| = (1 + \|e_0\|_c + \|e_1\|_c) +_c (\|e_0\|_p\ \|e_1\|_p)$.
In this case, \T{rec(...)} is $e_0$ and \T{Nil} is $e_1$. We translate \T{Nil} then
\T{rec(...)} seperately.
%
% APP ARGUMENT
%
The translation of a constructor applied to an expression is a tuple of the
cost of the translated expression and the corresponding complexity language
constructor applied to the potential of the translated expression. Since the
expression inside \T{Nil} is $\langle\rangle$, and
$\|\langle\rangle\| = \langle 0,\langle\rangle\rangle$, we have
%
\begin{flalign*}
  |\T{Nil}\| &= \langle\langle 0, \langle\rangle\rangle_c, \T{Nil}\langle 0,\langle\rangle\rangle_p\rangle \\
             &= \langle 0, \T{Nil}\langle\rangle\rangle
\end{flalign*}
%
% BEGIN REC
%
The rule for translating a \T{rec} expression is
\[
  \|\T{rec}(e,\overline{C \mapsto x.e_C})\| = \|e\|_c +_c \T{rec}(\|e\|_p, \overline{C \mapsto x.\|e_C\|})
\]
%
\begin{flalign*}
  &\|\T{rec}(xs, \T{Nil}\mapsto\lambda a.a, \\
  &\qquad \T{Cons}\mapsto b.\T{split}(b,x.c.\T{split}(c,xs'.r.\lambda a.\T{force}(r)\ \T{Cons}\langle x,a\rangle)))\| \\
  &= \|xs\|_c +_c \T{rec}(\|xs\|_p, \T{Nil} \mapsto 1 +_c \|\lambda a.a\| \\
  &\quadthree \T{Cons}\mapsto b. 1 +_c \|\T{split}(b,x.c.\T{split}(c,xs'.r.\lambda a.\T{force}(r)\ \T{Cons}\langle x,a\rangle))\|) \\
  &= \langle 0, xs \rangle_c +_c \T{rec}(\langle 0, xs\rangle_p, \T{Nil} \mapsto 1 +_c \|\lambda a.a\| \\
  &\quadthree \T{Cons}\mapsto b. 1 +_c \|\T{split}(b,x.c.\T{split}(c,xs'.r.\lambda a.\T{force}(r)\ \T{Cons}\langle x,a\rangle))\|) \\
  &\text{The term $xs$ is a variable and the rule for translating variables is $\|xs\| = \langle 0, xs\rangle$.} \\
  &= \T{rec}(xs, \T{Nil} \mapsto 1 +_c \|\lambda a.a\| \\
  &\quadthree \T{Cons}\mapsto b. 1 +_c \|\T{split}(b,x.c.\T{split}(c,xs'.r.\lambda a.\T{force}(r)\ \T{Cons}\langle x,a\rangle))\|)
\end{flalign*}
%
% NIL BRANCH
%
The translation of the \T{Nil} branch is
simple application of the $\|\lambda x.e\| = \langle 0, \lambda
x.\|e\|\rangle$ and the variable translation rule.
%
\begin{flalign*}
  &1 +_c \|\lambda a.a\| \\
  &= 1  +_c   \langle 0, \lambda a. \| a \|\rangle \\
  &=  \langle 1, \lambda a. \langle 0,a \rangle\rangle 
\end{flalign*}
%
% BEGIN CONS BRANCH
%
The translation of the \T{Cons} branch is a slightly more involved. The rule
for translating \T{split} is
%
\[ \|\T{split}(e_0,x_0.x_1.e_1)\| = \|e_0\|_c +_c \|e_1\|[\pi_0\|e_0\|_p/x_0, \pi_1\|e_0\|_p/x_1] \]
%
After applying the rule to the \T{Cons} branch we get
%
\begin{flalign*}
  &1 +_c \|\T{split}(b,x.c.\T{split}(c,xs'.r.\lambda a.\T{force}(r)\ \T{Cons} \langle x,a \rangle )) \| \\
  &= 1 +_c \|b\|_c +_c \|\T{split}(c,xs'.r.\lambda a.\T{force}(r)\ \T{Cons}\langle x,a \rangle) \|[\pi_0 \| b \|_p/x,\pi_1 \| b \|_p/c] 
\end{flalign*}
%
Remember that $b$ is a variable and has type
$\phi_\T{Cons}[\T{list} \times \T{susp (list} \to \T{list)}]$.
The translation of this type is 
$\textbf{C} \times \llangle \phi_\T{Cons} \rrangle [\T{list} \times \langle \T{list} \to \langle \textbf{C} \times \T{list} \rangle\rangle]$.
We can say that \T{$\pi_0\|$b$\|_p$} is the head of the list \T{xs},
\T{$\pi_0\pi_1\|$b$\|_p$} is the tail of the list \T{xs}, and
\T{$\pi_1\pi_1\|$b$\|_p$} is the result of the recursive call.
The translation of $b$ is $\langle 0, b\rangle$.
%
\begin{flalign*}
  &1 +_c \|b\|_c +_c \|\T{split}(c,xs'.r.\lambda a.\T{force}(r)\ \T{Cons}\langle x,a \rangle) \|[\pi_0 \| b \|_p/x,\pi_1 \| b \|_p/c] \\
  &=1 +_c \|\T{split}(c,xs'.r.\lambda a.\T{force}(r)\ \T{Cons}\langle x,a \rangle) \|[\pi_0 \| b \|_p/x,\pi_1 \| b \|_p/c] \\
  %
  &\qquad \text{We apply the rule for \T{split} again.} \\
  %
  &=1 +_c (\|c\|_c +_c \|\lambda a.\T{force}(r)\ \T{Cons}\langle x,a \rangle\|[\pi_0 \|c\|_p/xs', \pi_1\|c\|_p/r][\pi_0 \| b \|_p/x,\pi_1 \| b \|_p/c] \\
  %
  &\qquad \text{$c$ is a variable, so its translation is $\langle 0, c \rangle$.} \\
  %
  &=1 +_c \|\lambda a.\T{force}(r)\ \T{Cons}\langle x,a \rangle\|[\pi_0 \|c\|_p/xs', \pi_1\|c\|_p/r][\pi_0 \| b \|_p/x,\pi_1 \| b \|_p/c] \\
  %
  &\qquad \text{We apply the rule for abstraction.} \\
  %
  &=1 +_c \langle 0, \lambda a.\|\T{force}(r)\ \T{Cons}\langle x,a \rangle\|[\pi_0 \|c\|_p/xs', \pi_1\|c\|_p/r][\pi_0 \| b \|_p/x,\pi_1 \| b \|_p/c] \\
  %
  &\qquad \text{Recall $C +_c E$ is a macro for $\langle C + E_c, E_p\rangle$. We use this to eliminate the $+_c$.} \\
  &\qquad \text{We also apply the translation rule for application.} \\
  %
  &=\langle 1, \lambda a.(1 + \|\T{force}(r)\|_c + \|\T{Cons}\langle x,a\rangle\|_c) \\
  &\quadfive +_c \|\T{force}(r)\|_p \|\T{Cons}\langle x,a \rangle\|_p\rangle[\pi_0 \|c\|_p/xs', \pi_1\|c\|_p/r][\pi_0 \| b \|_p/x,\pi_1 \| b \|_p/c] \\
\end{flalign*}
%
% COMPOSE SUBSTITUTIONS
%
\begin{flalign*}
  &\text{We will translate $\T{force}(r)$ and $\T{Cons}\langle x,a\rangle$ individually.} \\
  &\text{First we compose the two substitutions.} \\
  %
  &\text{let } \Theta = [\pi_0 \|c\|_p/xs', \pi_1\|c\|_p/r][\pi_0 \| b \|_p/x,\pi_1 \| b \|_p/c] \\
  &\quadthree = [\pi_0\pi_1 \|b\|_p/xs', \pi_1\pi_1\|b\|_p/r, \pi_0 \| b \|_p/x] \\
  &\quad \text{Since $b$ is a variable, the potential  of its translation is $b$.} \\
  &\Theta = [\pi_0\pi_1 b/xs', \pi_1\pi_1 b/r, \pi_0 b/x] \\
\end{flalign*}
%
% FORCE
%
\begin{flalign*}
  &\quad \text{In translation of $\T{force}(r)$ we apply the rule $\|\T{force}(e)\| = \|e\|_c +_c \|e\|_p$.}\\
  &\quadthree \|\T{force}(r)\|\Theta = \|r\|_c\Theta +_c \|r\|_p\Theta \\
  %
  &\quad \text{We apply the variable translation rule to $r$, then apply the substitution $\Theta$.}\\
  %
  &\quadfive = \langle 0, r \rangle_c\Theta +_c \langle 0, r \rangle_p \Theta \\
  &\quadfive = r\Theta = \pi_1\pi_1 b \\
\end{flalign*}
%
% CONS
%
\begin{flalign*}
  &\quad \text{Next we do the translation of $\T{Cons}\langle x,a \rangle$.} \\
  &\quadthree \|\T{Cons}\langle x, a\rangle\| = \langle \|\langle x,a \rangle\|_c, \T{Cons} \|\langle x,a \rangle\|_p\rangle \\
  %
  &\quad \text{Notice the translation of $\langle x,a \rangle$ appears twice, so we will do this seperately.} \\
  %
  &\quadfour \|\langle x,a \rangle\| = \langle \|x\|_c + \|a\|_c, \langle \|x\|_p, \|a\|_p \rangle \rangle \Theta\\
  %
  &\quad \text{Both $x$ and $a$ are variables, so they have $0$ cost.}\\
  %
  &\quadsix = \langle 0, \langle x, a \rangle\rangle \Theta\\
  %
  &\quad \text{We apply the substitution $\Theta$.}\\
  %
  &\quadsix = \langle 0, \langle \pi_1 b, \pi_1\pi_1 b \rangle\rangle \\
  %
  &\quadsix = \langle 0, \langle \pi_1 b, \pi_1\pi_1 b \rangle\rangle \\
  %
  &\quad \text{We complete the translation of $\T{Cons}\langle x, a\rangle$ using $\langle x, a \rangle$.} \\
  %
  &\quadthree \|\T{Cons}\langle x, a\rangle\| = \langle \|\langle x,a \rangle\|_c, \T{Cons} \|\langle x,a \rangle\|_p\rangle \\
  &\quadfive = \langle 0, \T{Cons} \langle \pi_1 b, \pi_1\pi_1 b \rangle\rangle \\
\end{flalign*}
%
% END CONS BRANCH
%
\begin{flalign*}
  &\quad \text{We use substitute in the translations of $\T{force}(r)$ and $\T{Cons}\langle x, a\rangle$.}\\
  &\quad \text{$\|\T{force}(r)\|$ has cost $(\pi_1\pi_1 b)_c$ and $\|\T{Cons}\langle x,a\rangle\|$ has cost $0$.}\\
  &\langle 1, \lambda a.(1 + \|\T{force}(r)\|_c + \|\T{Cons}\langle x,a\rangle\|_c) +_c \|\T{force}(r)\|_p \|\T{Cons}\langle x,a \rangle\|_p\rangle \rangle\Theta \\
  %
  &= \langle 1, \lambda a.(1 + (\pi_1\pi_1 b)_c) +_c (\pi_1\pi_1 b)_p\ \T{Cons}\langle \pi_1 b, a \rangle\rangle \\
\end{flalign*}
%
% END REC
%
\begin{flalign*}
  &\text{We can now complete the translation of the \T{rec} expression.} \\
  &\|\T{rec}(xs, \T{Nil}\mapsto\lambda a.a, \\
  &\qquad \T{Cons}\mapsto b.\T{split}(b,x.c.\T{split}(c,xs'.r.\lambda a.\T{force}(r)\ \T{Cons}\langle x,a\rangle)))\| \\
  &= \T{rec}(xs, \T{Nil} \mapsto 1 +_c \|\lambda a.a\| \\
  &\quadthree \T{Cons}\mapsto b. 1 +_c \|\T{split}(b,x.c.\T{split}(c,xs'.r.\lambda a.\T{force}(r)\ \T{Cons}\langle x,a\rangle))\|) \\
  &= \T{rec}(xs, \T{Nil} \mapsto \langle 1, \lambda a. \langle 0,a \rangle\rangle \\
  &\quadthree \T{Cons}\mapsto b.\langle 1, \lambda a.(1 + (\pi_1\pi_1 b)_c) +_c (\pi_1\pi_1 b)_p\ \T{Cons}\langle \pi_1 b, a \rangle\rangle) \\
\end{flalign*}
%
% END APP FUNCTION
%
\begin{flalign*}
  &\text{We substitute the translation of \T{rec} and \T{Nil} into the translation of the application.}\\
  &\text{Let }R = \T{rec}(xs, \T{Nil} \mapsto \langle 1, \lambda a. \langle 0,a \rangle\rangle \\
  &\quadfive \T{Cons}\mapsto b.\langle 1, \lambda a.(1 + (\pi_1\pi_1 b)_c) +_c (\pi_1\pi_1 b)_p\ \T{Cons}\langle \pi_1 b, a \rangle\rangle) \\
  &\|\T{rec}(xs, \T{Nil}\mapsto\lambda a.a, \\
  &\qquad \T{Cons}\mapsto b.\T{split}(b,x.c.\T{split}(c,xs'.r.\lambda a.\T{force}(r)\ \T{Cons}\langle x,a\rangle)))\ \T{Nil}\| \\
  &\text{Substituting $R$ for the translation of \T{rec} and $\langle 0, \T{Nil}\rangle$ for the translation of \T{Nil}.} \\
  &\quad = (1 + R_c) +_c R_p\ \T{Nil} \rangle\\
  &\text{Recall }C +_c E = \langle C + E_c, E_p \rangle, \text{ so } (1 + E_c) +_c E_p = 1 +_c E \\
  &\quad = 1 +_c \T{rec}(xs, \T{Nil} \mapsto \langle 1, \lambda a. \langle 0,a \rangle\rangle \\
  &\quadthree \T{Cons}\mapsto b.\langle 1, \lambda a.(1 + (\pi_1\pi_1 b)_c) +_c (\pi_1\pi_1 b)_p\ \T{Cons}\langle \pi_1 b, a \rangle\rangle)\ \T{Nil}\\
\end{flalign*}
%
% END APP
%
\begin{flalign*}
  &\text{Finally, we substitute this into the translation of \T{rev}.} \\
  \|\T{rev}\| &= \|(\lambda\T{xs.rec(xs, Nil}\mapsto\lambda\T{a.a,} \\
              &\quad \T{Cons}\mapsto\T{b.split(b,x.c.split(c,xs'.r.}\lambda\T{a.force(r) Cons}\langle\T{x,a}\rangle\T{)))) Nil}\| \\
  &\quad = \langle 0, \lambda xs. 1 +_c \T{rec}(xs, \T{Nil} \mapsto \langle 1, \lambda a. \langle 0,a \rangle\rangle \\
  &\quadthree \T{Cons}\mapsto b.\langle 1, \lambda a.(1 + (\pi_1\pi_1 b)_c) +_c (\pi_1\pi_1 b)_p\ \T{Cons}\langle \pi_1 b, a \rangle\rangle)\ \T{Nil}\rangle\\
\end{flalign*}
%
% END REV TRANSLATION SPLIT
%
Observe that $\|\T{rev}\|$ admits the same syntactic sugar as \T{rev}. In the
complexity language, instead of taking projections of $b$, we can use the same
pattern matching syntactic sugar as in the source language.

\begin{flalign*}
  &\|\T{rev}\| = \langle 0, \lambda xs. 1 +_c \T{rec}(xs, \T{Nil} \mapsto \langle 1, \lambda a. \langle 0,a \rangle\rangle \\
  &\quadthree \T{Cons}\mapsto \langle x, \langle xs', r\rangle\rangle.\langle 1, \lambda a.(1 + r_c) +_c r_p\ \T{Cons}\langle \pi_1 x, a \rangle\rangle)\ \T{Nil}\rangle\\
\end{flalign*}


%
% FAST REVESE TRANSLATION USING SYNTACTIC SUGAR
%
We walk through the same translation of fast reverse, but we use the syntactic
sugar for matching introducted earlier. Recall the implementation of fast using
syntactic sugar. The translation is almost identicall to the translation of \T{rev}
written without syntactic sugar until we translate the \T{Cons} branch of the
\T{rec}.
%
\begin{flalign*}
  \|\T{rev}\| &= \|\lambda\T{xs.rec(xs, Nil}\mapsto\lambda\T{a.a,} \\
              &\quad \T{Cons}\mapsto\langle x, \langle xs', r\rangle\rangle.\lambda a.\T{force}(r)\ \T{Cons}\langle x, a\rangle)\ \T{Nil}\|
\end{flalign*}
%
First we apply the rule for translating an abstraction. The rule is
$\|\lambda x. e\| = \langle 0, \lambda x. \|e\|\rangle$.
%
\begin{flalign*}
  \|\T{rev}\| &= \langle 0, \lambda xs. \|\T{rec}(xs, \T{Nil}\mapsto\lambda a.a, \\
              &\quad \T{Cons}\mapsto\langle x, \langle xs', r\rangle\rangle.\lambda a.\T{force}(r)\ \T{Cons}\langle x, a\rangle))\ \T{Nil}\|\rangle
\end{flalign*}
%
%
% BEGIN APP
%
Next we apply the rule for translating an application. The rule is
$\|e_0\ e_1\| = (1 + \|e_0\|_c + \|e_1\|_c) +_c (\|e_0\|_p\ \|e_1\|_p)$.
In this case, \T{rec(...)} is $e_0$ and \T{Nil} is $e_1$. We translate
\T{Nil} then \T{rec(...)} seperately.
%
% APP ARGUMENT
%
The translation of a constructor applied to an expression is a tuple of the
cost of the translated expression and the corresponding complexity language
constructor applied to the potential of the translated expression. Since the
expression inside \T{Nil} is $\langle\rangle$, and
$\|\langle\rangle\| = \langle 0,\langle\rangle\rangle$, we have
%
\begin{flalign*}
  |\T{Nil}\| &= \langle\langle 0, \langle\rangle\rangle_c, \T{Nil}\langle 0,\langle\rangle\rangle_p\rangle \\
             &= \langle 0, \T{Nil}\langle\rangle\rangle
\end{flalign*}
%
% BEGIN REC
%
The rule for translating a \T{rec} expression is
\[
  \|\T{rec}(e,\overline{C \mapsto x.e_C})\| = \|e\|_c +_c \T{rec}(\|e\|_p, \overline{C \mapsto x.\|e_C\|})
\]
%
\begin{flalign*}
  &\|\T{rec}(xs, \T{Nil}\mapsto\lambda a.a, \\
  &\qquad \T{Cons}\mapsto b.\T{split}(b,x.c.\T{split}(c,xs'.r.\lambda a.\T{force}(r)\ \T{Cons}\langle x,a\rangle)))\| \\
  &= \|xs\|_c +_c \T{rec}(\|xs\|_p, \T{Nil} \mapsto 1 +_c \|\lambda a.a\| \\
  &\quadthree \T{Cons}\mapsto b. 1 +_c \|\T{split}(b,x.c.\T{split}(c,xs'.r.\lambda a.\T{force}(r)\ \T{Cons}\langle x,a\rangle))\|) \\
  &= \langle 0, xs \rangle_c +_c \T{rec}(\langle 0, xs\rangle_p, \T{Nil} \mapsto 1 +_c \|\lambda a.a\| \\
  &\quadthree \T{Cons}\mapsto b. 1 +_c \|\T{split}(b,x.c.\T{split}(c,xs'.r.\lambda a.\T{force}(r)\ \T{Cons}\langle x,a\rangle))\|) \\
  &\text{The term $xs$ is a variable and the rule for translating variables is $\|xs\| = \langle 0, xs\rangle$.} \\
  &= \T{rec}(xs, \T{Nil} \mapsto 1 +_c \|\lambda a.a\| \\
  &\quadthree \T{Cons}\mapsto b. 1 +_c \|\T{split}(b,x.c.\T{split}(c,xs'.r.\lambda a.\T{force}(r)\ \T{Cons}\langle x,a\rangle))\|)
\end{flalign*}
%
% NIL BRANCH
%
The translation of the \T{Nil} branch is the same as before.
%
\begin{flalign*}
  &1 +_c \|\lambda a.a\| =  \langle 1, \lambda a. \langle 0,a \rangle\rangle 
\end{flalign*}
%
% BEGIN CONS BRANCH
%
The translation of the \T{Cons} branch where the difference lies.
%
\begin{flalign*}
  &1 +_c \|\lambda a.\T{force}(r)\ \T{Cons}\langle x, a\rangle)\| \\
  &\quad = 1 +_c \langle 0, \lambda a.\|\T{force}(r)\ \T{Cons}\langle x, a\rangle)\| \\
  &\quad = \langle 1, \lambda a.(1 + \|\T{force}(r)\|_c + \|\T{Cons}\langle x, a\rangle)\|_c) +_c \|\T{force}(r)\|_p\ \|\T{Cons}\langle x, r \rangle\|_p \rangle \\
\end{flalign*}
%
The translation of $\T{force}(r)$ and $\T{Cons}\langle x, a \rangle$
are the same as before, except we do not have a substitution to apply.
%
\begin{flalign*}
  &\|\T{force}(r)\| = \|r\|_c +_c \|r\|_p = \langle 0, r \rangle_c +_c \langle 0, r \rangle_p = 0 +_c r = r
\end{flalign*}
%
\begin{flalign*}
  &\|\T{Cons}\langle x, a\rangle\| =  \langle 0, \T{Cons} \langle x, a \rangle\rangle \\
\end{flalign*}
%
So the complete translation of the \T{Cons} branch is
%
\begin{flalign*}
  &1 +_c \|\lambda a.\T{force}(r)\ \T{Cons}\langle x, a\rangle)\| \\
  &\quad = 1 +_c \langle 0, \lambda a.\|\T{force}(r)\ \T{Cons}\langle x, a\rangle)\| \\
  &\quad = \langle 1, \lambda a.(1 + \|\T{force}(r)\|_c + \|\T{Cons}\langle x, a\rangle)\|_c) +_c \|\T{force}(r)\|_p\ \|\T{Cons}\langle x, r \rangle\|_p \rangle \\
  &\quad = \langle 1, \lambda a.(1 + r_c + 0) +_c r_p\ \T{Cons}\langle x, r \rangle \rangle \\
  &\quad = \langle 1, \lambda a.(1 + r_c) +_c r_p\ \T{Cons}\langle x, r \rangle \rangle \\
\end{flalign*}
%
The complete translation of the \T{rec} becomes
%
\begin{flalign*}
  &\|\T{rec}(xs, \T{Nil}\mapsto\lambda a.a, \\
  &\qquad \T{Cons}\mapsto  \langle x, \langle xs', r\rangle\rangle.\lambda a.\T{force}(r)\ \T{Cons}\langle x, a\rangle)\| \\
  &= \T{rec}(xs, \T{Nil} \mapsto 1 +_c \|\lambda a.a\| \\
  &\quadthree \T{Cons}\mapsto \langle x, \langle xs', r\rangle\rangle.1 +_c \|\lambda a.\T{force}(r)\ \T{Cons}\langle x, a\rangle\|) \\
  &= \T{rec}(xs, \T{Nil} \mapsto \langle 0, \lambda a. \langle 0, a \rangle \rangle \\
  &\quadthree \T{Cons}\mapsto \langle x, \langle xs', r\rangle\rangle. \langle 1, \lambda a.(1 + r_c) +_c r_p\ \T{Cons}\langle x, r \rangle \rangle \\
\end{flalign*}
%
We substitute the translations of \T{rec(..)} and \T{Nil} into the application.
%
\begin{flalign*}
  &\text{Let }R = \T{rec}(xs, \T{Nil} \mapsto \langle 0, \lambda a. \langle 0, a \rangle \rangle \\
  &\quadthree \T{Cons}\mapsto \langle x, \langle xs', r\rangle\rangle. \langle 1, \lambda a.(1 + r_c) +_c r_p\ \T{Cons}\langle x, r \rangle \rangle \\
  &\|\T{rec}(xs, \T{Nil}\mapsto\lambda a.a, \\
  &\qquad \T{Cons}\mapsto \langle x, \langle xs',r \rangle\rangle.\lambda a.\T{force}(r)\ \T{Cons}\langle x, a \rangle)\ \T{Nil}\| \\
  &\text{Substituting $R$ for the translation of \T{rec} and $\langle 0, \T{Nil}\rangle$ for the translation of \T{Nil}.} \\
  &\quad = (1 + R_c) +_c R_p\ \T{Nil} \rangle\\
  &\quad = 1 +_c \T{rec}(xs, \T{Nil} \mapsto \langle 0, \lambda a. \langle 0, a \rangle \rangle \\
  &\quadthree \T{Cons}\mapsto \langle x, \langle xs', r\rangle\rangle. \langle 1, \lambda a.(1 + r_c) +_c r_p\ \T{Cons}\langle x, r \rangle \rangle)\ \T{Nil} \\
\end{flalign*}
%
And our complete translation of \T{rev} is
%
\begin{flalign*}
  \|\T{rev}\| &= \|\lambda\T{xs.rec(xs, Nil}\mapsto\lambda\T{a.a,} \\
              &\qquad \T{Cons}\mapsto\langle x, \langle xs', r\rangle\rangle.\lambda a.\T{force}(r)\ \T{Cons}\langle x, a\rangle)\ \T{Nil}\| \\
              &= \langle 0, \lambda xs. \|\T{rec}(xs, \T{Nil}\mapsto\lambda a.a, \\
              &\quad \T{Cons}\mapsto\langle x, \langle xs', r\rangle\rangle.\lambda a.\T{force}(r)\ \T{Cons}\langle x, a\rangle)\ \T{Nil}\| \rangle \\
              &= \langle 0, \lambda xs. 1 +_c \T{rec}(xs, \T{Nil} \mapsto \langle 0, \lambda a. \langle 0, a \rangle \rangle \\
              &\qquad \T{Cons}\mapsto \langle x, \langle xs', r\rangle\rangle. \langle 1, \lambda a.(1 + r_c) +_c r_p\ \T{Cons}\langle x, r \rangle \rangle)\ \T{Nil} \rangle\\
\end{flalign*}
%
This is the same as the translation of \T{rev} without the syntactic sugar. We
will use the syntactic sugar for the rest of this thesis.
%
% END FAST REVERSE SYNTACTIC SUGAR TRANSLATION
%
%
% FAST REVERSE INTERPRETATION
%
The interpretation of \T{rev} is rather dull as the cost of \T{rev} is always null.
Instead of interpreting \T{rev}, we will interpret \T{rev} applied to a list \T{xs}.
Below is the translation of \T{rev xs}.
%
\begin{flalign*}
  &\|\T{rev xs}\| = (1 + \|\T{rev}\|_c + \|xs\|_c) +_c \|\T{rev}\|_p\ \|xs\|_p \\
  &\text{The cost of $\|\T{rev}\|$ is $0$, and we will let \T{xs} be a value, which has $0$ cost.} \\
  &= (1 + 0 + 0) +_c \|\T{rev}\|_p\ xs \\
  &= 1 +_c (\lambda xs. 1 +_c \T{rec}(xs, \T{Nil} \mapsto \langle 0, \lambda a. \langle 0, a \rangle \rangle \\
  &\qquad \T{Cons}\mapsto \langle x, \langle xs', r\rangle\rangle. \langle 1, \lambda a.(1 + r_c) +_c r_p\ \T{Cons}\langle x, r \rangle \rangle)\ \T{Nil})\ xs\\
\end{flalign*}
%
We intepret the size of an \T{list} to be the number of list constructors.
%
\begin{flalign*}
  \llbracket \T{list} \rrbracket &= \mathbb{N}^\infty\\
  D^{list} &= \{\ast\} + \{1\} \times \mathbb{N}^\infty\\
  size_{list}(\T{Nil}) &= 1\\
  size_{list}(\T{Cons(1,n)}) &= 1 + n\\
\end{flalign*}
%
The interpretation of \T{rev xs} proceeds as follows.
\begin{lstlisting}
  $\llbracket \|$rev xs$\| \rrbracket = \llbracket 1 +_c $rec(xs,$\dots$)$_p$ Nil$ \rrbracket$
  = $\llbracket \langle 1 + ($rec(xs,$\dots$)$_p$ Nil$)_c, ($rec(xs,$\dots$)$_p$ Nil$)_p \rangle \rrbracket$
  = $\langle 1 + \llbracket ($rec(xs,$\dots$)$_p$ Nil$)_c \rrbracket, \llbracket ($rec(xs,$\dots$)$_p$ Nil$)_p \rrbracket \rangle$
\end{lstlisting}

We will focus on the interpretation of the auxilary function \T{rec(xs,$\dots$)}.

Let $g(n) = \llbracket \T{rec(xs,$\dots$)} \rrbracket \{xs \mapsto n\}$

\[g(n) = \bigvee_{size\ ys \leq n} case(ys, Nil \mapsto \langle1,\lambda a.\langle 0,a\rangle\rangle, Cons \mapsto \langle 1,m \rangle.\langle 1, \lambda a. 2 +_c \pi_1g(m) (a+1)\rangle)\]

For $n=0$, $g(0) = \langle 1,\lambda a.\langle 0,a\rangle\rangle$.

For $n>0$,
\[g(n+1) = \bigvee_{size\ ys \leq n+1} case(ys, Nil \mapsto \langle1,\lambda a.\langle 0,a\rangle\rangle, Cons \mapsto \langle 1,m \rangle.\langle 1, \lambda a. 2 +_c \pi_1g(m) (a+1)\rangle)\]

\[ g(n+1) = \bigvee_{size\ ys \leq n} case(ys, Nil \mapsto \langle1,\lambda a.\langle 0,a\rangle\rangle, Cons \mapsto \langle 1,m \rangle.\langle 1, \lambda a. 2 +_c \pi_1g(m) (a+1)\rangle) \]
\[ \vee \bigvee_{size\ ys = n+1} case(ys, Nil \mapsto \langle1,\lambda a.\langle 0,a\rangle\rangle, Cons \mapsto \langle 1,m \rangle.\langle 1, \lambda a. 2 +_c \pi_1g(m) (a+1)\rangle) \]
 
\[ g(n+1) = g(n) \vee \bigvee_{size\ ys = n+1} case(ys, Nil \mapsto \langle1,\lambda a.\langle 0,a\rangle\rangle, Cons \mapsto \langle 1,m \rangle.\langle 1, \lambda a. 2 +_c \pi_1g(m) (a+1)\rangle) \]

\[ g(n+1) = g(n) \vee \langle 1, \lambda a. 2 +_c \pi_1g(n) (a+1)\rangle)\]

We want to show that $g$ is monotonically increasing; $\forall n.g(n) \leq g(n+1)$.
By definition of $\leq$, $g(n) \leq g(n+1) \Leftrightarrow \pi_0 g(n) \leq \pi_0 g(n+1) \land \pi_1 g(n) \leq \pi_1 g(n+1)$.
First we will show $\forall n. \pi_0 g(n) = 1$, the immediate corollary of which is $\forall n. \pi_0 g(n) \leq \pi_0 g(n+1)$.
\begin{proof}
We prove this by induction on $n$.
  \begin{description}
    \item[Base case: $n=0$]\hfill \\
      By definition, $\pi_0 g(0) = 1$.
    \item[Induction step: $n>0$]\hfill \\
      By definition $\pi_0 g(n+1) = \pi_0 (g(n) \vee \langle 1, \lambda a. 2 +_c \pi_1g(n) (a+1)\rangle)$.
      We distribute the projection over the max: $\pi_0 g(n+1) = \pi_0 g(n) \vee 1$.
      By the induction hypothesis, $\pi_0 g(n) = 1$, so $\pi_0 g(n+1) = 1$.
  \end{description}
\end{proof}
Now we argue that $\pi_1g(n) \leq \pi_1 g(n+1)$.
First we prove the lemma $\forall n.\pi_1 g(n) a \leq \pi_1 g(n) (a+1)$.
\begin{proof}
  We prove this by induction on $n$.
  \begin{description}
    \item[$n=0$]\hfill \\
      $\pi_1 g(0) a = \langle 0,a\rangle \leq \pi_1 g(0) (a+1) = \langle 0,a+1 \rangle$.
    \item[$n>0$]\hfill \\
      We assume $\pi_1 g(n) a \leq \pi_1 g(n) (a+1)$.
      \[ \pi_1 g(n) a \leq \pi_1 g(n) (a+1) \]
      \[ \pi_1 g(n) a \vee 2 +_c g(n) a \leq \pi_1 g(n) (a+1) \vee 2 +_c g(n) (a+1) \]
      \[ \pi_1 g(n+1) a \leq \pi_1 g(n+1) (a+1) \]
  \end{description}
\end{proof}

Now we show $\pi_1 g(n) \leq \pi_1 g(n+1)$.
\begin{proof}
  By reflexivity, $\pi_1 g(n) \leq \pi_1 g(n)$.
  By the lemma we just proved:
  \[ \pi_1 g(n) a \leq \pi_1 g(n) (a+1) \]
  \[ \pi_1 g(n) a \leq 2 +_c \pi_1 g(n) (a+1) \]
  %\[ \pi_1 g(n) a \leq \langle 2 + \pi_0 (\pi_1 g(n) (a+1)), \pi_1 (\pi_1 g(n) (a+1)) \rangle \]
  \[ \lambda a.\pi_1 g(n) a \leq \lambda a. 2 +_c \pi_1 g(n) (a+1) \]
\end{proof}

So since for all $n$, $\pi_0 g(n) = 1$ and $\pi_1 g(n) \leq \lambda a. 2 +_c \pi_1 g(n) (a+1)$, we can say

\[ g(n) \leq \langle 1, \lambda a. 2 +_c \pi_1g(n) (a+1)\rangle) \]

So 
\[ g(n+1) = \langle 1, \lambda a. 2 +_c \pi_1g(n) (a+1)\rangle\]


To extract a recurrence from $g$, we apply $g$ to the interpretation of a list $a$.

Let $h(n,a) = \pi_1 g(n) a$

For $n=0$
\begin{align*}
h(0,a) &= \pi_1 g(0) a \\
&= (\lambda a.\langle 0,a\rangle) a \\
&= \langle 0, a\rangle
\end{align*}
For $n>0$
\begin{align*}
h(n,a) &= \pi_1 g(n) a \\ 
&= (\lambda a. 2 +_c \pi_1g(n-1) (a+1)) a \\
&= 2 +_c \pi_1 g(n-1) (a+1)) \\
&= 2 +_c h(n-1,a+1) \\
&= \langle 2 + \pi_0 h(n-1,a+1), \pi_1 h(n-1,a+1)\rangle
\end{align*}

From this recurrence, we can extract a recurrence for the cost. Let $h_c = \pi_0 \circ h$.

For $n=0$
\begin{align*}
h_c(0,a) &= \pi_0 h(0,a)\\
&= \pi_0 \langle 0, a\rangle\\
&= 0
\end{align*}
For $n>0$
\begin{align*}
h_c(n,a) &= \pi_0 \langle 2 + \pi_0 h(n-1,a+1), \pi_1 h(n-1,a+1)\rangle\\
&= 2 + \pi_0 h(n-1,a+1)\\
&= 2 + h_c(n-1,a+1)
\end{align*}

We now have a recurrence for the cost of the auxilary function \T{rec(xs,$\dots$)}:
\begin{framed}
  \begin{equation}
    h_c(n,a) = \begin{cases}
      0 & n = 0 \\
      2 + h_c(n-1,a+1) & n > 0
    \end{cases}
  \end{equation}
\end{framed}

\textbf{Theroem: $h_c(n,a) = 2n$}
\begin{proof}
  We prove this by induction on $n$.
  \begin{description}
    \item{Base case: $n=0$}\hfill \\
      \[ h_c(0,a) = 0 = 2\cdot0 \]
    \item{Induction case:}\hfill \\
      We assume $h_c(n,a+1) = 2n$.\[h_c(n+1,a) = 2 + h_c(n,a+1) = 2 + 2n = 2(n+1)\]
  \end{description}
\end{proof}  

The solution to the recurrence for the cost of the auxilary function \T{rec(xs,$\dots$)} is:
\begin{framed}
  \[h_c(n,a) = 2n \]
\end{framed}


We can also extract a recurrence for the potential. Let $h_p = \pi_1 \circ h$.

For $n=0$
\begin{align*}
h_p(0,a) &= \pi_1 h(0,a)\\
&= \pi_1 \langle 0, a\rangle\\
&= a
\end{align*}
For $n>0$
\begin{align*}
h_p(n,a) &= \pi_1 \langle 2 + \pi_0 h(n-1,a+1), \pi_1 h(n-1,a+1)\rangle\\
&= \pi_1 h(n-1,a+1)\\
&= h_p (n-1,a+1)
\end{align*}

We now have a recurrence for the potential of the auxilary function in \T{rev xs}:
\begin{framed}
  \begin{equation}
    h_p(n,a) = \begin{cases}
      a & n = 0 \\
      h_p(n-1,a+1) & n > 0
    \end{cases}
  \end{equation}
\end{framed}

\textbf{Theroem: $h_p(n,a) = n + a$}
\begin{proof}
  We prove this by induction on $n$.
  \begin{description}
    \item{Base case: $n=0$}\hfill \\
      \[ h_p(0,a) = a \]
    \item{Induction case:}\hfill \\
      \[h_p(n,a) = h_p(n-1,a+1) = n - 1 + a + 1 = n + a\]
  \end{description}
\end{proof}  

So the solution to the recurrence for the potential of the auxilary function.
\begin{framed}
  \[h_p(n,a) = n + a \]
\end{framed}


We return to our interpretation of \T{rev xs}.
\begin{lstlisting}
  $\llbracket \|$rev xs$\| \rrbracket = \langle 1 + \llbracket ($rec(xs,$\dots$)$_p$ Nil$)_c \rrbracket, \llbracket ($rec(xs,$\dots$)$_p$ Nil$)_p \rrbracket \rangle$
  $= \langle 1 + \pi_0 (\llbracket ($rec(xs,$\dots$)$_p\rrbracket 0) , \pi_1 (\llbracket $rec(xs,$\dots$)$_p\rrbracket 0)\rangle$
  $= \langle 1 + \pi_0 (\pi_1 g(n)\ 0) , \pi_1 (\pi_1g(n)\ 0)\rangle \text{ where }n\text{ is the length of}$ xs
  $= \langle 1 + \pi_0 h(n,0) , \pi_1 h(n,0)\rangle$
  $= \langle 1 + h_c(n,0) , h_p(n,0)\rangle$
  $= \langle 1 + 2n , n\rangle$
\end{lstlisting}

This result tells us the cost of applying \T{rev} to a list \T{xs} of length $n$ is $1+2n$, and the resulting list has size $n$.
So \T{rev} = $\Theta(n)$.




\subsection*{Translation of \T{rev} using \T{split}}

\subsection*{Interpretation}

We intepret the size of an \T{list} to be the number of list constructors.
\begin{framed}
$\llbracket$ \T{list} $\rrbracket$ = $\mathbb{N}^\infty$\\
$D^{list} = \{\ast\} + \{1\} \times \mathbb{N}^\infty$\\
$size_{list}(\T{Nil}) = 1$\\
$size_{list}(\T{Cons(1,n)}) = 1 + n$\\
\end{framed}

The interpretation of \T{rev xs} proceeds as follows.
\begin{lstlisting}
  $\llbracket \|$rev xs$\| \rrbracket = \llbracket 1 +_c $rec(xs,$\dots$)$_p$ Nil$ \rrbracket$
  = $\llbracket \langle 1 + ($rec(xs,$\dots$)$_p$ Nil$)_c, ($rec(xs,$\dots$)$_p$ Nil$)_p \rangle \rrbracket$
  = $\langle 1 + \llbracket ($rec(xs,$\dots$)$_p$ Nil$)_c \rrbracket, \llbracket ($rec(xs,$\dots$)$_p$ Nil$)_p \rrbracket \rangle$
\end{lstlisting}

We will focus on the interpretation of the auxilary function \T{rec(xs,$\dots$)}.

Let $g(n) = \llbracket \T{rec(xs,$\dots$)} \rrbracket \{xs \mapsto n\}$

\[g(n) = \bigvee_{size\ ys \leq n} case(ys, Nil \mapsto \langle1,\lambda a.\langle 0,a\rangle\rangle, Cons \mapsto \langle 1,m \rangle.\langle 1, \lambda a. 2 +_c \pi_1g(m) (a+1)\rangle)\]

For $n=0$, $g(0) = \langle 1,\lambda a.\langle 0,a\rangle\rangle$.

For $n>0$,
\[g(n+1) = \bigvee_{size\ ys \leq n+1} case(ys, Nil \mapsto \langle1,\lambda a.\langle 0,a\rangle\rangle, Cons \mapsto \langle 1,m \rangle.\langle 1, \lambda a. 2 +_c \pi_1g(m) (a+1)\rangle)\]

\[ g(n+1) = \bigvee_{size\ ys \leq n} case(ys, Nil \mapsto \langle1,\lambda a.\langle 0,a\rangle\rangle, Cons \mapsto \langle 1,m \rangle.\langle 1, \lambda a. 2 +_c \pi_1g(m) (a+1)\rangle) \]
\[ \vee \bigvee_{size\ ys = n+1} case(ys, Nil \mapsto \langle1,\lambda a.\langle 0,a\rangle\rangle, Cons \mapsto \langle 1,m \rangle.\langle 1, \lambda a. 2 +_c \pi_1g(m) (a+1)\rangle) \]
 
\[ g(n+1) = g(n) \vee \bigvee_{size\ ys = n+1} case(ys, Nil \mapsto \langle1,\lambda a.\langle 0,a\rangle\rangle, Cons \mapsto \langle 1,m \rangle.\langle 1, \lambda a. 2 +_c \pi_1g(m) (a+1)\rangle) \]

\[ g(n+1) = g(n) \vee \langle 1, \lambda a. 2 +_c \pi_1g(n) (a+1)\rangle)\]

We want to show that $g$ is monotonically increasing; $\forall n.g(n) \leq g(n+1)$.
By definition of $\leq$, $g(n) \leq g(n+1) \Leftrightarrow \pi_0 g(n) \leq \pi_0 g(n+1) \land \pi_1 g(n) \leq \pi_1 g(n+1)$.
First we will show $\forall n. \pi_0 g(n) = 1$, the immediate corollary of which is $\forall n. \pi_0 g(n) \leq \pi_0 g(n+1)$.
\begin{proof}
We prove this by induction on $n$.
  \begin{description}
    \item[Base case: $n=0$]\hfill \\
      By definition, $\pi_0 g(0) = 1$.
    \item[Induction step: $n>0$]\hfill \\
      By definition $\pi_0 g(n+1) = \pi_0 (g(n) \vee \langle 1, \lambda a. 2 +_c \pi_1g(n) (a+1)\rangle)$.
      We distribute the projection over the max: $\pi_0 g(n+1) = \pi_0 g(n) \vee 1$.
      By the induction hypothesis, $\pi_0 g(n) = 1$, so $\pi_0 g(n+1) = 1$.
  \end{description}
\end{proof}
Now we argue that $\pi_1g(n) \leq \pi_1 g(n+1)$.
First we prove the lemma $\forall n.\pi_1 g(n) a \leq \pi_1 g(n) (a+1)$.
\begin{proof}
  We prove this by induction on $n$.
  \begin{description}
    \item[$n=0$]\hfill \\
      $\pi_1 g(0) a = \langle 0,a\rangle \leq \pi_1 g(0) (a+1) = \langle 0,a+1 \rangle$.
    \item[$n>0$]\hfill \\
      We assume $\pi_1 g(n) a \leq \pi_1 g(n) (a+1)$.
      \[ \pi_1 g(n) a \leq \pi_1 g(n) (a+1) \]
      \[ \pi_1 g(n) a \vee 2 +_c g(n) a \leq \pi_1 g(n) (a+1) \vee 2 +_c g(n) (a+1) \]
      \[ \pi_1 g(n+1) a \leq \pi_1 g(n+1) (a+1) \]
  \end{description}
\end{proof}

Now we show $\pi_1 g(n) \leq \pi_1 g(n+1)$.
\begin{proof}
  By reflexivity, $\pi_1 g(n) \leq \pi_1 g(n)$.
  By the lemma we just proved:
  \[ \pi_1 g(n) a \leq \pi_1 g(n) (a+1) \]
  \[ \pi_1 g(n) a \leq 2 +_c \pi_1 g(n) (a+1) \]
  %\[ \pi_1 g(n) a \leq \langle 2 + \pi_0 (\pi_1 g(n) (a+1)), \pi_1 (\pi_1 g(n) (a+1)) \rangle \]
  \[ \lambda a.\pi_1 g(n) a \leq \lambda a. 2 +_c \pi_1 g(n) (a+1) \]
\end{proof}

So since for all $n$, $\pi_0 g(n) = 1$ and $\pi_1 g(n) \leq \lambda a. 2 +_c \pi_1 g(n) (a+1)$, we can say

\[ g(n) \leq \langle 1, \lambda a. 2 +_c \pi_1g(n) (a+1)\rangle) \]

So 
\[ g(n+1) = \langle 1, \lambda a. 2 +_c \pi_1g(n) (a+1)\rangle\]


To extract a recurrence from $g$, we apply $g$ to the interpretation of a list $a$.

Let $h(n,a) = \pi_1 g(n) a$

For $n=0$
\begin{align*}
h(0,a) &= \pi_1 g(0) a \\
&= (\lambda a.\langle 0,a\rangle) a \\
&= \langle 0, a\rangle
\end{align*}
For $n>0$
\begin{align*}
h(n,a) &= \pi_1 g(n) a \\ 
&= (\lambda a. 2 +_c \pi_1g(n-1) (a+1)) a \\
&= 2 +_c \pi_1 g(n-1) (a+1)) \\
&= 2 +_c h(n-1,a+1) \\
&= \langle 2 + \pi_0 h(n-1,a+1), \pi_1 h(n-1,a+1)\rangle
\end{align*}

From this recurrence, we can extract a recurrence for the cost. Let $h_c = \pi_0 \circ h$.

For $n=0$
\begin{align*}
h_c(0,a) &= \pi_0 h(0,a)\\
&= \pi_0 \langle 0, a\rangle\\
&= 0
\end{align*}
For $n>0$
\begin{align*}
h_c(n,a) &= \pi_0 \langle 2 + \pi_0 h(n-1,a+1), \pi_1 h(n-1,a+1)\rangle\\
&= 2 + \pi_0 h(n-1,a+1)\\
&= 2 + h_c(n-1,a+1)
\end{align*}

We now have a recurrence for the cost of the auxilary function \T{rec(xs,$\dots$)}:
\begin{framed}
  \begin{equation}
    h_c(n,a) = \begin{cases}
      0 & n = 0 \\
      2 + h_c(n-1,a+1) & n > 0
    \end{cases}
  \end{equation}
\end{framed}

\textbf{Theroem: $h_c(n,a) = 2n$}
\begin{proof}
  We prove this by induction on $n$.
  \begin{description}
    \item{Base case: $n=0$}\hfill \\
      \[ h_c(0,a) = 0 = 2\cdot0 \]
    \item{Induction case:}\hfill \\
      We assume $h_c(n,a+1) = 2n$.\[h_c(n+1,a) = 2 + h_c(n,a+1) = 2 + 2n = 2(n+1)\]
  \end{description}
\end{proof}  

The solution to the recurrence for the cost of the auxilary function \T{rec(xs,$\dots$)} is:
\begin{framed}
  \[h_c(n,a) = 2n \]
\end{framed}


We can also extract a recurrence for the potential. Let $h_p = \pi_1 \circ h$.

For $n=0$
\begin{align*}
h_p(0,a) &= \pi_1 h(0,a)\\
&= \pi_1 \langle 0, a\rangle\\
&= a
\end{align*}
For $n>0$
\begin{align*}
h_p(n,a) &= \pi_1 \langle 2 + \pi_0 h(n-1,a+1), \pi_1 h(n-1,a+1)\rangle\\
&= \pi_1 h(n-1,a+1)\\
&= h_p (n-1,a+1)
\end{align*}

We now have a recurrence for the potential of the auxilary function in \T{rev xs}:
\begin{framed}
  \begin{equation}
    h_p(n,a) = \begin{cases}
      a & n = 0 \\
      h_p(n-1,a+1) & n > 0
    \end{cases}
  \end{equation}
\end{framed}

\textbf{Theroem: $h_p(n,a) = n + a$}
\begin{proof}
  We prove this by induction on $n$.
  \begin{description}
    \item{Base case: $n=0$}\hfill \\
      \[ h_p(0,a) = a \]
    \item{Induction case:}\hfill \\
      \[h_p(n,a) = h_p(n-1,a+1) = n - 1 + a + 1 = n + a\]
  \end{description}
\end{proof}  

So the solution to the recurrence for the potential of the auxilary function.
\begin{framed}
  \[h_p(n,a) = n + a \]
\end{framed}


We return to our interpretation of \T{rev xs}.
\begin{lstlisting}
  $\llbracket \|$rev xs$\| \rrbracket = \langle 1 + \llbracket ($rec(xs,$\dots$)$_p$ Nil$)_c \rrbracket, \llbracket ($rec(xs,$\dots$)$_p$ Nil$)_p \rrbracket \rangle$
  $= \langle 1 + \pi_0 (\llbracket ($rec(xs,$\dots$)$_p\rrbracket 0) , \pi_1 (\llbracket $rec(xs,$\dots$)$_p\rrbracket 0)\rangle$
  $= \langle 1 + \pi_0 (\pi_1 g(n)\ 0) , \pi_1 (\pi_1g(n)\ 0)\rangle \text{ where }n\text{ is the length of}$ xs
  $= \langle 1 + \pi_0 h(n,0) , \pi_1 h(n,0)\rangle$
  $= \langle 1 + h_c(n,0) , h_p(n,0)\rangle$
  $= \langle 1 + 2n , n\rangle$
\end{lstlisting}


This result tells us the cost of applying \T{rev} to a list \T{xs} of length $n$ is $1+2n$, and the resulting list has size $n$.
So \T{rev} = $\Theta(n)$.

\section{Reverse}

\subsection*{Source Language}

\begin{lstlisting}
datatype intlist = Nil of unit | Cons of int $\times$ intlist
\end{lstlisting}
The following function reverses a list on $\Theta(n^2)$ time.
It walks down a list, appending the head of the list to the end of the result of recursively calling itself on the tail of the list.
\begin{lstlisting}[frame=single]
rev = $\lambda$xs.rec(xs, Nil $\mapsto$ Nil
                , Cons $\mapsto \langle$x,$\langle$xs',r$\rangle \rangle$.
                       rec(force(r), Nil $\mapsto$ Cons$\langle$x, Nil$\rangle$
                                   , Cons $\mapsto \langle$y,$\langle$ys,rys$\rangle \rangle$. Cons$\langle$y,force(rys)$\rangle$))
\end{lstlisting}


\subsection*{Complexity Language}
The translation into the complexity language is
\begin{lstlisting}
rev = $\langle$0,$\lambda$xs.rec(xs, Nil $\mapsto$ Nil
                  , Cons $\mapsto \langle$x,$\langle$xs',r$\rangle \rangle$.
                       rec(r, Nil $\mapsto$ Cons$\langle$x, Nil$\rangle$
                            , Cons $\mapsto \langle$y,$\langle$ys,rys$\rangle \rangle$. Cons$\langle$y,rys$\rangle$))$\rangle$
\end{lstlisting}

It is more interesting if we consider the translation of \texttt{rev} applied to some \texttt{xs:intlist}.
The translation of this function into the complexity language proceeds as follows.
We begin with translating the  outer \texttt{rec} construct.

\begin{lstlisting}
rev xs  = 1 $+$ $\|$xs$\|_c$ $+_c$ rec($\|$xs$\|_p$
                         , Nil $\mapsto$ 1 $+_c$ $\|$Nil$\|$
                         , Cons $\mapsto \langle$x,$\langle$xs',r$\rangle \rangle$. 1 $+_c$ $\|$rec($\dots$)$\|$)
\end{lstlisting}

The cost of the translation of an \texttt{intlist} is zero, and the potential of the translation of an \texttt{intlist} is the list itself.
\begin{lstlisting}
rev xs = 1 $+_c$ rec(xs, Nil $\mapsto$ $\langle$1,Nil$\rangle$ , Cons $\mapsto \langle$x,$\langle$xs',r$\rangle \rangle$. 1 $+_c$ $\|$rec($\dots$)$\|$)
\end{lstlisting}

Next we translate the inner \texttt{rec}.
\begin{lstlisting}
$\|$rec(force(r), Nil $\mapsto$ Cons($\langle$x, Nil$\rangle$)
                , Cons $\mapsto \langle$y,$\langle$ys,rys$\rangle \rangle$. Cons$\langle$y,force(rys)$\rangle$))$\|$
\end{lstlisting}

Since \texttt{x}, \texttt{xs'}, \texttt{r} are terms in the complexity language, they do not need to be translated.
First we apply the rules for \texttt{rec} and \texttt{force}.
\begin{lstlisting}
r$_c$ $+_c$ rec(r$_p$, Nil $\mapsto$ 1 $+_c$ $\|$Cons($\langle$x, Nil$\rangle$)$\|$
          , Cons $\mapsto \langle$y,$\langle$ys,rys$\rangle \rangle$. 1 $+_c$ $\|$Cons$\langle$y,force(rys)$\rangle\|$)
\end{lstlisting}

The cost of the \texttt{Cons} constructor is zero, and the translation of \texttt{force} is just the translation of its argument.
\begin{lstlisting}
r$_c$ $+_c$ rec(r$_p$, Nil $\mapsto$ $\langle$1,Cons$\langle$x$_p$, Nil$\rangle\rangle$
           , Cons $\mapsto \langle$y,$\langle$ys,rys$\rangle \rangle$. $\langle$1 $+$ rys$_c$,Cons$\langle$y$_p$,rys$_p$$\rangle\rangle$)
\end{lstlisting}

Putting the pieces together, we get

\begin{lstlisting}[frame=single]
$\|$rev xs$\|$ = 1 $+_c$ rec(xs$_p$
                   , Nil $\mapsto$ $\langle$1,Nil$\rangle$
                   , Cons $\mapsto \langle$x,$\langle$xs',r$\rangle \rangle$. 1 $+$ r$_c$ $+_c$
                        rec(r$_p$, Nil $\mapsto$ $\langle$1,Cons$\langle$x$_p$, Nil$\rangle\rangle$
                             , Cons $\mapsto \langle$y,$\langle$ys,rys$\rangle \rangle$. $\langle$1 $+$ rys$_c$,Cons$\langle$y$_p$,rys$_p$$\rangle\rangle$))
\end{lstlisting}


\subsection*{Interpretation}
We intepret the size of an \texttt{intlist} to be the number of constructors.

$\llbracket$ \texttt{intlist} $\rrbracket$ = $\mathbb{N}^\infty$\\
$D^{intlist} = \{\ast\} + \{1\} \times \mathbb{N}^\infty$\\
$size_{intlist}(\texttt{Nil}) = 0$\\
$size_{intlist}(\texttt{Cons(1,n)}) = 1 + n$\\

Then $\llbracket \| \texttt{rev xs} \|_c \rrbracket = 1 + g(\|xs\|_p)$, where
\[g(n) = \llbracket rec(z, Nil \mapsto 1, Cons \mapsto \langle x, \langle xs',r\rangle \rangle.1 + r_c + h(r_p))\rrbracket \{z \mapsto n\}\]
\[h(n) = \llbracket rec(z, Nil \mapsto 1, Cons \mapsto \langle y, \langle ys',r\rangle \rangle.1 + r_c \rrbracket \{z \mapsto n\}\]

We calculate that $h(0)=1$ and for $n > 0$, $h(n) = 1 + h(n-1)$.
$g(0) = 1$ and for $n > 0$, $g(n) = 1 + g(n-1) + h(n-1)$

\chapter{Parametric Insertion Sort}
%
Parametric insertion sort is a higher order algorithm which sorts a list using
a comparison function which is passed to it as an argument.  The running time
of insertion sort is $\mathcal{O}(n^2)$.  This characterization of the
complexity of parametric insertion sort does not capture role of the comparison
function in the running time.  When sorting a list of integers, where
comparison between any two integers takes constant time, this does not matter.
However, when sorting a list of strings, where the complexity of comparison is
order the length of the string, the length of the strings may influence the
running time more than the length of the list when sorting small lists of large
strings.
%
\begin{figure}[H]
\caption{Parametric insertion sort in the source language}
\begin{lstlisting}
data list = Nil of unit | Cons of int $\times$ list

insert = $\lambda$f.$\lambda$x.$\lambda$xs.rec(xs, Nil$\mapsto$ Cons$\langle$x,Nil$\rangle$,
                      Cons$\mapsto \langle$y,$\langle$ys,r$\rangle\rangle$.rec(f x y, True $\mapsto$ Cons$\langle$x,Cons$\langle$y,ys$\rangle\rangle$,
                                                 False $\mapsto$ Cons$\langle$y,force(r)$\rangle$))

sort = $\lambda$f.$\lambda$xs.rec(xs, Nil$\mapsto$ Nil, Cons$\mapsto \langle$y,$\langle$ys,r$\rangle\rangle$.insert f y force(r))
\end{lstlisting}
\end{figure}
%
%
\subsubsection{Insert}
%
The function \T{sort} relies on the function \T{insert} to insert the head of
the list into the result of recursively sorted tail of the list.  We will begin
with a translation and interpretation of \T{insert}.
%
%
\subsubsection{Translation}
%
The translation of \T{insert} is broken into chunks to make it more manageable.
Figure \ref{fig:fxy} steps through the translation of the comparison function
\T{f} applied to variables \T{x} and \T{y}.
%
\begin{figure}[H]
\caption{Translation of \T{(f:Int$\to$Int$\to$Bool) (x:Int) (y:Int)}.
\label{fig:fxy}
We assume \T{f}, \T{x} and \T{y} are variables, so the cost of their translation is 0.
Since \T{f} is a function of two arguments, the cost of \T{f} is 0 unless \T{f} is fully applied.
Notice that \T{f} in \T{$\|$f$\|$} is a source variable while plain \T{f} is a potential variable.
}
%
\begin{lstlisting}
$\|$f x y$\| = (1 + \|$f x$\|_c + \|$y$\|_c) +_c \|$f x$\|_p \|$y$\|_p$

  $= (1 + 1 + \|$f$\|_c + \|$x$\|_c + (\|$f$\|_p \|$x$\|_p)_c + \|$y$\|_c) +_c \|$f$\|_p \|$x$\|_p \|$y$\|_p$

  $= (2 + \langle$0,f$\rangle_c + \langle$0,x$\rangle_c + \langle$0,y$\rangle_c) +_c \langle$0,f$\rangle_p \langle$0,x$\rangle_p \langle$0,y$\rangle_p$

  $= (2 + 0 + 0 + 0) +_c $(f x y)

  $= \langle 2 + (($f x$)_p$ y$)_c, (($f x$)_p$ y$)_p\rangle$
\end{lstlisting}
  %$= (1 + \langle 1 + \|$f$\|_c + \|$x$\|_c + (\|$f$\|_p\|$x$\|_p)_c,(\|$f$\|_p\|$x$\|_p)_p \rangle_c + \langle$0,y$\rangle_c) +_c$
  %     $(\langle 1 + \|$f$\|_c + \|$x$\|_c + (\|$f$\|_p\|$x$\|_p)_c,(\|$f$\|_p\|$x$\|_p)_p \rangle_p  \langle$0,y$\rangle_p)$
\end{figure}
%
The translation \T{true} and \T{false} branches are given in figures
\ref{fig:insert_true} and \ref{fig:insert_false} respectively.
%
\begin{figure}[H]
\caption{Translation of \T{True$\to$Cons$\langle$x,Cons$\langle$y,ys$\rangle\rangle$} in the inner \T{rec} of \T{insert}.
In this case the element we are inserting into the list comes before the head of the list under the ordering given by \T{f}.
}
\label{fig:insert_true}
\begin{lstlisting}
$\|$True$\mapsto $Cons$\langle$x,Cons$\langle$y,ys$\rangle\rangle\|$

  $=$ True$\mapsto 1 +_c \|$Cons$\langle$x,Cons$\langle$y,ys$\rangle\rangle\|$

  $=$ True$\mapsto 1 +_c \langle \|\langle$x,Cons$\langle$y,ys$\rangle\rangle\|_c,$Cons$\|\langle$x,Cons$\langle$y,ys$\rangle\rangle\|_p\rangle$

  $=$ True$\mapsto 1 +_c \langle \langle \|$x$\|_c + \|$Cons$\langle$y,ys$\rangle\|_c,\langle\|$x$\|_p,\|$Cons$\langle$y,ys$\rangle\|_p\rangle\rangle_c$,
           Cons$\langle \|$x$\|_c + \|$Cons$\langle$y,ys$\rangle\|_c,\langle\|$x$\|_p,\|$Cons$\langle$y,ys$\rangle\|_p\rangle\rangle_p\rangle$

  $=$ True$\mapsto 1 +_c \langle \|$x$\|_c + \|$Cons$\langle$y,ys$\rangle\|_c,$Cons$\langle\|$x$\|_p,\|$Cons$\langle$y,ys$\rangle\|_p\rangle\rangle$

  $=$ True$\mapsto 1 +_c \langle \langle$0,x$\rangle_c + \langle\|\langle$y,ys$\rangle\|_c,$Cons$\|\langle$y,ys$\rangle\|_p\rangle_c$,
          Cons$\langle\langle$0,x$\rangle_p,\langle\|\langle$y,ys$\rangle\|_c,$Cons$\|\langle$y,ys$\rangle\|_p\rangle_p\rangle\rangle$

  $=$ True$\mapsto 1 +_c \langle 0 + \|\langle$y,ys$\rangle\|_c$,Cons$\langle$x,Cons$\|\langle$y,ys$\rangle\|_p\rangle\rangle$

  $=$ True$\mapsto 1 +_c \langle \langle\|$y$\|_c + \|$ys$\|_c,\langle\|$y$\|_p,\|$ys$\|_p\rangle\rangle_c$,Cons$\langle$x,Cons$\langle\|$y$\|_c + \|$ys$\|_c,\langle\|$y$\|_p,\|$ys$\|_p\rangle\rangle_p\rangle\rangle$

  $=$ True$\mapsto 1 +_c \langle\|$y$\|_c + \|$ys$\|_c$,Cons$\langle$x,Cons$\langle\|$y$\|_p,\|$ys$\|_p\rangle\rangle\rangle$

  $=$ True$\mapsto 1 +_c \langle\langle$0,y$\rangle_c + \langle$0,ys$\rangle_c$,Cons$\langle$x,Cons$\langle\langle$0,y$\rangle_p,\langle$0,ys$\rangle_p\rangle\rangle\rangle$

  $=$ True$\mapsto 1 +_c \langle0,$Cons$\langle$x,Cons$\langle$y,ys$\rangle\rangle\rangle$

  $=$ True$\mapsto \langle 1, $Cons$\langle$x,Cons$\langle$y,ys$\rangle\rangle\rangle$
\end{lstlisting}
\end{figure}
%
\begin{figure}[H]
\caption{Translation of the \T{False} branch of the inner \T{rec} of \T{insert}.
\label{fig:insert_false}
\T{r} stands for the recursive call, and has type \T{susp list}.
In this case the element we are inserting into the list comes after the head of the list under the ordering given by \T{f}.
}
\begin{lstlisting}
$\|$False$\mapsto $Cons$\langle$y,force(r)$\rangle\|$
  $=$ False$\mapsto 1 +_c \|$Cons$\langle$y,force(r)$\rangle\|$

  $=$ False$\mapsto 1 +_c \langle \|\langle$y,force(r)$\rangle\|_c$,Cons$\|\langle$y,force(r)$\rangle\|_p\rangle$

  $=$ False$\mapsto 1 +_c \langle \langle \|$y$\|_c + \|$force(r)$\|_c,\langle \|$y$\|_p,\|$force(r)$\|_p\rangle\rangle_c$,
            Cons$\langle \|$y$\|_c + \|$force(r)$\|_c,\langle \|$y$\|_p,\|$force(r)$\|_p\rangle\rangle_p\rangle$

  $=$ False$\mapsto 1 +_c  \langle\|$y$\|_c + \|$force(r)$\|_c$,Cons$\langle \|$y$\|_p,\|$force(r)$\|_p\rangle\rangle$

  $=$ False$\mapsto 1 +_c  \langle\langle$0,y$\rangle_c + (\|$r$\|_c +_c \|$r$\|_p)_c,$Cons$\langle \langle$0,y$\rangle_p,(\|$r$\|_c +_c \|$r$\|_p)\rangle\rangle$

  $=$ False$\mapsto 1 +_c  \langle0 + $r$_c,$Cons$\langle$y$,$r$_p\rangle\rangle$

  $=$ False$\mapsto \langle1 + $r$_c,$Cons$\langle$y$,$r$_p\rangle\rangle$
\end{lstlisting}
\end{figure}
%
Figure \ref{fig:insert_inner_rec} uses the translation of \T{f x y} and the
\T{true} and \T{false} branches to construct the translation of the inner
\T{rec} construct.
%
\begin{figure}[H]
\caption{Translation of the inner \T{rec} in \T{insert}.
In the \T{True} case, we have found the place of \T{x} in the list and we so stop.
In the \T{False} case, \T{x} comes after the head of list under the ordering given by \T{f} and we must recurse on the tail of the list.
}
\label{fig:insert_inner_rec}
\begin{lstlisting}
$\|$rec(f x y,True$\mapsto$Cons$\langle$x,Cons$\langle$y,ys$\rangle\rangle$,False$\mapsto$Cons$\langle$y,force(r)$\rangle$)$\|$
  $= \|$f x y$\|_c +_c$rec($\|$f x y$\|_p$,True$\mapsto 1 +_c \|$Cons$\langle$x,Cons$\langle$y,ys$\rangle\rangle\|$,
                          False$\mapsto 1 +_c \|$Cons$\langle$y,force(r)$\rangle\|$)

  $= 2 + ((\|$f$\|_p$ x$)_p $ y$)_c +_c$rec($((\|$f$\|_p$ x$)_p $ y$)_p$,
                        True$\mapsto 1 +_c \|$Cons$\langle$x,Cons$\langle$y,ys$\rangle\rangle\|$,
                        False$\mapsto 1 +_c \|$Cons$\langle$y,force(r)$\rangle\|$)

  $= 2 + ((\|$f$\|_p$ x$)_p $ y$)_c +_c$rec($((\|$f$\|_p$ x$)_p $ y$)_p$,
                        True$\mapsto \langle 1, $Cons$\langle$x,Cons$\langle$y,ys$\rangle\rangle\rangle$
                        False$\mapsto 1 +_c \|$Cons$\langle$y,force(r)$\rangle\|$)

  $= 2 + ((\|$f$\|_p$ x$)_p $ y$)_c +_c$rec($((\|$f$\|_p$ x$)_p $ y$)_p$,
                        True$\mapsto \langle 1, $Cons$\langle$x,Cons$\langle$y,ys$\rangle\rangle\rangle$
                        False$\mapsto \langle1 + $r$_c,$Cons$\langle$y$,$r$_p\rangle\rangle$)

\end{lstlisting}
\end{figure}
%
%
The \T{Nil} and \T{Cons} branches of the outer \T{rec} construct are given in
figures \ref{fig:insert_nil} and \ref{fig:insert_cons}, respectively.
%
\begin{figure}[H]
\caption{Translation of the \T{Nil} branch of the outer \T{rec} in \T{insert}.
The insertion of an element into an empty list results in a singleton list containing only the element.
This branch is also reached when the ordering given by \T{f} dictates \T{x} comes after than everything in the list,
  and should be placed at the back of the list.
}
\label{fig:insert_nil}
\begin{lstlisting}
$\|$Nil$\mapsto$Cons$\langle$x,Nil$\rangle\|$
  $=$ Nil$\mapsto 1 +_c \|$Cons$\langle$x,Nil$\rangle\|$

  $=$ Nil$\mapsto 1 +_c \langle\|\langle$x,Nil$\rangle\|_c$,Cons$\|\langle$x,Nil$\rangle\|_p\rangle$

  $=$ Nil$\mapsto 1 +_c \langle\langle\|$x$\|_c + \|$Nil$\|_c,\langle\|$x$\|_p,\|$Nil$\|_p\rangle\rangle_c$,Cons$\langle\|$x$\|_c + \|$Nil$\|_c,\langle\|$x$\|_p,\|$Nil$\|_p\rangle\rangle_p\rangle$

  $=$ Nil$\mapsto 1 +_c \langle\|$x$\|_c + \|$Nil$\|_c$,Cons$\langle\|$x$\|_p,\|$Nil$\|_p\rangle\rangle$

  $=$ Nil$\mapsto 1 +_c \langle\langle$0,x$\rangle_c + \langle$0,Nil$\rangle_c$,Cons$\langle\langle$0,x$\rangle_p,\langle$0,Nil$\rangle_p\rangle\rangle$

  $=$ Nil$\mapsto 1 +_c \langle0 + 0$,Cons$\langle$x,Nil$\rangle\rangle$

  $=$ Nil$\mapsto \langle1$,Cons$\langle$x,Nil$\rangle\rangle$
\end{lstlisting}
\end{figure}
%
\begin{figure}[H]
\caption{Translation of the \T{Cons} branch of the outer \T{rec} in \T{insert}.
In this branch we recurse on a nonempty list.
We check if \T{x} is comes before the head of the list under the ordering given by \T{f}, in which case we are done, 
  otherwise we recurse on the tail of the list.
}
\label{fig:insert_cons}
\begin{lstlisting}
$\|$Cons$\mapsto \langle$y,$\langle$ys,r$\rangle\rangle$.rec(f x y, True $\mapsto$ Cons$\langle$x,Cons$\langle$y,ys$\rangle\rangle$,
                             False $\mapsto$ Cons$\langle$y,force(r)$\rangle$)$\|$

  $=$ Cons$\mapsto \langle$y,$\langle$ys,r$\rangle\rangle$.$1 +_c \|$rec(f x y, True $\mapsto$ Cons$\langle$x,Cons$\langle$y,ys$\rangle\rangle$,
                             False $\mapsto$ Cons$\langle$y,force(r)$\rangle$)$\|$

  $=$ Cons$\mapsto \langle$y,$\langle$ys,r$\rangle\rangle$.$1 +_c (2 + (($f x$)_p$ y$)_c) +_c$rec($(($f x$)_p$ y$)_p$,
                                              True$\mapsto \langle 1, $Cons$\langle$x,Cons$\langle$y,ys$\rangle\rangle\rangle$
                                              False$\mapsto \langle1 + $r$_c,$Cons$\langle$y$,$r$_p\rangle\rangle$)$)$

  $=$ Cons$\mapsto \langle$y,$\langle$ys,r$\rangle\rangle$.$(3 + (($f x$)_p $ y$)_c) +_c$rec($(($f x$)_p$ y$)_p$,
                                           True$\mapsto \langle 1, $Cons$\langle$x,Cons$\langle$y,ys$\rangle\rangle\rangle$
                                           False$\mapsto \langle1 + $r$_c,$Cons$\langle$y$,$r$_p\rangle\rangle$)$)$
\end{lstlisting}
\end{figure}
%
We put these together to give the translation of \T{insert}.
%
\begin{figure}[H]
\caption{Translation of \T{insert}}
\label{fig:insert}
\begin{lstlisting}
$\|$insert$\|$ = $\|\lambda$f.$\lambda$x.$\lambda$xs.rec(xs, Nil$\mapsto$ Cons$\langle$x,Nil$\rangle$,
                      Cons$\mapsto \langle$y,$\langle$ys,r$\rangle\rangle$.rec(f x y, True $\mapsto$ Cons$\langle$x,Cons$\langle$y,ys$\rangle\rangle$,
                                                 False $\mapsto$ Cons$\langle$y,force(r)$\rangle$))$\|$

  $= \langle$0,$\lambda$f.$\langle$0,$\lambda$x.$\langle0,\lambda$xs.$\|$rec(xs, Nil$\mapsto$ Cons$\langle$x,Nil$\rangle$,
                          Cons$\mapsto \langle$y,$\langle$ys,r$\rangle\rangle$.rec(f x y, True $\mapsto$ Cons$\langle$x,Cons$\langle$y,ys$\rangle\rangle$,
                                                     False $\mapsto$ Cons$\langle$y,force(r)$\rangle$))$\|\rangle\rangle\rangle$

  $= \langle$0,$\lambda$f.$\langle$0,$\lambda$x.$\langle0,\lambda$xs.$\langle$0,xs$\rangle_c +_c $
          rec($\langle$0,xs$\rangle_p$,
              Nil$\mapsto \langle1$,Cons$\langle$x,Nil$\rangle\rangle$
              Cons$\mapsto \langle$y,$\langle$ys,r$\rangle\rangle$.$(3 + ((\langle$0,f$\rangle_p$ x$)_p $ y$)_c) +_c$rec($((\langle$0,f$\rangle_p$ x$)_p $ y$)_p$,
                                                   True$\mapsto \langle 1, $Cons$\langle$x,Cons$\langle$y,ys$\rangle\rangle\rangle$
                                                   False$\mapsto \langle1 + $r$_c,$Cons$\langle$y$,$r$_p\rangle\rangle$)$)$

  $= \langle$0,$\lambda$f.$\langle$0,$\lambda$x.$\langle0,\lambda$xs.
          rec(xs,
              Nil$\mapsto \langle1$,Cons$\langle$x,Nil$\rangle\rangle$
              Cons$\mapsto \langle$y,$\langle$ys,r$\rangle\rangle$.$(3 + (($f x$)_p $ y$)_c) +_c$rec($($f x$)_p $ y$)_p$,
                                                 True$\mapsto \langle 1, $Cons$\langle$x,Cons$\langle$y,ys$\rangle\rangle\rangle$
                                                 False$\mapsto \langle1 + $r$_c,$Cons$\langle$y$,$r$_p\rangle\rangle$)$)$
\end{lstlisting}
\end{figure}
%
Finally we give a translation of \T{insert f x xs} in figure
\ref{fig:insert_applied} because this is the term we will interpret in a
size-based semantics.
%
\begin{figure}[H]
  \caption{The translation of \T{insert f x xs}.
  Unlike before, we do not assume that \T{f}, \T{x}, \T{xs} are variables.
  They may be expressions with non-zero costs.}
  \label{fig:insert_applied}
  \begin{lstlisting}
  $\|$insert f x xs$\| = (1 + \|$insert f x$\|_c + \|$xs$\|_c) +_c \|$insert f x$\|_p \|$xs$\|_p$

     $= (1 + \|$insert f x$\|_c + \|$xs$\|_c) +_c \|$insert f x$\|_p  \|$xs$\|_p$

     $= (2 + \|$insert f$\|_c + \|$x$\|_c + (\|$insert f$\|_p \|$x$\|_p)_c + \|$xs$\|_c) +_c \|$insert f$\|_p \|$x$\|_p \|$xs$\|_p$

     $= (2 + \|$insert f$\|_c + \|$x$\|_c + \|$xs$\|_c) +_c \|$insert f$\|_p \|$x$\|_p \|$xs$\|_p$

     $= (3 + \|$insert$\|_c + \|$f$\|_c+ (\|$insert$\|_p \|$f$\|_p \|$x$\|_p)_c + \|$x$\|_c  + \|$xs$\|_c) +_c \|$insert$\|_p \|$f$\|_p \|$x$\|_p \|$xs$\|_p$

     $= (3 + \|$f$\|_c + \|$x$\|_c + \|$xs$\|_c) +_c \|$insert$\|_p \|$f$\|_p \|$x$\|_p \|$xs$\|_p$

     $= (3 + \|$f$\|_c + \|$x$\|_c + \|$xs$\|_c) +_c $rec($\|$xs$\|_p$,
              Nil$\mapsto \langle1$,Cons$\langle$x,Nil$\rangle\rangle$
              Cons$\mapsto \langle$y,$\langle$ys,r$\rangle\rangle$.$(3 + ((\|$f$\|_p \|$x$\|_p)_p $ y$)_c) +_c$rec($(\|$f$\|_p \|$x$\|_p)_p $ y$)_p$,
                                                 True$\mapsto \langle 1, $Cons$\langle \|$x$\|_p$,Cons$\langle$y,ys$\rangle\rangle\rangle$
                                                 False$\mapsto \langle1 + $r$_c,$Cons$\langle$y$,$r$_p\rangle\rangle$)$)$
  \end{lstlisting}
\end{figure}
%
The result is:
%
\begin{lstlisting}
$\|$insert f x xs$\|= (3 + \|$f$\|_c + \|$x$\|_c + \|$xs$\|_c)$
        $+_c $rec($\|$xs$\|_p$,
              Nil$\mapsto \langle1$,Cons$\langle$x,Nil$\rangle\rangle$
              Cons$\mapsto \langle$y,$\langle$ys,r$\rangle\rangle$.$(3 + ((\|$f$\|_p \|$x$\|_p)_p $ y$)_c) +_c$rec($(\|$f$\|_p \|$x$\|_p)_p $ y$)_p$,
                                                 True$\mapsto \langle 1, $Cons$\langle \|$x$\|_p$,Cons$\langle$y,ys$\rangle\rangle\rangle$
                                                 False$\mapsto \langle1 + $r$_c,$Cons$\langle$y$,$r$_p\rangle\rangle$)$)$
\end{lstlisting}
%


\subsubsection{Interpretation}
%
We well use an interpretation of lists as a pair of their greatest element and
their length.  Figure \ref{fig:interp_sizes} formalizes this interpretation.
%
\begin{figure}[H]
  \caption{Interpretation of lists as lengths}
  \label{fig:interp_sizes}
  \begin{align*}
    \llbracket list \rrbracket &= \mathbb{Z} \times \mathbb{N}^\infty \\
    D^{list} &= \{\ast\} + \{\mathbb{Z}\} \times \mathbb{N}^\infty \\
    size_{list} (Nil) &= (-\infty,0) \\
    size_{list} (Cons(i,(j,n))) &= (max\{i,j\},1 + n)
  \end{align*}
\end{figure}
%
We use the mutual ordering on pairs.  That is, $(s,n) \leq (s',n')$ if
$n \leq n'$ and $s < s'$ or $n < n'$ and $s \leq s'$.

First we interpret the \T{rec}, which drives of the cost of \T{insert}.  As in
the translation, we break the interpretation up to make it more manageable.  We
will write $map, \lambda$ and $+_c$ in the semantics, which stand for the
semantic equivalents of the syntactic \T{map}, $\lambda$ and $+_c$.  The
definitions of these semantic functions mirror the definitions of their
syntactic equivalents.  Figures \ref{fig:interp_sizes_inner_rec} and
\ref{fig:interp_sizes_outer_rec} walk through the interpretation.
%
\begin{figure}[H]
  \caption{Interpretation of the inner \T{rec} of \T{insert} with lists abstracted to sizes}
  \label{fig:interp_sizes_inner_rec}
  \begin{lstlisting}
    $\llbracket$rec($($f x$)_p $ y$)_p$,
       True$\mapsto \langle 1, $Cons$\langle$x,Cons$\langle$y,ys$\rangle\rangle\rangle$
       False$\mapsto \langle1 + $r$_c,$Cons$\langle$y$,$r$_p\rangle\rangle$)$)$ $\rrbracket \xi \{$f$ \mapsto f, $x$ \mapsto x, $y$ \mapsto y, $ys$ \mapsto (i,n), $r$ \mapsto r\}$

    $f_{True} (\langle\rangle) = \llbracket \langle$1,Cons$\langle$x,Cons$\langle$y,ys$\rangle\rangle\rangle\rrbracket \xi \{$f$ \mapsto f, $x$ \mapsto x, $y$ \mapsto y, $ys$ \mapsto (i,n), $r$ \mapsto r\}$

           $= (1, (max\{x,y,i\},2 + n))$

    $f_{False} (\langle\rangle) = \llbracket \langle1 + $r$_c,$Cons$\langle$y$,$r$_p\rangle\rangle$)$)$ $\rrbracket \xi \{$f$ \mapsto f, $x$ \mapsto x, $y$ \mapsto y, $ys$ \mapsto (i,n), $r$ \mapsto r\}$

           $= (1 + \pi_0 r, (max\{y,\pi_0\pi_1 r\}, 1 + \pi_1\pi_1 r))$
    
    $= \bigvee_{size(w) \leq \pi_1(\pi_1(f\ x)\ y)} case(w, (f_{True}, f_{False}))$

    $= \bigvee_{size(w) \leq \pi_1(\pi_1(f\ x)\ y)} case(w, (\lambda \langle\rangle.(1, (max\{x,y,i\},2 + n)), \lambda \langle\rangle.(1 + \pi_0 r, (max\{y,\pi_0\pi_1 r\}, 1 + \pi_1\pi_1 r))$

    $= (1, (max\{x,y,i\},2 + n)) \vee (1 + \pi_0 r, (max\{y,\pi_0\pi_1 r\}, 1 + \pi_1\pi_1 r))$
  \end{lstlisting}
\end{figure}
%
\begin{figure}[H]
  \caption{Interpretation of \T{rec} in \T{insert}.}
  \label{fig:interp_sizes_outer_rec}
  \begin{lstlisting}
   $g(i,n) = \llbracket$rec(xs,
        Nil$\mapsto \langle1$,Cons$\langle$x,Nil$\rangle\rangle$
        Cons$\mapsto \langle$y,$\langle$ys,r$\rangle\rangle$.$(3 + (($f x$)_p $ y$)_c) +_c$rec($($f x$)_p $ y$)_p$,
                                         True$\mapsto \langle 1, $Cons$\langle$x,Cons$\langle$y,ys$\rangle\rangle\rangle$
                                         False$\mapsto \langle1 + $r$_c,$Cons$\langle$y$,$r$_p\rangle\rangle$)$)\rrbracket \xi \{$f$\mapsto f, $x$\mapsto x,$xs$\mapsto (i,n)\}$

       $f_{Nil}(\langle\rangle) = \llbracket \langle1$,Cons$\langle$x,Nil$\rangle\rangle \rrbracket \xi \{$f$\mapsto f, $x$\mapsto x,$xs$\mapsto (i,n)\}$

       $f_{Nil}(\langle\rangle) = (1,(x,1))$

       $f_{Cons}((j,(j,m))) = \llbracket (3 + $((f x)$_p$ y)$_c) +_c $rec($\dots$)$ \rrbracket \xi $
            $\{$f$\mapsto f, $x$\mapsto x,$xs$\mapsto (i,n),\langle$y,$\langle$ys,r$\rangle\rangle\mapsto (map^{\mathbb{Z} \times \mathbb{N}^\infty} (\lambda a.(a,\llbracket$rec($w, \dots$)$\rrbracket \xi \{w \mapsto a\}),(j,(j,m))))\}$

           $= \llbracket \dots \rrbracket \xi \{\dots \langle$y,$\langle$ys,r$\rangle\rangle\mapsto (j,map^{\mathbb{N}^\infty} (\lambda a.(a,\llbracket$rec($w, \dots$)$\rrbracket \xi \{w \mapsto a\}),(j,m)))\}$

           $= \llbracket \dots \rrbracket \xi \{\dots \langle$y,$\langle$ys,r$\rangle\rangle\mapsto (j,((j,m),\llbracket$rec($w, \dots$)$\rrbracket \xi \{w \mapsto (j,m)\}))\}$

           $= \llbracket \dots \rrbracket \xi \{\dots \langle$y,$\langle$ys,r$\rangle\rangle\mapsto (j,((j,m),g(j,m)))\}$

           $= (3 + \pi_0(\pi_1(f\ x)\ j)) +_c$
                $((1,(max\{x,j\},2+m))) \vee (1+\pi_0g(j,m),(max\{j,\pi_0\pi_1g(j,m)\},1 + \pi_1\pi_1g(j,m)))$

           $= (3 + \pi_0(\pi_1(f\ x)\ j)) +_c$
                $(1 \vee (1+\pi_0g(j,m)), (max\{x,j,\pi_0\pi_1g(j,m)\}, 2 + m \vee 1 + \pi_1\pi_1g(j,m)))$

           $= (4 + \pi_0(\pi_1(f\ x)\ j) + \pi_0g(j,m), (max\{x,j\pi_0\pi_1g(j,m)\}, 2+m \vee 1 + \pi_1\pi_1g(j,m)))$

       $g(i,n) = \bigvee_{size(z) \leq (i,n)} case(z, (f_{Nil},f_{Cons}))$
  \end{lstlisting}
\end{figure}
%
The initial result is given in equation \ref{eq:insert_initial_recurrence}.
%
\begin{align*}
  &f_{Nil}(\langle\rangle) = (1,(x,1)) \\ 
  &f_{Cons}(j,(j,m)) = (4 + \pi_0(\pi_1(f\ x)\ j) + \pi_0g(j,m), \\
  &\ \ \ \ (max\{x,j,\pi_0\pi_1g(j,m)\}, 2+m \vee 1 + \pi_1\pi_1g(j,m))) \\
  \label{eq:insert_initial_recurrence}
  &g(i,n) = \bigvee_{size(z) \leq (i,n)} case(z, (f_{Nil},f_{Cons})) \numberthis
\end{align*}
%
This recurrence is difficult to work with.  Specifically, we cannot apply
traditional methods of solving it.  We will manipulate it into a more usable
form by eliminating the arbitrary maximum.  We will separate the recurrence
into a recurrence for the cost and a recurrence for the potential, and solve
those independently.
%
\begin{lemma}
  \label{lem:insert_rec_cost}
  $g_c(i,n) \leq (4 + ((f\ x)_p\ i)_c n + 1$
\end{lemma}
%
\begin{proof}
  We prove this by induction on $n$.
  Recall we use the mutual ordering on pairs.
  \begin{description}
    \item[case $n=0$]\hfill \\
      $g_c(i,n) = (1, (x, 1))_c = 1$
    \item[case$n>0$]\hfill \\
      \begin{align*}
        &= \bigvee_{size(z) \leq (i,n)} case(z, (f_{Nil}, f_{Cons})) &&\\
        &= \bigvee_{j < i, m \leq n \text{ or } j \leq i, m < n} case((j, m), (f_{Nil}, f_{Cons})) &&\\
        &= \bigvee_{j < i, m \leq n \text{ or } j \leq i, m < n} 4 + ((f\ x)_p\ j)_c + g_c(j, m')) &&\text{where $m' = m - 1$}\\
        &= \bigvee_{j < i, m \leq n \text{ or } j \leq i, m < n} 4 + ((f\ x)_p\ j)_c + (4 + ((f\ x)_p\ j)_c)m' + 1 &&\text{by the induction hypothesis}\\
        &= \bigvee_{j < i, m \leq n \text{ or } j \leq i, m < n} (4 + ((f\ x)_p\ j)_c) (m' + 1) + 1 &&\\
        &= \bigvee_{j < i, m \leq n \text{ or } j \leq i, m < n} (4 + ((f\ x)_p\ j)_c) m + 1 &&\\
        &\leq \bigvee_{i < j, m \leq n \text{ or } i \leq j, m < n} (4 + ((f\ x)_p\ i)_c) n + 1 &&\\
        &\leq (4 + ((f\ x)_p\ i)_c) n + 1&&
      \end{align*}
  \end{description}
\end{proof}
%
As expected, we find the cost of insert is bounded by the length of the list and the largest element.
%
\begin{lemma}
  \label{lem:insert_rec_potential}
  $g_p(i,n) \leq (max\{x, i\}, n+1)$
\end{lemma}
%
\begin{proof}
  We prove this by induction on $n$.
  \begin{description}
    \item[case $n=0$]\hfill \\
      $g_p(i,n) = (1, (x, 1))_p = (x, 1)$.
    \item[case $n>0$]\hfill \\
      \begin{align*}
        &= \bigvee_{size(z) \leq (i,n)} case(z, (f_{Nil}, f_{Cons}) &&\\
        &= \bigvee_{j < i, m \leq n \text{ or } j \leq i, m < n} (max\{x, j, \pi_0\pi_1g(j, m')\}, 2 + m' \vee 1 + \pi_1\pi_1g(j, m')) && \text{where $m' = m - 1$}\\
        &\leq \bigvee_{j < i, m \leq n \text{ or } j \leq i, m < n} (max\{x, j\}, 2 + m')&&\text{by the induction hypothesis}\\
        &\leq \bigvee_{j < i, m \leq n \text{ or } j \leq i, m < n} (max\{x,i\}, 1 + n)&&\\
        &\leq (max\{x,i\}, 1 + n)&&
      \end{align*}
  \end{description}
\end{proof}
%
We find the length of the potential is bounded by one plus the length of the
input, and the largest element in the output is bounded by the maximum of the
element being inserted and the largest element in the input.  This is somewhat
unsatisfactory, since we would expect the relationship to be equality.  What
happens if we try to prove the equality?
%
\begin{lemma}
  \label{lem:insert_rec_potential_wrong}
  $g_p(i,n) = (max\{x, i\}, n+1)$
\end{lemma}
\begin{proof}
  We attempt to prove this by induction on $n$.
  The first steps proceed similarly to \ref{lem:insert_rec_potential}.
  \begin{description}
    \item[case $n=0$]\hfill \\
      $g_p(i,n) = (1, (x, 1))_p = (x, 1)$.
    \item[case $n>0$]\hfill \\
      \begin{align*}
        &= \bigvee_{size(z) \leq (i,n)} case(z, (f_{Nil}, f_{Cons}) &&\\
        &= \bigvee_{j < i, m \leq n \text{ or } j \leq i, m < n} (max\{x, j, \pi_0\pi_1g(j, m')\}, 2 + m' \vee 1 + \pi_1\pi_1g(j, m')) && \text{where $m' = m - 1$}\\
        &= \bigvee_{j < i, m \leq n \text{ or } j \leq i, m < n} (max\{x, j\}, 2 + m')&&\text{by the induction hypothesis}\\
        &= \bigvee_{j < i, m \leq n} (max\{x, j\}, 1 + m) \vee \bigvee_{j \leq i, m < n} (max\{x, j\}, 1 + m)&&
      \end{align*}
  \end{description}
\end{proof}
%
We see that we get stuck.  Because of the mutual ordering on pairs, our big
maximum is over all $z$ such that $size(z) < (i, n)$.  This includes $(j, m)$
such that $j < i ^ m \leq n$.  We have no way of reasoning about the potential
of $g(i-1, n) \vee g(i, n -1)$.  So we cannot prove equality for
\ref{lem:insert_rec_potential}.  This indicates we may not have the optimal
ordering on pairs.

Using lemmas \ref{lem:insert_rec_cost} and \ref{lem:insert_rec_potential}, we
can express the cost and potential of \T{insert} in terms of its arguments.
%
\begin{equation}
  \label{eq:insert_interp}
  insert\ f\ x\ xs \leq (4 + ((f\ x)_p\ i)_c n + 1, (max\{x, i\}, n+1))
\end{equation}
%

\subsubsection{Sort}
%
\subsubsection{Translation}
The translation of sort is shown in figure \ref{fig:sort}.  The translation of
the \T{Nil} and \T{Cons} branches in the \T{rec} are walked through in figures
\ref{fig:sort_nil} and \ref{fig:sort_cons}, respectively.  The translation of
\T{sort} applied to its arguments is given in figure \ref{fig:sort_applied}.
%
\begin{figure}[H]
  \caption{Translation of \T{Nil} branch of \T{sort}.}
  \label{fig:sort_nil}
  \begin{lstlisting}
  $\|$Nil$\mapsto$Nil$\|$

  $= $Nil$\mapsto 1 +_c \|$Nil$\|$

  $= $Nil$\mapsto 1 +_c \langle$0,Nil$\rangle$

  $= $Nil$\mapsto \langle$1,Nil$\rangle$
  \end{lstlisting}
\end{figure}
%
\begin{figure}[H]
  \caption{Translation of \T{Cons} branch of \T{sort}.}
  \label{fig:sort_cons}
  \begin{lstlisting}
  $\|$Cons$\mapsto\langle$y,$\langle$ys,r$\rangle\rangle$.insert f y force(r)

  $= $Cons$\mapsto 1 +_c \|$insert f y force(r)$\|$

  $= $Cons$\mapsto\langle$y,$\langle$ys,r$\rangle\rangle. 1 +_c (\|$force(r)$\|_c) +_c \|$insert f y$\|_p \|$force(r)$\|_p$

  $= $Cons$\mapsto\langle$y,$\langle$ys,r$\rangle\rangle. 1 +_c ((\|$r$\|_c +_c \|$r$\|_p)_c) +_c \|$insert f y$\|_p (\|$r$\|_c +_c \|$r$\|_p)_p$

  $= $Cons$\mapsto\langle$y,$\langle$ys,r$\rangle\rangle. 1 +_c $r$_c +_c \|$insert f y$\|_p $r$_p$

  $= $Cons$\mapsto\langle$y,$\langle$ys,r$\rangle\rangle. 1 +_c $r$_c +_c 3 +_c \|$insert$\|_p$ f y r$_p$

  $= $Cons$\mapsto\langle$y,$\langle$ys,r$\rangle\rangle. (4 + $r$_c) +_c \|$insert$\|_p$ f y r$_p$

  \end{lstlisting}
\end{figure}
%
\begin{figure}[H]
\caption{Translation of \T{sort}}
\label{fig:sort}
\begin{lstlisting}
$\|$sort$\|$ = $\langle0,\lambda$f.$\langle0,\lambda$xs.$\|$rec(xs, Nil$\mapsto$ Nil,
                        Cons$\mapsto \langle$y,$\langle$ys,r$\rangle\rangle$.insert f y force(r))$\|\rangle\rangle$

      = $\langle0,\lambda$f.$\langle0,\lambda$xs.rec(xs, Nil$\mapsto \langle$1,Nil$\rangle$
                        Cons$\mapsto \langle$y,$\langle$ys,r$\rangle\rangle.4 +_c $r$_c +_c \|$insert$\|_p$ f y r$_p$
\end{lstlisting}
\end{figure}
%
\begin{figure}[H]
  \caption{Translation of \T{sort} applied to variables \T{f} and \T{xs}}
\label{fig:sort_applied}
\begin{lstlisting}
$\|$sort f xs$\| = (1 + \|$sort f$\|_c + \|$xs$\|_c) +_c \|$sort f$\|_p \|$xs$\|_p$

          $ = (1 + (1 + \|$sort$\|_c + \|$f$\|_c + \|$xs$\|_c)) +_c \|$sort$\|_p \|$f$\|_p \|$xs$\|_p$

          $ = (1 + (1 + 0 + 0 + 0)) +_c \|$sort$\|_p \|$f$\|_p \|$xs$\|_p$

          $ = 2 +_c \|$sort$\|_p $f$\|_p \|$xs$\|_p$

          $ = 2 +_c $rec($\|$xs$\|_p$, Nil$\mapsto \langle$1,Nil$\rangle$
                        Cons$\mapsto \langle$y,$\langle$ys,r$\rangle\rangle.(4 + $r$_c) +_c \|$insert$\|_p$ f y r$_p$
\end{lstlisting}
\end{figure}
%
%
\subsubsection{Interpretation}
%
The \T{rec} construct again drives the cost and potential of \T{sort}.  The
walk through of the interpretation of the \T{rec} is given in figure
\ref{fig:sort_rec_interp}.
%
\begin{figure}[H]
  \caption{Interpretation of \T{rec} in \T{sort}.TO DO FIX THIS}
  \label{fig:sort_rec_interp}
  \begin{lstlisting}
  $g(i, n) = \llbracket $rec($\|$xs$\|_p$, Nil$\mapsto \langle$1,Nil$\rangle$
                Cons$\mapsto \langle$y,$\langle$ys,r$\rangle\rangle.(4 + $r$_c) +_c \|$insert$\|_p$ f y r$_p$)$\rrbracket \xi \{ xs \mapsto n\}$

     $ = \llbracket $rec($\|$xs$\|_p$, Nil$\mapsto \langle$1,Nil$\rangle$
                Cons$\mapsto \langle$y,$\langle$ys,r$\rangle\rangle.(4 + $r$_c) +_c \|$insert$\|_p$ f y r$_p$)$\rrbracket \xi \{ xs \mapsto n\}$

     $ = \llbracket $rec($\|$xs$\|_p$, Nil$\mapsto \langle$1,Nil$\rangle$
                Cons$\mapsto \langle$y,$\langle$ys,r$\rangle\rangle.(4 + $r$_c) +_c \|$insert$\|_p$ f y r$_p$)$\rrbracket \xi \{ xs \mapsto n\}$

     $ = \bigvee_{size(z)\leq n} case(z,(f_{Nil},f_{Cons}))$

  $f_{Nil}(\langle\rangle) = \llbracket \langle$1,Nil$\rangle \rrbracket \xi$
     $ = f_{Nil}(\langle\rangle) = (1,(\neg\infty,0))$

  $f_{Cons}((j,m)) = \llbracket \dots \rrbracket \xi \{\langle$y,$\langle$ys,r$\rangle\rangle\mapsto map^{j\times\mathbb{N}^\infty} (\lambda a.(a,\llbracket$rec($w, \dots$)$\rrbracket \xi \{w \mapsto a\}), (j,m)) \}$

  $ = \llbracket \dots \rrbracket \xi \{\langle$y,$\langle$ys,r$\rangle\rangle\mapsto (map^{int} (\lambda a.(\dots), j),map^{\mathbb{N}^\infty} (\lambda a.(a,\llbracket$rec($w, \dots$)$\rrbracket \xi \{w \mapsto a\}), m)) \}$

    $ = \llbracket \dots \rrbracket \xi \{\langle$y,$\langle$ys,r$\rangle\rangle\mapsto (j,(m,\llbracket$rec($w, \dots$)$\rrbracket \xi \{w \mapsto m\})) \}$

    $ = \llbracket (4 + $r$_c) +_c \|$insert$\|_p$ f y r$_p \rrbracket \xi \{\langle$y,$\langle$ys,r$\rangle\rangle\mapsto (j,(m,g(j, m)) \}$

    $ = (4 + \pi_0 g(j, m)) +_c insert\ f\ j\ \pi_1g(j,m)$

  $g(i, n) = \bigvee_{size(z)\leq n} case(z,(\lambda(\langle\rangle).(1,(\neg\infty,0)),\lambda(j,m).(4 + \pi_0 g(j,m)) +_c insert\ f\ j\ \pi_1g(j, m)))$
  \end{lstlisting}
\end{figure}
%
Equation \ref{eq:sort_interp0_init} shows the initial recurrence extracted.
%
\begin{equation}
  \label{eq:sort_interp0_init}
  g(i,n) = \bigvee_{size(z)\leq (i,n)} case(z,(\lambda(\langle\rangle).(1,(\neg\infty,0),\lambda(j,m).4 + \pi_0 g(j,m)) +_c(insert\ f\ j\ \pi_1g(j, m)))
\end{equation}
%
Observe that in equation \ref{eq:sort_interp0_init}, the cost is depends on the
potential of the recursive call.  Therefore we must solve the recurrence for
the potential first.
%
\begin{lemma}
  \label{lem:sort_interp_potential}
  $\pi_1g(n) \leq (j, n)$
\end{lemma}
\begin{proof}
  We prove this by induction on $n$.
  We use equation \ref{eq:insert_interp} to determine the potential of the $insert$ function.
  \begin{description}
    \item[case $n=0$]$\pi_1g(i,n) = (i, 0)$
    \item[case $n>0$]\hfill \\
      \begin{align*}
        \pi_1g(i,n) &= \pi_1 \bigvee_{size(z)\leq n} case(z,(\lambda(\langle\rangle).(1,(\neg\infty,0)),\lambda(j,m).4 + \pi_0 g(j,m)) +_c(insert\ f\ j\ \pi_1g(j, m)))\\
        &= \bigvee_{j \leq i, m < n \text{ or } j < i, m \leq n} \pi_1 (insert\ f\ j\ \pi_1g(j', m'))\ \ \ \ \ j' \leq j, m' = m - 1\\
        &\leq \bigvee_{j \leq i, m < n \text{ or } j < i, m \leq n} \pi_1 (insert\ f\ j\ (j', m'))\\
        &\leq \bigvee_{j \leq i, m < n \text{ or } j < i, m \leq n} (max\{j, j'\}, m' + 1\\
        &\leq \bigvee_{j \leq i, m < n \text{ or } j < i, m \leq n} (j, m)\\
        &\leq \bigvee_{j \leq i, m < n \text{ or } j < i, m \leq n} (i, n)\\
        &\leq (i, n)
      \end{align*}
  \end{description}
\end{proof}
%
As in the interpretation of \T{insert} we are left with a less than
satisfactory bound on the potential of \T{sort}.  It would grievous mistake to
write a sorting function whose output was smaller than its input.  Under the
current interpretation of lists, this would mean either the length of the list
decreased or the size of the largest element in the list decreased.
Unfortunately we are stuck with an upper bound on the size of the output
because or interpretation of \T{insert} only provides an upper bound on the
potential of its output. We may solve the recurrence for the cost of \T{sort}.
%
\begin{lemma}
  \label{lem:sort_interp_cost}
  $\pi_0g(n) \leq (4 + \pi_0(\pi_1(f\ x)\ i)n^2 + 5n + 1$
\end{lemma}
%
\begin{proof}
  We prove this by induction on $n$.
  \begin{description}
    \item[case $n=0$] $\pi_0 g(i,n) = 1$
    \item[case $n>0$] \hfill \\
      \begin{align*}
        \pi_0g(i,n) &= \pi_0 \bigvee_{size(z) \leq (i,n)} case(z, (\lambda(\langle\rangle).(1,(\neg\infty,0)),\lambda(j,m).4 + \pi_0 g(j, m)) +_c (insert\ f\ j\ \pi_1g(j, m)))\\
        &= \bigvee_{j < i, m \leq n \text{ or } j \leq i, m < n} 4 + \pi_0 g(j, m - 1) + \pi_0(insert\ f\ j\ \pi_1g(j, m - 1))\\
        &\leq \bigvee_{j < i, m \leq n \text{ or } j \leq i, m < n} 4 + \pi_0 g(j, m - 1) + \pi_0(insert\ f\ j\ (j, m - 1))\\
        &\leq \bigvee_{j < i, m \leq n \text{ or } j \leq i, m < n} 4 + \pi_0 g(j, m - 1) + (4 + \pi_0(\pi_1(f\ j)\ j))(m - 1) + 1\\
        & \text{let $c_1 = (4 + \pi_0(\pi_1(f\ j)\ j))$}\\
        &\leq \bigvee_{j < i, m \leq n \text{ or } j \leq i, m < n} 4 + c_1(m-1)^2 + 5(m-1) + 1 + c_1(m - 1) + 1\\
        &\leq \bigvee_{j < i, m \leq n \text{ or } j \leq i, m < n} 4 + c_1m^2 - 2c_1m +c_1 + 5m-5 + 1 + c_1m - c_1 + 1\\
        &\leq \bigvee_{j < i, m \leq n \text{ or } j \leq i, m < n} c_1m^2 - c_1m + 5m + 1\\
        &\leq \bigvee_{j < i, m \leq n \text{ or } j \leq i, m < n} (4 + \pi_0(\pi_1(f\ i)\ i))n^2 + 5n + 1\\
        &\leq (4 + \pi_0(\pi_1(f\ i)\ i))n^2 + 5n + 1
      \end{align*}
  \end{description}
\end{proof}
%
As expected the cost of \T{sort} is $\mathcal{O}(n^2)$ where $n$ is the length
of the list.  It is clear from the analysis how the cost of the comparison
function determines the running time of \T{sort}.  We can see that the
comparison function is called order $n^2$ times.

%\section{Insertion Sort}
%
Insertion sort is a quadratic time sorting algorithm which sorts a list by
inserting an element from an unsorted segment of a container into a sorted
segment of the container.  Although the asymptotic complexity of insertion sort
is less than the optimal $\mathcal{O}(nlog_2n)$, insertion sort does have
redeeming attributes.  Insertion sort has small constant factors, making it
more efficient on small datasets. Insertion sort may be done in-place
(\citet{Cormen2001}).  The running time of insertion sort is
$\mathcal{O}(n^2)$.
%
\begin{flalign*}
  \T{data list} &= \T{Nil of unit | Cons of int}\times\T{list}
\end{flalign*}
%
\begin{flalign*}
  \T{insert} &= \lambda x.\lambda xs.\T{rec}(xs, \T{Nil} \mapsto \T{Cons}\LP x,\T{Nil}\RP, \\
             &\quadthree \T{Cons} \mapsto \LP y,\LP ys,r \RP\RP.\T{rec}(x \leq y, \T{True} \mapsto \T{Cons} \LP x,\T{Cons}\LP y,ys\RP\RP, \\
             &\quadsix \T{False} \mapsto \T{Cons} \LP y,\T{force}(r)\RP))
\end{flalign*}
%
\subsection{Translation}

The translation of the function \T{insert} is shown below.
%
\begin{flalign*}
  \T{sort} &= \lambda xs.\T{rec}(xs,\T{Nil} \mapsto \T{Nil}, \T{Cons} \mapsto \LP y,\LP ys,r\RP\RP.\T{insert}\ y\ \T{force}(r))
\end{flalign*}
%
The translation of the function \T{sort} is shown below.
%
\begin{flalign*}
  \|\T{sort}\| &= \LP 0,\lambda xs.\T{rec}(xs, \T{Nil} \mapsto \T{Nil}, \\
               &\quadthree \T{Cons} \mapsto \LP y,\LP ys,r\RP\RP.(3 + r_c) +_c (\|\T{insert}\|_p\ y)_p r_p)
\end{flalign*}
%

\subsection{Interpretation}

We will interpret lists as their lengths.
%
\begin{align*}
    \llbracket list \rrbracket &= \mathbb{N}^\infty \\
    D^{list} &= \{\ast\} + \{1\} \times \mathbb{N}^\infty \\
    size_{list} (\ast) &= 0 \\
    size_{list} ((1,n)) &= 1 + n
\end{align*}
%
The interpretation of the \T{rec} that drives the cost of insert is given below.
%
\begin{align}
  \label{eq:insert_initial_recurrence}
  g(n) &= \bigvee_{size(z) \leq n} case(z, (f_{Nil},f_{Cons})) \\
  &\text{where} \\
  f_{Nil}() &= (1,1) \\
  f_{Cons}((1,m)) &= (3 + (1 \leq\ 1)_c + g_c(m), (2+m) \vee (1+g_p(m)))
\end{align}
%
We can extract the recurrence for the cost and eliminate the big maximum operator.
\begin{equation*}
\label{eq:insert_cost}
c(n) = \begin{cases}
  1 & n = 0 \\
  4 + \pi_0(\pi_1(f\ 1)\ 1) + c(n-1) & n > 0
\end{cases}
\end{equation*}
%
The closed form solution to this recurrence is:
\begin{lemma}
\label{lem:insert_cost}
  $c(n) = (3 + (1 \leq 1)_c)n + 1$
\end{lemma}
%
We also extract a recurrence for the potential and eliminate the big maximum
operator.
%
\begin{equation*}
  \label{eq:insert_potential}
  p(n) = \begin{cases}
    1 & n = 0 \\
    1 + p(n-1) & n > 0
  \end{cases}
\end{equation*}
%
The closed form solution to this recurrence is:
\begin{equation*}
  p(n) = 1 + n
\end{equation*}
%
The interpretation of \T{sort} is:
%
\begin{equation}
  \label{eq:sort_interp0_init}
  g(n) = \bigvee_{size(z)\leq n} case(z,(\lambda(\LP\RP).(1,0),\lambda(1,m).3 + g_c(m)) +_c \llbracket\|\T{insert}\|\rrbracket\ 1\ g_p(m))
\end{equation}
%
The recurrence with the big maximum operator eliminated is:
%
\begin{equation}
  \label{eq:sort_rec_final}
  g(n) = \begin{cases}
    (1,0) & n=0 \\
    (3 + g_c(n-1) + (\llbracket\|\T{insert}\|\rrbracket\ 1\ g_p(n-1))_c, (\llbracket\|\T{insert}\|\rrbracket\ 1\ g_p(n-1))) & n > 0
  \end{cases}
\end{equation}
%
We extract the recurrence for the potential.
Let $p = \pi_1 \circ g$.
%
\begin{equation}
  \label{eq:sort_rec_potential}
  p(n) = \begin{cases}
    0 & n=0 \\
  (\llbracket\|\T{insert}\|\rrbracket\ 1\ g_p(n-1))) & n > 0
  \end{cases}
\end{equation}
%
The closed form solution to this recurrence is given below.
%
\begin{lemma}
  \label{lem:sort_potential}
  $p(n) = n$
\end{lemma}
%
We extract the recurrence for the cost.
Let $c = \pi_0 \circ g$.
%
\begin{equation}
  \label{eq:sort_rec_cost}
  c(n) = \begin{cases}
    1 & n=0 \\
    3 + g_c(n-1) + (\llbracket\|\T{insert}\|\rrbracket\ 1\ g_p(n-1) & n > 0
  \end{cases}
\end{equation}
%

% ================================================================================
%This recurrence is difficult to work with.
%Specifically, we cannot apply traditional methods of solving it.
%We will manipulate it into a more usable form by eliminating the arbitrary maximum.
%Observe that for $n=0$, $g(n) = f_{Nil}(\LP\RP) = (1,1)$.
%For $n>0$,
%\begin{align*}
%  g(n) &= \bigvee_{size(z) \leq n} case(z, (f_{Nil},f_{Cons})) &&\\
%  &= g(n-1) \vee \bigvee_{size(z) = n} case(z, (f_{Nil},f_{Cons})) &&\\
%  &= g(n-1) \vee f_{Cons}(n) &&\\
%  &= g(n-1) \vee (4 + \pi_0(\pi_1(f\ 1)\ 1) + \pi_0g(n-1), (1+n) \vee (1+\pi_1g(n-1))) &&\text{$m=n-1$}\\
%  &= (4 + \pi_0(\pi_1(f\ 1)\ 1) + \pi_0g(n-1), (1+n) \vee (1+\pi_1g(n-1))) &&\text{lemma \ref{lem:insert_g_monotonicity}}\\
%  &= (4 + \pi_0(\pi_1(f\ 1)\ 1) + \pi_0g(n-1), 1+\pi_1g(n-1)) &&\text{lemma \ref{lem:insert_potential_inc}}
%\end{align*}
%\begin{lemma}
%  \label{lem:insert_g_monotonicity}
%  $g(n) > g(n-1)$
%\end{lemma}
%\begin{proof}
%  TODO
%\end{proof}
%
%\begin{lemma}
%\label{lem:insert_potential_inc}
%$\pi_1 g(n) > n$
%\end{lemma}
%\begin{proof}
%We prove this by induction on $n$.
%\begin{description}
%  \item[case $n=0$:] $\pi_1g(0) = 1$
%  \item[case $n>0$:]\hfill
%    \begin{align*}
%      \pi_1g(n) &= \pi_1(g(n-1) \vee (4 + \pi_0(\pi_1(f\ 1)\ 1) + \pi_0g(n-1),(1+n) \vee (1 + \pi_1 g(n-1)))) \\
%      &= \pi_1g(n-1) \vee (1+n) \vee (1 + \pi_1 g(n-1)) \\
%      &\geq n-1 \vee (1+n) \vee (1 + n - 1) \\
%      &\geq 1+n \\
%      &> n
%    \end{align*}
%\end{description}
%\end{proof}
%
%Equation \ref{eq:insert_recurrence} shows the extracted recurrence.
%Without the arbitrary maximum, it is much more obvious how to find a solution to the recurrence.
%The recurrence is from a potential to a complexity, consequently we can extract a recurrence for the cost,
%equation \ref{eq:insert_cost}, and a recurrence for the potential, equation \ref{eq:insert_potential}, simply by taking the projections of equation \ref{eq:insert_recurrence}.
%The extracted recurrences for the cost and potential can then be solved by the substitution method.
%\begin{equation}
%  \label{eq:insert_recurrence}
%  g(n) = \begin{cases}
%    (1,1) & n = 0 \\
%    (4 + \pi_0(\pi_1(f\ 1)\ 1) + \pi_0g(n-1), 1+\pi_1g(n-1)) & n > 0
%  \end{cases}
%\end{equation}
%
%The cost recurrence is given by $\pi_0 \circ g$.
%\begin{equation}
%\label{eq:insert_cost}
%c(n) = \begin{cases}
%  1 & n = 0 \\
%  4 + \pi_0(\pi_1(f\ 1)\ 1) + c(n-1) & n > 0
%\end{cases}
%\end{equation}
%
%This recurrence is quite simple to solve.
%The solution and proof of the solution are given in theorem \ref{thm:insert_cost}.
%\begin{theorem}
%\label{thm:insert_cost}
%  $c(n) = (4 + \pi_0(\pi_1(f\ 1)\ 1))n + 1$
%\end{theorem}
%\begin{proof}
%  We prove this by induction on $n$.
%  \begin{description}
%    \item[case $n=0$] $c(0) = \pi_0g(0) = 1$
%    \item[case $n>0$]
%      \begin{align*}
%        c(n) &= 4 + \pi_0(\pi_1(f\ 1)\ 1) + c(n-1) \\
%        &= 4 + \pi_0(\pi_1(f\ 1)\ 1) + (4 + \pi_0(\pi_1(f\ 1)\ 1))(n-1) + 1 \\
%        &= (4 + \pi_0(\pi_1(f\ 1)\ 1)) n + 1
%      \end{align*}
%  \end{description}
%\end{proof}
%The solution tells us the cost of the \T{rec} construct in insert is linear in the size of the list.
%The constant factor cannot be determined because we do not know the cost of applying $f$ to its arguments.
%
%The potential recurrence is given by $\pi_1 \circ g$.
%\begin{equation}
%  \label{eq:insert_potential}
%  p(n) = \begin{cases}
%    1 & n = 0 \\
%    1 + p(n-1) & n > 0
%  \end{cases}
%\end{equation}
%
%\begin{theorem}
%  $p(n) = 1 + n$
%\end{theorem}
%\begin{proof}
%  We prove this by induction on $n$.
%  \begin{description}
%    \item[case $n=0$] $p(0) = 1$
%    \item[case $n>0$] $p(n) = 1 + p(n - 1) = 1 + n$
%  \end{description}
%\end{proof}
%
%
%\subsubsection{Interpretation}
%The \T{rec} construct again drives the cost and potential of \T{sort}.
%The walkthrough of the interpretation of the \T{rec} is given in figure \ref{fig:sort_rec_interp}.
%Equation \ref{eq:sort_interp0_init} shows the initial recurrence extracted.
%\begin{equation}
%  \label{eq:sort_interp0_init}
%  g(n) = \bigvee_{size(z)\leq n} case(z,(\lambda(\LP\RP).(1,0),\lambda(1,m).4 + \pi_0 g(m)) +_c insert\ f\ 1\ \pi_1g(m))
%\end{equation}
%
%We work the recurrence into a more recognisable form using some manipulation of the big max operator and some facts about $insert$.
%Observe for $n=0$, $g(n) = (1,0)$ and for $n>0$
%\begin{align*}
%  g(n) &= \bigvee_{size(z)\leq n} case(z,(\lambda(\LP\RP).(1,0),\lambda(1,m).4 + \pi_0 g(m)) +_c insert\ f\ 1\ \pi_1g(m)) \\
%  &= g(n-1) \vee \bigvee_{size(z) = n} case(z,(\lambda(\LP\RP).(1,0),\lambda(1,m).4 + \pi_0 g(m)) +_c insert\ f\ 1\ \pi_1g(m)) \\
%  &= g(n-1) \vee (4 + \pi_0 g(n-1)) +_c insert\ f\ 1\ \pi_1g(n-1) \\
%  &= g(n-1) \vee (4 + \pi_0 g(n-1) + \pi_0 (insert\ f\ 1\ \pi_1g(n-1)), \pi_1 (insert\ f\ 1\ \pi_1g(n-1))) \\
%  &\text{since $\pi_0 (insert\ f\ 1\ m) > 0$ and $\pi_1 (insert\ f\ 1\ m) = 1 + m$} \\
%  &= (4 + \pi_0 g(n-1) + \pi_0 (insert\ f\ 1\ \pi_1g(n-1)), \pi_1 (insert\ f\ 1\ \pi_1g(n-1)))
%\end{align*}
%
%So our simplified recurrence is
%\begin{equation}
%  \label{eq:sort_rec_final}
%  g(n) = \begin{cases}
%    (1,0) & n=0 \\
%    (4 + \pi_0 g(n-1) + \pi_0 (insert\ f\ 1\ \pi_1g(n-1)), \pi_1 (insert\ f\ 1\ \pi_1g(n-1))) & n > 0
%  \end{cases}
%\end{equation}
%
%From this we can extract recurrences for the cost and the potential simply by taking projections from $g$.
%We begin with the potential because we will require the solution to the potential recurrence to solve the cost recurrence.
%
%Let $p = \pi_1 \circ g$.
%\begin{equation}
%  \label{eq:sort_rec_potential}
%  p(n) = \begin{cases}
%    0 & n=0 \\
%    \pi_1 (insert\ f\ 1\ \pi_1g(n-1))) & n > 0
%  \end{cases}
%\end{equation}
%
%We prove the size of the potential of the output is same as the size of the input.
%In other words, \T{sort} does not change the size of the list.
%\begin{theorem}
%  \label{thm:sort_potential}
%  $p(n) = n$
%\end{theorem}
%\begin{proof}
%  We prove this by straightforward induction on $n$.
%  \begin{description}
%    \item[case $n=0$] $p(0) = 0$\hfill \\
%    \item[case $n>0$] $p(n) = \pi_1(insert\ f\ 1\ pi_g(n-1)) = \pi_1(insert\ f\ 1\ (n-1)) = n$
%  \end{description}
%\end{proof}
%
%Let $c = \pi_0 \circ g$.
%\begin{equation}
%  \label{eq:sort_rec_cost}
%  c(n) = \begin{cases}
%    1 & n=0 \\
%    4 + \pi_0 g(n-1) + \pi_0 (insert\ f\ 1\ \pi_1g(n-1) & n > 0
%  \end{cases}
%\end{equation}
%
%\begin{theorem}
%  \label{thm:sort_cost}
%\end{theorem}
%\begin{proof}
%  We prove this by straightforward induction on $n$.
%  \begin{description}
%    \item[case $n=0$:] $c(0) = 1$
%    \item[case $n>0$:]
%      \begin{align*}
%        c(n) &= 4 + \pi_0 g(n-1) + \pi_0(insert\ f\ 1\ \pi_1g(n-1)) \\
%        &= 4 + \pi_0 g(n-1) + \pi_0(insert\ f\ 1\ (n-1) \\
%        &= 4 + (4 + \pi_0(\pi_1(f\ 1)\ 1))(n-1) + 1) + \pi_0 g(n-1)
%      \end{align*}
%      TODO COMPLETE
%  \end{description}
%\end{proof}
%
%

\section{Sequential List Map}
\label{sec:sequential_list_map}

This example is provided for comparison with the parallel list map example
given later in this thesis. We use the familiar \T{list} datatype.
%
\begin{align*}
\T{list } &= \T{Nil of unit | Cons of int $\times$ list}
\end{align*}
%
The \T{map} function recurses on the list, applying its first argument to each
element.
%
\begin{equation*}
  \T{map} = \lambda f. \lambda xs . \T{rec}(xs, \T{Nil} \mapsto \T{Nil}, \T{Cons} \mapsto \langle y \langle ys, y \rangle\rangle. \T{Cons}\langle f\ y, \T{force}(r)\rangle)
\end{equation*}

\subsection{Translation}
The translation
\begin{align*}
  &\text{We apply the rule for translating an abstraction twice.}\\
  \|\T{map}\| &= \|\lambda f. \lambda xs . \T{rec}(xs, \T{Nil} \mapsto \T{Nil}, \T{Cons} \mapsto \langle y \langle ys, y \rangle\rangle. \T{Cons}\langle f\ y, \T{force}(r)\rangle)\| \\
              &= \langle 0, \lambda f. \langle 0, \lambda xs . \T{rec}(xs, \T{Nil} \mapsto \T{Nil}, \T{Cons} \mapsto \langle y \langle ys, y \rangle\rangle. \T{Cons}\langle f\ y, \T{force}(r)\rangle)\rangle\rangle \\
              &\text{We apply the rule for translating a \T{rec} construct. We use $\|xs\| = \langle 0,xs\rangle$.}\\
              &= \langle 0, \lambda f. \langle 0, \lambda xs .\T{rec}(xs, \T{Nil} \mapsto 1 +_c \|\T{Nil}\|, \\
              &\quadten \T{Cons} \mapsto \langle y \langle ys, y \rangle\rangle.1 +_c \|\T{Cons}\langle f\ y, \T{force}(r)\rangle\|)\rangle\rangle \\
  %
              &\text{The translation of \T{Nil} is $\langle 0, \T{Nil}\rangle$.}\\
  %
              &\text{To translate the \T{Cons} branch we translate the subexpressions first.}\\
              &\quadthree\|\T{Cons}\langle f\ x,\T{force}(r)\rangle\| = \langle \|\langle f\ x,\T{force}(r)\rangle_c, \T{Cons}\|\langle f\ x,\T{force}(r)\rangle\|\rangle \\
              &\quadfive \|\langle f\ x,\T{force}(r)\rangle\| = \langle\|f\ x\|_c + \|\T{force}(r)\|_c, \langle \|f\ x\|_p,\|\T{force}(r)\|_p\rangle\rangle \\
  %
              &\quadsix \|f\ x\| = (1 + \|f\|_c + \|x\|_c) +_c \|f\|_p\|x\|_p = \langle 1 + (f\ x)_c, (f\ x)_p\rangle \\
  %
              &\quadsix \|\T{force}(r)\| = \langle 0,r\rangle +_c \langle 0,r\rangle_p = r \\
  %
              &\quadfive \|\langle f\ x,\T{force}(r)\rangle\| = \langle 1 + (f\ x)_c + r_c, \langle (f\ x)_p,r_p\rangle\rangle \\
              &\quadthree \|\T{Cons}\langle f\ x,\T{force}(r)\rangle\| = \langle 1 + (f\ x)_c + r_c, \T{Cons}\langle (f\ x)_p,r_p\rangle\rangle \\
              &= \langle 0, \lambda f. \langle 0, \lambda xs . \T{rec}(xs, \T{Nil} \mapsto \langle 1,\T{Nil}\rangle, \\
              &\quadten \T{Cons} \mapsto \langle y \langle ys, y \rangle\rangle.\langle 2 + (f\ x)_c + r_c, \T{Cons}\langle (f\ x)_p,r_p\rangle\rangle \\
\end{align*}
%
We will translate \T{map} applied to some function \T{f} and a list \T{xs}.
%
\begin{align*}
  \|\T{map f xs}\| &= (1 + \|\T{map}\ f\|_c + \|xs\|_c) +_c \|\T{map}\ f\|_p \|xs\|_p \\
                   &\text{The translation of \T{map} partially applied to map is:}\\
                   &\quadthree \|\T{map}\ f\| = (1 + \|\T{map}\|_c + \|f\|_c) +_c \|\T{map}\|_p \|f\|_p \\
                   &\text{The cost of partially applied \T{map} is 0.} \\
                   &= (2 + \|f\|_c + \|xs\|_c) +_c (\|\T{map}\|_p\ \|f\|_p)_p \|xs\|_p
\end{align*}

\subsection{Interpretation}

We interpret lists as a pair of their largest element and length.
%
\begin{align*}
  \llbracket \T{list} \rrbracket &= \mathbb{Z} \times \mathbb{N} \\
  D^\T{list} &= \{*\} + (\llbracket \mathbb{Z} \rrbracket \times \llbracket \T{list} \rrbracket) \\
  size_\T{list}(*) &= (0, 0) \\
  size_\T{list}((x, (m, n))) &= (max(x, m), 1 + n)
\end{align*}
%
The recursor of \T{map} drives the cost, so we will interpret the recursor
first.
%
\begin{align*}
  g(i, n) &= \llbracket \T{rec}(xs, \T{Nil} \mapsto \langle 1,\T{Nil}\rangle, \\
       &\quadfive \T{Cons} \mapsto \langle 2 + (f\ x)_c + r_c,\T{Cons}\langle (f\ x)_p,r_p\rangle\rangle \rrbracket \{xs \mapsto n, f \mapsto f\}\\
       &= \bigvee\limits_{size(z) \leq n} case(z, f_{Nil}, f_{Cons}) \\
       &\text{where} \\
  f_{Nil}(\ast) &= \llbracket \langle 1,\T{Nil}\rangle\rrbracket \\
                &= (1,0) \\
  f_{Cons}(i, m) &= \llbracket \langle 2 + (f\ x)_c + r_c,\T{Cons}\langle (f\ x)_p,r_p\rangle\rangle\rrbracket \xi\\
                 &\qquad \text{where } \xi = \{xs \mapsto n, f \mapsto f, y \mapsto i, ys \mapsto m, r \mapsto g(i, m) \} \\
                 &= (2 + (f\ i)_c + g_c(i,m), (max((f\ i)_p, \pi_0g_p(i,m)), 1 + \pi_1 g_p(i,m)))
\end{align*}
%
We break the recurrence into the cases of $n=0$ and $n>0$
%
\begin{description}
  \item[case $n=0$]\hfill \\
    $g(i,0) = \bigvee\limits_{size(z) \leq 0} case(z, f_{Nil}, f_{Cons}) = (1,0)$
  \item[case $n>0$]\hfill \\
    \begin{align*}
      g(i,n) &= \bigvee\limits_{size(z) \leq n} case(z, f_{Nil}, f_{Cons}) \\
             &= \bigvee\limits_{size(z) \leq n-1} case(z,f_{Nil},f_{Cons}) \vee \bigvee\limits_{size(z) = n} case(z,f_{Nil}, f_{Cons}) \\
             &= g(i,n-1) \vee \\
             &\qquad (2 + (f\ i)_c + g_c(i,n-1), (max((f\ i)_p,\pi_0g_p(i,n-1)), 1 + \pi_1 g_p(i,n-1))) \\
             &\text{Since $g_c(i,n-1) \leq g_c(i,n-1)$, then $g_c(i,n-1)\leq 2 + (f\ i)_c + g_c(i,n-1)$.} \\
             &\text{Since $g_p(i,n-1) \leq g_p(i,n-1)$, then $\pi_0 g_p(i,n-1) \leq max((f\ i)_p, \pi_0g_p(i,n-1))$,} \\
             &\text{and $\pi_1 g_p(i,n-1) \leq 1 + \pi_1 g_p(i,n-1)$.} \\
             &= (2 + (f\ i)_c + g_c(i,n-1), (max((f\ i)_p,\pi_0g_p(i,n-1)), 1 + \pi_1 g_p(i,n-1))) \\
    \end{align*}
\end{description}
%
We obtain a closed form solution for the cost by using the substitution method.
%
\begin{lemma}
  \label{lem:listmap_g_cost}
  $g_c(i,n) = (2 + (f\ i)_c) n + 1$
\end{lemma}
%
\begin{proof}
  The proof is by induction on $n$.
  \begin{description}
    \item[case $n=0$]\hfill \\
      $g_c(i,0) = (1,0)_c = 1$
    \item[case $n>0$]\hfill \\
      \begin{align*}
        g_c(i,n) &= 2 + (f\ i)_c + g_c(i,n-1)\\
                 &= 2 + (f\ i)_c + (2 + (f\ i)_c)(n-1) + 1\\
                 &= 2 + (f\ i)_c + (2 + (f\ i)_c) n - 2 - (f\ i)_c + 1\\
                 &= (2 + (f\ i)_c) n + 1
      \end{align*}
    \end{description}
\end{proof}
%
We obtain a closed form of the solution for the potential of the recursor.
%
\begin{lemma}
  \label{lem:listmap_g_potential}
  $g_p(i,n) = ((f\ i)_p, n)$
\end{lemma}
%
\begin{proof}
  The proof is by induction on $n$.
  \begin{description}
    \item[case $n=0$]\hfill \\
      $g_p(i,0) = (1,0)_p = 0$
    \item[case $n>0$]\hfill \\
      \begin{align*}
        g_p(i,n) &= (max((f\ i)_p,\pi_0g_p(i,n-1)), 1 + \pi_1 g_p(i,n-1)) \\
                 &= (max((f\ i)_p, (f\ i)_p), 1 + n - 1) \\
                 &= ((f\ i)_p, n)
      \end{align*}
  \end{description}
\end{proof}
%
Using the results from lemmas \ref{lem:listmap_g_cost} and
\ref{lem:listmap_g_potential}, we interpret the translation of \T{map f xs}.
Recall the translation is $(2 + \|f\|_c + \|xs\|_c) +_c (\|\T{map}\|_p\ \|f\|_p)_p \|xs\|_p$.
So the interpretation is $2 \pplus_c (\llbracket\|\T{map}\|\rrbracket f)_p (i,n)$,
where $(i,n)$ is the interpretation of $xs$ and we assume $f$ and $xs$ have $0$
cost.
%
\begin{lemma}
  \label{lem:listmap_cost}
  $\llbracket\|\T{map f xs}\|_c\rrbracket = 3 + (2 + (f\ i)_c)n$ where $f$ is the
  interpretation of the translation of \T{f} and $(i,n)$ is the interpretation
  of the translation of \T{xs}.
\end{lemma}
%
\begin{lemma}
  \label{lem:listmap_potential}
  $\llbracket\|\T{map f xs}\|_p\rrbracket = ((f\ i)_p,n)$ where $f$ is the
  interpretation of the translation of \T{f} and $(i,n)$ is the interpretation
  of the translation of \T{xs}.
\end{lemma}
%
Lemma \ref{lem:listmap_cost} shows that the cost of \T{map f xs} is linear in
the size of the list but also depends on the cost of applying \T{f} to the
elements of \T{xs}. Lemma \ref{lem:listmap_potential} shows the cost of future
uses of \T{map f xs} depends on the length of \T{xs} and the size of the result
of applying \T{f} to the elements of \T{xs}.

\section{Sequential Tree Map}

This example is presented for comparison with the parallel tree map given in
chapter 4.
We use \T{int} labelled binary trees.
%
\begin{equation*}
  \T{datatype tree} = \T{E of Unit | N of int$\times$tree$\times$tree}
\end{equation*}
%
Tree \T{map f t} applies the function \T{f} to every label in the tree \T{t}.
%
\begin{align*}
  \T{map} &= \lambda f.\lambda t.\T{rec}(t, \T{E} \mapsto \T{E}, \T{N} \mapsto \LP x,\LP t_0,r_0\RP,\LP t_1,r_1\RP\RP. \T{N}\LP f\ x, \T{force}(r_0), \T{force}(r_1)\RP)
\end{align*}

\subsection{Translation}
The translation
\begin{align*}
  &\text{We apply the abstraction rule twice.} \\
  \|\T{map}\| &= \LP 0,\lambda f.\LP 0,\lambda t.\|\T{rec}(t, \T{E} \mapsto \T{E},\\
              &\quadsix\T{N} \mapsto \LP x,\LP t_0,r_0\RP,\LP t_1,r_1\RP\RP.\T{N}\LP f\ x,\T{force}(r_0),\T{force}(r_1)\RP)\|\RP\RP \\
              &\text{We apply the recursor translation rule.} \\
              &= \LP 0,\lambda f.\LP 0,\lambda t.\|t\|_c +_c \T{rec}(\|t\|_p, \T{E} \mapsto 1 +_c \|\T{E}\|,\\
              &\quadsix \T{N} \mapsto \LP x,\LP t_0,r_0\RP,\LP t_1,r_1\RP\RP.1 +_c \|\T{N}\LP f\ x,\T{force}(r_0),\T{force}(r_1)\RP\|)\RP\RP \\
  %
              &\text{The translation of the variable $t$ is $\LP 0,t\RP$.} \\
              &\text{The translation of the constructor \T{E} with $\LP\RP$ as its argument is $\LP 0,\T{E}\RP$.} \\
              &\text{The translation of the constructor \T{N}$\LP e\RP$ is $\LP \|e\|_c,\T{N}\LP\|e\|_p\RP\RP$.} \\
  %
              &\text{$f$ and $x$ are variables, so their translations are $\LP 0,f\RP$ and $\LP 0,x\RP$ respectively.}\\
              &\quadthree \|f\ x\| = \LP 1 + \|f\|_c + \|x\|_c +_c \|f\|_p\|x\|_p = \LP 1 + (f\ x)_c,(f\ x)_c\RP \\
              &\text{$r_0$ and $r_1$ are also variables.} \\
              &\quadthree \|\T{force}(r_i)\| = \|r_i\|_c +_c \|r_i\|_p = \LP 0,r_i\RP +_c \LP 0,r_i\RP = r_i \\
  %
              &\text{We use this result to translate the argument to the \T{N} constructor.} \\
              &\quadthree \|\LP f\ x,\T{force}(r_0),\T{force}(r_1)\RP\| = \\
              &\quadsix \LP \|f\ x\|_c + \|\T{force}(r_0)\|_c + \|\T{force}(r_1)\|_c, \\
              &\quadseven \LP \|f\ x\|_p,\|\T{force}(r_0)\|_p\RP,\|\T{force}(r_1)\|_p\RP\RP \\
              &\quadthree = \LP 1 + (f\ x)_c + r_{0c} + r_{1c},\LP (f\ x)_p, r_{0p}, r_{1p}\RP\RP\\
  %
              &\text{We use to translate the \T{N} constructor.} \\
              &\T{N}\LP f\ x,\T{force}(r_0),\T{force}(r_1)\RP = \LP 1 + (f\ x)_c + r_{0c} + r_{1c}, \T{N}\LP(f\ x)_p,r_{0p},r_{1p}\RP\RP \\
  %
              &\text{We use this to complete the translation of \T{map}.} \\
              &= \LP 0,\lambda f.\LP 0,\lambda t.\T{rec}(t, \T{E} \mapsto \LP 1,\T{E}\RP,\\
              &\quadfive \T{N} \mapsto \LP x,\LP t_0,r_0\RP,\LP t_1,r_1\RP\RP.\LP 2 + (f\ x)_c + r_{0c} + r_{1c}, \T{N}\LP (f\ x)_p,r_{0p},r_{1p}\RP\RP)\RP\RP \\
\end{align*}

\subsection{Interpretation}

\chapter{Work and Span}

\section{Work and span}
Work and span is a method of calculating the cost of programs that may be run on multiple machines.
The work of a program corresponds to the total number of steps needed to run.
The span of a program is the steps in the critical path.
The critical path is the largest number of steps that must be executed sequentially.
The length of the critical path determines how much a program can be parallelized.
If the span is equal to the work, than every step in the computation depends on the previous step, and the program cannot be parallelized.

Instead of calculating the cost of program, we will construct a cost graph.
The cost graph represents dependencies between computations in a program and may be used to determine optimal execution strategies.

A cost graph is defined as follows.

\[ \mathcal{C} ::= 0 | 1 | \mathcal{C} \oplus \mathcal{C} | \mathcal{C} \otimes \mathcal{C} \]

The operator $\oplus$ connects to cost graphs who must be combined sequentially.
The operator $\otimes$ connects cost graphs which may be combined in parallel.

The work of a cost graph is defined as 
\begin{equation*}
  work(c) = \begin{cases}
    0 &\text{if } c = 0 \\
    1 &\text{if } c = 1 \\
    work(c_0) + work(c_1) &\text{if } c = c_0 \otimes c_1 \\
    work(c_0) + work(c_1) &\text{if } c = c_0 \oplus c_1
  \end{cases}
\end{equation*}

The span of a cost graph is defined as
\begin{equation*}
  span(c) = \begin{cases}
    0 &\text{if } c = 0 \\
    1 &\text{if } c = 1 \\
    max(span(c_0), span(c_1)) &\text{if } c = c_0 \otimes c_1 \\
    span(c_0) + span(c_1) &\text{if } c = c_0 \oplus c_1
  \end{cases}
\end{equation*}


We alter the operational semantics of the source language slightly to reflect that the cost of evaluating an expression is a cost graph instead of an integer.
Figure \ref{fig:ws_srclang_oper_sem} shows the new operational semantics.
For tuples, the subexpressions may be evaluated in parallel, so the cost of evaluating a tuple is the cost graphs of the subexpressions connected by $\otimes$.
For \T{split}, the second subexpression depends on the result of the first subexpression, so the cost of evaluating the \T{split} is the cost graphs of the subexpression connected by $\oplus$.


\begin{figure}
\label{fig:ws_srclang_oper_sem}
\caption{Source language operational semantics}
\AxiomC{$e_0 \downarrow^{n_0} v_0$}
\AxiomC{$e_1 \downarrow^{n_1} v_1$}
\BinaryInfC{$\langle e_0, e_1 \rangle \downarrow^{n_0 \otimes n_1} \langle v_0, v_1 \rangle$}
\DisplayProof

\AxiomC{$e_0 \downarrow^{n_0} \langle v_0, v_1 \rangle$}
\AxiomC{$e_1[v_0/x_0, v_1/x_1] \downarrow^{n_1} v$}
\BinaryInfC{$split(e_0, x_0.x_1.e_1) \downarrow^{n_0 \oplus n_1} v$}
\DisplayProof

\AxiomC{$e_0 \downarrow^{n_0} \lambda x.e_0'$}
\AxiomC{$e_1 \downarrow^{n_1} v_1$}
\AxiomC{$e_0'[v_1/x] \downarrow^n v$}
\TrinaryInfC{$e_0\ e_1 \downarrow^{(n_0 \otimes n_1) \oplus n \oplus 1} v$}
\DisplayProof

\AxiomC{}
\UnaryInfC{$delay(e) \downarrow^0 delay(e)$}
\DisplayProof

\AxiomC{$e \downarrow^{n_0} delay(e_0)$}
\AxiomC{$e_0 \downarrow^{n_1} v$}
\BinaryInfC{$force(e) \downarrow^{n_0 \oplus n_1} v$}
\DisplayProof

\AxiomC{$e \downarrow^n v$}
\UnaryInfC{$C e \downarrow^n C v$}
\DisplayProof

\AxiomC{$e \downarrow^{n_0} C v_0$}
\AxiomC{$map^{\phi_C}(y.\langle y, delay(rec(y, \overline{C \mapsto x.e_C}))\rangle, v_0) \downarrow^{n_1} v_1$}
\AxiomC{$e_C[v_1/x] \downarrow^{n_2} v$}
\TrinaryInfC{$rec(e, \overline{C \mapsto x.e_C}) \downarrow^{1 \oplus n_0 \oplus n_1 \oplus n_2} v$}
\DisplayProof

\AxiomC{}
\UnaryInfC{$map^t(x.v, v_0) \downarrow^0 v[v_0/x]$}
\DisplayProof

\AxiomC{}
\UnaryInfC{$map^\tau(x.v, v_0) \downarrow^0 v_0$}
\DisplayProof

\AxiomC{$map^{\phi_0}(x.v, v_0) \downarrow^{n_0} v_0'$}
\AxiomC{$map^{\phi_1}(x.v, v_1) \downarrow^{n_1} v_1'$}
\BinaryInfC{$map^{\phi_0 \times \phi_1}(x.v, \langle v_0, v_1 \rangle) \downarrow^{n_0 \otimes n_1} \langle v_0', v_1'\rangle$}
\DisplayProof

\AxiomC{}
\UnaryInfC{$map^{\tau \to \phi}(x.v, \lambda y.e) \downarrow^0 \lambda y.let(e, z.map^\phi(x.v, z))$}
\DisplayProof

\AxiomC{$e_0 \downarrow^{n_0} v_0$}
\AxiomC{$e_1[v_0/x] \downarrow^{n_1} v$}
\BinaryInfC{$let(e_0, x.e_1) \downarrow^{n_0 \oplus n_1} v$}
\DisplayProof
\end{figure}


The complexity translation is given in figure \ref{fig:ws_complexity_translation}.
The operator $E_0 \oplus_c E_1$ is syntactic sugar for $\langle E_0 \oplus E_{1c}, E_{1p} \rangle$.
The translation is similar to the original translation except we replace the use of $+$ and $+_c$ with $\oplus$ and $\oplus_c$.
In the tuple case and function application case we use $\otimes$ since the arguments do not depend on each other and may be computed in parallel.


\begin{figure}
  \label{fig:ws_complexity_translation}
  \caption{Work and span translation from source language to compleity language}
  \begin{align*}
    \|x\| &= \langle 0, x \rangle \\
    \|\langle\rangle\| &= \langle 0, \langle \rangle \rangle \\
    \|\langle e_0, e_1 \rangle \| &= \langle \|e_0\|_c \otimes \|e_1\|_c, \langle \|e_0\|_p, \|e_1\|_p\rangle\rangle \\
    \|split(e_0, x_0.x_1.e_1)\| &= \|e_0\|_c \oplus_c \|e_1\|[\pi_0\|e_0\|_p/x_0, \pi_1\|e_1\|_p/x_1] \\
    \|\lambda x.e\| &= \langle 0, \lambda x.\|e\| \rangle \\
    \|e_0\ e_1\| &= 1 \oplus (\|e_0\|_c \otimes \|e_1\|_c) \oplus_c \|e_0\|_p\ \|e_1\|_p \\
    \|delay(e)\| &= \langle 0, \|e\|\rangle \\
    \|force(e)\| &= \|e\|_c \oplus_c \|e\|_p \\
    \|C_i^\delta e\| &= \langle \|e\|_c, C_i^\delta \|e\|_p \rangle \\
    \|rec^\delta(e, \overline{C \mapsto x.e_C})\| &= \|e\|_c \oplus_c rec^\delta(\|e\|_p, \overline{C \mapsto x.1 \oplus_c \|e_C\|}) \\
    \|map^\phi(x.v_0, v_1)\| &= \langle 0, map^{\langle\langle \phi \rangle \rangle} (x. \|v_0\|_p, \|v_1\|_p)\rangle \\
    \|let(e_0, x.e_1)\| &= \|e_0\|_c \oplus_c \|e_1\|[\|e_0\|_p/x]
  \end{align*}
\end{figure}

\section{Bounding Relation}
TODO


\section{Parallel List Map}
If we revisit \T{map} using the work and span translation, we will get a different result.


We use the same data type \T{list} and \T{map} function as in sequential list map.
\begin{equation*}
  \T{datatype list} = \T{Nil of Unit | Cons of int $\times$ list}
\end{equation*}

\begin{equation*}
  \T{map} = \lambda f. \lambda xs . \T{rec}(xs, \T{Nil} \mapsto \T{Nil}, \T{Cons} \mapsto \langle y \langle ys, y \rangle\rangle. \T{Cons}\langle f\ y, \T{force}(r)\rangle)
\end{equation*}

The derivation of the complexity expression is given in figure \ref{fig:ws_map_complexity_translation}.
\begin{figure}
  \label{fig:ws_map_complexity_translation}
  \caption{Work and span complexity translation of \T{map f xs}.}
  \[\begin{split}
    &\|\T{map f xs}\| = \\
    &  (3 \oplus f_c \otimes xs_c) \oplus_c \T{rec}(xs_p, \T{Nil}\mapsto 1 \oplus_c \|\T{Nil}\|, \\
    &\quadten\qquad \T{Cons} \mapsto \langle y, \langle ys, r \rangle \rangle. 1 \oplus_c \|\T{Cons}\langle f\ y, \T{force}(r)\rangle \|) \\
    %Nil branch
    &\quad 1 \oplus_c \|\T{Nil}\| = 1 \oplus_c \langle 0, Nil \rangle = \langle 1, Nil \rangle \\
    %Cons branch
    &\quad\|\T{Cons}\langle f\ y, \T{force}(r)\rangle \| = \langle \|\langle f\ y, \T{force}(r)\rangle\|_c, \T{Cons} \| \langle f\ y, \T{force}(r)\rangle \|_p\rangle \\
    &\qquad \|\langle f\ y, \T{force}(r)\rangle\| = \langle \|f\ y\|_c \otimes \|\T{force}(r)\|_c, \langle \|f\ y\|_p, \|\T{force}(r)\|_p\rangle\rangle \\
    %f y
    &\quadthree \|f\ y\| = (1 \oplus \|f\|_c \otimes \|y\|_c) \oplus_c \|f\|_p \|y\|_p \\
    &\quadfour = (1 \oplus \langle 0, f_p \rangle_c \otimes \langle 0, y \rangle_c) \oplus_c \langle 0, f_p \rangle_p \langle 0, y \rangle_p \\
    &\quadfour = 1 \oplus_c f_p\ y \\
    &\quadfour = \langle 1 \oplus (f_p y)_c, (f_p\ y)_p \rangle \\
    %force r
    &\quadthree \|\T{force}(r)\| = \|r\|_c \oplus_c \|r\|_p \\
    &\quadfour = \langle 0, r \rangle_c \oplus_c \langle 0, r \rangle_p \\
    &\quadfour = r \\
    %<...>
    &\qquad \|\langle f\ y, \T{force}(r)\rangle\| = \langle (1 \oplus (f_p\ y)_c) \otimes r_c, \langle (f_p\ y)_p, r_p \rangle \rangle \\
    %Cons<..>
    &\quad\|\T{Cons}\langle f\ y, \T{force}(r)\rangle \| = \langle (1 \oplus (f_p\ y)_c) \otimes r_c, \T{Cons} \langle (f_p\ y)_p, r_p \rangle \rangle \\
    %rec
    &\|\T{map f xs}\| = (3 \oplus f_c \otimes xs_c) \oplus_c \\
    &\quad \T{rec}(xs_p, \T{Nil}\mapsto \langle 1, \T{Nil}\rangle, \\
    &\quadfour \T{Cons} \mapsto \langle y, \langle ys, r \rangle \rangle. \langle 1 \oplus ((1 \oplus (f_p\ y)_c) \otimes r_c), \T{Cons} \langle (f_p\ y)_p, r_p \rangle \rangle)
  \end{split}\]
\end{figure}

The complexity language translation is
\begin{equation*}
\begin{split}
    &\|\T{map f xs}\| = (3 \oplus f_c \otimes xs_c) \oplus_c \\
    &\quad \T{rec}(xs_p, \T{Nil}\mapsto \langle 1, \T{Nil}\rangle, \\
    &\quadfour \T{Cons} \mapsto \langle y, \langle ys, r \rangle \rangle. \langle 1 \oplus ((1 \oplus (f_p\ y)_c) \otimes r_c), \T{Cons} \langle (f_p\ y)_p, r_p \rangle \rangle)
\end{split}
\end{equation*}


We interpret lists as a pair of their largest element and length.
\begin{align*}
  \llbracket \T{list} \rrbracket &= \mathbb{Z} \times \mathbb{N} \\
  D^\T{list} &= \{*\} + (\llbracket \mathbb{Z} \rrbracket \times \llbracket \T{list} \rrbracket) \\
  size_\T{list}(*) &= (0, 0) \\
  size_\T{list}((x, (m, n))) &= (max(x, m), 1 + n)
\end{align*}

The interpretation of the recurser is given in figure \ref{fig:ws_map_interpretation}.

\begin{figure}
  \label{fig:ws_map_interpretation}
  \caption{Interpretation of recurser in \T{map}}
  \[\begin{split}
      &\quad \text{Let } \eta = \{ xs \mapsto (0, (m, n)), f \mapsto (0, f)\}\\
      &\quad g(f, (m, n)) = \llbracket \T{rec}(xs_p, \T{Nil} \mapsto \langle 1, \T{Nil}\rangle,\\
      &\quadten \T{Cons} \mapsto \langle y, \langle ys, r \rangle \rangle . \langle 1 \oplus (1 \oplus (f_p\ y)_c \otimes r_c), \T{Cons}\langle (f_p\ y)_p, r_p\rangle\rangle) \rrbracket \eta\\
      &\qquad = \bigvee\limits_{(m_1, n_1) \leq (m, n)} case((m_1, n_1), \T{Nil} \mapsto \llbracket \langle 1, \T{Nil} \rangle \rrbracket \eta, \\
      &\quadten \T{N} \mapsto \langle y, \langle ys, r \rangle \rangle . \llbracket \langle 1 \oplus ((1 \oplus (f_p\ y)_c) \otimes r_c), \T{Cons} \langle (f_p\ y)_p, r_p \rangle \rangle \rrbracket \eta_c) \\
      &\qquad \text{where } \eta_c = \{xs \mapsto (0, (m, n)), f \mapsto (0, f), y \mapsto m, ys \mapsto (0, (m, n)), \\
      &\quadeight r \mapsto g(f, (m, n-1)))\} \\
      %Nil branch
      &\quadthree \text{\T{Nil} branch} \\
      &\quadthree \llbracket \langle 1, \T{Nil} \rangle \rrbracket \eta = (\llbracket 1 \rrbracket \eta , \llbracket \T{Nil} \rrbracket \eta) = (1, (0, 0)) \\
      %Cons branch
      &\quadthree \text{\T{Cons} branch} \\
      &\quadthree  \llbracket \langle 1 \oplus ((1 \oplus (f_p\ m)_c) \otimes r_c), \T{Cons} \langle (f_p\ m)_p, r_p \rangle \rangle \rrbracket \eta_c \\
      &\quadfour = (1 \oplus ((1 \oplus (f\ m)_c) \otimes g_c(f, (m, n-1))), ((f\ m)_p, \pi_1 g_p(f, (m, n-1)))) \\
      %putting branches together
      &\quad g(f, (m, n)) = \\
      &\qquad \bigvee\limits_{(m_1, n_1) \leq (m, n)} case((m_1, n_1), \T{Nil} \mapsto (1, (0, 0)), \\
      &\quadfour \T{Cons} \mapsto (1 \oplus ((1 \oplus (f\ m)_c) \otimes g_c(f, (m, n-1))), ((f\ m)_p, \pi_1 g_p(f, (m, n-1))))) 
  \end{split}\]
\end{figure}

The result is 
\begin{equation*}
  \begin{split}
  &g(f, (m, n)) = \\
  &\quad \bigvee\limits_{(m_1, n_1) \leq (m, n)} case((m_1, n_1), \T{Nil} \mapsto (1, (0, 0)), \\
  &\quadfour \T{Cons} \mapsto (1 \oplus ((1 \oplus (f\ y)_c) \otimes g_c(f, (m, n-1))), ((f\ y)_p, \pi_1 g_p(f, (m, n-1))))) 
  \end{split}
\end{equation*}

%In figure \ref{fig:ws_map_massage}, we massage the recurrence into a simpler form by eliminating the big max and the $case$.
%\begin{figure}
%  \label{fig:ws_map_massage}
%  \caption{Simplification of the interpretation of \T{map}}
%  \[\begin{split}
%  &g(f, (m, n)) = \\
%  &\quad \bigvee\limits_{(m_1, n_1) \leq (m, n)} case((m_1, n_1), \T{Nil} \mapsto (1, (0, 0)), \\
%  &\quadfive \T{Cons} \mapsto (1 \oplus ((1 \oplus (f\ y)_c) \otimes g_c(f, (m, n-1))), ((f\ y)_p, \pi_1 g_p(f, (m, n-1)))))  \\
%      %base case
%  &\text{For } n = 0 \\
%  &\quad g(f, (m, 0) = \\
%  &\qquad (1, (0, 0)) \vee \bigvee\limits_{1 + n_1 } \\
%  &\quadsix \T{Cons} \mapsto (1 \oplus ((1 \oplus (f\ y)_c) \otimes g_c(f, (m, -1))), ((f\ y)_p, \pi_1 g_p(f, (m, -1)))))  \\
%  &\qquad = (1, (0, 0)) \\
%      %inductive case
%  &\text{For } n > 0 \\
%  &fubar
%  \end{split} \]
%\end{figure}
%
%The result is
%\begin{equation*}
%  g(f, (m, n)) = \begin{cases}
%    (1, (0, 0)) &n \equiv 0 \\
%    (1 \oplus (1 \oplus (f\ m)_c \otimes g(f, (m, n-1))_c), ((f\ m)_p, g(f, (m, n-1))_p)) &n > 0
%  \end{cases}
%\end{equation*}

We compile the recurrence down to the work and the span to make it easier to manipulate.
The costs change from cost graphs to a tuple of the work and the span.
\begin{equation*}
  \begin{split}
  &g(f, (m, n)) = \\
  &\quad \bigvee\limits_{(m_1, n_1) \leq (m, n)} case((m_1, n_1), \\
  &\quadfive\quadfour \T{Nil} \mapsto ((1, 1), (0, 0)), \\
  &\quadfive\quadfour \T{Cons} \mapsto ((2 + \pi_0(f\ y)_c + \pi_0 g_c(f, (m, n-1)), \\
  &\quadfive\quadten                     1 + max(1 + \pi_1(f\ y)_c, \pi_1 g_c(f, (m, n-1)))), \\
  &\quadfive\quadeight                  ((f\ y)_p, \pi_1 g_p(f, (m, n-1))))) 
  \end{split}
\end{equation*}

We prove by induction bounds on the work and span of the cost of $g$.

\begin{theorem}
$\pi_0 g_c(f, (m, n)) \leq 1 + (2 + \pi_0(f\ m)_c)n$
\end{theorem}

\begin{proof}
   The proof is by induction on $n$.
  \begin{description}
    \item[case $n=0$]\mbox{}\\[-1.5\baselineskip]
      \begin{align*}
      \pi_0 g_c(f, (m, 0)) = \pi_0((1, 1), (0, 0))_c = \pi_0(1, 1) = 1 
      \end{align*}
    \item[case $n>0$]\mbox{}\\[-1.5\baselineskip]
      \[\begin{split}
        &\pi_0(g_c(f, (m, n))) = \\
        &\quad \pi_0(\bigvee\limits_{(m_1, n_1) \leq (m, n)} case((m_1, n_1), \\
        &\quadfive\quadfour \T{Nil} \mapsto ((1, 1), (0, 0)), \\
        &\quadfive\quadfour \T{Cons} \mapsto ((2 + \pi_0(f\ m_1)_c + \pi_0g_c(f, (m_1, n_1-1)), \\
        &\quadfive\quadten                     1 + max(1 + \pi_1(f\ m_1)_c, \pi_1g_c(f, (m_1, n_1-1)))), \\
        &\quadfive\quadeight                  ((f\ m_1)_p, \pi_1 g_p(f, (m_1, n_1-1))))) )_c \\
        &\quad = 1 \vee \pi_0(\bigvee\limits_{(m_1, n_1) \leq (m, n)} ((2 + \pi_0(f\ m_1)_c + \pi_0g_c(f, (m_1, n_1-1)), \\
        &\quadfour\quadten                     1 + max(1 + \pi_1(f\ m_1)_c, \pi_1g_c(f, (m_1, n_1-1)))), \\
        &\quadfour\quadeight                  ((f\ m_1)_p, \pi_1 g_p(f, (m_1, n_1-1))))) )_c \\
        &\quad = 1 \vee \bigvee\limits_{(m_1, n_1) \leq (m, n)} 2 + \pi_0 (f\ m_1)_c + \pi_0 g_c(f, (m_1, n_1-1)) \\
        &\quad \leq \bigvee\limits_{(m_1, n_1) \leq (m, n)} 2 + \pi_0 (f\ m_1)_c + (1 + (2 + \pi_0 (f\ m_1)_c)(n_1-1)) \\
        &\quad \leq 2 + \pi_0 (f\ m)_c + 1 + (2 + \pi_0 (f\ m)_c)(n - 1) \\
        &\quad \leq 1 + (2 + \pi_0 (f\ m)_c)n \\
      \end{split}\]
  \end{description}
\end{proof}

\begin{theorem}
  $\pi_1 g_c(f, (m, n)) \leq 1 + \pi_1 (f\ m)_c + n$
\end{theorem}
\begin{proof}
  \begin{description}
    \item[case $n=0$]\mbox{}\\[-1.5\baselineskip]
      \begin{align*}
      \pi_1 g_c(f, (m, 0)) = \pi_1((1, 1), (0, 0))_c = \pi_1(1, 1) = 1 
      \end{align*}
    \item[case $n>0$]\mbox{}\\[-1.5\baselineskip]
      \[\begin{split}
        &\pi_1(g_c(f, (m, n))) = \\
        &\quad \pi_1(\bigvee\limits_{(m_1, n_1) \leq (m, n)} case((m_1, n_1), \\
        &\quadfive\quadfour \T{Nil} \mapsto ((1, 1), (0, 0)), \\
        &\quadfive\quadfour \T{Cons} \mapsto ((2 + \pi_0(f\ m_1)_c + \pi_0g_c(f, (m_1, n_1-1)), \\
        &\quadfive\quadten                     1 + max(1 + \pi_1(f\ m_1)_c, \pi_1g_c(f, (m_1, n_1-1)))), \\
        &\quadfive\quadeight                  ((f\ m_1)_p, \pi_1 g_p(f, (m_1, n_1-1))))) )_c \\
        &\quad = 1 \vee \pi_1(\bigvee\limits_{(m_1, n_1) \leq (m, n)} ((2 + \pi_0(f\ m_1)_c + \pi_0g_c(f, (m_1, n_1-1)), \\
        &\quadfour\quadten                     1 + max(1 + \pi_1(f\ m_1)_c, \pi_1g_c(f, (m_1, n_1-1)))), \\
        &\quadfour\quadeight                  ((f\ m_1)_p, \pi_1 g_p(f, (m_1, n_1-1))))) )_c \\
        &\quad = 1 \vee \bigvee\limits_{(m_1, n_1) \leq (m, n)} 1 + max(1 + \pi_1 (f\ m_1)_c, \pi_1 g_c(f, (m_1, n_1-1))) \\
        &\quad \leq \bigvee\limits_{(m_1, n_1) \leq (m, n)} 1 + max(1 + \pi_1 (f\ m_1)_c, 1 + \pi_1 (f\ m_1)_c + n_1 - 1) \\
        &\quad \leq 1 + max(1 + \pi_1 (f\ m)_c, 1 + \pi_1 (f\ m)_c + n - 1) \\
        &\quad \leq 1 + 1 + \pi_1 (f\ m)_c + n - 1 \\
        &\quad \leq 1 + \pi_1 (f\ m)_c + n
      \end{split}\]
  \end{description}
\end{proof}

Compare these results with sequential map.


\section{Parallel Tree Map}
A program which is embarrasingly parallel is tree map.
When a function $f$ is mapped over a tree $t$, each application of $f$ to the label at each node can be done independently.
Furthermore, the tree data structure itself is dividable by construction.
Dividing the work requires only destruction of the node constructor to yield the left and right subtrees.

We will use \T{int} labelled binary trees.
\begin{equation*}
  \T{datatype tree} = \T{E of Unit | N of int$\times$tree$\times$tree}
\end{equation*}

\T{map} simply deconstructs each node, applies the function to the label, recuses on the children, and reconstructs a node using the results.
\begin{equation*}
  \T{map} = \lambda f.\lambda t.\T{rec}(t, \T{E} \mapsto \T{E}, \T{N} \mapsto \langle x, \langle t_0, r_0 \rangle, \langle t_1, r_1 \rangle\rangle.\T{N} \langle f\ x, \T{force}(r_0), \T{force}(r_1)\rangle)
\end{equation*}

\begin{figure}
  \label{fig:ws_treemap_complexity_translation}
  \caption{Complexity translation of \T{map f t}}
  \begin{equation*}
    \begin{split}
      &3 \oplus (f_c \otimes t_c) \oplus_c \T{rec}(t_p, \T{E} \mapsto 1 \oplus_c \|\T{E}\|, \\
      &\quadten \T{N} \mapsto \langle y, \langle t_0, r_0 \rangle \langle t_1, r_1 \rangle \rangle 1 \oplus_c \|\T{N}\langle f\ x, \T{force}(r_0) \T{force}(r_1)\rangle \| ) \\
      %E branch
      &\quad 1 \oplus_c \|E\| = 1 \oplus \langle 0, E \rangle = \langle 1, E \rangle \\
      %N branch
      &\quad \|\T{N}\langle f\ x, \T{force}(r_0) \T{force}(r_1)\rangle \| = \\
      &\qquad \langle \|\langle f\ x, \T{force}(r_0) \T{force}(r_1)\rangle \|_c, \T{N} \|\langle f\ x, \T{force}(r_0) \T{force}(r_1)\rangle\|_p\rangle \\
      &\quadthree \|\langle f\ x, \T{force}(r_0) \T{force}(r_1)\rangle \| = \\
      &\quadfour \langle \|f\ x\|_c \otimes \|\T{force}(r_0)\|_c \otimes \|\T{force}(r_1)\|_c, \langle \|f\ x\|_p, \|\T{force}(r_0)\|_p, \|\T{force}(r_1)\|_p \rangle \rangle \\
      % \|f x\|
      &\quadfive  \|f\ x\| = 1 \oplus \|f\|_c \otimes \|x\|_c \oplus_c \|f\|_p \|x\|_p \\
      &\quadeight = 1 \oplus \langle 0, f \rangle_c \otimes \langle 0, x \rangle_c \oplus_c \langle 0, f \rangle_p \langle 0, x \rangle_p \\
      &\quadeight = 1 \oplus_c (f_p\ x) \\
      &\quadeight = \langle 1 \oplus (f_p\ x)_c, (f_p\ x)_p \rangle \\
      % \|force(r0)\|
      &\quadfive \|\T{force}(r_0)\| = \|r_0\|_c \oplus_c \|r_0\|_p \\
      &\quadeight = \langle 0, r_0 \rangle \oplus_c \langle 0, r_0 \rangle_p \\
      &\quadeight = \langle 0 + r_{0c}, r_{0p} \rangle \\
      &\quadeight = r_0 \\
      % \|force(r1)\|
      &\quadfive \|\T{force}(r_1)\| = \|r_1\|_c \oplus_c \|r_1\|_p \\
      &\quadeight = \langle 0, r_1 \rangle \oplus_c \langle 0, r_1 \rangle_p \\
      &\quadeight = \langle 0 + r_{1c}, r_{1p} \rangle \\
      &\quadeight = r_1  \\
      %\|<...>\|
      &\quadthree \|\langle f\ x, \T{force}(r_0) \T{force}(r_1)\rangle \| = \\
      &\quadsix = \langle 1 \oplus (f_p\ x)_c \otimes r_{0c} \otimes r_{1c}, \langle (f_p\ x)_p, r_{0p}, r_{1p}\rangle\rangle \\
      %\|N<...>\|
      &\qquad \|\T{N}\langle f\ x, \T{force}(r_0) \T{force}(r_1)\rangle \| = \\
      &\quadfive = \langle 1 \oplus (f_p\ x)_c \otimes r_{0c} \otimes r_{1c}, \T{N} \langle (f_p\ x)_p, r_{0p}, r_{1p}\rangle\rangle \\
      %rec
      &\|\T{map f t}\| = \\
      &2 \oplus (f_c \otimes t_c) \oplus 1 \oplus_c \T{rec}(t_p, \T{E} \mapsto \langle 1, \T{E}\rangle, \\
      &\quadeight \T{N} \mapsto \langle y, \langle t_0, r_0 \rangle \langle t_1, r_1 \rangle \rangle.  \langle 2 \oplus (f_p\ x)_c \otimes r_{0c} \otimes r_{1c}, \T{N} \langle (f_p\ x)_p, r_{0p}, r_{1p}\rangle\rangle
    \end{split}
  \end{equation*}
\end{figure}

The translation of \T{map f t} is given in figure \ref{fig:ws_treemap_complexity_translation}.
\begin{equation}
  \label{eq:ws_treemap_complexity}
  \begin{split}
    &\|\T{map f t}\| = 2 \oplus (f_c \otimes t_c) \oplus 1 \oplus_c \\
    &\quadfour \T{rec}(t_p, \T{E} \mapsto \langle 1, \T{E}\rangle, \\
    &\quadseven \T{N} \mapsto \langle y, \langle t_0, r_0 \rangle \langle t_1, r_1 \rangle \rangle.  \langle 2 \oplus (f_p\ x)_c \otimes r_{0c} \otimes r_{1c}, \T{N} \langle (f_p\ x)_p, r_{0p}, r_{1p}\rangle\rangle
\end{split}
\end{equation}

The result is shown in equation \ref{eq:ws_treemap_complexity}.

We interpret trees as the number of \T{N} constructors and the maximum label.
\begin{align*}
  \llbracket tree \rrbracket &= \mathbb{Z} \times \mathbb{Z} \\
  D_\T{tree} &= \{\ast\} + \mathbb{Z} \times \llbracket \T{tree} \rrbracket \times \llbracket \T{tree} \rrbracket \\
  size_\T{tree}(\ast) &= (0, 0) \\
  size_\T{tree}(x, (m_0, n_0), (m_1, n_1)) &= (max(x, m_0, m_1), 1 + n_0 + n_1)
\end{align*}

\begin{figure}
  \label{fig:ws_map_interpretation_derivation}
  \caption{Deriviation of interpretation of \T{map f t}}
\[ \begin{split}
\end{split} \]
\end{figure}


\subsection*{Source Language}
Recall the definition of the tree data type and the function map.

\chapter{Mutual Recurrence}

\section{Motivation}
The interpretation of a recursive function can be seperated into a recurrence
for the cost and a recurrence for the potential. The recurrence for the cost
depends on the recurrence for the potential. However, the recurrence for the
potential does not depend on the cost. We prove this by designing a pure
potential translation. The pure potential translation is identical to the
complexity translation except that it does not keep track of the cost.

We then show by logical relations that the potential of the complexity
translation is related to the pure potential relation.

\section{Pure Potential Translation}
Our pure potential translation is defined below. The translation of an
expression is essientially the expression itself, without suspensions.
%
\begin{align*}
  |x| &= x                                                                                     \\
  |\langle\rangle| &= \langle\rangle                                                           \\
  |\langle e_0, e_1 \rangle | &= \langle |e_0|, |e_1| \rangle                                  \\
  |\texttt{split}(e_0, x_0. x_1. e_1)| &= |e_1|[\pi_0|e_0|/x_0, \pi_0|e_0|/x_1]                \\
  |\lambda x.e | &= \lambda x.|e|                                                              \\
  |e_0\ e_1| &= |e_0|\ |e_1|                                                                   \\
  |delay(e)| &= |e|                                                                            \\
  |force(e)| &= |e|                                                                            \\
  |C_i^\delta e| &= C_i^\delta |e|                                                             \\
  |rec^\delta(e, \overline{C \mapsto x.e_C})| &= rec^\delta(|e|, \overline{C \mapsto x.|e_C|}) \\
  |map^\phi(x.v_0, v_1)| &= map^{|\phi|}(x.|v_0|, |v_1|)                                       \\
  |let(e_0, x.e_1)| &= |e_1|[|e_0|/x]
\end{align*}
%
\section{Logical Relation}
We define our logical relation below.
%
\begin{align*}
  E &\sim_{\texttt{\tiny{Unit}}} E' \text{always}  \\
  E &\sim_{\tau_0 \times \tau_1} E' \Leftrightarrow \forall k. \langle k, \pi_0 E_p\rangle \sim_{\tau_0} \pi_0 E', \forall k. \langle k, \pi_1 E_p\rangle \sim_{\tau_1} \pi_1 E' \\
  E &\sim_{\texttt{\tiny{susp }} \tau} E' \Leftrightarrow E_p \sim_\tau E' \\
  E &\sim_{\sigma \to \tau} E' \Leftrightarrow \forall E_0 \sim_\sigma E'_0. E_p E_{0p} \sim_\tau E' E'_0 \\
  E &\sim_\delta E' \Leftrightarrow \exists k, k', C, V, V'. V \sim_{\phi[\delta]} V', E \downarrow \langle k, C V_p \rangle, E' \downarrow C V'
\end{align*}
%
The relation is defined on closed terms, but we extend it to open terms.
Let $\Theta$ and $\Theta'$ be any substitutions such that $\forall x : \|\tau\|, \forall k, \langle k, \Theta(x) \rangle \sim_\tau \Theta'(x)$.
If $E\ \Theta \sim_\tau E'\ \Theta'$, then $E \sim_\tau E'$.

\section{Proof}

We require some lemmas.

The first states we can always ignore the cost of related terms.
%
\begin{lemma}[Ignore Cost]
  \label{lem:ignorecost}
\[
  E \sim_\tau E' \Leftrightarrow \forall k, \langle k, E_p \rangle \sim_\tau E'
\]
\end{lemma}
%
\begin{proof}
  We proceed by induction on type.

  Case $E \sim_{\texttt{Unit}} E'$. 
  Then $\forall k, \langle k, E_p \rangle \sim_{\texttt{\tiny{Unit}}} E'$ by definition.

  Case $E \sim_{\tau_0 \times \tau_1} E'$.
  By definition for $i\in{0,1}, \forall k_i, \langle k_i, \pi_i E_p \rangle \sim_{\tau_i} \pi_i E'$.
  Let $k$ be some cost.
  Then $\langle k, E_p \rangle \sim_{\tau_0 \times \tau_1} E'$ by definition.

  Case $E \sim_{\texttt{\tiny{susp }} \tau} E'$.
  By definition $E_p \sim_\tau E'$.
  Let $k$ be some cost.
  Then $\langle k, E_p \rangle \sim_{\texttt{\tiny{susp }} \tau} E'$.

  Case $E \sim_{\sigma \to \tau} E'$.
  Let $E_0, E_0'$ by some complexity language terms such that $E_0 \sim_\sigma E_0'$.
  Let $k$ be some cost.
  Then, $E_p\ E_0 \sim_\tau E'\ E_0'$.
  So $\langle k, E_p \rangle \sim_{\sigma \to \tau} E'$.

  Case $E \sim_\delta E'$.
  Then by definition there exists costs $k$ and $k'$, a constructor $C$, and complexity language values $V$ and $V'$ such that $V \sim_{\Phi[\delta]} V', E \downarrow \langle k, C V_p \rangle$, and $E' \downarrow C V'$.
  Since $E \downarrow \langle k, C V_p \rangle$, we know $\forall k_0, \exists k_0'. \langle k_0, E_p \rangle \downarrow \langle k_0', C V_p \rangle$.
  So by definition we have $\forall k_0, \langle k_0, E_p \rangle \sim_\Phi E'$.
\end{proof}
%
The next lemma states that if two terms step to related terms, then those terms are related.
%
\begin{lemma}[Related Step Back]
  \label{lem:relatedstepback}
  \[
    E \to F, E' \to F', F \sim_\sigma F' \implies E \sim_\sigma E'
  \]
\end{lemma}
%
\begin{proof}
  The proof proceeds by induction on type.
  
  Case \texttt{Unit}. Trivial since $E \sim_{\texttt{\tiny{Unit}}} E'$ always.

  Case $\delta$.
  By definition $\exists C, U, U', k, k'$ such that $F \downarrow \langle k, C U_p \rangle, F' \downarrow C U', U \sim_{\phi[\delta]} U'$.
  Since $E \to F$ and $E' \to F'$, $E \downarrow \langle k, C U_p \rangle$ and $E' \downarrow C U'$.
  Therefore since $U \sim_{\phi[\delta]} U'$, we have $E \sim_\delta E'$.

  Case $\sigma \to \tau$.
  Let $E_0 \sim_\sigma E'_0$.
  By definition, $F\ E_0 \sim_\tau F'\ E'_0$.
  Since $E \to F$ and $E' \to F'$, $E\ E_0 \to F\ E_0$ and $E'\ E'_0 \to F'\ E'_0$.
  So by the induction hypothesis, $E\ E_0 \sim_\tau E'\ E_0'$.
  So by definition, $E \sim_{\sigma \to \tau} E'$.

  Case $\tau_0 \times \tau_1$.
  Since $F \sim_{\tau_0 \times \tau_1} F'$, for $i\in\{0, 1\}$,  $\forall k_i, \langle k_i, \pi_i F_p \rangle \sim_{\tau_i} \pi_i F'$, by definition.
  From $E \to F$, we get $\langle k_i, \pi_i E_p \rangle \to \langle k_i', \pi_i F_p \rangle$.
  From $E' \to F'$, we get $\pi_i E' \to \pi_i F'$.
  We can apply our induction hypothesis to get $\langle k_i, \pi_i E_p \rangle \sim_{\tau_i} \pi_i E'$.
  By \ref{lem:ignorecost}, $\forall k_i, \langle k_i, \pi_i E_p \rangle \sim_{\tau_i} \pi_i E$.
  So by definition $E \sim_{\tau_0 \times \tau_1} E'$.

  Case $\texttt{susp }\tau$.
  Since $F \sim_{\texttt{\tiny{susp }}\tau} F'$, by definition $F_p \sim_\tau F'$.
  Since $E \to F$, $E_p \to F_p$.
  So by the induction hypothesis, since $E_p \to F_p, E' \to F', F_p \sim_\tau F'$, $E_p \sim_\tau E'$.
  So by definition $E \sim_{\texttt{\tiny{susp }}\tau} E'$.

\end{proof}
%
The next lemma states that related terms step to related terms
%
\begin{lemma}
  \label{lem:relatedstep}[Related Step]
  \[ E \to F, E' \to F', E \sim_\sigma E' \implies F \sim_\sigma F' \]
\end{lemma}
%
\begin{proof}
  The proof is by induction on type.

  Case \texttt{Unit}.
  $F \sim_{\texttt{\tiny{Unit}}} F'$ always.

  Case $\delta$.
  By definition, $E \sim_\delta E'$ implies $\exists C, V, V', k$ such that $E \downarrow \langle k, C V_p \rangle, E' \downarrow C V', V \sim_{\phi[\delta]} V'$.
  Since $E \to F$, $F \downarrow \langle k, C V_p \rangle$; and since $E \to F'$, $F' \downarrow C V'$.
  By \ref{lem:ignorecost}, $\langle k, V_p \rangle \sim_{\phi[\delta]} V'$.
  So because $F \downarrow \langle k, C V_p \rangle, F' \downarrow C V', \langle k, V_p \rangle \sim_{\phi[\delta]} V'$, we can apply our induction hypothesis to get $F \sim_\delta F'$.

  Case $\tau_0 \times \tau_1$.
  By definition $E \sim_{\tau_0 \times \tau_1} \implies \forall i \in \{0, 1\}, \forall k, \langle k_i, \pi_i E_p \rangle \sim_{\tau_i} \pi_i E'$.
  Fix some $k_i$.
  Since $E \to F$, $\langle k_i, \pi_i E_p \rangle \to \langle k_i, \pi_i F_p \rangle$.
  Since $E' \to F'$, $\pi_i E' \to \pi_i F'$.
  From $\langle k_i, \pi_i E_p \rangle \to \langle k_i, \pi_i F_p \rangle, \langle k_i, \pi_i E_p \rangle \sim_{\tau_i} \pi_i E'$, the induction hypothesis tells us $\langle k_i, \pi_i F_p \rangle \sim_{\tau_i} \pi_i F'$.
  So by definition $F \sim_{\tau_0 \times \tau_1} F'$.

  Case $\texttt{susp } \tau$.
  By definition $E \sim_{\texttt{susp }\tau} E' \implies E_p \sim_\tau E'$.
  Since $E \to F$, $E_p \to F_p$.
  From $E_p \to F_p, E' \to F', E_p \sim_\tau E'$, the induction hypothesis gives us $F_p \sim_\tau F'$.
  So by definition $F \sim_{\texttt{susp }\tau} F'$.

  Case $\sigma \to \tau$.
  Let $E_0 \sim_\sigma E_0'$.
  By definition, $E\ E_0 \sim_\tau E'\ E_0'$.
  Since $E \to F$, $E\ E_0 \to F\ E_0$.
  Since $E' \to F'$, $E'\ E_0' \to F'\ E_0'$.
  From $E\ E_0 \to F\ E_0, E'\ E_0' \to F'\ E_0', E\ E_0 \sim_\tau E'\ E_0'$, the induction hypothesis tells us $F\ E_0 \sim_\tau F'\ E_0'$.
  So by definition $F \sim_{\sigma \to \tau} F'$.
\end{proof}
%
The next lemma states that if the arguments to $map$ are related, then $map$ preserves the relatedness.
%
\begin{lemma}
  \label{lem:relatedmap}[Related Map]
  \[ E \sim_{\tau_1} E', E_0 \sim_{\tau_0} E_0' \implies \forall k. \langle k, map^\Phi(x, E_p, E_{0p})\rangle \sim_{\Phi[\tau_1]} map^\Phi(x, E', E_0') \]
\end{lemma}
%
\begin{proof}
  The proof proceeds by induction on type.

  Recall the definition of the $map$ macro.
  \begin{align*}
    map^t(x.E, E_0) &= E[E_0/x]                                                                                       \\
    map^T(x.E, E_0) &= E_0                                                                                            \\ 
    map^{\Phi_0 \times \Phi_1}(x.E, E_0) &= \langle map^{\Phi_0}(x.E, \pi_0 E_0), map^{\Phi_1}(x.E, \pi_1 E_0 \rangle \\
    map^{T \to \Phi}(x.E, E_0) &= \lambda y.map^\Phi(x.E, E_0\ y)
  \end{align*}

  Case $\Phi = t$.
  Then $map^t(x.E_p, E_{0p}) = E_p[E_{0p}/x]$ and $map^t(x.E', E_0') = E'[E_0'/x]$.
  Let $k$ be some cost.
  By \ref{lem:ignorecost}, $E \sim_{\tau_1} E'$ implies $\langle k, E_p \rangle \sim_{\tau_1} E'$.
  Since $\langle k, E_p \rangle \sim_{\tau_1} E'$ and $E_0 \sim_{\tau_0} E_0'$, $\langle k, E_p \rangle [E_{0p}/x] \sim_{\phi[\tau_0]} E'[E_0'/x]$.
  So $\forall k, \langle k, map^t(x.E_p, E_{0p}) \rangle \sim_{\Phi[\tau_1]} map^t(x.E', E_0')$.

  Case $\Phi = T$.
  Then $map^T(x.E_p, E_{0p}) = E_{0p}$ and $map^T(x.E', E_0') = E_0'$.
  By \ref{lem:ignorecost} $\forall k, \langle k, E_{0p} \rangle \sim_{\tau_0} E_0'$.
  So $\forall k, \langle k, map^T(x.E_p, E_{0p}) \rangle \sim_{\Phi[\tau_1]} map^T(x.E', E_0')$.

  Case $\Phi = \Phi_0 \times \Phi_1$.
  Then $map^{\Phi_0 \times \Phi_1}(x. E_p, E_{0p}) = \langle map^{\Phi_0}(x. E_p, \pi_0 E_{0p}), map^{\Phi_1}(x. E_p, \pi_1 E_{0p}) \rangle$.
  Similarly $map^{\Phi_0 \times \Phi_1}(x. E', E_0') = \langle map^{\Phi_0}(x. E', \pi_0 E_0'), map^{\Phi_1}(x. E', \pi_1 E_0') \rangle$.
  By definition, $\forall k, \langle k, \pi_0 E_{0p} \rangle \sim_{\Phi_0[\tau_0]} \pi_0 E_0'$.
  By the induction hypothesis, $\forall k, \langle k, map^{\Phi_0}(x. E_p, \pi_0 E_{0p}) \sim_{\Phi_0[\tau_1]} map^{\Phi_0[\tau_1]}(x. E', E_0')$.
  By definition, $\forall k, \langle k, \pi_1 E_{0p} \rangle \sim_{\Phi_1[\tau_0]} \pi_1 E_0'$.
  By the induction hypothesis, $\forall k, \langle k, map^{\Phi_1}(x. E_p, \pi_1 E_{0p}) \sim_{\Phi_1[\tau_1]} map^{\Phi_1[\tau_1]}(x. E', E_0')$.
  So by definition, $\forall k, \langle k, \langle map^{\Phi_0}(x. E_p, \pi_0 E_{0p}), map^{\Phi_1}(x. E_p, \pi_1 E_{0p}) \rangle \rangle \sim_{\Phi[\tau_1]} \langle \langle map^{\Phi_0[\tau_1]}(x. E', E_0'), map^{\Phi_1[\tau_1]}(x. E', E_0') \rangle \rangle$.

  Case $T \to \Phi$.
  Then $map^{T \to \Phi}(x. E_p, E_{0p}) = \lambda y.map^\Phi(x.E_p, E_{0p}\ y)$ and $map^{T \to \Phi}(x. E', E_0') = \lambda y.map^\Phi(x.E', E_0'\ y)$.
  Let $E_1 : T$.
  Then $\lambda y.map^\Phi(x.E_p, E_{0p}\ y)\ E_1 \to map^\Phi(x.E_p, E_{0p}\ E_1)$.
  Similarly, $\lambda y.map^\Phi(x.E', E_0'\ y)\ E_1' \to map^\Phi(x. E', E_0'\ E_1')$.
  Since $E_0 \sim E_0'$ and $E_1 \sim E_1'$, we have $E_{0p}\ E_1 \sim E_0'\ E_1'$.
  So by our induction hypothesis, $map^\Phi(x.E_p, E_{0p}\ E_1) \sim map^\Phi(x. E', E_0'\ E_1')$.
  So by \ref{lem:relatedstepback}, $\lambda y.map^\Phi(x.E_p, E_{0p}\ y)\ E_1 \sim \lambda y.map^\Phi(x.E', E_0'\ y)\ E_1'$.
  So by definition, $\lambda y.map^\Phi(x.E_p, E_{0p}\ y) \sim \lambda y.map^\Phi(x.E', E_0'\ y)$.
  So $map^{T \to \Phi}(x. E_p, E_{0p}) \sim map^{T \to \Phi}(x. E', E_0')$.
\end{proof}
%
Our last lemma is about the relatedness of $rec$ terms.
%
\begin{lemma}[Related Rec]
  \label{lem:relatedrec}
  \[ E \sim_\delta E', \forall C, E_C \sim_\tau E_C' \implies rec(E_p, \overline{C \mapsto x.E_c}) \sim_\tau rec(E', \overline{C \mapsto x.E_c'}) \]
\end{lemma}
%
\begin{proof}
  Recall the rule for evaluating $rec$ in the complexity language:
  \begin{prooftree}
    \AxiomC{$E \downarrow C V_0$}
    \AxiomC{$map^\Phi(y,\langle y, rec(y, \overline{C \mapsto x.E_C})\rangle, V_0) \downarrow V_1$}
    \AxiomC{$E_C[V_1/x] \downarrow V$}
    \TrinaryInfC{$rec(E, \overline{C \mapsto x.E_C}) \downarrow V$}
  \end{prooftree}
  By definition of $\sim_\delta$, $\exists k, C, V_0, V_0'$ such that $E \downarrow \langle k, C V_{0p} \rangle, E' \downarrow C V_0'$, and $V_0 \sim_\delta V_0'$.
  Our proof proceeds by induction on the number of constructors in $C V_{0p}$.
  If $\Phi = T$, then $map^\Phi(y, \langle y, rec(y, \overline{C \mapsto x.E_C})\rangle, V_{0p}) = \langle y, rec(y, \overline{C \mapsto x.E_C})\rangle[V_{0p}/y] = \langle V_{0p}, rec(V_{0p}, \overline{C \mapsto x.E_C}) \rangle$.
  Similarly for the pure potential, $map^\Phi(y, \langle y, rec(y, \overline{C \mapsto x.E_C'})\rangle, V_{0p}') = \langle y, rec(y, \overline{C \mapsto x.E_C'})\rangle [V_0'/y] = \langle V_0', rec(V_0', \overline{C \mapsto x.E_C'}) \rangle$.
  By the induction hypothesis, $rec(V_{0p}, \overline{C \mapsto x.E_C}) \sim_\tau rec(V_0', \overline{C \mapsto x.E_C'})$.
  By definition of $\sim_{\texttt{susp }\tau}$, for any $k$, $\langle k, rec(V_{0p}, \overline{C \mapsto x.E_C}) \rangle \sim_{\texttt{susp }\tau} rec(V_0', \overline{C \mapsto x.E_C'})$.
  So by definition of $\sim_{\tau_0 \times \tau_1}$, $\langle 0, \langle V_{0p}, rec(V_{0p}, C \mapsto x.E_C) \rangle\rangle \sim_{\phi[\delta \times \texttt{susp }\tau]} \langle V_0', rec(V_0', \overline{C \mapsto x.E_C'})\rangle$.
  So by \ref{lem:relatedmap}, $\forall k. \langle k, map^\Phi(y, \langle y, rec(y, \overline{C \mapsto x.E_C})\rangle, V_{0p}) \sim_{\phi[\delta \times \texttt{susp }\tau]} map^\Phi(y, \langle y, rec(y, \overline{C \mapsto x.E_C'}) \rangle, V_0')$.
  Let $\langle 0, map^\Phi(y, \langle y, rec(y, \overline{C \mapsto x.E_C})\rangle, V_{0p}) \downarrow V_1$.
  Let $map^\Phi(y, \langle y, rec(y, \overline{C \mapsto x.E_C'}) \rangle, V_0') \downarrow V_1'$.
  By \ref{lem:relatedstep}, $V_1 \sim_{\phi[\delta \times \texttt{susp }\tau]} V_1'$.

  If $\Phi = t$, then $map^\Phi(y, \langle y, rec(y, \overline{C \mapsto x.E_C})\rangle, V_{0p}) = V_{0p}$.
  Similarly, $map^\Phi(y, \langle y, rec(y, \overline{C \mapsto x.E_C'})\rangle V_0') = V_0'$.
  So in this case $V_0 = V_1$ and $V_0' = V_1'$.
  We have already established $V_0 \sim_\tau V_0'$.

  So in both cases $V_1 \sim_{\phi[\delta \times \texttt{susp }\tau]} V_1'$.

  By definition of the relation $E_C[V_{1p}/x] \sim_\tau E_C'[V_1'/x]$.
  Let $E_C[V_{1p}/x] \downarrow V_2$ and $E_C'[V_1'/x] \downarrow V_2'$.
  By \ref{lem:relatedstep}, $V_2 \sim_\tau V_2'$.
  So by \ref{lem:relatedstepback}, $rec(E_p, \overline{C \mapsto x.E_C}) \sim_\tau rec(E', \overline{C \mapsto x.E_C'})$.
\end{proof}
%
Our theorem is that for all well-typed terms in the source language, the
complexity translation of the term is related to the pure potential translation
of that term.
%
\begin{theorem}[Distinct Recurrence]
  \[ \gamma \vdash e : \tau \implies \|e\| \sim_\tau |e| \]
\end{theorem}
%
\begin{proof}
  Our proof is by induction on the typing derivation $\gamma \vdash e : \tau$.

  Case \AxiomC{}\UnaryInfC{$\gamma, x : \sigma \vdash x : \sigma$}\DisplayProof.
  Then by definition of the logical relation, $\forall k, \langle k, \Theta(x) \rangle \sim_\sigma \Theta'(x)$.
  Since $\|x\| = \langle 0, x \rangle$ and $|x| = x$, we have $\langle 0, x \rangle \sim_\sigma x$.

  Case \AxiomC{}\UnaryInfC{$\gamma \vdash e : Unit$}\DisplayProof.
  By definition, $\|e\| \sim_{\texttt{Unit}} |e|$ always.

  Case \AxiomC{$\gamma \vdash e_0 : \tau_0$}\AxiomC{$\gamma \vdash e_1 : \tau_1$}\BinaryInfC{$\gamma \vdash \langle e_0, e_1 \rangle : \tau_0 \times \tau_1$}\DisplayProof
  By the induction hypothesis, $\|e_0\| \sim_{\tau_0} |e_0|$ and $\|e_1\| \sim_{\tau_1} |e_1|$.
  By \ref{lem:ignorecost}, $\forall k, \langle k, \|e_0\|_p \rangle \sim_{\tau_0} |e_0|$
  and $\forall k, \langle k, \|e_1\|_p \rangle \sim_{\tau_1} |e_1|$.
  So by definition, $\|\langle e_0, e_1 \rangle \| \sim_{\tau_0 \times \tau_1} |\langle e_0, e_1 \rangle |$.

  Case \AxiomC{$\gamma \vdash e_0 : \tau_0 \times \tau_1$}\AxiomC{$\gamma, x_0 : \tau_0, x_1 : \tau_1 \vdash e_1 : \tau$}\BinaryInfC{$\gamma \vdash split(e_0, x_0.x_1.e_1) : \tau$}\DisplayProof
  By the induction hypothesis, $\|e_0\| \sim_{\tau_0 \times \tau_1} |e_0|$ and $\|e_1\| \sim_\tau |e_1|$.
  From $\|e_0\| \sim_{\tau_0 \times \tau_1} |e_0|$ it follows by definition that 
    $\forall k, \langle k, \pi_0 \|e_0\|_p \rangle \sim_{\tau_0} \pi_0 |e_0|$ and
    $\forall k, \langle k, \pi_1 \|e_1\|_p \rangle \sim_{\tau_1} \pi_1 |e_1|$.
  The complixity translation is $\|split(e_0, x_0.x_1.e_1)\| = \|e_0\|_c +_c \|e_1\|[\pi_0\|e_0\|_p/x_0, \pi_1\|e_1\|_p/x_1]$.
  The pure potential translation is $|split(e_0, x_0.x_1.e_1)| = |e_1|[\pi_0|e_0|/x_0, \pi_1|e_0|/x_1]$.
  By \ref{lem:ignorecost}, it suffices to show $\|e_1\|[\pi_0\|e_0\|_p/x_0, \pi_1\|e_1\|_p/x_1] \sim_\tau |e_1|[\pi_0|e_0|/x_0, \pi_1|e_0|/x_1]$
  By definition of the relation, it suffices to show $\|e_1\| \sim_\tau |e_1|$,
    $\forall k, \langle k, \pi_0 \|e_0\|_p \rangle \sim_{\tau_0} \pi_0 |e_0|$,
    and $\forall k, \langle k, \pi_1 \|e_0\|_p \rangle \sim_{\tau_1} \pi_1 |e_0|$.
  Since we have already establed all three conditions, we have $\|split(e_0, x_0. x_1.e_1)\| \sim_\tau |split(e_0,x_0.x_1.e_1)|$.


  Case \AxiomC{$\gamma, x : \sigma \vdash e : \tau$}\UnaryInfC{$\gamma \vdash \lambda x.e : \sigma \to \tau$}\DisplayProof
  By the induction hypothesis $\|e\|\sim_\tau|e|$.
  The complexity translation is $\|\lambda x.e\| = \langle 0, \lambda x.\|e\|\rangle$.
  The pure potential translation is $|\lambda x.e| = \lambda x.|e|$.
  Let $E_0 : \|\sigma\|$ and $E_0' : |\sigma|$ be complexity language terms such that $E_0 \sim_\sigma E_0'$.
  Then $\langle 0, \lambda x.\|e\|\rangle\ E_0 \to \langle 0 + E_{0c}, \|e\|[x \mapsto E_0]\rangle$
    and $\lambda x.|e|\ E_0' \to |e|[x \mapsto E_0']$.
  Since $\|e\| \sim_\tau |e|$ and $E_0 \sim_\sigma E_0'$, $\|e\|[x \mapsto E_0] \sim_\tau |e|[x \mapsto E_0']$.
  By \ref{lem:relatedstepback}, $\langle 0, \lambda x. \|e\| \rangle\ E_0 \sim_\tau (\lambda x.|e|)\ E_0'$.
  So by definition $\langle 0, \lambda x. \|e\| \rangle \sim_{\sigma \to \tau} \lambda x. |e|$.
  So $\|\lambda x.e\| \sim_{\sigma \to \tau} |\lambda x.e|$.

  Case \AxiomC{$\gamma \vdash e_0 : \sigma \to \tau$}\AxiomC{$\gamma \vdash e_1 : \sigma$}\BinaryInfC{$\gamma \vdash e_0\ e_1 : \tau$}\DisplayProof
  The complexity translation is $\|e_0\ e_1\| = (1 + \|e_0\|_c + \|e_1\|_c) +_c \|e_0\|_p \|e_1\|_p$.
  The pure potential translation is $|e_0\ e_1| = |e_0| |e_1|$.
  By \ref{lem:ignorecost}, it suffices to show $\|e_0\|_p \|e_1\|_p \sim_\tau |e_0||e_1|$.
  By the induction hypothesis, $\|e_0\| \sim_{\sigma \to \tau} |e_0|$ and $\|e_1\| \sim_\sigma |e_1|$.
  By definition, $\|e_0\|_p \|e_1\|_p \sim_{\tau} |e_0| |e_1|$.

  Case \AxiomC{$\gamma \vdash e : \tau$}\UnaryInfC{$\gamma \vdash delay(e) : susp\ \tau$}\DisplayProof
  By the induction hypothesis $\|e\| \sim_\tau |e|$.
  So $\langle 0, \|e\|\rangle \sim_{\texttt{susp }\tau} |e|$.
  The complexity translation is $\|delay(e)\| = \langle 0, \|e\|\rangle$.
  The pure potential translatio is $|delay(e)| = |e|$.
  So $\|delay(e)\| \sim_{\texttt{susp }\tau} |delay(e)|$.

  Case \AxiomC{$\gamma \vdash e : susp\ \tau$}\UnaryInfC{$\gamma \vdash force(e) : \tau$}\DisplayProof
  By the induction hypothesis $\|e\| \sim_{\texttt{susp }\tau} |e|$.
  So by definition of the relation at \texttt{susp} type, $\|e\|_p \sim_\tau |e|$.
  By \ref{lem:ignorecost}, $\forall k, \langle k, \|e\|_{pp} \rangle \sim_\tau |e|$.
  The complexity translation is $\|force(e)\| = \|e\|_c +_c \|e\|_p$.
  The pure potential translation is $|force(e)| = |e|$.
  So $\|e\|_c +_c \|e\|_p \sim_\tau |e|$.
  So $\|force(e)\| \sim_\tau |force(e)|$.

  Case \AxiomC{$\gamma \vdash e_0 : \sigma$}\AxiomC{$\gamma, x : \sigma \vdash e_1 : \tau$}\BinaryInfC{$\gamma \vdash let(e_0, x.e_1) : \tau$}\DisplayProof
  By the induction hypothesis $\|e_0\| \sim_\sigma |e_0|$ and $\|e_1\| \sim_\tau |e_1|$.
  So $\|e_1\|[\|e_0\|_p/x] \sim_\tau |e_1|[|e_0|/x]$.
  By \ref{lem:ignorecost}, $\forall k, \langle k, \|e_1\|_p[\|e_0\|_p/x] \rangle \sim_\tau |e_1|[|e_0|/x]$.
  The complexity translation is $\|let(e_0, x.e_1)\| = \|e_0\|_c +_c \|e_1\|[\|e_0\|_p/x]$.
  The pure potential translation is $|let(e_0, x.e_1)| = |e_1|[|e_0|/x]$.
  So $\|let(e_0, x.e_1)\|  \sim_\tau |let(e_0, x.e_1)|$.

  Case \AxiomC{$\gamma, x : \tau_0 \vdash v_1 : \tau_1$}\AxiomC{$\gamma \vdash v_0 : \phi[\tau_0]$}\BinaryInfC{$\gamma \vdash map^\phi(x.v_1, v_0) : \phi[\tau_1]$}\DisplayProof
  By the induction hypothesis $\|v_1\| \sim_{\tau_1} |v_1|$ and $\|v_0\| \sim_{\phi[\tau_0]} |v_0|$.
  By \ref{lem:relatedmap}, $\forall k, \langle k, map^\Phi(x.\|v_1\|_p, \|v_0\|_p)\rangle \sim_{\phi[\tau_1]} map^\Phi(x.|v_1|, |v_0|)$.
  The complexity translation is $\|map^\phi(x.v_1, v_0)\| = \langle 0, map^\Phi(x.\|v_0\|_p, \|v_1\|_p)\rangle$.
  The pure potential translation is $|map^\phi(x.v_1, v_0| = map^\Phi(x, |v_0|, |v_1|)$.
  So we have $\|map^\phi(x.v_1, v_0)\| \sim_{\phi[\tau_1]} |map^\phi(x.v_1, v_0|$.

  Case \AxiomC{$\gamma \vdash e_0 : \delta$}\AxiomC{$\forall C (\gamma, x: \phi_C[\delta \times susp\ \tau] \vdash e_c : \tau$}\BinaryInfC{$\gamma \vdash rec^\delta(e_0, \overline{C \mapsto x.e_C}) : \tau$}\DisplayProof
  By the induction hypothesis $\|e_0\| \sim_\delta |e_0|$ and $\forall C, \|e_c\| \sim_\tau |e_c|$.
  By \ref{lem:ignorecost}, $\forall k, \langle k \|e_C\| \sim_\tau |e_c$, so $1 +_c \|e_C\| \sim_\tau |e_c|$.
  So by \ref{lem:relatedrec}, $rec(\|e_0\|_p, \overline{C \mapsto x.1 +_c \|e_C\|}) \sim_\tau rec(|e_0|, \overline{C \mapsto x.1 +_c |e_C|})$.
  So by \ref{lem:ignorecost}, $\|e_0\|_c +_c rec(\|e_0\|_p, \overline{C \mapsto x.1 +_c \|e_C\|}) \sim_\tau rec(|e_0|, \overline{C \mapsto x.1 +_c |e_C|})$

  Case \AxiomC{$\gamma \vdash e : \phi[\delta]$}\UnaryInfC{$\gamma \vdash C e : \delta$}\DisplayProof
  By the induction hypothesis, $\|e\| \sim_{\phi[\delta]} |e|$.
  There exists $V, V'$ such that $\|e\| \downarrow V$ and $|e| \downarrow V'$.
  By \ref{lem:relatedstep} $V \sim_{\phi[\delta]} V'$.
  Since $\|e\| \downarrow V$, $\langle k, C\ \|e\| \rangle \downarrow \langle k, C\ V_p$.
  Similarly, since $|e| \downarrow V'$, $C |e| \downarrow C V'$.
  So by definition we have $\langle k, C\ \|e\|\rangle \sim_\delta C |e|$.
  The complexity translation is $\|C e\| = \langle \|e\|, C\|e\|_p\rangle$.
  The pure potential translation is $|C e| = C |e|$.
  Therefore by \ref{lem:ignorecost}, $\|C e\| \sim_\delta |C e|$.
\end{proof}


\chapter{Conclusions and Future Work}

This is the conclusion.


\bibliography{bibliography}
\bibliographystyle{plainnat}
\end{document}
