\documentclass[11pt, ma]{westhesis}
\usepackage[utf8]{inputenc}
\usepackage{natbib}
\usepackage{amsmath}
\usepackage{amsfonts}
\usepackage{amssymb}
\usepackage{bussproofs}
\usepackage{listings}
%\usepackage{ntheorem}
\usepackage{float}
\usepackage{upgreek}
\usepackage{stmaryrd}

% ND comments.
\usepackage[normalem]{ulem}
\setlength{\marginparwidth}{1.25in}
\reversemarginpar
\newcommand{\ndComment}[2][\relax]{\ifx#1\relax\else\uline{#1}\fi\marginpar{\renewcommand{\baselinestretch}{1.0}\tiny ND: {#2}}}
\newcommand{\ndTypo}[1]{\ndComment[#1]{$\longrightarrow$}}
\newcommand{\ndDel}[1]{\sout{#1}\ndComment{$\longrightarrow$}}

\newcommand{\LP}{\langle}
\newcommand{\RP}{\rangle}
\newcommand{\LB}{\llbracket}
\newcommand{\RB}{\rrbracket}
\newcommand{\quadthree}{\qquad\quad}
\newcommand{\quadfour}{\quadthree\quad}
\newcommand{\quadfive}{\quadfour\quad}
\newcommand{\quadsix}{\quadfive\quad}
\newcommand{\quadseven}{\quadsix\quad}
\newcommand{\quadeight}{\quadseven\quad}
\newcommand{\quadten}{\quadfive\quadfive}

\newcommand{\llangle}{\LP\!\LP}
\newcommand{\rrangle}{\RP\!\RP}
%\newcommand{\llambda}{\lambda\!\!\lambda}
%\newcommand{\pplus}{\raisebox{0.25ex}{+}\!\!\!\!\!+}

\makeatletter
\def\old@comma{,}
\catcode`\,=13
\def,{%
  \ifmmode%
    \old@comma\discretionary{}{}{}%
  \else%
    \old@comma%
  \fi%
}
\makeatother

\allowdisplaybreaks


\department{Mathematics and Computer Science}
\submitdate{TDS}
\advisor{Norman Danner}

\title{Extracting Cost Recurrences from Sequential and Parallel Functional Programs}
\author{Justin Raymond}

\theoremstyle{plain}
\newtheorem{theorem}{Theorem}[chapter]
\newtheorem{corollary}{Corollary}[theorem]
\newtheorem{lemma}[theorem]{Lemma}

\theoremstyle{definition}
\newtheorem{defn}{Definition}[chapter]
%\newenvironment{definition}[1][Definition]{\begin{trivlist}
%\item[\hskip \labelsep {\bfseries #1}]}{\end{trivlist}}

\newcommand{\T}[1]{\texttt{#1}}
\lstset{mathescape=true,basicstyle=\ttfamily}

\newcommand\numberthis{\addtocounter{equation}{1}\tag{\theequation}}
\begin{document}
%
\begin{abstract}
Complexity analysis aims to predict the resources, most often time and space,
which a program requires.  We build on previous work by \cite{Danner2013} and
\cite{Danner2015} which formalizes the extraction of recurrences for evaluation
cost from higher order functional programs. Source language programs are
translated into a complexity language. The translation of an expression is a
pair of a cost, a bound on the cost of evaluating the program to a value, and a
potential, the cost of future use of the value. We use the formalization to
analyze the time complexity of higher order functional programs. We also
demonstrate the flexibility of the method by extending it to parallel cost
semantics. Costs are cost graphs, which express dependencies between
subcomputations in the program. We prove by logical relations that the
extracted recurrences are an upper bound on the evaluation cost of the original
program. We also give examples of the analysis of higher order functional
programs under the parallel evaluation semantics.
\end{abstract}
%
\begin{acknowledgements}
  Thank you to my adviser Norman Danner for having the patience to put up with
  me the past year. Without him, this thesis would not have made it past the
  title page. Thanks also to Jim Lipton, Dan Licata, and Danny Krizanc who,
  along with Norman Danner, taught me everything I know about computer science.
\end{acknowledgements}
%

\frontmatter
\maketitle
\makededication
\makeack
\makeabstract        %TODO

\tableofcontents

\mainmatter
\chapter{Introduction}

\section{Complexity Analysis}

The efficiency of programs is categorized by how the resource usage of a
program increases with the input size in the limit.  This is often called the
asymptotic efficiency or complexity of a program.  Asymptotic efficiency
abstracts away the details of efficiency, allowing programs to be compared
without knowledge of specific hardware architecture or the size and shape of
the programs input (\citet{Cormen2001}).  However, traditional complexity
analysis is first-order; the asymptotic efficiency of a program is only
expressed in terms of its input.  Consider the following function.
%
\lstset{language=[Objective]Caml}
\begin{lstlisting}
let rec map f xs =
  match xs with
  | [] -> []
  | (x::xs') -> f x :: map f xs'
\end{lstlisting}
%
The function \T{map} applies a function to every element in a list.
Traditional analysis assumes the cost of applying its first argument is
constant.




Traditional complexity analysis proceeds as follows.  First we write a
recurrence for the cost.  \[ T(n) = c + T(n-1) \] The variable $n$ is the
length of the list and the constant \T{c} is the cost of applying the function
$f$ to an element of the list and then applying the cons function \T{::}. The
result is the asymptotic efficiency of \T{map} is linear in the length of the
list.



The are two problems with this approach.  The first is that the analysis
assumes the cost of applying the function \T{f} to each element in the list has
a constant cost. If the cost function is has a constant cost, such as fixed
width integer addition, then this first-order analysis is sufficient.  The cost
of mapping a constant cost function over a list will increase linearly in the
size of the list.  However, first-order complexity analysis will not accurately
describe the cost of mapping a nontrivial function over a list. The cost of
mapping a quadratic time function such as insertion sort over a list of lists
depends not only on the length of the list, but also on the length of the
sublists.  A more accurate prediction of the cost of this function can be
obtained by taking into account the cost of insertion sort.



The second is that there is no formal connection between the implementation of
\T{map} and the recurrence $T(n)$. The consequence is extraction of the correct
recurrence relies on the absence of human error, which the author of this
thesis can attest the difficulty of doing. A formalization of the connection
between the source program and the recurrence would prevent this. The
translation from the source program to the recurrence could be done using
application of a series of rules and the translation could even be automated.




For an example such \T{map}, it is simple enough to change our analysis to
reflect that applying the functions \T{f} does not have constant cost $c$, but
instead has cost $f_c(x)$, where $x$ is some notion of size of the elements of
the list. If the elements of the list are fixed width integers, then all $x$
would be equivalent, and $f_c(x)$ would be constant. If the elements of the
list are strings or lists than the notion of size could be their
length. If we only interpret the size of lists to be their lengths,
because then we have no information about the size of the elements we apply $f$
to. So we interpret lists as a pair of their largest element and their length. The
recurrence for the cost of \T{map} becomes $T(n,x) = (f_c(x) + c)n$. Our
analysis of the cost of \T{map} is now parameterized by the cost applying \T{f}
to the elements of the list. However, this does not allow us to analyze the
composition of functions. For example to analyze the cost of $g \circ f$, we
need to have a notion of the size of the result of $f$, as well as the cost. We
need to have a notion of the size of the result of $f$ in order to analyze the
cost of applying $g$ to the result of applying $f$ to some value.


The term we will this notion of the size of the result is potential. Potential
represents the cost of future use of an expression. As mentioned above,
potential is necessary to compose the analysis of functions. Consider this
implementation of \T{fromList} which creates a set from a list of items.
%
\lstset{language=Haskell}
\begin{lstlisting}
fromList xs = foldr `insert` empty xs
\end{lstlisting}
%
The \T{insert} function takes an element and a set and adds the element to the
set. \T{empty} is the empty set. The \T{insert} function will be with
increasing sized sets each step of the fold. To correctly analyze \T{fromList},
our analysis of \T{insert} must include both a cost of inserting an element
into a set, and a potential with which we can use to analyze the cost of the
next application of \T{insert} by \T{foldr}.

\section{Previous Work}

As we have just seen, traditional complexity analysis does not have a formal
connection between the programs and the extracted recurrences. Traditional
complexity analysis is also not compositional.

%The method of extracting recurrences presented in \citet{Danner2013} and
%\citet{Danner2015} addresses these problems. The first problem with traditional
%analysis is there is no formal connection between a program and the recurrence
%for its cost. There is a language which programs are written in, which we will
%refer to as the source language, and a language for recurrences, which we will
%refer to as the complexity language. In traditional complexity analysis, the
%source language is some programming language and the complexity language is
%mathematics, and there is no formalization of the translation from the source
%language to the complexity language.  There is also no proof that the
%mathematical recurrence for the cost of the source language program is an upper
%bound on the cost of executing the source language program.  \citet{Danner2013}
%address this problem by formally defining a translation from a source language
%to a complexity language and proving the recurrence in the complexity language
%is an upper bound on the cost of executing the source language program. The
%complexity language is also higher order. So if the source language program is
%higher order, such as \T{map}, the recurrence in the complexity language will
%reflect the cost of applying the higher order function. Finally, functions in
%the complexity language are from potentials to complexities, so the result of
%applying a function results in both a cost of the function and a potential
%which can be used to analyze the cost of applying another function to the
%result.


\citet{Danner2007}, building on the work of others, introduced the
idea that the complexity of an expression consists of a cost, representing an
upper bound on the time it takes to evaluate the expression, and a potential,
representing the cost of future uses of the expression.  The notion of a
potential allows the analysis of higher-order expressions.  The complexity of a
higher-order function such as \T{map} depends on the potential of its argument
function.  They developed a type system for ATR, a call-by-value version of
System T, that consists of a part restricting the sizes of values of
expressions and a part restricting the cost of evaluating a expression.
Programs written in ATR are constrained by the type system to run in less
than type-2 polynomial time.  \citet{Danner2009} extended this work to express
more forms of recursion, in particular those required by insertion sort and
selection sort.



\citet{Danner2013} utilized the notion of thinking of the complexity of an
expression as a pair of a cost and a potential to statically analyze the
complexity of a higher-order functional language with structural list
recursion.  The expressions in the higher-order functional language with
structural list recursion, referred to as the source language, are mapped to
expressions in a complexity language by a translation function.  The translated
expression describes an upper bound on the complexity of the original programs.



\citet{Danner2015} built on this work to formalize the extraction
of recurrences from a higher-order functional language with structural
recursion on arbitrary inductive data types. Programs are written in a higher
order functional language, referred to as the source language. The programs are
translated into a complexity language, which is essentially for recurrences.
The result of the translation of an expression is a pair of a cost and a
potential. The cost is a bound on the steps required to evaluate the expression
to a value, the potential is a size which represents cost of future use of the
value. A bounding relation is used to prove the translation and denotational
semantics of the complexity language give an upper bound on the operational
cost of running the source program. The paper also presents a syntactic
bounding theorem, where the abstraction of values to sizes done syntactically
instead of semantically.  Arbitrary inductive data types are handled
semantically using programmer specified sizes of data types. Sizes must be
programmer specified because the structure of a data type does not always
determine the interpretation of the size of a data type. There also exist
different reasonable interpretations of size, and some may be preferable to
others depending on what is being analyzed.


\section{Contribution}

This thesis comes in three parts.



Chapter 2 contains a catalog of examples of the extraction of recurrences
from functional programs using the approach given by \citet{Danner2015}. These
examples illustrate how to apply the method to nontrivial examples. They also
serve to demonstrate common techniques for solving the extracted recurrences.
The examples include reversing a list in quadratic, reversing a list in linear
time, insertion sort, parametric insertion sort, list map, and tree map.



Chapter 3 extends the analysis to parallel programs. The source language syntax
remains unchanged, but the operational semantics change to allow binary
fork-join parallelism, also called nested parallelism. The subexpressions to
tuples and function application may be evaluated in parallel. The
subexpressions themselves may have subexpressions which may also be evaluated
in parallel. We change costs from from natural numbers to the cost graphs
described in \citet{Harper2012PFPL}. A cost graph represents the dependencies
between subcomputations in a program.  The nodes of the graph are
subcomputations of the program and an edge between two nodes indicates the
result of one computation is an input to the other. The cost graph can be used
to determine an optimal strategy for scheduling the computation on multiple
processors. The cost graph has two properties that we are interested in, work
and span. The work is the total steps required to run the program, which
corresponds to the steps a single processor must execute to run the program.
The span is the critical path; the longest number of steps from the start to
the end of the cost graph.



Chapter 4 demonstrates the recurrence for the potential does not depend on
the recurrence for the cost. Consequently, we can extract the recurrence for the
potential and analyze it independently. This is useful because it is often
easier to solve the cost and potential recurrences independently than it is to
solve the initial recurrence. We are also sometimes only interested in just the
potential or just the cost of a recurrence.

\chapter{Higher Order Complexity Analysis}
\paragraph{}
Programs are written in the source language. Then the program is translated
to a complexity language. The semantic interpretation of the complexity
language program may be used to analyse the complexity of the original
program.

\section{Source Language}
\paragraph{}
The source language is the simply typed lambda calculus with \T{Unit},
products, suspensions, programmer-defined inductive datatypes and a recursion
construct. Valid signatures, types, and constructor arguments are given in
Figure \ref{fig:source_lang_sigs_types_constructors}. The types, expressions,
and typing judgments of the source language are given in Figure
\ref{fig:source_lang_syntax_types}. Evaluation is call-by-value and the rules
for evaluation are given in Figure \ref{fig:source_lang_oper_sem}.

\paragraph{}
We use big-step operational cost semantics. Small-step operational semantics
provide an indirect notion of number of steps required to evaluate a program to
a value. Big-step operational semantics do not allow this since intermediate
evaluation steps are suppressed. Big-step operational semantics introduce a
notion of cost by using evaluation judgements of the form $e \downarrow^n v$.
For example the evaluation judgement for a tuple is
%
\begin{prooftree}
  \AxiomC{$e_0 \downarrow^{n_0} v_0$}
  \AxiomC{$e_1 \downarrow^{n_1} v_1$}
  \BinaryInfC{$\LP e_0,e_1 \RP \downarrow^{n_0 + n_1} \LP v_0,v_1\RP$}
\end{prooftree}
%
This judgement reads if $e_0$ evaluates to a value $v_0$ in $n_0$ steps and
$e_1$ evaluates to a value $v_1$ in $n_1$ steps, then the tuple $\LP
e_0,v_0\RP$ evaluates to the value $\LP v_0,v_1\RP$ in $n_0+n_1$ steps.

\paragraph{}
A program using datatypes must have a top-level signature $\psi$ consisting of
datatype declarations of the form
%
\[ \T{datatype} \delta = C^\delta_0 \T{of} \phi_{C_0}[\delta] \ |\ ...\ |\ C^\delta_{n-1} \T{of} \phi_{C_{n-1}}[\delta] \]
%
Each datatype may only refer to datatypes declared earlier in the signature.
This prevents general recursive datatypes.  The argument to each constructor is
given by a strictly positive functor $\phi$, which is one of $t$, $\tau$,
$\phi_0 \times \phi_1$, and $\tau \rightarrow \phi$.  The identity functor $t$
represents recursive occurrence of the datatype.  The constant functor $\tau$
represents a non-recursive type.  The product functor $\phi_0 \times \phi_1$
represents a pair of arguments.  The constant exponential $\tau \rightarrow \phi$
represents a function type.  The introduction forms for datatypes are the
constructors.  The elimination form for a datatype is the \T{rec} construct.

\paragraph{}
To give the reader a better understanding of the source language, we will
implement a small program, explaining the syntax and semantics we need as we
go.  We define an \T{list} datatype in the source language below.
%
\[
  \T{datatype list = Nil of unit | Cons of int$\times$list}
\]
%
\T{unit} is a singleton type with only one inhabitant, the value
$\LP\RP$, also called unit.


The \T{listmap} function applies a function to each element in a list.
%
\[
  \T{listmap f xs} = \T{rec}(xs, \T{Nil} \mapsto z.\T{Nil}, \T{Cons} \mapsto z.\T{Cons}\LP\pi_0 z, \T{force}(\pi_1\pi_1 z)\RP)
\]
%
This function uses the \T{rec} construct, which is how we do structural
recursion on datatypes.
%
\begin{prooftree}
\AxiomC{$\gamma \vdash e : \delta$}
\AxiomC{$\forall C . \gamma, x : \phi_C[\delta \times \T{susp}\ \tau] \vdash e_C : \tau$}
\BinaryInfC{$\gamma \vdash \T{rec}^\delta(e, \overline{C \mapsto x.e_C}) : \tau$}
\end{prooftree}
%
The \T{rec} is a branch on an expression. The expression is evaluated to a
value, and the branch of the \T{rec} matching the outermost constructor of the
value is taken. Inside the each branch of the \T{rec}, the variable $x$ is a
value of type $\phi_C[\delta \times \T{susp } \tau]$. A suspension is an
unevaluated computation.  A suspension has type \T{susp} $\tau$ where $\tau$ is
the type of the suspended computation.

\paragraph{}
Suspensions are introduced using the \T{delay}$(e)$ operator.  Suspensions are
eliminated using the \T{force}$(e)$ operator, which evaluates the suspended
computation. The \T{rec} construct makes available all recursive calls.
Suspensions are necessary to avoid charging for recursive calls that are not
actually used.

\paragraph{}
The operational semantics for \T{rec} are
%
\begin{prooftree}
  \AxiomC{$e \downarrow^{n_0} C v_0$}
  \AxiomC{$\T{map}^{\phi_C}(y.\LP y, \T{delay}(\T{rec}(y, \overline{C \mapsto x.e_C}))\RP, v_0) \downarrow^{n_1} v_1$}
  \AxiomC{$e_C[v_1/x] \downarrow^{n_2} v$}
  \TrinaryInfC{$rec(e, \overline{C \mapsto x.e_C}) \downarrow^{1 + n_0 + n_1 + n_2} v$}
\end{prooftree}
%
\T{map} is used to lift functions from $\sigma \rightarrow \tau$ to
$\phi[\sigma] \rightarrow \phi[\tau]$. To understand the role of \T{map} in
\T{rec}, let us consider the two branches of \T{rec} in \T{listmap}.
%
\[
  \text{Let } E = \T{rec}(y, \T{Nil} \mapsto \T{Nil}, \T{Cons} \mapsto z.\T{Cons}\LP\pi_0 z, \T{force}(\pi_1\pi_1 z)\RP)
\]
%

\paragraph{}
The first case, $xs$ is \T{Nil}, so according to the operational semantics,
$e \downarrow \T{Nil} \LP\RP$ in $0$ steps. Next
%
\[
  \T{map}^{\phi_\T{Nil}}(y,\LP y, \T{delay}(E)\RP, \LP\RP)
\]
%
is evaluated to a value $v_1$. We substitute $v_1$ for $z$ in the body of the
\T{Nil} branch and evaluate the body to a value to get our result.
%
\begin{prooftree}
  \AxiomC{}
  \UnaryInfC{$\T{map}^\tau(x.v, v_0) \downarrow^0 v_0$}
\end{prooftree}
%
So the \T{map} evaluates to $\LP\RP$.


\paragraph{}
In the second case, the outermost constructor of $xs$ is \T{Cons}. Let the
argument to this constructor be the tuple $\LP x,xs'\RP$. So the map expression
is
%
\[
  \T{map}^{\phi_\T{Cons}}(y,\LP y,\T{delay}(E)\RP, (x, xs') \RP)
\]
%
To evaluate the \T{map} expression, we use the rule for mapping over a pair.
%
\begin{prooftree}
  \AxiomC{$\T{map}^{\phi_0}(x.v, v_0) \downarrow^{n_0} v_0'$}
  \AxiomC{$\T{map}^{\phi_1}(x.v, v_1) \downarrow^{n_1} v_1'$}
  \BinaryInfC{$\T{map}^{\phi_0 \times \phi_1}(x.v, \LP v_0, v_1 \RP) \downarrow^{n_0 + n_1} \LP v_0', v_1'\RP$}
\end{prooftree}
%
We apply the rule for mapping over a pair.
%
\[
  \LP \T{map}^\T{int}(y,\LP y,\T{delay}(E)\RP, x), \T{map}^{\phi_\T{susp list}}(y,\LP y,\T{delay}(E)\RP, xs')\RP
\]
%
The first \T{map} is over a non-recursive occurrence of \T{list}. Recall the rule
for evaluating this \T{map}.
%
\begin{prooftree}
  \AxiomC{}
  \UnaryInfC{$\T{map}^\tau(x.v, v_0) \downarrow^0 v_0$}
\end{prooftree}
%
So the first \T{map} evaluates to $x$.


The second map is over a recursive occurrence of \T{list}. The rule to
evaluate this \T{map} is
%
\begin{prooftree}
  \AxiomC{}
  \UnaryInfC{$\T{map}^t(x.v, v_0) \downarrow^0 v[v_0/x]$}
\end{prooftree}
%
The result of the \T{map} over the second element of the tuple is $\LP y,
\T{delay}(E) \RP[xs'/y] = \LP xs',\T{delay}(E[xs'/y])\RP$. So the result of the
\T{map} over the tuple is $\LP x, \LP xs', \T{delay}(E[xs'/y])\RP\RP$. Recall
the body of the \T{Cons} branch of the \T{rec} is $\T{Cons}\LP f \pi_0 z,
\T{force}(\pi_1\pi_1 z)\RP$. We have just shown how the \T{map} expression
results in the term $\LP x, \LP xs',\T{delay}(E[xs'/z])\RP\RP$. This is the
term $z$ is bound to inside the body of the \T{Cons} branch. So $\pi_0 z$ is
the head of the list and $\pi_1\pi_1 z$ is a suspended computation representing
the recursive call on the tail of the list. Since it is  suspended, we need to
use the \T{force} function to evaluate it.


\paragraph{}
The $\T{let}(e_0,x.e_1)$ syntactic construct serves the purpose avoiding
recomputation of values. If $e_0$ is an expensive computation that occurs more
than once in $e_1$, we can use \T{let} to compute $e_0$ and use the result
inside $e_1$ multiple times without paying cost multiple times.


\begin{figure}
  \caption{Source language syntax and types}
  \label{fig:source_lang_syntax_types}
  Types
  \begin{align*}
    \tau &::= \T{unit}\ |\ \tau \times \tau\ |\ \tau \rightarrow \tau\ |\ \T{susp}\ \tau\ |\ \delta \\
    \phi &::= t\ |\ \tau\ |\ \phi \times \phi\ |\ \tau \rightarrow \phi \\
    \T{datatype}\ \delta &= C^\delta_0 \T{of} \phi_{C_0}[\delta]\ |\ ...\ |\ C^\delta_{n-1} \T{of} \phi_{C_{n-1}}[\delta]
  \end{align*}

  Expressions
  \begin{align*}
    v &::= x\ |\ \LP\RP\ |\ \LP v, v \RP\ |\ \lambda x.e\ |\ \T{delay}(e)\ |\ C\ v \\
    e &::= x\ |\ \LP\RP\ |\ \LP e, e \RP\ |\ \T{split}(e, x.x.e)\ |\ \lambda x.e\ |\ e\ e \\
      &\quad\ |\ \T{delay}(e)\ |\ \T{force}(e)\ |\ C^\delta\ e\ |\ \T{rec}^\delta(e, \overline{C \mapsto x.e_C}) \\
      &\quad\ |\ \T{map}^\phi(x.v, v)\ |\ \T{let}(e, x.e) \\
    n &::= 0\ |\ 1\ |\ n + n
  \end{align*}

  Typing Judgments

  \AxiomC{}
  \UnaryInfC{$\gamma, x : \sigma \vdash x : \sigma$}
  \DisplayProof
  \AxiomC{}
  \UnaryInfC{$\gamma \vdash \LP \RP : \T{unit}$}
  \DisplayProof

  \bigskip

  \AxiomC{$\gamma \vdash e_0 : \tau_0$}
  \AxiomC{$\gamma \vdash e_1 : \tau_1$}
  \BinaryInfC{$\LP e_0, e_1 \RP : \tau_0 \times \tau_1$}
  \DisplayProof
  \AxiomC{$\gamma \vdash e_0 : \tau_0 \times \tau_1$}
  \AxiomC{$\gamma, x_0 : \tau_0, x_1 : \tau_1 \vdash e_1 : \tau$}
  \BinaryInfC{$\gamma \vdash \T{split}(e_0, x_0.x_1.e_1) : \tau$}
  \DisplayProof

  \bigskip

  \AxiomC{$\gamma, x : \sigma \vdash e : \tau$}
  \UnaryInfC{$\gamma \vdash \lambda x.e : \sigma \rightarrow \tau$}
  \DisplayProof
  \AxiomC{$\gamma \vdash e_0 : \sigma \rightarrow \tau$}
  \AxiomC{$\gamma \vdash e_1 : \sigma$}
  \BinaryInfC{$\gamma \vdash e_0\ e_1 : \tau$}
  \DisplayProof

  \bigskip

  \AxiomC{$\gamma \vdash e : \tau$}
  \UnaryInfC{$\gamma \vdash \T{delay}(e) : \T{susp}\ \tau$}
  \DisplayProof
  \AxiomC{$\gamma \vdash e : \T{susp}\ \tau$}
  \UnaryInfC{$\gamma \vdash \T{force}(e) : \tau$}
  \DisplayProof

  \bigskip

  \AxiomC{$\gamma \vdash e : \phi_C[\delta]$}
  \UnaryInfC{$\gamma \vdash C^\delta\ e : \delta$}
  \DisplayProof
  \AxiomC{$\gamma \vdash e : \delta$}
  \AxiomC{$\forall C . \gamma, x : \phi_C[\delta \times \T{susp}\ \tau] \vdash e_C : \tau$}
  \BinaryInfC{$\gamma \vdash \T{rec}^\delta(e, \overline{C \mapsto x.e_C}) : \tau$}
  \DisplayProof

  \bigskip

  \AxiomC{$\gamma, x : \tau_0 \vdash v_1 : \tau_1$}
  \AxiomC{$\gamma \vdash v_0 : \phi[\tau_0]$}
  \BinaryInfC{$\T{map}^\phi(x.v_1, v_0) : \phi[\tau_1]$}
  \DisplayProof
  \AxiomC{$\gamma \vdash e_0 : \sigma$}
  \AxiomC{$\gamma, x : \sigma \vdash e_1 : \tau$}
  \BinaryInfC{$\T{let}(e_0, x.e_1) : \tau$}
  \DisplayProof
\end{figure}


\begin{figure}
  \caption{Source language valid signatures, types, and constructor arguments}
  \label{fig:source_lang_sigs_types_constructors}

  Signatures: $\psi$ \T{sig}

  \bigskip

  \AxiomC{}
  \UnaryInfC{$\LP\RP$ \T{sig}}
  \DisplayProof
  \quadfive
  \AxiomC{$\delta \notin \forall\text{C}(\psi \vdash \phi_C \T{ok})$}
  \UnaryInfC{$\psi, \T{ datatype } \delta = \overline{C \T{ of } \phi_C[\delta]} \T{ sig}$}
  \DisplayProof

  \bigskip \bigskip

  Types : $\psi \vdash \tau\ \T{type}$

  \bigskip

  \AxiomC{}
  \UnaryInfC{$\psi \vdash \T{unit type}$}
  \DisplayProof
  \quad
  \AxiomC{$\psi \vdash \tau_0\ \T{type}$}
  \AxiomC{$\psi \vdash \tau_1\ \T{type}$}
  \BinaryInfC{$\psi \vdash \tau_0 \times \tau_1\ \T{type}$}
  \DisplayProof

  \bigskip

  \AxiomC{$\psi \vdash \tau_0\ \T{type}$}
  \AxiomC{$\psi \vdash \tau_1\ \T{type}$}
  \BinaryInfC{$\psi \vdash \tau_0 \rightarrow \tau_1\ \T{type}$}
  \DisplayProof
  \quad
  \AxiomC{$\psi \vdash \tau\ \T{type}$}
  \UnaryInfC{$\psi \vdash \T{susp}\ \tau\ \T{type}$}
  \DisplayProof
  \quad
  \AxiomC{$\delta \in \psi$}
  \UnaryInfC{$\psi \vdash \delta\ \T{type}$}
  \DisplayProof

  \bigskip

  Constructor arguments: $\psi \vdash \phi \T{ ok}$

  \bigskip

  \AxiomC{}
  \UnaryInfC{$\psi \vdash t\ \T{ ok}$}
  \DisplayProof
  \quadseven
  \AxiomC{$\psi \vdash \tau \T{ type}$}
  \UnaryInfC{$\psi \vdash \tau \T{ ok}$}
  \DisplayProof

  \bigskip

  \AxiomC{$\psi \vdash \phi_0\ \T{ok}$}
  \AxiomC{$\psi \vdash \phi_1\ \T{ok}$}
  \BinaryInfC{$\psi \vdash \phi_0 \times \phi_1\ \T{ok}$}
  \DisplayProof
  \quadfive
  \AxiomC{$\psi \vdash \tau\ \T{type}$}
  \AxiomC{$\psi \vdash \phi\ \T{ok}$}
  \BinaryInfC{$\psi \vdash \tau \rightarrow \phi\ \T{ok}$}
  \DisplayProof
\end{figure}


\begin{figure}
  \caption{Source language operational semantics}
  \label{fig:source_lang_oper_sem}

  \bigskip

  \AxiomC{$e_0 \downarrow^{n_0} v_0$}
  \AxiomC{$e_1 \downarrow^{n_1} v_1$}
  \BinaryInfC{$\LP e_0, e_1 \RP \downarrow^{n_0 + n_1} \LP v_0, v_1 \RP$}
  \DisplayProof
  \quad
  \AxiomC{$e_0 \downarrow^{n_0} \LP v_0, v_1 \RP$}
  \AxiomC{$e_1[v_0/x_0, v_1/x_1] \downarrow^{n_1} v$}
  \BinaryInfC{$\T{split}(e_0, x_0.x_1.e_1) \downarrow^{n_0 + n_1} v$}
  \DisplayProof

  \bigskip

  \AxiomC{$e_0 \downarrow^{n_0} \lambda x.e_0'$}
  \AxiomC{$e_1 \downarrow^{n_1} v_1$}
  \AxiomC{$e_0'[v_1/x] \downarrow^n v$}
  \TrinaryInfC{$e_0\ e_1 \downarrow^{1 + n_0 + n_1 + n} v$}
  \DisplayProof
  \quad
  \AxiomC{}
  \UnaryInfC{$\T{delay}(e) \downarrow^0 \T{delay}(e)$}
  \DisplayProof

  \bigskip

  \AxiomC{$e \downarrow^{n_0} \T{delay}(e_0)$}
  \AxiomC{$e_0 \downarrow^{n_1} v$}
  \BinaryInfC{$\T{force}(e) \downarrow^{n_0 + n_1} v$}
  \DisplayProof
  \quad
  \AxiomC{$e \downarrow^n v$}
  \UnaryInfC{$C e \downarrow^n C v$}
  \DisplayProof

  \bigskip

  \AxiomC{$e \downarrow^{n_0} C v_0$}
  \AxiomC{$\T{map}^{\phi_C}(y.\LP y, \T{delay}(rec(y, \overline{C \mapsto x.e_C}))\RP, v_0) \downarrow^{n_1} v_1$}
  \AxiomC{$e_C[v_1/x] \downarrow^{n_2} v$}
  \TrinaryInfC{$rec(e, \overline{C \mapsto x.e_C}) \downarrow^{1 + n_0 + n_1 + n_2} v$}
  \DisplayProof

  \bigskip

  \AxiomC{}
  \UnaryInfC{$\T{map}^t(x.v, v_0) \downarrow^0 v[v_0/x]$}
  \DisplayProof
  \quad
  \AxiomC{}
  \UnaryInfC{$\T{map}^\tau(x.v, v_0) \downarrow^0 v_0$}
  \DisplayProof

  \bigskip

  \AxiomC{$\T{map}^{\phi_0}(x.v, v_0) \downarrow^{n_0} v_0'$}
  \AxiomC{$\T{map}^{\phi_1}(x.v, v_1) \downarrow^{n_1} v_1'$}
  \BinaryInfC{$\T{map}^{\phi_0 \times \phi_1}(x.v, \LP v_0, v_1 \RP) \downarrow^{n_0 + n_1} \LP v_0', v_1'\RP$}
  \DisplayProof

  \AxiomC{}
  \UnaryInfC{$\T{map}^{\tau \to \phi}(x.v, \lambda y.e) \downarrow^0 \lambda y.\T{let}(e, z.\T{map}^\phi(x.v, z))$}
  \DisplayProof
  \quad
  \AxiomC{$e_0 \downarrow^{n_0} v_0$}
  \AxiomC{$e_1[v_0/x] \downarrow^{n_1} v$}
  \BinaryInfC{$\T{let}(e_0, x.e_1) \downarrow^{n_0 + n_1} v$}
  \DisplayProof
\end{figure}

\section{Complexity Language}

\paragraph{}
The types, expressions, and typing judgments of the complexity language are
given in Figure \ref{fig:complexity_lang}.  The complexity language is similar
to the source language with a few exceptions.

\paragraph{}
Suspensions are no longer present in the complexity language. Recall
suspensions served the purpose of avoiding charging costs in unused recursive
calls during the translation into the complexity language. Since the complexity
language program has already been translated, the complexity language does not
need suspensions.

\paragraph{}
Another difference is tuples are deconstructed using projection functions
instead of \T{split}. In the source language, to add two elements of a tuple
together we write
%
\[
  \lambda p.\T{split}(p,x_0.x_1.x_0 + x_1)
\]
%
In the complexity language we write
%
\[
  \lambda p.\pi_0 p + \pi_1 p
\]
%

\paragraph{}
The \T{map} function is treated as a macro $\T{map}^\Phi$ in the complexity
language. The macro is defined by $\Phi$ and the definition mirrors the
semantics of \T{map} in the source language.
%
\begin{align*}
  \T{map}^t(x.E,E_0) &= E[E_0/x] \\
  \T{map}^T(x.E,E_0) &= E_0 \\
  \T{map}^{\Phi_0\times\Phi_1}(x.E,E_0) &= \LP\T{map}^{\Phi_0}(x.E,\pi_0 E_0), \T{map}^{\Phi_1}(x.E,\phi_1 E_1)\RP \\
  \T{map}^{T \to \Phi}(x.E,E_0) = \lambda y.\T{map}^\Phi(x.E,E_0\ y)
\end{align*}
%


\begin{figure}
  \caption{Complexity language types, expressions, and typing judgments}
  \label{fig:complexity_lang}

  Types
  \begin{align*}
    T &::= \textbf{C} \ |\ \T{unit} \ |\ \Delta \ |\ T \times T \ |\ T \rightarrow T \\
    \Phi &::= t \ |\ T \ |\ \Phi \times \Phi \ |\ T \rightarrow \Phi \\
    \textbf{C} &::= 0\ |\ 1\ |\ 2\ |\ ... \\
    \T{datatype}\Delta &= C^\Delta_0 \T{of} \Phi_{C_0}[\Delta] \ |\ ... \ |\ C^\Delta_{n-1} \T{of} \Phi_{C_{n-1}}[\Delta]
  \end{align*}

  Expressions
  \begin{align*}
    E &::= x | 0 | 1 | E + E | \LP\RP | \LP E,E \RP | \\
      &\quad \pi_0 E | \pi_1 E | \lambda x.E | E\ E | C^\delta\ E | \text{rec}^\Delta(E, \overline{C \mapsto x.E_C})
  \end{align*}

  Typing Judgments

  \bigskip

  \AxiomC{}
  \UnaryInfC{$\Gamma, x : T \vdash x : T$}
  \DisplayProof
  \quad
  \AxiomC{}
  \UnaryInfC{$\Gamma \vdash 0 : \textbf{C}$}
  \DisplayProof
  \quad
  \AxiomC{}
  \UnaryInfC{$\Gamma \vdash 1 : \textbf{C}$}
  \DisplayProof
  \quad
  \AxiomC{}
  \UnaryInfC{$\Gamma \vdash \LP\RP : \textbf{unit}$}
  \DisplayProof

  \bigskip

  \AxiomC{$\Gamma \vdash E_0 : \textbf{C}$}
  \AxiomC{$\Gamma \vdash E_1 : \textbf{C}$}
  \BinaryInfC{$\Gamma \vdash E_0 + E_1 : \textbf{C}$}
  \DisplayProof
  \quad
  \AxiomC{$\Gamma \vdash E_0 : T_0$}
  \AxiomC{$\Gamma \vdash E_1 : T_1$}
  \BinaryInfC{$\Gamma \vdash \LP E_0, E_1 \RP : T_0 \times T_1$}
  \DisplayProof

  \bigskip

  \AxiomC{$\Gamma \vdash E : T_0 \times T_1$}
  \UnaryInfC{$\Gamma \vdash \pi_i E : T_i$}
  \DisplayProof
  \quad \AxiomC{$\Gamma, x : T_0 \vdash E : T_1$} \UnaryInfC{$\Gamma \vdash \lambda x.E : T_0 \rightarrow T_1$}
  \DisplayProof

  \bigskip

  \AxiomC{$\Gamma \vdash E_0 : T_0 \rightarrow T_1$}
  \AxiomC{$\Gamma \vdash E_1 : T_0$}
  \BinaryInfC{$\Gamma \vdash E_0\ E_1 : T_1$}
  \DisplayProof
  \quad
  \AxiomC{$\Gamma \vdash E : \Phi_C[\Delta]$}
  \UnaryInfC{$\Gamma \vdash C^\Delta E : \Delta$}
  \DisplayProof

  \bigskip

  \AxiomC{$\Gamma \vdash E : \Delta$}
  \AxiomC{$\forall C . \Gamma, x : \Phi_C[\Delta \times T] \vdash E_C : T$}
  \BinaryInfC{$\Gamma \vdash \text{rec}^\Delta(E, \overline{C \mapsto x.E_C}) : T$}
  \DisplayProof

\end{figure}

The translation from the source language to the complexity language is given in
Figure \ref{fig:complexity_translation_types} and Figure
\ref{fig:complexity_translation_expressions}.
%
\begin{figure}
  \caption{Translation from source language to complexity language types.}
  \label{fig:complexity_translation_types}
  %
  \begin{align*}
    \|\tau\| &= \textbf{C} \times \llangle \tau \rrangle \\
    \llangle\T{unit}\rrangle &= \T{unit} \\
    \llangle \sigma \times \tau \rrangle &= \llangle \sigma \rrangle \times \llangle \tau \rrangle \\
    \llangle \sigma \rightarrow \tau \rrangle &= \llangle \sigma \rrangle \rightarrow \|\tau\| \\
    \llangle \T{susp}\ \tau \rrangle &= \|\tau\| \\
    \llangle \delta \rrangle &= \delta \\
  \end{align*}
  %
  \begin{align*}
    \|\phi\| &= \textbf{C} \times \llangle \phi \rrangle \\
    \llangle t \rrangle &= t \\
    \llangle \tau \rrangle &= \llangle \tau \rrangle \\
    \llangle \phi_0 \times \phi_1 \rrangle &= \llangle \phi_0 \rrangle \times \llangle phi_1 \rrangle \\
    \llangle \tau \rightarrow \phi \rrangle &= \llangle \phi \rrangle \rightarrow \|\phi\| \\
  \end{align*}
  \begin{align*}
    \llangle \psi \rrangle &= \text{ for each }\delta \in \psi, \delta = C_0^\delta \T{ of } \llangle \phi_{C_0}\rrangle[\delta], . . . , C_{n-1}^\delta \T{ of } \llangle \phi_{n-1}\rrangle[\delta] \\
  \end{align*}
  %
\end{figure}
%
We denote the complexity translation of a source language expression $e$ as
$\|e\|$. We refer to complexity language expressions of type \textbf{C} as
\textit{costs}, complexity language expressions of type $\llangle \tau \rrangle$ as
\textit{potentials}, and complexity language expressions of type
$\textbf{C}\times \llangle \tau \rrangle$ as \textit{complexities}.
\paragraph{}
Examining the translation of source language types to complexity language types
in Figure \ref{fig:complexity_translation_types}, we see that the translation
of a source language expression of type $\tau$ is
$\textit{C}\times\llangle\tau\rrangle$.  The first element is the cost, a bound
on the cost of evaluating the expression, and the second element is the
potential, an expression for the size of the value.  The potential translation
of types \T{unit} and $\delta$ is the corresponding complexity language types
\T{unit} and $\delta$. The potential translation of a product type is the
complexity language type of a product of the potential translations of the
components of the product type.
TODO EXPLAIN TRANSLATION MORE
%
\begin{figure}
  \caption{Translation from source language to complexity language expressions.}
  \label{fig:complexity_translation_expressions}
  \begin{align*}
    \|x\| &= \LP 0, x \RP \\
    \|\LP\RP\| &= \LP 0, \LP\RP\RP \\
    \|\LP e_0,e_1 \RP\| &= \LP \|e_0\|_c + \|e_1\|_c, \LP \|e_0\|_p, \|e_1\|_p\RP\RP \\
    \|\T{split}(e_0, x_0.x_1.e_1)\| &= \|e_0\|_c +_c \|e_1\|[\pi_0\|e_0\|_p/x_0, \pi_1\|e_0\|_p/x_1] \\
    \|\lambda x.e\| &= \LP 0, \lambda x.\|e\| \RP \\
    \|e_0\ e_1\| &= (1 + \|e_0\|_c + \|e_1\|_c) +_c \|e_0\|_p \|e_1\|_p \\
    \|\T{delay}(e)\| &= \LP 0, \|e\|\RP \\
    \|\T{force}(e)\| &= \|e\|_c +_c \|e\|_p \\
    \|\text{C}^\delta_i\ e\| &= \LP \|e\|_c, \text{C}^\delta_i \|e\|_p \RP \\
    \|\T{rec}^\delta(e, \overline{C \mapsto x.e_C})\| &= \|e\|_c +_c \T{rec}^\delta(\|e\|_p, \overline{C \mapsto x.1 +_c \|e_C\|}) \\
    \|\T{map}^\phi(x.v_0, v_1)\| &= \LP 0, \T{map}^{\llangle \phi \rrangle}(x.\|v_0\|_p, \|v_1\|_p)\RP \\
    \|\T{let}(e_0, x.e_1)\rrangle &= \|e_0\|_c +_c \|e_1\|[\|e_0\|_p/x]
  \end{align*}
  %
\end{figure}



\section{Denotational Semantics}

\paragraph{}
The denotational interpretation of the complexity language is standard. We
interpret numbers as elements of $\mathbb{Z}$, tuples of type $\tau \times
\sigma$ as elements of the cross-product of the set of values of type $\tau$
and the set of values of type $\sigma$, lambda expressions of type $\tau \to
\sigma$ as mathematical functions from the set of values of type $\tau$ to the
set of values of type $\sigma$, and application as mathematical function
application. The nonstandard interpretations are those of datatype constructors
and \T{rec}. Since datatypes are programmer-defined and there are multiple
interpretations for a single datatype, the programmer must provide their own
translation. For example we may decide to interpret \T{list} as their length.
%
\[
  \LB \T{list} \RB = \mathbb{N}
\]
%
\paragraph{}
Semantically, we need to distinguish between the constructors of a datatype, so
we also define a semantic value $D^\T{list}$. $D^\T{list}$ is a sum type of the
arguments to the \T{list} constructors. \T{list} has two constructors, \T{Nil},
which has argument of type \T{unit}, and \T{Cons}, which has argument of type
$\T{int} \times \T{list}$.
%
\[
  D^\T{list} = \ast + \mathbb{Z} \times \mathbb{N}
\]
%
We will write $C_i$ for the $i^{th}$ injection into $D^\T{list}$. $C_i$ takes
us from the interpretation of the argument of a constructor to a value of type
$D^\T{list}$. We also need a function $size_{list}$ which takes us from
$D^\T{list}$ back to $\llbracket \T{list} \rrbracket$. $size_\Delta$ is the
programmers notion of size for programmer-defined datatypes. In this case, we want
the size of a list to be its length. So our $size$ function is defined as
follows.
%
\begin{align*}
  size_{list} (\ast) &= 0 \\
  size_{list}(i,n) &= 1 + n
\end{align*}
%
There is a restriction on the definition of the $size$ function. The size of a
value must be strictly greater than the size of any of its substructures of the
same type. In the case of \T{list}, the restriction means the size of a $(1,
n)$ must be strictly greater than the size of $(j, n-1)$.

\paragraph{}
In general, when we interpret a source language with programmer-defined
datatypes, for each datatype $\Delta$ we must define an interpretation $\LB
\Delta \RB$, a set $D^\Delta$, and a function $size : D^\Delta -> \LB \Delta
\RB$.


\paragraph{}
The interpretation of a datatype with constructor $C$ under the environment
$\xi$ is
%
\[
  \llbracket C\ e\rrbracket \xi = size(C (\llbracket e \rrbracket \xi))
\]
%
In the \T{list} case, if $e$ is \T{Nil}, then the interpretation $\llbracket \T{Nil}
\rrbracket$ is $size_{list}(C_0 \LB\LP\RP\RB)$, since the argument to the \T{Nil}
constructor is $\LP\RP$. The interpretation of \T{unit} is $0$. So
$size_{list}(C_0 \LB\LP\RP\RB = size_{list}(C_0\ 0)$. $C_0$ is the $0^{th}$
injection from $\LB \Phi[\T{list}\ \RB$ to $D^\T{list}$.  So
$size_{list}(C_0\ 0) = size_{list}(\ast)$. By our definition of
$size_{list}$, $size_{list}(\ast) = 0$.


\paragraph{}
The interpretation of \T{rec} is also nonstandard. To interpret \T{rec}, we
introduce a semantic $case$ function.
%
\[
  case^\delta : D^\delta \times \Uppi_{C} (S^{\LB \Phi_C[\delta]\RB} \to S^\tau) \to S^\tau
\]
%
$S^T = \LB T \RB$ for each complexity type $T$. The interpretation of a \T{rec} is
%
\[
  \LB rec^\delta(E^\delta, \overline{C \mapsto x^{\phi_C[\delta \times \tau]}.E_C^\tau}) \RB \xi = \bigvee\limits_{size(z) \leq \LB E \RB \xi} case(z, \overline{f_C})
\]
%
where for each constructor $C$,
%
\[
  f_C(x) = \LB E_C \RB \xi \{ x \mapsto map^{\Phi_C}(a.(a, \LB rec(w, \overline{C \mapsto x.E_C})\RB \xi \{ w \mapsto a \}), x) \}
\]
%
Since we cannot predict which branch the \T{rec} will take, we must take the
maximum over all possible branches to obtain an upper bound. Recall our
restriction on the $size$ function that the size of a value must be strictly
greater than the size of any of its substructure of the same type. This ensures
the recursion used to interpret the \T{rec} expressions is well-defined.
Continuing with the \T{list} example, the interpretation of \T{rec} on a \T{list} is
%
\begin{align*}
  \LB\T{rec}(E_0, \T{Nil} \mapsto E_\T{Nil}, \T{Cons} \mapsto x.E_\T{Cons})\RB &= \bigvee\limits_{size(z) \leq \LB E_0 \RB} case(z, f_{Nil}, f_{Cons}) \\
  \text{where} \\
  f_{Nil}(\ast) &= \LB E_\T{Nil} \RB \\
  f_{Cons}(i,n) &= \LB E_\T{Cons} \RB
\end{align*}

\chapter{Sequential Recurrence Extraction Examples}
% ---------------------------------------- FAST REVERSE ----------------------------------------
\section{Fast Reverse}
Fast reverse is an implementation reverse in linear time complexity.
A naive implementation of reverse appends the head of the list to recursively
reversed tail of  the list. Fast reverse instead uses an abstraction to delay
the consing. As this is the first example, we will walk through the translation
and interpretation in gory detail. In following examples we will relegate the
walk-through of the translation to the appendices, where the reader can peruse
them, perhaps over a glass of carbenet sauvignon, as a relaxing end to a
stressful day.

The definition of the list datatype holds no suprises.
\[ \T{datatype list} = \T{Nil of unit}\ |\ \T{Cons of int} \times \T{list} \]

The implementation of fast reverse is not obvious. We write a function \T{rev}
that applies an auxilary function to an empty list to produce the result.  The
specification of reverse is \T{rev [$x_0,\dots,x_{n-1}$] =
[$x_{n-1},\dots,x_0$]}. The specification of the auxilary function
\T{rec(xs,$\dots$)} is \T{rec($[x_0,\dots,x_{n-1}],\dots$)
[$y_0,\dots,y_{m-1}$] = [$x_{n-1},\dots,x_0,y_0,\dots,y_{m-1}$]}.

\begin{lstlisting}
rev xs = $\lambda$xs.rec(xs,
               Nil $\mapsto$ $\lambda$a.a,
               Cons$\mapsto$b.split(b,x.c.split(c,xs'.r.
                        $\lambda$a.force(r) Cons$\LP$x,a$\RP$))) Nil
\end{lstlisting}

Notice that the implementation of \T{rev} would be much cleaner if we where
able to pattern match on cases of the \T{rec}. Below is \T{rev} written with
this syntactic sugar.

\begin{lstlisting}
rev = $\lambda$xs.rec(xs, Nil $\mapsto \lambda$a.a,
              Cons$\mapsto\LP$y$\LP$ys,r$\RP\RP$.$\lambda$b.force(r) Cons$\LP$x,b$\RP$) Nil
\end{lstlisting}

Each recursive call creates an abstraction that applies the recursive call on
the tail of the list to the list created by consing the head of the list onto
the abstraction argument. The recursive calls builds nested abstractions as
deep as the length of the list which is collapsed by application of the
outermost abastraction to \T{Nil}. Below we show the evaluation of \T{rev}
applied to a small list of just two elements.

\begin{lstlisting}
rev (Cons$\LP$0,Cons$\LP$1, Nil$\RP\RP$) $\to$
  rec(Cons$\LP$0,Cons$\LP$1,Nil$\RP\RP$,
      Nil $\mapsto\lambda$a.a
      Cons$\mapsto$b.split(b,x.c.split(c,xs'.r.
               $\lambda$a.force(r) Cons$\LP$x,a$\RP$))) Nil
  $\to^* (\lambda$a0.($\lambda$a1.($\lambda$a2.a2) Cons$\LP$1,a1$\RP$) Cons$\LP$0,a0$\RP$) Nil
  $\to_\beta$ ($\lambda$a1.($\lambda$a2.a2) Cons$\LP$1,a1$\RP$) Cons$\LP$0,Nil$\RP$
  $\to_\beta$ ($\lambda$a2.a2) Cons$\LP$1,Cons$\LP$0,Nil$\RP\RP$
  $\to_\beta$ Cons$\LP$1,Cons$\LP$0,Nil$\RP\RP$
\end{lstlisting}

% -------------------- BEGIN REV TRANSLATION SPLIT --------------------
\subsection{Translation}
We will walk through the translation from the source language to the complexity
language.
%
\begin{flalign*}
  \|\T{rev}\| &= \|\lambda xs.\T{rec}(xs, \T{Nil}\mapsto\lambda\T{a.a,} \\
              &\quad \T{Cons}\mapsto b.\T{split}(b,x.c.\T{split}(c,xs'.r.\lambda a.\T{force}(r)\ \T{Cons}\LP\T{x,a}\RP)))\ \T{Nil}\| \\
\end{flalign*}
%
% BEGIN ABSTRACTION
%
First we apply the rule for translating an abstraction. The rule is
$\|\lambda x. e\| = \LP 0, \lambda x. \|e\|\RP$.
%
\begin{flalign*}
  \|\T{rev}\| &= \|\lambda xs.\T{rec}(xs, \T{Nil} \mapsto\lambda a.a, \\
              &\quadthree \T{Cons}\mapsto b.\T{split}(b,x.c.\T{split}(c,xs'.r.\lambda a.\T{force}(r) \T{Cons}\LP x,a\RP)))\ \T{Nil}\| \\
              &\quad = \LP 0, \lambda xs.\|\T{rec}(xs, \T{Nil}\mapsto\lambda a.a, \\
              &\quadthree \T{Cons}\mapsto b.\T{split}(b,x.c.\T{split}(c,xs'.r.\lambda a.\T{force}(r) \T{Cons}\LP x,a\RP)))\ \T{Nil}\|\RP \\
\end{flalign*}
%
The next translation is an application. The rule for translating an application is
$\|e_0\ e_1\| = (1 + \|e_0\|_c + \|e_1\|_c) +_c (\|e_0\|_p\ \|e_1\|_p)$.
In this case, \T{rec(...)} is $e_0$ and \T{Nil} is $e_1$. We translate \T{Nil} then
\T{rec(...)} seperately.
%
% APP ARGUMENT
%
The translation of a constructor applied to an expression is a tuple of the
cost of the translated expression and the corresponding complexity language
constructor applied to the potential of the translated expression. Since the
expression inside \T{Nil} is $\LP\RP$, and
$\|\LP\RP\| = \LP 0,\LP\RP\RP$, we have
%
\begin{flalign*}
  |\T{Nil}\| &= \LP\LP 0, \LP\RP\RP_c, \T{Nil}\LP 0,\LP\RP\RP_p\RP \\
             &= \LP 0, \T{Nil}\LP\RP\RP
\end{flalign*}
%
% BEGIN REC
%
The rule for translating a \T{rec} expression is
\[
  \|\T{rec}(e,\overline{C \mapsto x.e_C})\| = \|e\|_c +_c \T{rec}(\|e\|_p, \overline{C \mapsto x.\|e_C\|})
\]
%
\begin{flalign*}
  &\|\T{rec}(xs, \T{Nil}\mapsto\lambda a.a, \\
  &\qquad \T{Cons}\mapsto b.\T{split}(b,x.c.\T{split}(c,xs'.r.\lambda a.\T{force}(r)\ \T{Cons}\LP x,a\RP)))\| \\
  &= \|xs\|_c +_c \T{rec}(\|xs\|_p, \T{Nil} \mapsto 1 +_c \|\lambda a.a\| \\
  &\quadthree \T{Cons}\mapsto b. 1 +_c \|\T{split}(b,x.c.\T{split}(c,xs'.r.\lambda a.\T{force}(r)\ \T{Cons}\LP x,a\RP))\|) \\
  &= \LP 0, xs \RP_c +_c \T{rec}(\LP 0, xs\RP_p, \T{Nil} \mapsto 1 +_c \|\lambda a.a\| \\
  &\quadthree \T{Cons}\mapsto b. 1 +_c \|\T{split}(b,x.c.\T{split}(c,xs'.r.\lambda a.\T{force}(r)\ \T{Cons}\LP x,a\RP))\|) \\
  &\text{The term $xs$ is a variable and the rule for translating variables is $\|xs\| = \LP 0, xs\RP$.} \\
  &= \T{rec}(xs, \T{Nil} \mapsto 1 +_c \|\lambda a.a\| \\
  &\quadthree \T{Cons}\mapsto b. 1 +_c \|\T{split}(b,x.c.\T{split}(c,xs'.r.\lambda a.\T{force}(r)\ \T{Cons}\LP x,a\RP))\|)
\end{flalign*}
%
% NIL BRANCH
%
The translation of the \T{Nil} branch is
simple application of the $\|\lambda x.e\| = \LP 0, \lambda
x.\|e\|\RP$ and the variable translation rule.
%
\begin{flalign*}
  &1 +_c \|\lambda a.a\| \\
  &= 1  +_c   \LP 0, \lambda a. \| a \|\RP \\
  &=  \LP 1, \lambda a. \LP 0,a \RP\RP
\end{flalign*}
%
% BEGIN CONS BRANCH
%
The translation of the \T{Cons} branch is a slightly more involved. The rule
for translating \T{split} is
%
\[ \|\T{split}(e_0,x_0.x_1.e_1)\| = \|e_0\|_c +_c \|e_1\|[\pi_0\|e_0\|_p/x_0, \pi_1\|e_0\|_p/x_1] \]
%
After applying the rule to the \T{Cons} branch we get
%
\begin{flalign*}
  &1 +_c \|\T{split}(b,x.c.\T{split}(c,xs'.r.\lambda a.\T{force}(r)\ \T{Cons} \LP x,a \RP )) \| \\
  &= 1 +_c \|b\|_c +_c \|\T{split}(c,xs'.r.\lambda a.\T{force}(r)\ \T{Cons}\LP x,a \RP) \|[\pi_0 \| b \|_p/x,\pi_1 \| b \|_p/c]
\end{flalign*}
%
Remember that $b$ is a variable and has type
$\phi_\T{Cons}[\T{list} \times \T{susp (list} \to \T{list)}]$.
The translation of this type is
$\textbf{C} \times \llangle \phi_\T{Cons} \rrangle [\T{list} \times \LP \T{list} \to \LP \textbf{C} \times \T{list} \RP\RP]$.
We can say that \T{$\pi_0\|$b$\|_p$} is the head of the list \T{xs},
\T{$\pi_0\pi_1\|$b$\|_p$} is the tail of the list \T{xs}, and
\T{$\pi_1\pi_1\|$b$\|_p$} is the result of the recursive call.
The translation of $b$ is $\LP 0, b\RP$.
%
\begin{flalign*}
  &1 +_c \|b\|_c +_c \|\T{split}(c,xs'.r.\lambda a.\T{force}(r)\ \T{Cons}\LP x,a \RP) \|[\pi_0 \| b \|_p/x,\pi_1 \| b \|_p/c] \\
  &=1 +_c \|\T{split}(c,xs'.r.\lambda a.\T{force}(r)\ \T{Cons}\LP x,a \RP) \|[\pi_0 \| b \|_p/x,\pi_1 \| b \|_p/c] \\
  %
  &\qquad \text{We apply the rule for \T{split} again.} \\
  %
  &=1 +_c (\|c\|_c +_c \|\lambda a.\T{force}(r)\ \T{Cons}\LP x,a \RP\|[\pi_0 \|c\|_p/xs', \pi_1\|c\|_p/r][\pi_0 \| b \|_p/x,\pi_1 \| b \|_p/c] \\
  %
  &\qquad \text{$c$ is a variable, so its translation is $\LP 0, c \RP$.} \\
  %
  &=1 +_c \|\lambda a.\T{force}(r)\ \T{Cons}\LP x,a \RP\|[\pi_0 \|c\|_p/xs', \pi_1\|c\|_p/r][\pi_0 \| b \|_p/x,\pi_1 \| b \|_p/c] \\
  %
  &\qquad \text{We apply the rule for abstraction.} \\
  %
  &=1 +_c \LP 0, \lambda a.\|\T{force}(r)\ \T{Cons}\LP x,a \RP\|[\pi_0 \|c\|_p/xs', \pi_1\|c\|_p/r][\pi_0 \| b \|_p/x,\pi_1 \| b \|_p/c] \\
  %
  &\qquad \text{Recall $C +_c E$ is a macro for $\LP C + E_c, E_p\RP$. We use this to eliminate the $+_c$.} \\
  &\qquad \text{We also apply the translation rule for application.} \\
  %
  &=\LP 1, \lambda a.(1 + \|\T{force}(r)\|_c + \|\T{Cons}\LP x,a\RP\|_c) \\
  &\quadfive +_c \|\T{force}(r)\|_p \|\T{Cons}\LP x,a \RP\|_p\RP[\pi_0 \|c\|_p/xs', \pi_1\|c\|_p/r][\pi_0 \| b \|_p/x,\pi_1 \| b \|_p/c] \\
\end{flalign*}
%
% COMPOSE SUBSTITUTIONS
%
\begin{flalign*}
  &\text{We will translate $\T{force}(r)$ and $\T{Cons}\LP x,a\RP$ individually.} \\
  &\text{First we compose the two substitutions.} \\
  %
  &\text{let } \Theta = [\pi_0 \|c\|_p/xs', \pi_1\|c\|_p/r][\pi_0 \| b \|_p/x,\pi_1 \| b \|_p/c] \\
  &\quadthree = [\pi_0\pi_1 \|b\|_p/xs', \pi_1\pi_1\|b\|_p/r, \pi_0 \| b \|_p/x] \\
  &\quad \text{Since $b$ is a variable, the potential  of its translation is $b$.} \\
  &\Theta = [\pi_0\pi_1 b/xs', \pi_1\pi_1 b/r, \pi_0 b/x] \\
\end{flalign*}
%
% FORCE
%
\begin{flalign*}
  &\quad \text{In translation of $\T{force}(r)$ we apply the rule $\|\T{force}(e)\| = \|e\|_c +_c \|e\|_p$.}\\
  &\quadthree \|\T{force}(r)\|\Theta = \|r\|_c\Theta +_c \|r\|_p\Theta \\
  %
  &\quad \text{We apply the variable translation rule to $r$, then apply the substitution $\Theta$.}\\
  %
  &\quadfive = \LP 0, r \RP_c\Theta +_c \LP 0, r \RP_p \Theta \\
  &\quadfive = r\Theta = \pi_1\pi_1 b \\
\end{flalign*}
%
% CONS
%
\begin{flalign*}
  &\quad \text{Next we do the translation of $\T{Cons}\LP x,a \RP$.} \\
  &\quadthree \|\T{Cons}\LP x, a\RP\| = \LP \|\LP x,a \RP\|_c, \T{Cons} \|\LP x,a \RP\|_p\RP \\
  %
  &\quad \text{Notice the translation of $\LP x,a \RP$ appears twice, so we will do this seperately.} \\
  %
  &\quadfour \|\LP x,a \RP\| = \LP \|x\|_c + \|a\|_c, \LP \|x\|_p, \|a\|_p \RP \RP \Theta\\
  %
  &\quad \text{Both $x$ and $a$ are variables, so they have $0$ cost.}\\
  %
  &\quadsix = \LP 0, \LP x, a \RP\RP \Theta\\
  %
  &\quad \text{We apply the substitution $\Theta$.}\\
  %
  &\quadsix = \LP 0, \LP \pi_1 b, \pi_1\pi_1 b \RP\RP \\
  %
  &\quadsix = \LP 0, \LP \pi_1 b, \pi_1\pi_1 b \RP\RP \\
  %
  &\quad \text{We complete the translation of $\T{Cons}\LP x, a\RP$ using $\LP x, a \RP$.} \\
  %
  &\quadthree \|\T{Cons}\LP x, a\RP\| = \LP \|\LP x,a \RP\|_c, \T{Cons} \|\LP x,a \RP\|_p\RP \\
  &\quadfive = \LP 0, \T{Cons} \LP \pi_1 b, \pi_1\pi_1 b \RP\RP \\
\end{flalign*}
%
% END CONS BRANCH
%
\begin{flalign*}
  &\quad \text{We use substitute in the translations of $\T{force}(r)$ and $\T{Cons}\LP x, a\RP$.}\\
  &\quad \text{$\|\T{force}(r)\|$ has cost $(\pi_1\pi_1 b)_c$ and $\|\T{Cons}\LP x,a\RP\|$ has cost $0$.}\\
  &\LP 1, \lambda a.(1 + \|\T{force}(r)\|_c + \|\T{Cons}\LP x,a\RP\|_c) +_c \|\T{force}(r)\|_p \|\T{Cons}\LP x,a \RP\|_p\RP \RP\Theta \\
  %
  &= \LP 1, \lambda a.(1 + (\pi_1\pi_1 b)_c) +_c (\pi_1\pi_1 b)_p\ \T{Cons}\LP \pi_1 b, a \RP\RP \\
\end{flalign*}
%
% END REC
%
\begin{flalign*}
  &\text{We can now complete the translation of the \T{rec} expression.} \\
  &\|\T{rec}(xs, \T{Nil}\mapsto\lambda a.a, \\
  &\qquad \T{Cons}\mapsto b.\T{split}(b,x.c.\T{split}(c,xs'.r.\lambda a.\T{force}(r)\ \T{Cons}\LP x,a\RP)))\| \\
  &= \T{rec}(xs, \T{Nil} \mapsto 1 +_c \|\lambda a.a\| \\
  &\quadthree \T{Cons}\mapsto b. 1 +_c \|\T{split}(b,x.c.\T{split}(c,xs'.r.\lambda a.\T{force}(r)\ \T{Cons}\LP x,a\RP))\|) \\
  &= \T{rec}(xs, \T{Nil} \mapsto \LP 1, \lambda a. \LP 0,a \RP\RP \\
  &\quadthree \T{Cons}\mapsto b.\LP 1, \lambda a.(1 + (\pi_1\pi_1 b)_c) +_c (\pi_1\pi_1 b)_p\ \T{Cons}\LP \pi_1 b, a \RP\RP) \\
\end{flalign*}
%
% END APP FUNCTION
%
\begin{flalign*}
  &\text{We substitute the translation of \T{rec} and \T{Nil} into the translation of the application.}\\
  &\text{Let }R = \T{rec}(xs, \T{Nil} \mapsto \LP 1, \lambda a. \LP 0,a \RP\RP \\
  &\quadfive \T{Cons}\mapsto b.\LP 1, \lambda a.(1 + (\pi_1\pi_1 b)_c) +_c (\pi_1\pi_1 b)_p\ \T{Cons}\LP \pi_1 b, a \RP\RP) \\
  &\|\T{rec}(xs, \T{Nil}\mapsto\lambda a.a, \\
  &\qquad \T{Cons}\mapsto b.\T{split}(b,x.c.\T{split}(c,xs'.r.\lambda a.\T{force}(r)\ \T{Cons}\LP x,a\RP)))\ \T{Nil}\| \\
  &\text{Substituting $R$ for the translation of \T{rec} and $\LP 0, \T{Nil}\RP$ for the translation of \T{Nil}.} \\
  &\quad = (1 + R_c) +_c R_p\ \T{Nil} \RP\\
  &\text{Recall }C +_c E = \LP C + E_c, E_p \RP, \text{ so } (1 + E_c) +_c E_p = 1 +_c E \\
  &\quad = 1 +_c \T{rec}(xs, \T{Nil} \mapsto \LP 1, \lambda a. \LP 0,a \RP\RP \\
  &\quadthree \T{Cons}\mapsto b.\LP 1, \lambda a.(1 + (\pi_1\pi_1 b)_c) +_c (\pi_1\pi_1 b)_p\ \T{Cons}\LP \pi_1 b, a \RP\RP)\ \T{Nil}\\
\end{flalign*}
%
% END APP
%
\begin{flalign*}
  &\text{Finally, we substitute this into the translation of \T{rev}.} \\
  \|\T{rev}\| &= \|(\lambda\T{xs.rec(xs, Nil}\mapsto\lambda\T{a.a,} \\
              &\quad \T{Cons}\mapsto\T{b.split(b,x.c.split(c,xs'.r.}\lambda\T{a.force(r) Cons}\LP\T{x,a}\RP\T{)))) Nil}\| \\
  &\quad = \LP 0, \lambda xs. 1 +_c \T{rec}(xs, \T{Nil} \mapsto \LP 1, \lambda a. \LP 0,a \RP\RP \\
  &\quadthree \T{Cons}\mapsto b.\LP 1, \lambda a.(1 + (\pi_1\pi_1 b)_c) +_c (\pi_1\pi_1 b)_p\ \T{Cons}\LP \pi_1 b, a \RP\RP)\ \T{Nil}\RP\\
\end{flalign*}
%
% END REV TRANSLATION SPLIT
%
Observe that $\|\T{rev}\|$ admits the same syntactic sugar as \T{rev}. In the
complexity language, instead of taking projections of $b$, we can use the same
pattern matching syntactic sugar as in the source language.

\begin{flalign*}
  &\|\T{rev}\| = \LP 0, \lambda xs. 1 +_c \T{rec}(xs, \T{Nil} \mapsto \LP 1, \lambda a. \LP 0,a \RP\RP \\
  &\quadthree \T{Cons}\mapsto \LP x, \LP xs', r\RP\RP.\LP 1, \lambda a.(1 + r_c) +_c r_p\ \T{Cons}\LP \pi_1 x, a \RP\RP)\ \T{Nil}\RP\\
\end{flalign*}


%
% FAST REVESE TRANSLATION USING SYNTACTIC SUGAR
%
\subsection{Syntactic Sugar Translation}
%
We walk through the same translation of fast reverse, but we use the syntactic
sugar for matching introducted earlier. Recall the implementation of fast using
syntactic sugar. The translation is almost identical to the translation of \T{rev}
written without syntactic sugar until we translate the \T{Cons} branch of the
\T{rec}.
%
\begin{flalign*}
  \|\T{rev}\| &= \|\lambda\T{xs.rec(xs, Nil}\mapsto\lambda\T{a.a,} \\
              &\quad \T{Cons}\mapsto\LP x, \LP xs', r\RP\RP.\lambda a.\T{force}(r)\ \T{Cons}\LP x, a\RP)\ \T{Nil}\|
\end{flalign*}
%
First we apply the rule for translating an abstraction. The rule is
$\|\lambda x. e\| = \LP 0, \lambda x. \|e\|\RP$.
%
\begin{flalign*}
  \|\T{rev}\| &= \LP 0, \lambda xs. \|\T{rec}(xs, \T{Nil}\mapsto\lambda a.a, \\
              &\quad \T{Cons}\mapsto\LP x, \LP xs', r\RP\RP.\lambda a.\T{force}(r)\ \T{Cons}\LP x, a\RP)\ \T{Nil}\|\RP
\end{flalign*}
%
%
% BEGIN APP
%
Next we apply the rule for translating an application. The rule is
$\|e_0\ e_1\| = (1 + \|e_0\|_c + \|e_1\|_c) +_c (\|e_0\|_p\ \|e_1\|_p)$.
In this case, \T{rec(...)} is $e_0$ and \T{Nil} is $e_1$. We translate
\T{Nil} then \T{rec(...)} seperately.
%
% APP ARGUMENT
%
The translation of a constructor applied to an expression is a tuple of the
cost of the translated expression and the corresponding complexity language
constructor applied to the potential of the translated expression. Since the
expression inside \T{Nil} is $\LP\RP$, and
$\|\LP\RP\| = \LP 0,\LP\RP\RP$, we have
%
\begin{flalign*}
  \|\T{Nil}\| &= \LP\LP 0, \LP\RP\RP_c, \T{Nil}\LP 0,\LP\RP\RP_p\RP \\
             &= \LP 0, \T{Nil}\LP\RP\RP
\end{flalign*}
%
% BEGIN REC
%
The rule for translating a \T{rec} expression is
\[
  \|\T{rec}(e,\overline{C \mapsto x.e_C})\| = \|e\|_c +_c \T{rec}(\|e\|_p, \overline{C \mapsto x.\|e_C\|})
\]
%
\begin{flalign*}
  &\|\T{rec}(xs, \T{Nil}\mapsto\lambda a.a, \\
  &\qquad \T{Cons}\mapsto \LP x,\LP xs',r \RP\RP.\lambda a.\T{force}(r)\ \T{Cons}\LP x,a\RP)\| \\
  &= \|xs\|_c +_c \T{rec}(\|xs\|_p, \T{Nil} \mapsto 1 +_c \|\lambda a.a\| \\
  &\quadthree \T{Cons}\mapsto \LP x,\LP xs',r \RP\RP.1 +_c \|\lambda a.\T{force}(r)\ \T{Cons}\LP x,a\RP\|) \\
  &= \LP 0, xs \RP_c +_c \T{rec}(\LP 0, xs\RP_p, \T{Nil} \mapsto 1 +_c \|\lambda a.a\| \\
  &\quadthree \T{Cons}\mapsto \LP x,\LP xs',r \RP\RP.1 +_c \|\lambda a.\T{force}(r)\ \T{Cons}\LP x,a\RP\|) \\
  &\text{The term $xs$ is a variable and the rule for translating variables is $\|xs\| = \LP 0, xs\RP$.} \\
  &= \T{rec}(xs, \T{Nil} \mapsto 1 +_c \|\lambda a.a\| \\
  &\quadthree \T{Cons}\mapsto \LP x,\LP xs',r \RP\RP.1 +_c \|\lambda a.\T{force}(r)\ \T{Cons}\LP x,a\RP\|) \\
\end{flalign*}
%
% NIL BRANCH
%
The translation of the \T{Nil} branch is the same as before.
%
\begin{flalign*}
  &1 +_c \|\lambda a.a\| =  \LP 1, \lambda a. \LP 0,a \RP\RP
\end{flalign*}
%
% BEGIN CONS BRANCH
%
The translation of the \T{Cons} branch is much simpler without the two \T{split}s.
%
\begin{flalign*}
  &1 +_c \|\lambda a.\T{force}(r)\ \T{Cons}\LP x, a\RP\| \\
  &\quad = 1 +_c \LP 0, \lambda a.\|\T{force}(r)\ \T{Cons}\LP x, a\RP\|\RP \\
  &\quad = \LP 1, \lambda a.(1 + \|\T{force}(r)\|_c + \|\T{Cons}\LP x, a\RP)\|_c) +_c \|\T{force}(r)\|_p\ \|\T{Cons}\LP x, a \RP\|_p \RP \\
\end{flalign*}
%
The translation of $\T{force}(r)$ and $\T{Cons}\LP x, a \RP$
are the same as before, except we do not have a substitution to apply.
%
\begin{flalign*}
  &\|\T{force}(r)\| = \|r\|_c +_c \|r\|_p = \LP 0, r \RP_c +_c \LP 0, r \RP_p = 0 +_c r = r
\end{flalign*}
%
\begin{flalign*}
  &\|\T{Cons}\LP x, a\RP\| =  \LP 0, \T{Cons} \LP x, a \RP\RP \\
\end{flalign*}
%
So the complete translation of the \T{Cons} branch is
%
\begin{flalign*}
  &1 +_c \|\lambda a.\T{force}(r)\ \T{Cons}\LP x, a\RP\| \\
  &\quad = 1 +_c \LP 0, \lambda a.\|\T{force}(r)\ \T{Cons}\LP x, a\RP\|\RP \\
  &\quad = \LP 1, \lambda a.(1 + \|\T{force}(r)\|_c + \|\T{Cons}\LP x, a\RP)\|_c) +_c \|\T{force}(r)\|_p\ \|\T{Cons}\LP x, a \RP\|_p \RP \\
  &\quad = \LP 1, \lambda a.(1 + r_c + 0) +_c r_p\ \T{Cons}\LP x, a \RP \RP \\
  &\quad = \LP 1, \lambda a.(1 + r_c) +_c r_p\ \T{Cons}\LP x, a \RP \RP \\
\end{flalign*}
%
The complete translation of the \T{rec} becomes
%
\begin{flalign*}
  &\|\T{rec}(xs, \T{Nil}\mapsto\lambda a.a, \\
  &\qquad \T{Cons}\mapsto  \LP x, \LP xs', r\RP\RP.\lambda a.\T{force}(r)\ \T{Cons}\LP x, a\RP)\| \\
  &= \T{rec}(xs, \T{Nil} \mapsto 1 +_c \|\lambda a.a\| \\
  &\quadthree \T{Cons}\mapsto \LP x, \LP xs', r\RP\RP.1 +_c \|\lambda a.\T{force}(r)\ \T{Cons}\LP x, a\RP\|) \\
  &= \T{rec}(xs, \T{Nil} \mapsto \LP 0, \lambda a. \LP 0, a \RP \RP \\
  &\quadthree \T{Cons}\mapsto \LP x, \LP xs', r\RP\RP. \LP 1, \lambda a.(1 + r_c) +_c r_p\ \T{Cons}\LP x, a \RP \RP \\
\end{flalign*}
%
We substitute the translations of \T{rec(..)} and \T{Nil} into the application.
%
\begin{flalign*}
  &\text{Let }R = \T{rec}(xs, \T{Nil} \mapsto \LP 1, \lambda a. \LP 0, a \RP \RP \\
  &\quadthree \T{Cons}\mapsto \LP x, \LP xs', r\RP\RP. \LP 1, \lambda a.(1 + r_c) +_c r_p\ \T{Cons}\LP x, a \RP \RP \\
  &\|\T{rec}(xs, \T{Nil}\mapsto\lambda a.a, \\
  &\qquad \T{Cons}\mapsto \LP x, \LP xs',r \RP\RP.\lambda a.\T{force}(r)\ \T{Cons}\LP x, a \RP)\ \T{Nil}\| \\
  &\text{Substituting $R$ for the translation of \T{rec} and $\LP 0, \T{Nil}\RP$ for the translation of \T{Nil}.} \\
  &\quad = (1 + R_c) +_c R_p\ \T{Nil} \RP\\
  &\quad = 1 +_c \T{rec}(xs, \T{Nil} \mapsto \LP 1, \lambda a. \LP 0, a \RP \RP \\
  &\quadthree \T{Cons}\mapsto \LP x, \LP xs', r\RP\RP. \LP 1, \lambda a.(1 + r_c) +_c r_p\ \T{Cons}\LP x, a \RP \RP)\ \T{Nil} \\
\end{flalign*}
%
And our complete translation of \T{rev} is
%
\begin{flalign*}
  \|\T{rev}\| &= \|\lambda\T{xs.rec(xs, Nil}\mapsto\lambda\T{a.a,} \\
              &\qquad \T{Cons}\mapsto\LP x, \LP xs', r\RP\RP.\lambda a.\T{force}(r)\ \T{Cons}\LP x, a\RP)\ \T{Nil}\| \\
              &= \LP 0, \lambda xs. \|\T{rec}(xs, \T{Nil}\mapsto\lambda a.a, \\
              &\quad \T{Cons}\mapsto\LP x, \LP xs', r\RP\RP.\lambda a.\T{force}(r)\ \T{Cons}\LP x, a\RP)\ \T{Nil}\| \RP \\
              &= \LP 0, \lambda xs. 1 +_c \T{rec}(xs, \T{Nil} \mapsto \LP 1, \lambda a. \LP 0, a \RP \RP \\
              &\qquad \T{Cons}\mapsto \LP x, \LP xs', r\RP\RP. \LP 1, \lambda a.(1 + r_c) +_c r_p\ \T{Cons}\LP x, a \RP \RP)\ \T{Nil} \RP\\
\end{flalign*}
%
This is the same as the translation of \T{rev} without the syntactic sugar. We
will use the syntactic sugar for the rest of this thesis.
%
% END FAST REVERSE SYNTACTIC SUGAR TRANSLATION
%
%
% ============= FAST REVERSE INTERPRETATION ====================
%
\subsection{Interpretation}
%
The interpretation of \T{rev} is not interesting as the cost of \T{rev} is always null.
Instead of interpreting \T{rev}, we will interpret \T{rev} applied to a list \T{xs}.
Below is the translation of \T{rev xs}.
%
\begin{flalign*}
  &\|\T{rev xs}\| = (1 + \|\T{rev}\|_c + \|xs\|_c) +_c \|\T{rev}\|_p\ \|xs\|_p \\
  &\text{The cost of $\|\T{rev}\|$ is $0$, and we will let \T{xs} be a value, which has $0$ cost.} \\
  &= (1 + 0 + 0) +_c \|\T{rev}\|_p\ xs \\
  &= 1 +_c (\lambda xs. 1 +_c \T{rec}(xs, \T{Nil} \mapsto \LP 1, \lambda a. \LP 0, a \RP \RP \\
  &\qquad \T{Cons}\mapsto \LP x, \LP xs', r\RP\RP. \LP 1, \lambda a.(1 + r_c) +_c r_p\ \T{Cons}\LP x, a \RP \RP)\ \T{Nil})\ xs\\
\end{flalign*}
%
The cost of \T{rev} is driven by the auxilary function \T{rec(...)}. The cost
of \T{rev} will be determined by the cost of the auxilary function \T{rec(...)}
applied to \T{Nil} plus some constant factor. We will interpret the auxilary
function in the following denotational semantics. We intepret the size of an
\T{list} to be the number of list constructors.
%
\begin{flalign*}
  \llbracket \T{list} \rrbracket &= \mathbb{N}^\infty\\
  D^{list} &= \{\ast\} + \{1\} \times \mathbb{N}^\infty\\
  size_{list}(\T{Nil}) &= 1\\
  size_{list}(\T{Cons(1,n)}) &= 1 + n\\
\end{flalign*}
%
We define the macro $R(xs)$ as the translation of the auxilary function
\T{rec(...)} to avoid repeated coping of the translation.
%
\begin{flalign*}
  &\text{Let } R(xs) = \T{rec}(xs, \T{Nil} \mapsto \LP 1, \lambda a. \LP 0, a \RP \RP \\
  &\quadfive \T{Cons}\mapsto b. \LP 1, \lambda a.(1 + \pi_1\pi_1 b_c) +_c \pi_1\pi_1 b_p\ \T{Cons}\LP \pi_0 b, a \RP \RP \\
\end{flalign*}
%
The recurrence $g(n)$ is the interpretation of the auxilary function $R(xs)$,
where $n$ is the interpretation of $xs$.
%
\begin{flalign*}
  &g(n) = \llbracket R(xs) \rrbracket \{xs \mapsto n\} \\
  &= \bigvee\limits_{size\ z \leq \llbracket xs \rrbracket \{xs \mapsto n\}} case(z, f_C, f_N) \\
  &\text{where} \\
  &f_{Nil}(x) = \llbracket \LP 1, \lambda a. \LP 0, a\RP\RP \rrbracket \{xs \mapsto n\} \\
  &\qquad = (1, \llambda a.(0, a)) \\
  &f_{Cons}(b) = \llbracket \LP 1, \lambda a.(1 + \pi_1\pi_1 b_c) +_c \pi_1\pi_1 b_p\ \T{Cons}\LP \pi_0 b, a \RP \RP \rrbracket \\
  &\quadfive \{xs \mapsto n, b \mapsto map^{\Phi_{Cons}}(\llambda d.(d, \llbracket R(w) \rrbracket \{ w \mapsto d, xs \mapsto n \}), b) \} \\
\end{flalign*}
%
Let us take a moment to analyze the semantic $map$. The definition mirrors the
definition of the \T{map} macro in the complexity language. Since $b$ is a
tuple, $map$ over a tuple is defined as the tuple of the $map$ over the
projections of the
tuple.
%
\begin{flalign*}
  &\qquad map^{\Phi_{Cons}}(\llambda d.(d, \llbracket R(w) \rrbracket \{ w \mapsto d \}), b) \\
  &\qquad = (map^{int}(\llambda d.(d, \llbracket R(w) \rrbracket \{w \mapsto d\}), \pi_0 b), \\
  &\quadfour map^{list}(\llambda d.(d, \llbracket R(w) \rrbracket \{w \mapsto d\}), \pi_1 b)) \\
  %
  &\text{The definition of $map$ over $int$ is $map^{int}(\llambda x.V_0, V_1) = V_1$.}\\
  &\qquad = (\pi_0 b, map^{list}(\llambda d.(d, \llbracket R(w) \rrbracket \{w \mapsto d\}), \pi_1 b)) \\
  %
  &\text{The definition of $map$ over a recursive occurence of a }\\
  &\text{a datatype is $map^T(\llambda x.V_0,V_1) = V_0[V_1/x]$.} \\
  &\qquad = (\pi_0 b, (\pi_1 b, \llbracket R(w) \rrbracket \{w \mapsto \pi_1 b\})) \\
  %
  &\text{Observe that we can substitute $g(\pi_1 b)$ for $\llbracket R(w) \rrbracket \{w \mapsto \pi_1 b\}$.} \\
  &\qquad = (\pi_0 b, (\pi_1 b, g(\pi_1 b))) \\
\end{flalign*}
%
Let us resume our interpretation of \T{rec(...)}.
%
\begin{flalign*}
  &f_{Cons}(b) = \llbracket \LP 1, \lambda a.(1 + \pi_1\pi_1 b_c) +_c \pi_1\pi_1 b_p\ \T{Cons}\LP \pi_0 b, a \RP \RP \rrbracket \\
  &\quadfive \{xs \mapsto n, b \mapsto map^{\Phi_{Cons}}(\llambda d.(d, \llbracket R(w) \rrbracket \{ w \mapsto d\}), b) \} \\
  &\quad = \llbracket \LP 1, \lambda a.(1 + \pi_1\pi_1 b_c) +_c \pi_1\pi_1 b_p\ \T{Cons}\LP \pi_0 b, a \RP \RP \rrbracket \\
  &\quadfive \{xs \mapsto n, b \mapsto (\pi_0 b, (\pi_1 b, g(\pi_1 b))) \} \\
  &\quad = (1, \llbracket \lambda a.(1 + \pi_1\pi_1 b_c) +_c \pi_1\pi_1 b_p\ \T{Cons}\LP \pi_0 b, a \RP \rrbracket \\
  &\quadfive \{xs \mapsto n, b \mapsto (\pi_0 b, (\pi_1 b, g(\pi_1 b))) \}) \\
  &\quad = (1, \llambda a. \llbracket (1 + \pi_1\pi_1 b_c) +_c \pi_1\pi_1 b_p\ \T{Cons}\LP \pi_0 b, a \RP \rrbracket \\
  &\quadfive \{xs \mapsto n, b \mapsto (\pi_0 b, (\pi_1 b, g(\pi_1 b))), a \mapsto a \}) \\
  &\quad = (1, \llambda a. (1 + g_c(\pi_1 b)) \pplus_c g_p(\pi_1 b)\ (1 + a))\\
\end{flalign*}
%
So the initial extracted recurrence from \T{rec} is
%
\begin{flalign*}
  &g(n) = \bigvee\limits_{size\ z \leq n} case(z, f_C, f_N) \\
  &\text{where} \\
  &f_{Nil}(x) = (1, \llambda a.(0, a)) \\
  &f_{Cons}(b) = (1, \llambda a. (1 + g_c(\pi_1 b)) \pplus_c g_p(\pi_1 b)\ (a + 1))\\
\end{flalign*}
%
To obtain a closed form solution for the recurrence, we must eliminate the big
maximum operator. To do so we break the defnition of $g$ into two cases.
%
\begin{description}
  \item[case $n=0$]\hfill \\
    For $n=0$, $g(0) = (1,\llambda a.(0,a))$.
  \item[case $n>0$]\hfill \\
    \begin{align*}
      g(n+1) &= \bigvee_{size\ ys \leq n+1} case(ys, f_{Nil}, f_{Cons}) \\
             &= \bigvee_{size\ ys \leq n} case(ys, f_{Nil}, f_{Cons}) \vee \bigvee_{size\ ys = n+1} case(ys, f_{Nil}, f_{Cons}) \\
             &= g(n) \vee \bigvee_{size\ ys = n+1} case(ys, \llambda().(1,\llambda a.( 0,a)), \\
             &\quadten \llambda (1,m).(1, \llambda a.(1 + g_c(m)) \pplus_c g_p(m) (a+1))) \\
             &= g(n) \vee \LP 1, \llambda a. (1 + g_c(n)) \pplus_c g_p(n) (a+1))) \\
    \end{align*}
\end{description}
%
In order to eliminate the remaining max operator, we want to show that $g$ is
monotonically increasing; $\forall n.g(n) \leq g(n+1)$.
By definition of $\leq$,
$g(n) \leq g(n+1) \Leftrightarrow g_c(n) \leq g_c(n+1) \land g_p(n) \leq g_p(n+1)$.
First we will show lemma \ref{lem:fr_interp_g_cost_one}, which states the cost
of $g(n)$ is always one.
%
\begin{lemma}
  \label{lem:fr_interp_g_cost_one}
  $\forall n. g_c(n) = 1$.
\end{lemma}
%
\begin{proof}
We prove this by induction on $n$.
  \begin{description}
    \item[Base case: $n=0$]\hfill \\
      By definition, $g_c(0) = (1, \llambda a.(0,a)) = 1$.
    \item[Induction step: $n>0$]\hfill \\
      By definition $g_c(n+1) = (g(n) \vee (1, \llambda a. (1 + g_c(n)) \pplus_c g_p(n)\ (a+1)))_c$.
      We distribute the projection over the max: $g_c(n+1) = g_c(n) \vee 1$.
      By the induction hypothesis, $g_c(n) = 1$, so $g_c(n+1) = 1$.
  \end{description}
\end{proof}
%
The immediate corollary of this is $g_c(n)$ is monotonically increasing.
%
\begin{corollary}
  \label{lem:fr_interp_g_cost_monotonically_increasing}
  $\forall n. g_c(n) \leq g_c(n+1)$.
\end{corollary}
%
First we prove the lemma stating the potential of $g(n)\ a$ is monotonically
increasing.
%
\begin{lemma}
  \label{lem:fr_interp_g_potential_monotonically_increasing}
  $\forall n.g_p(n)\ a \leq g_p(n)\ (a+1)$
\end{lemma}
%
\begin{proof}
  We prove this by induction on $n$.
  \begin{description}
    \item[$n=0$]\hfill \\
      $g_p(0)\ a = (\llambda a.(0,a))\ a = (0,a)$\\
      $g_p(0)\ (a+1) = (\llambda a.(0.a))\ (a+1) = (0,a+1)$\\
      $(0,a) \leq (0,a+1)$.
    \item[$n>0$]\hfill \\
      We assume $g_p(n)\ a \leq g_p(n)\ (a+1)$.
      \begin{align*}
      g_p(n)\ a &\leq g_p(n)\ (a+1)  \\
      g_p(n)\ a \vee (1 + g_c(n)) \pplus_c g_p(n)\ a &\leq g_p(n)\ (a+1) \vee (1 + g_c(n)) \pplus_c g_p(n)\ (a+1) \\
      g_p(n+1)\ a &\leq g_p(n+1)\ (a+1)
      \end{align*}
  \end{description}
\end{proof}
%
Now we show $g_p(n) \leq g_p(n+1)$.
%
\begin{proof}
  By reflexivity, $g_p(n) \leq g_p(n)$.
  By the lemma we just proved:
  \begin{align*}
  g_p(n)\ a &\leq g_p(n)\ (a+1) \\
  g_p(n)\ a &\leq (1 + g_c(n)) \pplus_c g_p(n)\ (a+1) \\
  %\[ \pi_1 g(n) a \leq \LP 2 + \pi_0 (\pi_1 g(n) (a+1)), \pi_1 (\pi_1 g(n) (a+1)) \RP \]
  \llambda a.g_p(n)\ a &\leq \llambda a. (1 + g_c(n)) \pplus_c g_p(n)\ (a+1)
  \end{align*}
\end{proof}
%
So since for all $n$, $g_c(n) = 1$ and $g_p(n) \leq \llambda a. (1 + g_c(n)) \pplus_c g_p(n)\ (a+1)$, we conclude
%
\begin{flalign*}
  g(n) &\leq \LP 1, \llambda a. (1 + g_c(n)) \pplus_c g_p(n)\ (a+1)\RP) \\
\end{flalign*}
%
So
%
\begin{flalign*}
  g(n+1) &= \LP 1, \llambda a. (1 + g_c(n)) \pplus_c g_p(n)\ (a+1)\RP\\
\end{flalign*}
%
%
To extract a recurrence from $g$, we apply $g$ to the interpretation of a list $a$.
%
Let $h(n,a) = g_p(n)\ a$.
%
For $n=0$
%
\begin{align*}
  h(0,a) &= g_p(0) a \\
         &= (\llambda a.(0,a)) a \\
         &= (0, a)
\end{align*}
%
For $n>0$
%
\begin{align*}
  h(n,a) &= g_p(n) a \\
         &= (\llambda a. (1 + g_c(n-1)) \pplus_c g_p(n-1)\ (a+1))\ a \\
         &= (1 + g_c(n-1)) +_c g_p(n-1) (a+1) \\
         &= (1 + 1) +_c h(n-1,a+1) \\
         &= (2 + h_c(n-1,a+1), h_p(n-1,a+1))
\end{align*}
%
From this recurrence, we can extract a recurrence for the cost.
%
For $n=0$
%
\begin{align*}
h_c(0,a) &= (0, a)_c = 0
\end{align*}
%
For $n>0$
%
\begin{align*}
  h_c(n,a) &= (2 + h_c(n-1,a+1), h_p(n-1,a+1))_c = 2 + h_c(n-1,a+1)
\end{align*}
%
We now have a recurrence for the cost of the auxilary function
\T{rec(xs,$\dots$)} when applied to some list:
%
\begin{equation}
  h_c(n,a) = \begin{cases}
    0 & n = 0 \\
    2 + h_c(n-1,a+1) & n > 0
  \end{cases}
\end{equation}
%
We state the solution to the recurrence $h_c$ is $2n$.
%
\begin{theorem}
  \label{lem:fr_interp_h_cost}
  $h_c(n,a) = 2n$
\end{theorem}
%
\begin{proof}
  We prove this by induction on $n$.
  \begin{description}
    \item{case $n=0$}\hfill \\
      $h_c(0,a) = 0 = 2 \cdot 0$
    \item{case $n>0$}\hfill \\
      We assume $h_c(n,a+1) = 2n$.
      \begin{align*}
        h_c(n+1,a) &= 2 + h_c(n,a+1) \\
                   &= 2 + 2n  \\
                   &= 2(n+1)
      \end{align*}
  \end{description}
\end{proof}
%
So we have proved the interpretation of applying the auxilary function of
\T{rev xs} to a list is linear in the length of \T{xs}.



We can also extract a recurrence for the potential.
For $n=0$
%
\begin{align*}
h_p(0,a) &= h_p(0,a) \\
         &= (0, a)_p \\
         &= a
\end{align*}
%
For $n>0$
%
\begin{align*}
  h_p(n,a) &= (2 + h_c(n-1,a+1), h_p(n-1,a+1))_p \\
           &= h_p(n-1,a+1)
\end{align*}
%
We now have a recurrence for the potential of the auxilary function in
\T{rev xs} when applied to some list $a$.
%
\begin{equation}
  h_p(n,a) = \begin{cases}
    a & n = 0 \\
    h_p(n-1,a+1) & n > 0
  \end{cases}
\end{equation}
%
\begin{theorem}
  \label{lem:fr_interp_h_potential}
  $h_p(n,a) = n + a$
\end{theorem}
%
\begin{proof}
  We prove this by induction on $n$.
  \begin{description}
    \item{case $n=0$}\hfill \\
      \[ h_p(0,a) = a \]
    \item{case $n>0$}\hfill \\
      \begin{align*}
      h_p(n,a) &= h_p(n-1,a+1) \\
               &= n - 1 + a + 1  \qquad \text{by the induction hypothesis} \\
               &= n + a
      \end{align*}
  \end{description}
\end{proof}
%

% ===================================================================================================

Now that we have obtained a closed form solution for the recurrence describing
the cost and potential of the auxilary function that drives the cost of
\T{rev}, we can obtain the interpretations for the cost and potential of \T{rev xs}.
Recall the translation of \T{rev xs}.
%
\begin{flalign*}
  \|\T{rev xs}\| &= 1 +_c (\lambda xs. 1 +_c \T{rec}(xs, \T{Nil} \mapsto \LP 1, \lambda a. \LP 0, a \RP \RP \\
  &\qquad \T{Cons}\mapsto \LP x, \LP xs', r\RP\RP. \LP 1, \lambda a.(1 + r_c) +_c r_p\ \T{Cons}\LP x, a \RP \RP)\ \T{Nil})\ xs
\end{flalign*}
%
We can obtain an interpretation of $\|\T{rev xs}\|$ by substituting our
interpretation of the auxilary function.
%
\begin{flalign*}
  &\text{Let $n = \llbracket \|xs\| \rrbracket.$}\\
  \llbracket\|\T{rev xs}\|\rrbracket &= \llbracket 1 +_c (\lambda xs. 1 +_c \T{rec}(xs, \T{Nil} \mapsto \LP 1, \lambda a. \LP 0, a \RP \RP\\
  &\qquad \T{Cons}\mapsto \LP x, \LP xs', r\RP\RP. \LP 1, \lambda a.(1 + r_c) +_c r_p\ \T{Cons}\LP x, a \RP \RP)\ \T{Nil})\ xs \rrbracket \{xs \mapsto n\}  \\
  &= 1 \pplus_c \llbracket \lambda xs. 1 +_c \T{rec}(xs, \T{Nil} \mapsto \LP 1, \lambda a. \LP 0, a \RP \RP\\
  &\qquad \T{Cons}\mapsto \LP x, \LP xs', r\RP\RP. \LP 1, \lambda a.(1 + r_c) +_c r_p\ \T{Cons}\LP x, a \RP \RP)\ \T{Nil} \rrbracket \{xs \mapsto n\}\ n  \\
  &= 1 \pplus_c (\llambda xs. \llbracket 1 +_c \T{rec}(xs, \T{Nil} \mapsto \LP 1, \lambda a. \LP 0, a \RP \RP\\
  &\qquad \T{Cons}\mapsto \LP x, \LP xs', r\RP\RP. \LP 1, \lambda a.(1 + r_c) +_c r_p\ \T{Cons}\LP x, a \RP \RP)\ \T{Nil}\rrbracket \{xs \mapsto n\})\ n  \\
  &= 1 \pplus_c (\llambda xs. 1 \pplus_c \llbracket \T{rec}(xs, \T{Nil} \mapsto \LP 1, \lambda a. \LP 0, a \RP \RP\\
  &\qquad \T{Cons}\mapsto \LP x, \LP xs', r\RP\RP. \LP 1, \lambda a.(1 + r_c) +_c r_p\ \T{Cons}\LP x, a \RP \RP)\rrbracket \{xs \mapsto n\}\ 0)\ n  \\
  &= 1 \pplus_c (\llambda xs. 1 \pplus_c h(xs,0))\ n  \\
  &= 1 \pplus_c (1 \pplus_c h(n,0)) \\
  &= 1 \pplus_c (1 \pplus_c (2n,n)) \\
  &= (2 + 2n,n)
\end{flalign*}

So we see that the cost of \T{rev xs} is linear in the length of the list, and
that the potential of the result is equal to the potential of the input.

\chapter{Reverse}
%
Here we present the naive implementation of list reverse. The naive
implementation reverses a list in quadratic time as opposed to linear time.
%
\begin{align*}
  \T{datatype list} &= \T{Nil of unit}\ |\ \T{Cons of int} \times \T{list}
\end{align*}
%
The implementation walks down a list, appending the head of the list to the end
of the result of recursively calling itself on the tail of the list. We use the
syntactic sugar introduced earlier. \T{rev} uses the auxilary function
\T{snoc}.  \T{snoc} appends an item to the end of a list.
%
\begin{flalign*}
  \T{snoc} &= \lambda xs.\lambda x.\T{rec}(xs, \T{Nil} \mapsto \T{Cons} \langle x, \T{Nil} \rangle, \\
           &\quadsix \T{Cons}  \mapsto \langle y, \langle ys,r \rangle \rangle . \T{Cons} \langle y,\T{force}(r) \rangle)
\end{flalign*}
%
The quadratic time implementation of reverse recurses on the list, appending
the head of the list to the recursively reversed tail of the list.
%
\begin{flalign*}
  rev &= \lambda xs.\T{rec}(xs, \T{Nil} \mapsto \T{Nil}, \\
      &\quadfive \T{Cons} \mapsto \langle x, \langle xs',r\rangle \rangle. \T{snoc}\ \T{force}(r)\ x)
\end{flalign*}
%
%
\section{Translation}
%
% BEGIN SNOC TRANSLATION ========================================
%
\subsection{\T{snoc} Translation}
%
First we translate the function \T{snoc}. To do so we apply the rule for
translating an abstraction two times. Recall the rule is
$\|\lambda x.e\| = \langle 0, \lambda x.\|e\|\rangle$.
%
\begin{flalign*}
  \|\T{snoc}\| &= \|\lambda xs.\lambda x.\T{rec}(xs, \T{Nil} \mapsto \T{Cons} \langle x, \T{Nil} \rangle, \\
               &\quadsix \T{Cons}  \mapsto \langle y, \langle ys,r \rangle \rangle . \T{Cons} \langle y,\T{force}(r) \rangle) \| \\
               &= \langle 0, \lambda xs.\|\lambda x.\T{rec}(xs, \T{Nil} \mapsto \T{Cons} \langle x, \T{Nil} \rangle, \\
               &\quadsix \T{Cons}  \mapsto \langle y, \langle ys,r \rangle \rangle . \T{Cons} \langle y,\T{force}(r) \rangle)\|\rangle \\
               &= \langle 0, \lambda xs.\langle 0, \lambda x.\|\T{rec}(xs, \T{Nil} \mapsto \T{Cons} \langle x, \T{Nil} \rangle, \\
               &\quadsix \T{Cons}  \mapsto \langle y, \langle ys,r \rangle \rangle . \T{Cons} \langle y,\T{force}(r) \rangle)\|\rangle\rangle \\
               %
               &\text{Next we apply the rule for translating a \T{rec}.} \\
               %
               &= \langle 0, \lambda xs.\langle 0, \lambda x.\|xs\|_c +_c \T{rec}(\|xs\|_p, \T{Nil} \mapsto 1 +_c \|\T{Cons} \langle x, \T{Nil} \rangle\|, \\
               &\quadsix \T{Cons}  \mapsto \langle y, \langle ys,r \rangle \rangle . 1 +_c \|\T{Cons} \langle y,\T{force}(r) \rangle\|)\rangle\rangle \\
               %
               &\text{$xs$ is a variable, so its translation is $\langle 0,xs \rangle$.} \\
               %
               &= \langle 0, \lambda xs.\langle 0, \lambda x.\langle 0,xs\rangle_c +_c \T{rec}(\langle 0,xs\rangle_p, \T{Nil} \mapsto 1 +_c \|\T{Cons} \langle x, \T{Nil} \rangle\|, \\
               &\quadsix \T{Cons}  \mapsto \langle y, \langle ys,r \rangle \rangle . 1 +_c \|\T{Cons} \langle y,\T{force}(r) \rangle\|)\rangle\rangle \\
               %
               &\text{We take the cost and potential projections of the translated term.} \\
               %
               &= \langle 0, \lambda xs.\langle 0, \lambda x.\T{rec}(xs, \T{Nil} \mapsto 1 +_c \|\T{Cons} \langle x, \T{Nil} \rangle\|, \\
               &\quadsix \T{Cons}  \mapsto \langle y, \langle ys,r \rangle \rangle . 1 +_c \|\T{Cons} \langle y,\T{force}(r) \rangle\|)\rangle\rangle \\
\end{flalign*}
%
We will translate $\T{Cons}\langle x,\T{Nil}\rangle$. In order to do so we will
first translate $\langle x,\T{Nil}\rangle$.
%
\begin{flalign*}
  \|\langle x,\T{Nil}\rangle\| &= \langle \|x\|_c + \|\T{Nil}\|_c, \langle\|x\|_p,\|\T{Nil}\|_p\rangle \\
       &\text{$x$ is a variable, so its translation is $\langle 0,x\rangle$.} \\
       &\text{The translation of \T{Nil} is $\langle 0,\T{Nil}\rangle$.} \\
       &= \langle \langle 0,x\rangle_c + \langle 0,\T{Nil}\rangle_c, \langle \langle 0,x\rangle_p,\langle 0,\T{Nil}\rangle_p\rangle\rangle \\
       &= \langle 0, \langle x,\T{Nil}\rangle\rangle
\end{flalign*}
%
We use the result in translation of $\T{Cons}\langle x,\T{Nil}\rangle$.
%
\begin{flalign*}
  \quad \|\T{Cons}\langle x,\T{Nil}\rangle\| &= \langle \|\langle x,\T{Nil}\rangle\|_c, \T{Cons} \|\langle x,\T{Nil}\rangle\|_p\rangle \\
     &= \langle \langle 0, \langle x,\T{Nil}\rangle\rangle_c, \T{Cons} \langle 0, \langle x,\T{Nil}\rangle\rangle_p\rangle \\
     &= \langle 0, \T{Cons} \langle x,\T{Nil}\rangle\rangle
\end{flalign*}
%
Now that we have translated $\T{Cons}\langle x,\T{Nil}\rangle$ we return can
substitute it in to the translation of \T{snoc} to complete the translation of
the \T{Nil} branch of the \T{rec}.
%
\begin{flalign*}
   &= \langle 0, \lambda xs.\langle 0, \lambda x.\T{rec}(xs, \T{Nil} \mapsto 1 +_c \langle 0, \T{Cons} \langle x,\T{Nil}\rangle\rangle \\
   &\quadsix \T{Cons}  \mapsto \langle y, \langle ys,r \rangle \rangle . 1 +_c \|\T{Cons} \langle y,\T{force}(r) \rangle\|)\rangle\rangle \\
   &\text{we can expand the $+_c$ macro to simplify the \T{Nil} branch.} \\
   &= \langle 0, \lambda xs.\langle 0, \lambda x.\T{rec}(xs, \T{Nil} \mapsto \langle 1, \T{Cons} \langle x,\T{Nil}\rangle\rangle \\
   &\quadsix \T{Cons}  \mapsto \langle y, \langle ys,r \rangle \rangle . 1 +_c \|\T{Cons} \langle y,\T{force}(r) \rangle\|)\rangle\rangle \\
\end{flalign*}
%
To complete the translation of \T{snoc} we must translate
$\T{Cons} \langle y,\T{force}(r) \rangle$. To do so we first translate
$\langle y,\T{force}(r)\rangle$.
%
\begin{flalign*}
  \|\langle y,\T{force}(r)\rangle\| &= \langle \|y\|_c + \|\T{force}(r)\|_c, \langle \|y\|_p,\|\T{force}(r)\|_p\rangle\rangle \\
                                    &\text{$y$ is a variable, so } \\
                                    &\quad \|y\| = \langle 0,y\rangle \\
                                    &\quad \T{force}(r) = \|r\|_c +_c \|r\|_p \\
                                    &\quad\text{$r$ is also a variable.} \\
                                    &\quad\quadthree = \langle 0,r\rangle_c +_c \langle 0,r\rangle_p \\
                                    &\quad\quadthree = 0 +_c r = r \\
                                    &= \langle \langle 0,y\rangle_c + r_c,\langle \langle 0,y\rangle_p,r_p\rangle\rangle \\
                                    &= \langle r_c,\langle y,r_p\rangle\rangle
\end{flalign*}
%
We use this in our translation of $\T{Cons} \langle y,\T{force}(r) \rangle$.
%
\begin{flalign*}
  \T{Cons} \langle y,\T{force}(r) \rangle &= \langle \|\langle y,\T{force}(r)\rangle\|_c, \T{Cons} \|\langle y,\T{force}(r)\rangle\|_p\rangle \\
                                          &= \langle \langle r_c,\langle y,r_p\rangle\rangle_c, \langle r_c,\langle y,r_p\rangle\rangle_p\rangle \\
                                          &= \langle r_c, \T{Cons} \langle y,r_p\rangle\rangle
\end{flalign*}
%
We substitute this result into our translation of \T{rev}
\begin{flalign*}
  \|\T{snoc}\| &= \langle 0, \lambda xs.\langle 0, \lambda x.\T{rec}(xs, \T{Nil} \mapsto \langle 1,\T{Cons} \langle x, \T{Nil} \rangle\rangle, \\
               &\quadsix \T{Cons}  \mapsto \langle y, \langle ys,r \rangle \rangle . 1 +_c \|\T{Cons} \langle y,\T{force}(r) \rangle\|)\rangle\rangle \\
               &= \langle 0, \lambda xs.\langle 0, \lambda x.\T{rec}(xs, \T{Nil} \mapsto \langle 1,\T{Cons} \langle x, \T{Nil} \rangle\rangle, \\
               &\quadsix \T{Cons}  \mapsto \langle y, \langle ys,r \rangle \rangle . 1 +_c \langle r_c, \T{Cons} \langle y,r_p\rangle\rangle)\rangle\rangle \\
               &= \langle 0, \lambda xs.\langle 0, \lambda x.\T{rec}(xs, \T{Nil} \mapsto \langle 1,\T{Cons} \langle x, \T{Nil} \rangle\rangle, \\
               &\quadsix \T{Cons}  \mapsto \langle y, \langle ys,r \rangle \rangle .\langle 1 + r_c, \T{Cons} \langle y,r_p\rangle\rangle)\rangle\rangle
\end{flalign*}
%
% END SNOC TRANSLATION
%
% ========================================
%
% BEGIN REV TRANSLATION
%
\subsection{\T{rev} Translation}
%
The translation into the complexity language follows
%
First we apply the abstraction translation rule:
$\|\lambda x.e\| = \langle 0, \lambda x.\|e\|\rangle$.
%
\begin{flalign*}
  \|rev\| &= \|\lambda xs.\T{rec}(xs, \T{Nil} \mapsto \T{Nil}, \\
          &\quadfive \T{Cons} \mapsto \langle x, \langle xs',r\rangle \rangle. \T{snoc}\ \T{force}(r)\ x)\\
          &= \langle 0,\lambda xs.\|\T{rec}(xs, \T{Nil} \mapsto \T{Nil}, \\
          &\quadfive \T{Cons} \mapsto \langle x, \langle xs',r\rangle \rangle. \T{snoc}\ \T{force}(r)\ x)\|\rangle\\
          &\text{Next we apply the \T{rec} translation rule.} \\
          &= \langle 0,\lambda xs.\|xs\|_c +_c \T{rec}(\|xs\|_p, \T{Nil} \mapsto 1 +_c \|\T{Nil}\|, \\
          &\quadfive \T{Cons} \mapsto \langle x, \langle xs',r\rangle \rangle. 1 +_c \|\T{snoc}\ \T{force}(r)\ x\|)\rangle\\
          &\text{As before, the translation of the variable $xs$ is $\langle 0, xs\rangle$,}\\
          &\text{and the translation of \T{Nil} is $\langle 0,\T{Nil}\rangle$.} \\
          &= \langle 0,\lambda xs.\langle 0,xs\rangle_c +_c \T{rec}(\langle 0, xs\rangle_p, \T{Nil} \mapsto 1 +_c \langle 0,\T{Nil}\rangle, \\
          &\quadfive \T{Cons} \mapsto \langle x, \langle xs',r\rangle \rangle. 1 +_c \|\T{snoc}\ xs\ x\|)\rangle\\
          &\text{We take the projections of the translated expressions and expand the $+_c$ macro.} \\
          &= \langle 0,\lambda xs.\T{rec}(xs, \T{Nil} \mapsto \langle 1,\T{Nil}\rangle, \\
          &\quadfive \T{Cons} \mapsto \langle x, \langle xs',r\rangle \rangle. 1 +_c \|\T{snoc}\ xs\ x\|)\rangle\\
          %
          &\text{Next we translate the application \T{snoc force(r) x}.}\\
          &\quad \|\T{snoc}\ \T{force}(r)\ x\| = (1 + \|\T{snoc force}(r)\|_c + \|x\|_c) +_c \|\T{snoc}\ \T{force}(r)\|_p \|x\|_p \\
          &\quad \|\T{snoc}\ \T{force}(r)\| = (1 + \|\T{snoc}\|_c + \|\T{force}(r)\|_c) +_c \|\T{snoc}\|_p \|\T{force}(r)\|_p \\
          %
          &\quad\text{Next we translate the \T{force}.} \\
          &\quadthree \|\T{force}(r)\| = \|r\|_c +_c \|r\|_p \\
          &\quadthree \text{$r$ is also a variable, so its translation is $\langle 0,xs\rangle$. The cost of $\|\T{snoc}\|$ is 0.} \\
          &\quadsix = \langle 0,r \rangle_c +_c \langle 0,r\rangle_p = r \\
          %
          &\quad \|\T{snoc}\ \T{force}(r)\| = (1 + 0 + r_c) +_c \|\T{snoc}\|_p\ r_p \\
          &\quad\text{$x$ is a variable so its translation is $\langle 0,x\rangle$.} \\
          &\quad \|\T{snoc force}(r)\ x\| = (1 + \|\T{snoc force}(r)\|_c + \|x\|_c) +_c \|\T{snoc}\ r\|_p \|x\|_pi \\
          &\quadfour = (1 + 1 + r_c + (\|\T{snoc}\|_p\ r_p)_c) +_c (\|\T{snoc}\|_p\ r_p)_p\ x \\
          &\quad\text{The cost of the partially applied function is 0.} \\
          &\quadfour = (2 + r_c) +_c (\|\T{snoc}\|_p\ r_p)_p\ x \\
          %
          &\text{We can use this to complete the translation of the \T{Cons} branch.}\\
          &= \langle 0,\lambda xs.\T{rec}(xs, \T{Nil} \mapsto \langle 1,\T{Nil}\rangle, \\
          &\quadfive \T{Cons} \mapsto \langle x, \langle xs',r\rangle \rangle. 1 +_c ((2 + r_c) +_c (\|\T{snoc}\|_p\ r_p)_p\ x)\rangle\\
          &= \langle 0,\lambda xs.\T{rec}(xs, \T{Nil} \mapsto \langle 1,\T{Nil}\rangle, \\
          &\quadfive \T{Cons} \mapsto \langle x, \langle xs',r\rangle \rangle. (3 + r_c) +_c (\|\T{snoc}\|_p\ r_p)_p\ x)\rangle\\
\end{flalign*}
%
%
It is more interesting if we consider the translation of \T{rev} applied to
some list \T{xs}. The translation of this function into the complexity
language proceeds as follows. First we apply the rule for translating an
application.
%
\begin{flalign*}\
  \|\T{rev}\ xs\| &= (1 + \|\T{rev}\|_c + \|xs\|_c) +_c \|\T{rev}\|_p \|xs\|_p \\
                  &= (1 + \|xs\|_c) +_c \|\T{rev}\|_p\ \|xs\|_p \\
                  &= (1 + \|xs\|_c) +_c (\lambda xs.\T{rec}(xs, \T{Nil} \mapsto \langle 1,\T{Nil}\rangle, \\
                  &\quadfive \T{Cons} \mapsto \langle x, \langle xs',r\rangle \rangle. (3 + r_c) +_c (\|\T{snoc}\|_p\ r_p)_p\ x))\ \|xs\|_p
\end{flalign*}
%
%
%
\section{Interpretation}
%
We intepret the size of an \T{list} to be the number of \T{Cons} constructors.
%
\begin{align*}
  \llbracket \T{list} \rrbracket &= \mathbb{N}^\infty \\
  D^{list} &= \{\ast\} + \{1\} \times \mathbb{N}^\infty \\
  size_{list}(\T{Nil}) &= 0 \\
  size_{list}(\T{Cons(1,n)}) &= 1 + n
\end{align*}
%
%
\subsection{\T{snoc} Interpretation}
%
We interpret $\|\T{snoc xs x}\|$. Recall the translation.
%
\begin{flalign*}
  \T{snoc}\ xs\ x &= (2 + \|xs\|_c + \|x\|_c) +_c (\T{snoc}_p\ \|xs\|_p)_p\ \|x\|_p
\end{flalign*}
%
The cost of \T{snoc} is driven by the recursion. We interpret the cost of the
\T{rec} by defining a recurrence $g(n)$. We add $x \mapsto x$ to the
environment, where $x$ is the interpretation of $x$.
%
\begin{flalign*}
  g(n) &= \llbracket \T{rec}(xs, \T{Nil} \mapsto \langle 1,\T{Cons} \langle x, \T{Nil} \rangle\rangle, \\
       &\quadsix \T{Cons}  \mapsto \langle y, \langle ys,r \rangle \rangle .\langle 1 + r_c, \T{Cons} \langle y,r_p\rangle\rangle)\rrbracket \{xs \mapsto n, x \mapsto x\}\\
       &= \bigvee\limits_{size\ z \leq n} case(z, f_{Nil}, f_{Cons}) \\
  \text{where}& \\
  f_{Nil}(\ast) &= \llbracket \langle 1,\T{Cons} \langle x, \T{Nil} \rangle\rangle \rrbracket \{xs \mapsto n, x \mapsto x\}\\
             &= (1, 1) \\
  f_{Cons}(1, m) &= \llbracket \langle 1 + r_c, \T{Cons} \langle y,r_p\rangle\rangle \rrbracket \{xs \mapsto n, x \mapsto x, y \mapsto 1, ys \mapsto m, r \mapsto g(m) \} \\
                 &= (1 + g_c(m), 1 + g_p(m))
\end{flalign*}
%
To eliminate the big maximum operator, we use the same technique as in fast
reverse, by splitting the big maximum into two cases: $size\ z < n$ and $size\
z = n$.
%
\begin{description}
  \item[case $n=0$]\hfill \\
    The only $z$ such that $size\ z \leq 0$ is $\ast$. So $g(0) = f_{Nil}(0) = (1,1)$.
  \item[case $n>0$]\hfill \\
    \begin{align*}
      g(n) &= \bigvee\limits_{size z < n} case(z,f_{Nil},f_{Cons}) \vee \bigvee\limits_{size\ z = n} case(z,f_{Nil},f_{Cons}) \\
           &= g(n-1) \vee (1 + g_c(n-1), 1 + g_p(n-1)) \\
           &\text{Since $\leq$ is symmetric, $g(n-1)\leq (g_c(n-1), g_p(n-1))$, and} \\
           &\text{$(g_c(n-1),g_p(n-1)) < (1 + g_c(n-1),g_p(n-1))$.}\\
           &\leq (1 + g_c(n-1), 1 + g_p(n-1))
    \end{align*}
\end{description}
%
The solution to this recurrence is given in lemma \ref{thm:rev_interp}.
%
\begin{lemma}
  $g(n) = (1 + n, 1 + n)$.
\end{lemma}
\begin{proof}
  We prove this by induction on $n$.
  \begin{description}
    \item[case $n=0$]\hfill \\
      $g(0) = (1, 1)$.
    \item[case $n>0$]\hfill \\
      \begin{align*}
        g(n) &= (1 + g_c(n-1), 1 + g_p(n-1)) \\
             &= (1 + (n,n)_c, 1 + (n,n)_p) \\
             &= (1 + n, 1 + n)
      \end{align*}
  \end{description}
\end{proof}
%
This is a closed form solution for the recurrence describing the complexity of
the \T{rec} expression in the body of \T{snoc}.
%
\subsection{\T{rev} Interpretation}
%
Recall the translation of \T{rev xs}.
\begin{flalign*}
  \T{rev}\ xs &= (1 + \|xs\|_c) +_c (\lambda xs.\T{rec}(xs, \T{Nil} \mapsto \langle 1,\T{Nil}\rangle, \\
              &\quadfive \T{Cons} \mapsto \langle x, \langle xs',r\rangle \rangle. (3 + r_c) +_c (\|\T{snoc}\|_p\ r_p)_p\ x))\ \|xs\|_p
\end{flalign*}
%
%==============================================================================
%
Then $\llbracket \| \T{rev xs} \|_c \rrbracket = 1 + g(\|xs\|_p)$, where
\[g(n) = \llbracket rec(z, Nil \mapsto 1, Cons \mapsto \langle x, \langle xs',r\rangle \rangle.1 + r_c + h(r_p))\rrbracket \{z \mapsto n\}\]
\[h(n) = \llbracket rec(z, Nil \mapsto 1, Cons \mapsto \langle y, \langle ys',r\rangle \rangle.1 + r_c \rrbracket \{z \mapsto n\}\]
%
We calculate that $h(0)=1$ and for $n > 0$, $h(n) = 1 + h(n-1)$.
$g(0) = 1$ and for $n > 0$, $g(n) = 1 + g(n-1) + h(n-1)$
%

\section{Parametric Insertion Sort}
%
Parametric insertion sort is a higher order algorithm which sorts a list using
a comparison function which is passed to it as an argument.  The running time
of insertion sort is $\mathcal{O}(n^2)$.  This characterization of the
complexity of parametric insertion sort does not capture role of the comparison
function in the running time.  When sorting a list of integers, where
comparison between any two integers takes constant time, this does not matter.
However, when sorting a list of strings, where the complexity of comparison is
order the length of the string, the length of the strings may influence the
running time more than the length of the list when sorting small lists of large
strings.

We use the familiar \T{list} datatype.
%
\begin{flalign*}
  \T{data list} &= \T{Nil of unit | Cons of int $\times$ list}
\end{flalign*}
%
The function \T{sort} relies on the function \T{insert}. \T{insert} inserts an
element into a sorted list.
%
\begin{flalign*}
  \T{insert} &= \lambda f.\lambda x.\lambda xs.\T{rec}(xs, \T{Nil} \mapsto \T{Cons} \LP x, \T{Nil}\RP,\\
             &\quadeight \T{Cons}\mapsto \LP y, \LP ys,r \RP\RP.\T{rec}(f\ x\ y, \T{True}\mapsto \T{Cons}\LP x,\T{Cons}\LP y,ys \RP\RP, \\
             &\quadten\quadten\quad \T{False}\mapsto \T{Cons}\LP y,\T{force}(r)\RP))
\end{flalign*}
%
The \T{sort} function recurses on the list, using the \T{insert} function to
insert the head of the list into the recursively sorted tail of the list.
%
\begin{flalign*}
  \T{sort} &= \lambda f.\lambda xs.\T{rec}(xs, \T{Nil} \mapsto \T{Nil}, \T{Cons} \mapsto \LP y,\LP ys,r \RP\RP.\T{insert}\ f\ y\ \T{force}(r))
\end{flalign*}
%
\subsection{Translation of \T{insert}}
%
We walk through the translation of \T{insert}. We will translate from the
bottom up. We translate the \T{True} and \T{False} branches of the inner
\T{rec}, then we translate the inner \T{rec}, then we translate the \T{Nil} and
\T{Cons} branches of the outer \T{rec}, and finally complete the translation of
\T{insert}.
%
% BEGIN INSERT TRANSLATION
%
%
% BEGIN TRUE BRANCH
%
The translation of the \T{True} branch of the inner \T{rec} is given below.
The translation of a datatype is the cost of translating its argument, and
complexity language constructor applied to the potential of the translated
argument.
%
\begin{flalign*}
  \|\T{Cons}\LP x,\T{Cons}\LP y, rs \RP\RP\| &= \LP \|\LP x, \T{Cons} \LP y,ys\RP\RP\|_c, \T{Cons}\|\LP x, \T{Cons}\LP y,ys\RP\|_p\RP
\end{flalign*}
%
The argument to the \T{Cons} constructor is a tuple. The cost of the
translation of a tuple is the cost of the translation of each element and the
potential is the tuple of the potentials of the translations of each element.
%
\begin{flalign*}
  \|\LP x, \T{Cons}\LP y,ys\RP\RP\| &= \LP \|x\|_c + \|\T{Cons}\LP y,ys\RP\|_c, \LP \|x\|_p,\T{Cons}\LP y,ys\RP\|_p\RP\RP
\end{flalign*}
%
The first element of the tuple is a variable, but the second element is another
\T{list}. So we translate the second element first. To do so we apply the rule
for translating a datatype.
%
\begin{flalign*}
  \|\T{Cons}\LP y,ys\RP\| &= \LP \|y,ys\|_c, \T{Cons}\|y,ys\|_p\RP
\end{flalign*}
%
The argument to the constructor is a tuple. We apply the rule for translating a
tuple again. Both element of the tuple are variables, so their translated cost
is $0$ and their translated potential is their corresponding variable in the
complexity language.
%
\begin{flalign*}
  \|\LP y,ys\RP\| &= \LP \|y\|_c + \|rs\|_c, \LP\|y\|_p,\|rs\|_p\RP\RP \\
                         &= \LP \LP 0,y\RP_c + \LP 0, rs\RP_c, \LP \LP0,y\RP_p, \LP 0,rs\RP\RP\RP \\
                         &= \LP 0, \LP y,ys\RP\RP
\end{flalign*}
%
We use this to complete the translation of $\T{Cons}\LP y,ys\RP$.
%
\begin{flalign*}
  \|\T{Cons}\LP y,ys\RP\| &= \LP \|y,ys\|_c, \T{Cons}\|y,ys\|_p\RP \\
                                  &= \LP 0, \T{Cons}\LP y,ys\RP\RP
\end{flalign*}
%
We use this result to complete the translation of $\LP x,\T{Cons}\LP y,ys\RP\RP$.
%
\begin{flalign*}
  \|\LP x, \T{Cons}\LP y,ys\RP\RP\| &= \LP \|x\|_c + \|\T{Cons}\LP y,ys\RP\|_c, \LP \|x\|_p,\T{Cons}\LP y,ys\RP\|_p\RP\RP \\
                                                    &= \LP 0, \LP x,\T{Cons}\LP y,ys\RP\RP\RP
\end{flalign*}
%
And finally we use this to complete the translation of $\T{Cons}\LP x,\T{Cons}\LP y,ys\RP\RP$.
%
\begin{flalign*}
  \|\T{Cons}\LP x,\T{Cons}\LP y,ys \RP\RP\| &= \LP \|\LP x, \T{Cons} \LP y,ys\RP\RP\|_c, \T{Cons}\|\LP x,\T{Cons}\LP y,ys\RP\RP\|_p\RP \\
                                                            &= \LP 0, \T{Cons}\LP x,\T{Cons}\LP y,ys\RP\RP\RP
\end{flalign*}
%
% END TRUE BRANCH
%
% BEGIN FALSE BRANCH
%
Next we will translate the \T{False} branch.
%
\begin{flalign*}
  \|\T{Cons}\LP y,\T{force}(r)\RP\| &= \LP \|\LP y,\T{force}(r)\RP\|_c, \T{Cons}\|\LP y,\T{force}(r)\RP\|_p\RP
\end{flalign*}
%
To complete this we must first translate the tuple. The two elements of the
tuple are $y$ and $\T{force}(r)$.  The translation of the variable $y$ is
$\LP 0, y\RP$. The translation of $\T{force}(r)$ is
$\|r\|_c +_c \|r\|_p$. Like $y$, $r$ is a variable so its translation is
$\LP 0,r\RP$. So the translation of $\T{force}(r)$ is $0 +_c r$ which
simplifies to $r$.
%
\begin{flalign*}
  \|\LP y,\T{force}(r)\RP\| &= \LP \|y\|_c + \|\T{force}(r)\|_c, \LP\|y\|_p,\|\T{force}(r)\|_p\RP\RP \\
                                    &= \LP 0 + r_c, \LP y,r_p\RP\RP \\
                                    &= \LP r_c, \LP y,r_p\RP\RP
\end{flalign*}
%
We substitute this into the translation of $\T{Cons}\LP y,\T{force}(r)\RP$.
%
\begin{flalign*}
  \|\T{Cons}\LP y,\T{force}(r)\RP\| &= \LP \|\LP y,\T{force}(r)\RP\|_c, \T{Cons}\|\LP y,\T{force}(r)\RP\|_p\RP \\
                                            &= \LP r_c, \T{Cons}\LP y,r_p\RP\RP
\end{flalign*}
%
% END FALSE BRANCH
%
% BEGIN INNER REC
%
We put together the translations of the \T{True} and \T{False} branches to
translate the inner \T{rec}.  We will need the translation of the comparison
function $f$ applied to $x$ and $y$.  To translate $f\ x\ y$ we apply the
function application rule twice. First we apply the rule to $(f\ x)\ y$. Then
we apply the rule to $f\ x$. Then we expand the $+_c$ macro to simplify the
result.
%
\begin{flalign*}
\numberthis \label{eq:fxy}
\|f\ x\ y\| &= (1 + \|f\ x\|_c + \|y\|_c) +_c \|f\ x\|_p \|y\|_p \\
            &\qquad \|f\ x\| = (1 + \|f\|_c + \|x\|_c) +_c \|f\|_p \|x\|_p \\
            &\quadfour = (1 + \|f\|_c + \|x\|_c + (\|f\|_p \|x\|_p)_c, (\|f\|_p \|x\|_p)_p) \\
            &= (1 + (1 + \|f\|_c + \|x\|_c +_c (\|f\|_p \|x\|_p)_c) + \|y\|_c) +_c (\|f\|_p \|x\|_p)_p \|y\|_p \\
            &= (2 + \|f\|_c + \|x\|_c + \|y\|_c + (\|f\|_p \|x\|_p)_c) +_c (\|f\|_p \|x\|_p)_p \|y\|_p
\end{flalign*}
%
We use the translation of \T{f x y} and the \T{True} and \T{False}
branches to construct the translation of the inner \T{rec} construct.
%
\begin{flalign*}
  \|\T{rec}&(f\ x\ y, \T{True}\mapsto\T{Cons}\LP x,\T{Cons}\LP y,ys\RP\RP, \T{False}\mapsto \T{Cons}\LP y,\T{force}(r)\RP)\| \\
           &= \|f\ x\ y\|_c +_c \T{rec}(\|f\ x\ y\|_p, \T{True}\mapsto 1 +_c \|\T{Cons}\LP x,\T{Cons}\LP y,ys\RP\RP\|, \\
           &\quadsix \T{False}\mapsto 1 +_c \|\T{Cons}\LP y,\T{force}(r)\RP\|) \\
           &= \|f\ x\ y\|_c +_c \T{rec}(\|f\ x\ y\|_p, \T{True}\mapsto 1 +_c \LP 0, \T{Cons}\LP x,\T{Cons}\LP y,ys\RP\RP\RP, \\
           &\quadsix \T{False}\mapsto 1 +_c \LP r_c, \T{Cons}\LP y,r_p\RP\RP) \\
           &= (2 + \|f\|_c + \|x\|_c + \|y\|_c + (\|f\|_p \|x\|_p)_c) \\
           &\quadthree +_c \T{rec}((\|f\|_p\ \|x\|_p)_p \|y\|_p, \T{True}\mapsto \LP 1, \T{Cons}\LP x,\T{Cons}\LP y,ys\RP\RP\RP, \\
           &\quadsix \T{False}\mapsto \LP 1 + r_c, \T{Cons}\LP y,r_p\RP\RP) \\
\end{flalign*}
%
% END INNER REC
%
% BEGIN NIL
%
Next we translate the \T{Nil} and \T{Cons} branches of the outer \T{rec} of
\T{insert}. In this branch we append the element to an empty list.
%
\begin{flalign*}
  \|\T{Cons} \LP x, \T{Nil}\RP\| &= \LP \|\LP x,\T{Nil}\RP\|_c,\T{Cons}\|\LP x,\T{Nil}\RP\|_p\RP \\
                                         &= \LP 0,\T{Cons}\LP x,\T{Nil}\RP\RP
\end{flalign*}
%
% END NIL
%
% BEGIN CONS
%
Translation of the \T{Cons} branch of the outer \T{rec} in \T{insert}.  In this
branch we recurse on a nonempty list.  We check if \T{x} is comes before the
head of the list under the ordering given by \T{f}, in which case we are done,
otherwise we recurse on the tail of the list.
%
\begin{flalign*}
  \|\T{rec}&(f\ x\ y, \T{True}\mapsto \T{Cons}\LP x,\T{Cons}\LP y,ys \RP\RP, \T{False}\mapsto \T{Cons}\LP y,\T{force}(r)\RP)\| \\
           &= \|f\ x\ y\|_c +_c \T{rec}(f\ x\ y, \T{True}\mapsto 1 +_c \|\T{Cons}\LP x,\T{Cons}\LP y,ys \RP\RP\|)  \\
           &= (2 + \|f\|_c + \|x\|_c + \|y\|_c + (\|f\|_p \|x\|_p)_c) \\
           &\quadthree +_c \T{rec}((\|f\|_p\ \|x\|_p)_p \|y\|_p, \T{True}\mapsto \LP 1, \T{Cons}\LP x,\T{Cons}\LP y,ys\RP\RP\RP, \\
           &\quadsix \T{False}\mapsto \LP 1 + r_c, \T{Cons}\LP y,r_p\RP\RP) \\
           &\text{We know that $f, x, $ and $y$ are variables, so their translations have $0$ cost.} \\
           &= (2 + (f\ x)_c) \\
           &\quadthree +_c \T{rec}(((f\ x)_p\ y)_p, \T{True}\mapsto \LP 1, \T{Cons}\LP x,\T{Cons}\LP y,ys\RP\RP\RP, \\
           &\quadsix \T{False}\mapsto \LP 1 + r_c, \T{Cons}\LP y,r_p\RP\RP)
\end{flalign*}
%
% END CONS
%
We complete the translation of the outer \T{rec} using the translated \T{Nil}
and \T{Cons}.
%
\begin{flalign*}
  \|\T{rec}&(xs, \T{Nil} \mapsto \T{Cons} \LP x, \T{Nil}\RP,\\
             &\quad \T{Cons}\mapsto \LP y, \LP ys,r \RP\RP.\T{rec}(f\ x\ y, \T{True}\mapsto \T{Cons}\LP x,\T{Cons}\LP y,ys \RP\RP, \\
             &\quadfour \T{False}\mapsto \T{Cons}\LP y,\T{force}(r)\RP))\| \\
             &= \|xs\|_c +_c \T{rec}(\|xs\|_p, \T{Nil} \mapsto 1 +_c \|\T{Cons} \LP x, \T{Nil}\RP\|,\\
             &\quad \T{Cons}\mapsto \LP y, \LP ys,r \RP\RP. 1 +_c \|\T{rec}(f\ x\ y, \T{True}\mapsto \T{Cons}\LP x,\T{Cons}\LP y,ys \RP\RP, \\
             &\quadfour \T{False}\mapsto \T{Cons}\LP y,\T{force}(r)\RP)\|)\| \\
             %
             &\text{We substitute in our translations of the branches. Also note that $xs$}\\
             &\text{is a variable, so its translation is $\LP 0,xs\RP$.} \\
             &= \T{rec}(xs, \T{Nil} \mapsto 1 +_c \LP 0,\T{Cons}\LP x,\T{Nil}\RP\RP, \\
             &\quad \T{Cons}\mapsto \LP y, \LP ys,r \RP\RP. 1 +_c ((2 + (f x)_c) \\
             &\quadthree +_c \T{rec}(((f\ x)_p\ y)_p, \T{True}\mapsto \LP 1, \T{Cons}\LP x,\T{Cons}\LP y,ys\RP\RP\RP, \\
             &\quadsix \T{False}\mapsto \LP 1 + r_c, \T{Cons}\LP y,r_p\RP\RP))) \\
             &= \T{rec}(xs, \T{Nil} \mapsto \LP 1,\T{Cons}\LP x,\T{Nil}\RP\RP, \\
             &\quad \T{Cons}\mapsto \LP y, \LP ys,r \RP\RP. (3 + (f x)_c) \\
             &\quadthree +_c \T{rec}(((f\ x)_p\ y)_p, \T{True}\mapsto \LP 1, \T{Cons}\LP x,\T{Cons}\LP y,ys\RP\RP\RP, \\
             &\quadsix \T{False}\mapsto \LP 1 + r_c, \T{Cons}\LP y,r_p\RP\RP))) \\
\end{flalign*}
%
% END OUTER REC
%
The translation of \T{insert} is just three applications of the application
rule.
%
\begin{flalign*}
  \T{insert} &= \|\lambda f.\lambda x.\lambda xs.\T{rec}(xs, \T{Nil} \mapsto \T{Cons} \LP x, \T{Nil}\RP,\\
             &\quadeight \T{Cons}\mapsto \LP y, \LP ys,r \RP\RP.\T{rec}(f\ x\ y, \T{True}\mapsto \T{Cons}\LP x,\T{Cons}\LP y,ys \RP\RP, \\
             &\quadten\quadeight \T{False}\mapsto \T{Cons}\LP y,\T{force}(r)\RP))\| \\
             &= \LP 0, \lambda f. \LP 0, \lambda x.\LP 0,\lambda xs.\|\T{rec}(xs, \T{Nil} \mapsto \T{Cons} \LP x, \T{Nil}\RP,\\
             &\quadeight \T{Cons}\mapsto \LP y, \LP ys,r \RP\RP.\T{rec}(f\ x\ y, \T{True}\mapsto \T{Cons}\LP x,\T{Cons}\LP y,ys \RP\RP, \\
             &\quadten\quadeight \T{False}\mapsto \T{Cons}\LP y,\T{force}(r)\RP))\|\RP\RP\RP \\
             &= \LP 0, \lambda f. \LP 0, \lambda x.\LP 0,\lambda xs. \T{rec}(xs, \T{Nil} \mapsto \LP 1,\T{Cons}\LP x,\T{Nil}\RP\RP, \\
             &\quad\quad \T{Cons}\mapsto \LP y, \LP ys,r \RP\RP. (3 + (f x)_c) +_c \T{rec}(((f\ x)_p\ y)_p, \\
             &\quadten\quadten \T{True}\mapsto \LP 1, \T{Cons}\LP x,\T{Cons}\LP y,ys\RP\RP\RP, \\
             &\quadten\quadten \T{False}\mapsto \LP 1 + r_c, \T{Cons}\LP y,r_p\RP\RP))) \\
\end{flalign*}
%
% END INSERT
%
We are interested in the interpretation of applying \T{insert}. So we will give
a translation of \T{insert f x xs}.
%
\begin{flalign*}
  \|\T{insert}\ f\ x\ xs\| &= (1 + \|\T{insert}\ f\ x\|_c + \|xs\|_c) +_c \|\T{insert}\ f\ x\|_p \|xs\|_p \\
     &= (1 + \|\T{insert}\ f\ x\|_c + \|xs\|_c) +_c \|\T{insert}\ f\ x\|_p  \|xs\|_p \\
     &= (2 + \|\T{insert}\ f\|_c + \|x\|_c + (\|\T{insert}\ f\|_p \|x\|_p)_c + \|xs\|_c) \\
     &\quadfour +_c \|\T{insert}\ f\|_p \|x\|_p \|xs\|_p \\
     &= (2 + (\|\T{insert}\|_p\ f)_c + \|x\|_c + \|xs\|_c) +_c \|\T{insert}\ f\|_p \|x\|_p \|xs\|_p \\
     &= (3 + \|\T{insert}\|_c + \|f\|_c+ (\|\T{insert}\|_p \|f\|_p \|x\|_p)_c + \|x\|_c + \|xs\|_c) \\
     &\quadfour +_c \|\T{insert}\|_p \|f\|_p \|x\|_p \|xs\|_p \\
     &= (3 + \|f\|_c + \|x\|_c + \|xs\|_c) +_c \|\T{insert}\|_p \|f\|_p \|x\|_p \|xs\|_p \\
     &= (3 + \|f\|_c + \|x\|_c + \|xs\|_c) +_c \T{rec}(\|xs\|_p, \\
     &\quadfour \T{Nil}\mapsto \LP 1,\T{Cons}\LP x,\T{Nil}\RP\RP  \\
     &\quadfour \T{Cons}\mapsto \LP y,\LP ys,r\RP\RP.(3 + ((\|f\|_p \|x\|_p)_p  y)_c) \\
     &\quadsix +_c \T{rec}((\|f\|_p \|x\|_p)_p  y)_p, \\
     &\quadeight \T{True}\mapsto \LP 1, \T{Cons}\LP \|x\|_p,\T{Cons}\LP y,ys\RP\RP\RP \\
     &\quadeight \T{False}\mapsto \LP 1 + r_c,\T{Cons}\LP y,r_p\RP\RP))
\end{flalign*}
%
% END INSERT APPLIED
%
% BEGIN SORT TRANSLATION
%
\subsection{Translation of \T{sort}}
%
We walk through the translation of \T{sort}. Recall the definition of \T{sort}.
%
\begin{flalign*}
  \T{sort} &= \lambda f.\lambda xs.\T{rec}(xs, \T{Nil} \mapsto \T{Nil}, \T{Cons} \mapsto \LP y,\LP ys,r \RP\RP.\T{insert}\ f\ y\ \T{force}(r))
\end{flalign*}
%
The translation of \T{sort} begins with two applications of the rule for translating abstractions.
%
\begin{flalign*}
  \|\T{sort}\| &= \| \lambda f.\lambda xs.\T{rec}(xs, \T{Nil} \mapsto \T{Nil}, \T{Cons} \mapsto \LP y,\LP ys,r \RP\RP.\T{insert}\ f\ y\ \T{force}(r))\| \\
               &= \LP 0, \lambda f.\LP 0,\lambda xs.\|\T{rec}(xs, \T{Nil} \mapsto \T{Nil}, \\
               &\quadthree \T{Cons} \mapsto \LP y,\LP ys,r \RP\RP.\T{insert}\ f\ y\ \T{force}(r))\|\RP\RP \\
               &= \LP 0, \lambda f.\LP 0,\lambda xs.\|xs\|_c +_c \T{rec}(\|xs\|_p, \T{Nil} \mapsto 1 +_c \|\T{Nil}\|, \\
               &\quadthree \T{Cons} \mapsto \LP y,\LP ys,r \RP\RP.1 +_c \|\T{insert}\ f\ y\ \T{force}(r)\|)\RP\RP \\
               &\text{As $xs$ is a variable, $\|xs\| = \LP 0,xs \RP$.} \\
               &\text{We have seen before $\|\T{Nil}\| = \LP 0,\T{Nil}\RP$.}\\
               &= \LP 0, \lambda f.\LP 0,\lambda xs.\T{rec}(xs, \T{Nil} \mapsto 1 +_c \LP 0,\T{Nil}\RP, \\
               &\quadthree \T{Cons} \mapsto \LP y,\LP ys,r \RP\RP.1 +_c \|\T{insert}\ f\ y\ \T{force}(r)\|)\RP\RP \\
               &\text{We can use our translation of \T{insert} applied to three arguments.} \\
               &= \LP 0, \lambda f.\LP 0,\lambda xs.\T{rec}(xs, \T{Nil} \mapsto 1 +_c \LP 0,\T{Nil}\RP, \\
               &\quad \T{Cons} \mapsto \LP y,\LP ys,r \RP\RP.(4 + \|f\|_c + \|y\|_c + \|force(r)\|_c) \\
               &\quadthree +_c ((\|\T{insert}\|_p \|f\|_p)_p \|y\|_p)_p \|force(r)\|_p)\RP\RP \\
               &\text{The variables $f$, $y$, and $r$ translate to $\LP 0,f\RP$, $\LP 0,y\RP$, and $\LP 0,r\RP$ respectively.}\\
               &\text{The expression $\T{force}(r)$ translates to $\|r\|_c +_c \|r\|$, which is just $r$.} \\
               &= \LP 0, \lambda f.\LP 0,\lambda xs.\T{rec}(xs, \T{Nil} \mapsto 1 +_c \LP 0,\T{Nil}\RP, \\
               &\quad \T{Cons} \mapsto \LP y,\LP ys,r \RP\RP.(4 + r_c) +_c ((\|\T{insert}\|_p f)_p y)_p r_p)\RP\RP
\end{flalign*}
%
We also give the translation of \T{sort} applied to two arguments.
%
\begin{flalign*}
  \|\T{sort}\ f\ xs\| &= (1 + \|\T{sort}\ f\|_c + \|xs\|_c) +_c (\|\T{sort}\ f\|_p)_p \|xs\|_p \\
                      &= (1 + (1 + \|\T{sort}\|_c + \|f\|_c) + \|xs\|_c) +_c (\|\T{sort}\|_p \|f\|_p)_p \|xs\|_p \\
                      &= (2 + \|f\|_c + \|xs\|_c) +_c (\|\T{sort}\|_p \|f\|_p)_p \|xs\|_p
\end{flalign*}
%
%
\subsection{Interpretation of \T{insert}}
%
We well use an interpretation of lists as a pair of their greatest element and
their length.
%
\begin{align*}
   \LB list \RB &= \mathbb{Z} \times \mathbb{N}^\infty \\
                     D^{list} &= \{\ast\} + \{\mathbb{Z}\} \times \mathbb{N}^\infty \\
            size_{list} (\ast) &= (-\infty,0) \\
  size_{list} ((i,(j,n))) &= (max\{i,j\},1 + n)
\end{align*}
%
We use the mutual ordering on pairs.  That is, $(s,n) \leq (s',n')$ if
$n \leq n'$ and $s < s'$ or $n < n'$ and $s \leq s'$.
%
First we interpret the \T{rec}, which drives of the cost of \T{insert}.  As in
the translation, we break the interpretation up to make it more manageable.  We
will write $map, \lambda$ and $+_c$ in the semantics, which stand for the
semantic equivalents of the syntactic \T{map}, $\lambda$ and $+_c$.  The
definitions of these semantic functions mirror the definitions of their
syntactic equivalents.

First we interpret the inner \T{rec} of $\|\T{insert}\|$.
%
\begin{flalign*}
  \LB \T{rec}&(((f\ x)_p\ y)_p, \T{True} \mapsto \LP 1, \T{Cons}\LP x, \T{Cons}\LP y,ys\RP\RP\RP, \T{False}\mapsto \LP 1 + r_c,\T{Cons}\LP y,r_p\RP\RP)\RB \xi \\
                    &= \bigvee\limits_{size(z) \leq ((f\ x)_p\ y)_p} case(z, f_{True}, f_{False}) \\
                    &\text{where} \\
  \xi &= \{f \mapsto f, x \mapsto x, y\mapsto j, ys\mapsto (j, m), r \mapsto r \} \\
  f_{True}(\ast) &= \LB \LP 1, \T{Cons}\LP x, \T{Cons}\LP y,ys\RP\RP\RP \RB\xi \\
                 &= (1, \LB \T{Cons}\LP x, \T{Cons}\LP y,ys\RP\RP\RB)\xi \\
                 &= (1, (max(x,j), 2 + m)) \\
  f_{False}(\ast) &= \LB \LP 1 + r_c,\T{Cons}\LP y,r_p\RP\RP\RB \xi \\
                  &= (1 + r_c, (max(j,\pi_0 r_p), 1 + \pi_1 r_p))
\end{flalign*}
%
Since there are only two $z$, we can simplify the big maximum to a maximum.
%
\begin{flalign*}
  \bigvee\limits_{size(z) \leq ((f\ x)_p\ y)_p} &case(z, f_{True}, f_{False}) \\
                                   &= (1, (max(x,j), 2 + m)) \vee (1 + r_c, (max(j,\pi_0 r_p), 1 + \pi_1 r_p))
\end{flalign*}
%
Using this, we proceed to interpret the outer recurrence.
%
\begin{flalign*}
  g(i, n) &= \LB\T{rec}(xs, \T{Nil} \mapsto \LP 1,\T{Cons}\LP x,\T{Nil}\RP\RP, \\
            &\quad \T{Cons}\mapsto \LP y, \LP ys,r \RP\RP. (3 + (f x)_c) \\
            &\quadthree +_c \T{rec}(((f\ x)_p\ y)_p, \T{True}\mapsto \LP 1, \T{Cons}\LP x,\T{Cons}\LP y,ys\RP\RP\RP, \\
            &\quadsix \T{False}\mapsto \LP 1 + r_c, \T{Cons}\LP y,r_p\RP\RP))\RB \xi \\
            &\text{where} \\
  \xi &= \{xs \mapsto (i, n), x \mapsto x\} \\
  g(i,n) &= \bigvee\limits_{size(z) \leq (i,n)} case(z, f_{Nil}, f_{Cons}) \\
         &\text{where}\\
  f_{Nil}(\ast) &= \LB \LP 1,\T{Cons}\LP x,\T{Nil}\RP\RP \RB \xi \\
                &= (1, (x, 1)) \\
  f_{Cons}(j,m) &= \LB (4 + (f\ x)_c) +_c \T{rec}(((f\ x)_p\ y)_p, \T{True}\mapsto \LP 1, \T{Cons}\LP x,\T{Cons}\LP y,ys\RP\RP\RP, \\
                &\quadthree \T{False}\mapsto \LP 1 + r_c, \T{Cons}\LP y,r_p\RP\RP)\RB \xi \{ y \mapsto j, ys \mapsto (j, m), r \mapsto g(j, m)\} \\
                &= (4 + (f\ x)_c) +_c \LB\T{rec}(((f\ x)_p\ y)_p, \T{True}\mapsto \LP 1, \T{Cons}\LP x,\T{Cons}\LP y,ys\RP\RP\RP, \\
                &\quadthree \T{False}\mapsto \LP 1 + r_c, \T{Cons}\LP y,r_p\RP\RP)\RB \xi \{ y \mapsto j, ys \mapsto (j, m), r \mapsto g(j, m)\} \\
                &= (4 + (f\ x)_c) +_c ((1, (max(x,j), 2 + m)) \\
                &\quadthree \vee (1 + g_c(j,m), (max(j,\pi_0 g_p(j,m)), 1 + \pi_1 g_p(j,m))))
\end{flalign*}
%
So the interpretation of $\|\T{insert}\|$ is
%
\begin{flalign*}
  \|\T{insert}\| &= (0, \lambda f.(0,\lambda x.(0,\lambda (i,n). g(i,n)))) \\
  \text{where}& \\
  g(i,n) &= \bigvee\limits_{size(z) \leq (i,n)} case(z, f_{Nil}, f_{Cons}) \\
         &\text{where}\\
  f_{Nil}(\ast) &= (1, (x, 1)) \\
  f_{Cons}(j,m) &= (4 + (f\ x)_c) +_c ((1, (max(x,j), 2 + m)) \\
                &\quadthree \vee (1 + g_c(j,m), (max(j,\pi_0 r_p), 1 + \pi_1 g_p(j,m))))
\end{flalign*}
%
To obtain a closed form solution, we will seperate the recurrence into a
recurrence for the cost and a recurrence for the potential, and solve those
independently. We use the substitution method to prove a closed form solution
to the cost of the \T{rec} construct of \T{insert}.
%
\begin{lemma}
  \label{lem:insert_rec_cost}
  $g_c(i,n) \leq (4 + ((f\ x)_p\ i)_c n + 1$
\end{lemma}
%
\begin{proof}
  We prove this by induction on $n$.
  Recall we use the mutual ordering on pairs.
  \begin{description}
    \item[case $n=0$]\hfill \\
      $g_c(i,n) = (1, (x, 1))_c = 1$
    \item[case $n>0$]\hfill \\
      \begin{flalign*}
        g_c(i,n) &= \bigvee_{size(z) \leq (i,n)} case(z, (f_{Nil}, f_{Cons})) \\
                 &= \bigvee\limits_{\substack{j < i, m \leq n\\ \text{ or } j \leq i, m < n}} case((j, m), (f_{Nil}, f_{Cons})) \\
                 &= \bigvee\limits_{\substack{j < i, m \leq n\\ \text{ or } j \leq i, m < n}} 4 + ((f\ x)_p\ j)_c + g_c(j, m-1)) \\
                 &\leq \bigvee\limits_{\substack{j < i, m \leq n\\ \text{ or } j \leq i, m < n}} 4 + ((f\ x)_p\ j)_c + (4 + ((f\ x)_p\ j)_c)(m-1) + 1\\
                 &\leq \bigvee\limits_{\substack{j < i, m \leq n\\ \text{ or } j \leq i, m < n}} (4 + ((f\ x)_p\ j)_c) (m-1 + 1) + 1 \\
                 &\leq \bigvee\limits_{\substack{j < i, m \leq n\\ \text{ or } j \leq i, m < n}} (4 + ((f\ x)_p\ j)_c) m + 1 \\
                 &\leq \bigvee\limits_{\substack{i < j, m \leq n\\ \text{ or } i \leq j, m < n}} (4 + ((f\ x)_p\ i)_c) n + 1 \\
        &\leq (5 + ((f\ x)_p\ i)_c) n
      \end{flalign*}
  \end{description}
\end{proof}
%
As expected, we find the cost of insert is bounded by the length of the list
and the largest element. Now we use the same method to obtain a closed form
solution for the potential of the \T{rec} construct of \T{insert}. The
potential of the resulting list is bounded by the list with maximum element
equal to max of the inserted element an the maximum element of the original
list and one more the length of the original list.
%
\begin{lemma}
  \label{lem:insert_rec_potential}
  $g_p(i,n) \leq (max\{x, i\}, n+1)$
\end{lemma}
%
\begin{proof}
  We prove this by induction on $n$.
  \begin{description}
    \item[case $n=0$]\hfill \\
      $g_p(i,n) = (1, (x, 1))_p = (x, 1)$.
    \item[case $n>0$]\hfill \\
      \begin{flalign*}
        g_p(i,n) &= \bigvee_{size(z) \leq (i,n)} case(z, f_{Nil}, f_{Cons}) \\
                 &= \bigvee\limits_{\substack{j < i, m \leq n \\\text{ or } j \leq i, m < n}} (max\{x, j, \pi_0g_p(j, m-1)\}, 2 + (m-1) \vee 1 + \pi_1g_p(j,m-1))\\
                 &\leq \bigvee\limits_{\substack{j < i, m \leq n\\ \text{ or } j \leq i, m < n}} (max\{x, j\}, 2 + (m-1))\\
                 &\leq \bigvee\limits_{\substack{j < i, m \leq n\\ \text{ or } j \leq i, m < n}} (max\{x,i\}, 1 + n)\qquad\text{Since $m\leq n$.}\\
        &\leq (max\{x,i\}, 1 + n)
      \end{flalign*}
  \end{description}
\end{proof}
%
Using lemmas \ref{lem:insert_rec_cost} and \ref{lem:insert_rec_potential}, we
can express the cost and potential of \T{insert} in terms of its arguments.
%
\begin{equation}
  \label{eq:insert_interp}
  insert\ f\ x\ xs \leq (5 + ((f\ x)_p\ i)_c n, (max\{x, i\}, n+1))
\end{equation}
%
%
\subsection{Interpretation of \T{sort}}
%
We use the same denotational semantics as in the intepretation of \T{insert}.
We interpret \T{list} as a tuple of the largest element in the list and the
number of \T{Cons} constructors. As in \T{insert}, the \T{rec} construct again
drives the cost and potential of \T{sort}. We interpret the \T{rec} construct,
manipulate the recurrence to a closed form, and use the result to interpret the
cost and potential of applying \T{sort} to a comparison function $f$ and a list
$xs$.
%
\begin{flalign*}
  g(i, n) &= \LB \T{rec}(xs, \T{Nil} \mapsto \LP 1,\T{Nil} \RP,  \\
          &\quadthree \T{Cons} \mapsto \LP y, \LP ys,r \RP\RP.(4 + r _c) +_c ((\|\T{insert}\|_p f)_p y)_p r_p) \RB \{ xs \mapsto (i, n), f\mapsto f\} \\
          &= \bigvee\limits_{size(z)\leq n} case(z, f_{Nil}, f_{Cons}) \\
  %
          &\text{where}\\
  f_{Nil} &= \LB \LP 1,\T{Nil}\RP \RB \{ xs \mapsto (i,n)\} \\
          &= (1, (-\infty,0)) \\
f_{Cons} &= \LB (4 + r _c) +_c ((\|\T{insert}\|_p f)_p y)_p r_p) \RB \xi \\
         &\text{where}\quad \xi = \{xs \mapsto (i,n), f \mapsto f, y \mapsto j, ys \mapsto (j,m), r \mapsto g(j,m)\} \\
         &= (4 + g_c(j,m)) +_c ((\LB\|\T{insert}\|_p\RB\xi\ f)_p j)_p g_p(j,m) \\
         &= (5 + g_c(j,m)) +_c (((f\ j)_p\ j)_c g_p(j,m), (max\{j,j\},g_p(j, m)+1)) \\
         &= (5 + g_c(j,m) + (f\ j)_p\ j)_c g_p(j,m), (max\{j,j\},g_p(j,m) + 1))
\end{flalign*}
%
Observe that in equation \ref{eq:sort_interp0_init}, the cost is depends on the
potential of the recursive call. Therefore we must solve the recurrence for
the potential first.
%
\begin{lemma}
  \label{lem:sort_interp_potential}
  $g_p(n) \leq (j, n)$
\end{lemma}
\begin{proof}
  We prove this by induction on $n$.
  We use equation \ref{eq:insert_interp} to determine the potential of the $insert$ function.
  \begin{description}
    \item[case $n=0$]\hfill \\
      $g_p(i,n) = (i, 0)$
    \item[case $n>0$]\hfill \\
      \begin{flalign*}
        g_p(i,n) &= (\bigvee_{size(z)\leq n} case(z,f_{Nil}, f_{Cons}))_p \\
                 &= \bigvee\limits_{\substack{j \leq i, m < n\\ \text{ or } j < i, m \leq n}} (case(z,f_{Nil},f_{Cons}))_p \\
                 &= \bigvee\limits_{\substack{j \leq i, m < n\\ \text{ or } j < i, m \leq n}} (case(z,f_{Nil},f_{Cons}))_p \vee \bigvee\limits_{\substack{j \leq i, m < n \\ \text{ or } j < i, m \leq n}} (case(z,f_{Nil},f_{Cons}))_p \\
                 &\leq (i, n) % TODO
      \end{flalign*}
  \end{description}
\end{proof}
%
As in the interpretation of \T{insert} we are left with a less than
satisfactory bound on the potential of \T{sort}. It would grievous mistake to
write a sorting function whose output was smaller than its input.  Under the
current interpretation of lists, this would mean either the length of the list
decreased or the size of the largest element in the list decreased.
Unfortunately we are stuck with an upper bound on the size of the output
because the interpretation of $case$ is a maximum over a set of smaller terms.


Using the solution to the recurrence for the potential, we can solve the
recurrence for the cost of \T{sort}.
%
\begin{lemma}
  \label{lem:sort_interp_cost}
  $\pi_0g(n) \leq (3 + \pi_0(\pi_1(f\ x)\ i)n^2 + 5n + 1$
\end{lemma}
%
\begin{proof}
  We prove this by induction on $n$.
  \begin{description}
    \item[case $n=0$] $g_c(i,n) = 1$
    \item[case $n>0$] \hfill \\
      \begin{flalign*}
        g_c(i,n) &= (\bigvee_{size(z) \leq (i,n)} case(z, f_{Nil}, f_{Cons}))_c \\
                 &= \bigvee\limits_{\substack{j < i, m \leq n\\ \text{ or } j \leq i, m < n}} 3 + g_c(j, m - 1) + (insert\ f\ j\ \pi_1g(j, m - 1))_c\\
        &\leq \bigvee\limits_{\substack{j < i, m \leq n\\ \text{ or } j \leq i, m < n}} 3 + g_c(j, m - 1) + (insert\ f\ j\ (j, m - 1))_c\\
        &\leq \bigvee\limits_{\substack{j < i, m \leq n\\ \text{ or } j \leq i, m < n}} 3 + g_c(j, m - 1) + (3 + ((f\ j)\ j))(m - 1) + 1\\
        & \text{let $c_1 = (3 + \pi_0(\pi_1(f\ j)\ j))$}\\
        &\leq \bigvee\limits_{\substack{j < i, m \leq n\\ \text{ or } j \leq i, m < n}} 3 + c_1(m-1)^2 + 5(m-1) + 1 + c_1(m - 1) + 1\\
        &\leq \bigvee\limits_{\substack{j < i, m \leq n\\ \text{ or } j \leq i, m < n}} 3 + c_1m^2 - 2c_1m +c_1 + 5m-5 + 1 + c_1m - c_1 + 1\\
        &\leq \bigvee\limits_{\substack{j < i, m \leq n\\ \text{ or } j \leq i, m < n}} c_1m^2 - c_1m + 5m + 1\\
        &\leq \bigvee\limits_{\substack{j < i, m \leq n\\ \text{ or } j \leq i, m < n}} (3 + \pi_0(\pi_1(f\ i)\ i))n^2 + 5n + 1\\
        &\leq (4 + ((f\ i)\ i)_p)_cn^2 + 5n + 1
      \end{flalign*}
  \end{description}
\end{proof}
%
As expected the cost of \T{sort} is $\mathcal{O}(n^2)$ where $n$ is the length
of the list.  It is clear from the analysis how the cost of the comparison
function determines the running time of \T{sort}.  We can see that the
comparison function is called order $n^2$ times.

\section{Insertion Sort}
%
Insertion sort is a quadratic time sorting algorithm which sorts a list by
inserting an element from an unsorted segment of a container into a sorted
segment of the container.  Although the asymptotic complexity of insertion sort
is less than the optimal $\mathcal{O}(nlog_2n)$, insertion sort does have
redeeming attributes.  Insertion sort has small constant factors, making it
more efficient on small datasets. Insertion sort may be done in-place
(\citet{Cormen2001}).  The running time of insertion sort is
$\mathcal{O}(n^2)$.
%
\begin{flalign*}
  \T{data list} &= \T{Nil of unit | Cons of int}\times\T{list}
\end{flalign*}
%
\begin{flalign*}
  \T{insert} &= \lambda x.\lambda xs.\T{rec}(xs, \T{Nil} \mapsto \T{Cons}\LP x,\T{Nil}\RP, \\
             &\quadthree \T{Cons} \mapsto \LP y,\LP ys,r \RP\RP.\T{rec}(x \leq y, \T{True} \mapsto \T{Cons} \LP x,\T{Cons}\LP y,ys\RP\RP, \\
             &\quadsix \T{False} \mapsto \T{Cons} \LP y,\T{force}(r)\RP))
\end{flalign*}
%
\subsection{Translation}

The translation of the function \T{insert} is shown below.
%
\begin{flalign*}
  \T{sort} &= \lambda xs.\T{rec}(xs,\T{Nil} \mapsto \T{Nil}, \T{Cons} \mapsto \LP y,\LP ys,r\RP\RP.\T{insert}\ y\ \T{force}(r))
\end{flalign*}
%
The translation of the function \T{sort} is shown below.
%
\begin{flalign*}
  \|\T{sort}\| &= \LP 0,\lambda xs.\T{rec}(xs, \T{Nil} \mapsto \T{Nil}, \\
               &\quadthree \T{Cons} \mapsto \LP y,\LP ys,r\RP\RP.(3 + r_c) +_c (\|\T{insert}\|_p\ y)_p r_p)
\end{flalign*}
%

\subsection{Interpretation}

We will interpret lists as their lengths.
%
\begin{align*}
    \LB list \RB &= \mathbb{N}^\infty \\
    D^{list} &= \{\ast\} + \{1\} \times \mathbb{N}^\infty \\
    size_{list} (\ast) &= 0 \\
    size_{list} ((1,n)) &= 1 + n
\end{align*}
%
The interpretation of the \T{rec} that drives the cost of insert is given below.
%
\begin{align}
  \label{eq:insert_initial_recurrence}
  g(n) &= \bigvee_{size(z) \leq n} case(z, (f_{Nil},f_{Cons})) \\
  &\text{where} \\
  f_{Nil}() &= (1,1) \\
  f_{Cons}((1,m)) &= (3 + (1 \leq\ 1)_c + g_c(m), (2+m) \vee (1+g_p(m)))
\end{align}
%
We can extract the recurrence for the cost and eliminate the big maximum operator.
\begin{equation*}
\label{eq:insert_cost}
c(n) = \begin{cases}
  1 & n = 0 \\
  4 + \pi_0(\pi_1(f\ 1)\ 1) + c(n-1) & n > 0
\end{cases}
\end{equation*}
%
The closed form solution to this recurrence is:
\begin{lemma}
\label{lem:insert_cost}
  $c(n) = (3 + (1 \leq 1)_c)n + 1$
\end{lemma}
%
We also extract a recurrence for the potential and eliminate the big maximum
operator.
%
\begin{equation*}
  \label{eq:insert_potential}
  p(n) = \begin{cases}
    1 & n = 0 \\
    1 + p(n-1) & n > 0
  \end{cases}
\end{equation*}
%
The closed form solution to this recurrence is:
\begin{equation*}
  p(n) = 1 + n
\end{equation*}
%
The interpretation of \T{sort} is:
%
\begin{equation}
  \label{eq:sort_interp0_init}
  g(n) = \bigvee_{size(z)\leq n} case(z,(\lambda(\LP\RP).(1,0),\lambda(1,m).3 + g_c(m)) +_c \LB\|\T{insert}\|\RB\ 1\ g_p(m))
\end{equation}
%
The recurrence with the big maximum operator eliminated is:
%
\begin{equation}
  \label{eq:sort_rec_final}
  g(n) = \begin{cases}
    (1,0) & n=0 \\
    (3 + g_c(n-1) + (\LB\|\T{insert}\|\RB\ 1\ g_p(n-1))_c, (\LB\|\T{insert}\|\RB\ 1\ g_p(n-1))) & n > 0
  \end{cases}
\end{equation}
%
We extract the recurrence for the potential.
Let $p = \pi_1 \circ g$.
%
\begin{equation}
  \label{eq:sort_rec_potential}
  p(n) = \begin{cases}
    0 & n=0 \\
  (\LB\|\T{insert}\|\RB\ 1\ g_p(n-1))) & n > 0
  \end{cases}
\end{equation}
%
The closed form solution to this recurrence is given below.
%
\begin{lemma}
  \label{lem:sort_potential}
  $p(n) = n$
\end{lemma}
%
We extract the recurrence for the cost.
Let $c = \pi_0 \circ g$.
%
\begin{equation}
  \label{eq:sort_rec_cost}
  c(n) = \begin{cases}
    1 & n=0 \\
    3 + g_c(n-1) + (\LB\|\T{insert}\|\RB\ 1\ g_p(n-1) & n > 0
  \end{cases}
\end{equation}
%

% ================================================================================
%This recurrence is difficult to work with.
%Specifically, we cannot apply traditional methods of solving it.
%We will manipulate it into a more usable form by eliminating the arbitrary maximum.
%Observe that for $n=0$, $g(n) = f_{Nil}(\LP\RP) = (1,1)$.
%For $n>0$,
%\begin{align*}
%  g(n) &= \bigvee_{size(z) \leq n} case(z, (f_{Nil},f_{Cons})) &&\\
%  &= g(n-1) \vee \bigvee_{size(z) = n} case(z, (f_{Nil},f_{Cons})) &&\\
%  &= g(n-1) \vee f_{Cons}(n) &&\\
%  &= g(n-1) \vee (4 + \pi_0(\pi_1(f\ 1)\ 1) + \pi_0g(n-1), (1+n) \vee (1+\pi_1g(n-1))) &&\text{$m=n-1$}\\
%  &= (4 + \pi_0(\pi_1(f\ 1)\ 1) + \pi_0g(n-1), (1+n) \vee (1+\pi_1g(n-1))) &&\text{lemma \ref{lem:insert_g_monotonicity}}\\
%  &= (4 + \pi_0(\pi_1(f\ 1)\ 1) + \pi_0g(n-1), 1+\pi_1g(n-1)) &&\text{lemma \ref{lem:insert_potential_inc}}
%\end{align*}
%\begin{lemma}
%  \label{lem:insert_g_monotonicity}
%  $g(n) > g(n-1)$
%\end{lemma}
%\begin{proof}
%  TODO
%\end{proof}
%
%\begin{lemma}
%\label{lem:insert_potential_inc}
%$\pi_1 g(n) > n$
%\end{lemma}
%\begin{proof}
%We prove this by induction on $n$.
%\begin{description}
%  \item[case $n=0$:] $\pi_1g(0) = 1$
%  \item[case $n>0$:]\hfill
%    \begin{align*}
%      \pi_1g(n) &= \pi_1(g(n-1) \vee (4 + \pi_0(\pi_1(f\ 1)\ 1) + \pi_0g(n-1),(1+n) \vee (1 + \pi_1 g(n-1)))) \\
%      &= \pi_1g(n-1) \vee (1+n) \vee (1 + \pi_1 g(n-1)) \\
%      &\geq n-1 \vee (1+n) \vee (1 + n - 1) \\
%      &\geq 1+n \\
%      &> n
%    \end{align*}
%\end{description}
%\end{proof}
%
%Equation \ref{eq:insert_recurrence} shows the extracted recurrence.
%Without the arbitrary maximum, it is much more obvious how to find a solution to the recurrence.
%The recurrence is from a potential to a complexity, consequently we can extract a recurrence for the cost,
%equation \ref{eq:insert_cost}, and a recurrence for the potential, equation \ref{eq:insert_potential}, simply by taking the projections of equation \ref{eq:insert_recurrence}.
%The extracted recurrences for the cost and potential can then be solved by the substitution method.
%\begin{equation}
%  \label{eq:insert_recurrence}
%  g(n) = \begin{cases}
%    (1,1) & n = 0 \\
%    (4 + \pi_0(\pi_1(f\ 1)\ 1) + \pi_0g(n-1), 1+\pi_1g(n-1)) & n > 0
%  \end{cases}
%\end{equation}
%
%The cost recurrence is given by $\pi_0 \circ g$.
%\begin{equation}
%\label{eq:insert_cost}
%c(n) = \begin{cases}
%  1 & n = 0 \\
%  4 + \pi_0(\pi_1(f\ 1)\ 1) + c(n-1) & n > 0
%\end{cases}
%\end{equation}
%
%This recurrence is quite simple to solve.
%The solution and proof of the solution are given in theorem \ref{thm:insert_cost}.
%\begin{theorem}
%\label{thm:insert_cost}
%  $c(n) = (4 + \pi_0(\pi_1(f\ 1)\ 1))n + 1$
%\end{theorem}
%\begin{proof}
%  We prove this by induction on $n$.
%  \begin{description}
%    \item[case $n=0$] $c(0) = \pi_0g(0) = 1$
%    \item[case $n>0$]
%      \begin{align*}
%        c(n) &= 4 + \pi_0(\pi_1(f\ 1)\ 1) + c(n-1) \\
%        &= 4 + \pi_0(\pi_1(f\ 1)\ 1) + (4 + \pi_0(\pi_1(f\ 1)\ 1))(n-1) + 1 \\
%        &= (4 + \pi_0(\pi_1(f\ 1)\ 1)) n + 1
%      \end{align*}
%  \end{description}
%\end{proof}
%The solution tells us the cost of the \T{rec} construct in insert is linear in the size of the list.
%The constant factor cannot be determined because we do not know the cost of applying $f$ to its arguments.
%
%The potential recurrence is given by $\pi_1 \circ g$.
%\begin{equation}
%  \label{eq:insert_potential}
%  p(n) = \begin{cases}
%    1 & n = 0 \\
%    1 + p(n-1) & n > 0
%  \end{cases}
%\end{equation}
%
%\begin{theorem}
%  $p(n) = 1 + n$
%\end{theorem}
%\begin{proof}
%  We prove this by induction on $n$.
%  \begin{description}
%    \item[case $n=0$] $p(0) = 1$
%    \item[case $n>0$] $p(n) = 1 + p(n - 1) = 1 + n$
%  \end{description}
%\end{proof}
%
%
%\subsubsection{Interpretation}
%The \T{rec} construct again drives the cost and potential of \T{sort}.
%The walkthrough of the interpretation of the \T{rec} is given in figure \ref{fig:sort_rec_interp}.
%Equation \ref{eq:sort_interp0_init} shows the initial recurrence extracted.
%\begin{equation}
%  \label{eq:sort_interp0_init}
%  g(n) = \bigvee_{size(z)\leq n} case(z,(\lambda(\LP\RP).(1,0),\lambda(1,m).4 + \pi_0 g(m)) +_c insert\ f\ 1\ \pi_1g(m))
%\end{equation}
%
%We work the recurrence into a more recognisable form using some manipulation of the big max operator and some facts about $insert$.
%Observe for $n=0$, $g(n) = (1,0)$ and for $n>0$
%\begin{align*}
%  g(n) &= \bigvee_{size(z)\leq n} case(z,(\lambda(\LP\RP).(1,0),\lambda(1,m).4 + \pi_0 g(m)) +_c insert\ f\ 1\ \pi_1g(m)) \\
%  &= g(n-1) \vee \bigvee_{size(z) = n} case(z,(\lambda(\LP\RP).(1,0),\lambda(1,m).4 + \pi_0 g(m)) +_c insert\ f\ 1\ \pi_1g(m)) \\
%  &= g(n-1) \vee (4 + \pi_0 g(n-1)) +_c insert\ f\ 1\ \pi_1g(n-1) \\
%  &= g(n-1) \vee (4 + \pi_0 g(n-1) + \pi_0 (insert\ f\ 1\ \pi_1g(n-1)), \pi_1 (insert\ f\ 1\ \pi_1g(n-1))) \\
%  &\text{since $\pi_0 (insert\ f\ 1\ m) > 0$ and $\pi_1 (insert\ f\ 1\ m) = 1 + m$} \\
%  &= (4 + \pi_0 g(n-1) + \pi_0 (insert\ f\ 1\ \pi_1g(n-1)), \pi_1 (insert\ f\ 1\ \pi_1g(n-1)))
%\end{align*}
%
%So our simplified recurrence is
%\begin{equation}
%  \label{eq:sort_rec_final}
%  g(n) = \begin{cases}
%    (1,0) & n=0 \\
%    (4 + \pi_0 g(n-1) + \pi_0 (insert\ f\ 1\ \pi_1g(n-1)), \pi_1 (insert\ f\ 1\ \pi_1g(n-1))) & n > 0
%  \end{cases}
%\end{equation}
%
%From this we can extract recurrences for the cost and the potential simply by taking projections from $g$.
%We begin with the potential because we will require the solution to the potential recurrence to solve the cost recurrence.
%
%Let $p = \pi_1 \circ g$.
%\begin{equation}
%  \label{eq:sort_rec_potential}
%  p(n) = \begin{cases}
%    0 & n=0 \\
%    \pi_1 (insert\ f\ 1\ \pi_1g(n-1))) & n > 0
%  \end{cases}
%\end{equation}
%
%We prove the size of the potential of the output is same as the size of the input.
%In other words, \T{sort} does not change the size of the list.
%\begin{theorem}
%  \label{thm:sort_potential}
%  $p(n) = n$
%\end{theorem}
%\begin{proof}
%  We prove this by straightforward induction on $n$.
%  \begin{description}
%    \item[case $n=0$] $p(0) = 0$\hfill \\
%    \item[case $n>0$] $p(n) = \pi_1(insert\ f\ 1\ pi_g(n-1)) = \pi_1(insert\ f\ 1\ (n-1)) = n$
%  \end{description}
%\end{proof}
%
%Let $c = \pi_0 \circ g$.
%\begin{equation}
%  \label{eq:sort_rec_cost}
%  c(n) = \begin{cases}
%    1 & n=0 \\
%    4 + \pi_0 g(n-1) + \pi_0 (insert\ f\ 1\ \pi_1g(n-1) & n > 0
%  \end{cases}
%\end{equation}
%
%\begin{theorem}
%  \label{thm:sort_cost}
%\end{theorem}
%\begin{proof}
%  We prove this by straightforward induction on $n$.
%  \begin{description}
%    \item[case $n=0$:] $c(0) = 1$
%    \item[case $n>0$:]
%      \begin{align*}
%        c(n) &= 4 + \pi_0 g(n-1) + \pi_0(insert\ f\ 1\ \pi_1g(n-1)) \\
%        &= 4 + \pi_0 g(n-1) + \pi_0(insert\ f\ 1\ (n-1) \\
%        &= 4 + (4 + \pi_0(\pi_1(f\ 1)\ 1))(n-1) + 1) + \pi_0 g(n-1)
%      \end{align*}
%      TODO COMPLETE
%  \end{description}
%\end{proof}
%
%

\section{Sequential List Map}
\label{sec:sequential_list_map}

This example is provided for comparison with the parallel list map example
given later in this thesis. We use the familiar \T{list} datatype.
%
\begin{align*}
\T{list } &= \T{Nil of unit | Cons of int $\times$ list}
\end{align*}
%
The \T{map} function recurses on the list, applying its first argument to each
element.
%
\begin{equation*}
  \T{map} = \lambda f. \lambda xs . \T{rec}(xs, \T{Nil} \mapsto \T{Nil}, \T{Cons} \mapsto \LP y \LP ys, y \RP\RP. \T{Cons}\LP f\ y, \T{force}(r)\RP)
\end{equation*}

\subsection{Translation}
%
\begin{align*}
  &\text{We apply the rule for translating an abstraction twice.}\\
  \|\T{map}\| &= \|\lambda f. \lambda xs . \T{rec}(xs, \T{Nil} \mapsto \T{Nil}, \T{Cons} \mapsto \LP y \LP ys, y \RP\RP. \T{Cons}\LP f\ y, \T{force}(r)\RP)\| \\
              &= \LP 0, \lambda f. \LP 0, \lambda xs . \T{rec}(xs, \T{Nil} \mapsto \T{Nil}, \T{Cons} \mapsto \LP y \LP ys, y \RP\RP. \T{Cons}\LP f\ y, \T{force}(r)\RP)\RP\RP \\
              &\text{We apply the rule for translating a \T{rec} construct. We use $\|xs\| = \LP 0,xs\RP$.}\\
              &= \LP 0, \lambda f. \LP 0, \lambda xs .\T{rec}(xs, \T{Nil} \mapsto 1 +_c \|\T{Nil}\|, \\
              &\quadten \T{Cons} \mapsto \LP y \LP ys, y \RP\RP.1 +_c \|\T{Cons}\LP f\ y, \T{force}(r)\RP\|)\RP\RP \\
  %
              &\text{The translation of \T{Nil} is $\LP 0, \T{Nil}\RP$.}\\
  %
              &\text{To translate the \T{Cons} branch we translate the subexpressions first.}\\
              &\quadthree\|\T{Cons}\LP f\ x,\T{force}(r)\RP\| = \LP \|\LP f\ x,\T{force}(r)\RP_c, \T{Cons}\|\LP f\ x,\T{force}(r)\RP\|\RP \\
              &\quadfive \|\LP f\ x,\T{force}(r)\RP\| = \LP\|f\ x\|_c + \|\T{force}(r)\|_c, \LP \|f\ x\|_p,\|\T{force}(r)\|_p\RP\RP \\
  %
              &\quadsix \|f\ x\| = (1 + \|f\|_c + \|x\|_c) +_c \|f\|_p\|x\|_p = \LP 1 + (f\ x)_c, (f\ x)_p\RP \\
  %
              &\quadsix \|\T{force}(r)\| = \LP 0,r\RP +_c \LP 0,r\RP_p = r \\
  %
              &\quadfive \|\LP f\ x,\T{force}(r)\RP\| = \LP 1 + (f\ x)_c + r_c, \LP (f\ x)_p,r_p\RP\RP \\
              &\quadthree \|\T{Cons}\LP f\ x,\T{force}(r)\RP\| = \LP 1 + (f\ x)_c + r_c, \T{Cons}\LP (f\ x)_p,r_p\RP\RP \\
              &= \LP 0, \lambda f. \LP 0, \lambda xs . \T{rec}(xs, \T{Nil} \mapsto \LP 1,\T{Nil}\RP, \\
              &\quadten \T{Cons} \mapsto \LP y \LP ys, y \RP\RP.\LP 2 + (f\ x)_c + r_c, \T{Cons}\LP (f\ x)_p,r_p\RP\RP \\
\end{align*}
%
We will translate \T{map} applied to some function \T{f} and a list \T{xs}.
%
\begin{align*}
  \|\T{map f xs}\| &= (1 + \|\T{map}\ f\|_c + \|xs\|_c) +_c \|\T{map}\ f\|_p \|xs\|_p \\
                   &\text{The translation of \T{map} partially applied to map is:}\\
                   &\quadthree \|\T{map}\ f\| = (1 + \|\T{map}\|_c + \|f\|_c) +_c \|\T{map}\|_p \|f\|_p \\
                   &\text{The cost of partially applied \T{map} is 0.} \\
                   &= (2 + \|f\|_c + \|xs\|_c) +_c (\|\T{map}\|_p\ \|f\|_p)_p \|xs\|_p
\end{align*}

\subsection{Interpretation}

We interpret lists as a pair of their largest element and length.
%
\begin{align*}
  \LB \T{list} \RB &= \mathbb{Z} \times \mathbb{N} \\
  D^\T{list} &= \{*\} + (\LB \mathbb{Z} \RB \times \LB \T{list} \RB) \\
  size_\T{list}(*) &= (0, 0) \\
  size_\T{list}((x, (m, n))) &= (max(x, m), 1 + n)
\end{align*}
%
The recursor of \T{map} drives the cost, so we will interpret the recursor
first.
%
\begin{align*}
  g(i, n) &= \LB \T{rec}(xs, \T{Nil} \mapsto \LP 1,\T{Nil}\RP, \\
       &\quadfive \T{Cons} \mapsto \LP 2 + (f\ x)_c + r_c,\T{Cons}\LP (f\ x)_p,r_p\RP\RP \RB \{xs \mapsto n, f \mapsto f\}\\
       &= \bigvee\limits_{size(z) \leq n} case(z, f_{Nil}, f_{Cons}) \\
       &\text{where} \\
  f_{Nil}(\ast) &= \LB \LP 1,\T{Nil}\RP\RB \\
                &= (1,0) \\
  f_{Cons}(i, m) &= \LB \LP 2 + (f\ x)_c + r_c,\T{Cons}\LP (f\ x)_p,r_p\RP\RP\RB \xi\\
                 &\qquad \text{where } \xi = \{xs \mapsto n, f \mapsto f, y \mapsto i, ys \mapsto m, r \mapsto g(i, m) \} \\
                 &= (2 + (f\ i)_c + g_c(i,m), (max((f\ i)_p, \pi_0g_p(i,m)), 1 + \pi_1 g_p(i,m)))
\end{align*}
%
We break the recurrence into the cases of $n=0$ and $n>0$
%
\begin{description}
  \item[case $n=0$]\hfill \\
    $g(i,0) = \bigvee\limits_{size(z) \leq 0} case(z, f_{Nil}, f_{Cons}) = (1,0)$
  \item[case $n>0$]\hfill \\
    \begin{align*}
      g(i,n) &= \bigvee\limits_{size(z) \leq n} case(z, f_{Nil}, f_{Cons}) \\
             &= \bigvee\limits_{size(z) \leq n-1} case(z,f_{Nil},f_{Cons}) \vee \bigvee\limits_{size(z) = n} case(z,f_{Nil}, f_{Cons}) \\
             &= g(i,n-1) \vee \\
             &\qquad (2 + (f\ i)_c + g_c(i,n-1), (max((f\ i)_p,\pi_0g_p(i,n-1)), 1 + \pi_1 g_p(i,n-1))) \\
             &\text{Since $g_c(i,n-1) \leq g_c(i,n-1)$, then $g_c(i,n-1)\leq 2 + (f\ i)_c + g_c(i,n-1)$.} \\
             &\text{Since $g_p(i,n-1) \leq g_p(i,n-1)$, then $\pi_0 g_p(i,n-1) \leq max((f\ i)_p, \pi_0g_p(i,n-1))$,} \\
             &\text{and $\pi_1 g_p(i,n-1) \leq 1 + \pi_1 g_p(i,n-1)$.} \\
             &= (2 + (f\ i)_c + g_c(i,n-1), (max((f\ i)_p,\pi_0g_p(i,n-1)), 1 + \pi_1 g_p(i,n-1))) \\
    \end{align*}
\end{description}
%
We obtain a closed form solution for the cost by using the substitution method.
%
\begin{lemma}
  \label{lem:listmap_g_cost}
  $g_c(i,n) = (2 + (f\ i)_c) n + 1$
\end{lemma}
%
\begin{proof}
  The proof is by induction on $n$.
  \begin{description}
    \item[case $n=0$]\hfill \\
      $g_c(i,0) = (1,0)_c = 1$
    \item[case $n>0$]\hfill \\
      \begin{align*}
        g_c(i,n) &= 2 + (f\ i)_c + g_c(i,n-1)\\
                 &= 2 + (f\ i)_c + (2 + (f\ i)_c)(n-1) + 1\\
                 &= 2 + (f\ i)_c + (2 + (f\ i)_c) n - 2 - (f\ i)_c + 1\\
                 &= (2 + (f\ i)_c) n + 1
      \end{align*}
    \end{description}
\end{proof}
%
We obtain a closed form of the solution for the potential of the recursor.
%
\begin{lemma}
  \label{lem:listmap_g_potential}
  $g_p(i,n) = ((f\ i)_p, n)$
\end{lemma}
%
\begin{proof}
  The proof is by induction on $n$.
  \begin{description}
    \item[case $n=0$]\hfill \\
      $g_p(i,0) = (1,0)_p = 0$
    \item[case $n>0$]\hfill \\
      \begin{align*}
        g_p(i,n) &= (max((f\ i)_p,\pi_0g_p(i,n-1)), 1 + \pi_1 g_p(i,n-1)) \\
                 &= (max((f\ i)_p, (f\ i)_p), 1 + n - 1) \\
                 &= ((f\ i)_p, n)
      \end{align*}
  \end{description}
\end{proof}
%
Using the results from lemmas \ref{lem:listmap_g_cost} and
\ref{lem:listmap_g_potential}, we interpret the translation of \T{map f xs}.
Recall the translation is $(2 + \|f\|_c + \|xs\|_c) +_c (\|\T{map}\|_p\ \|f\|_p)_p \|xs\|_p$.
So the interpretation is $2 +_c (\LB\|\T{map}\|\RB f)_p (i,n)$,
where $(i,n)$ is the interpretation of $xs$ and we assume $f$ and $xs$ have $0$
cost.
%
\begin{lemma}
  \label{lem:listmap_cost}
  $\LB\|\T{map f xs}\|_c\RB = 3 + (2 + (f\ i)_c)n$ where $f$ is the
  interpretation of the translation of \T{f} and $(i,n)$ is the interpretation
  of the translation of \T{xs}.
\end{lemma}
%
\begin{lemma}
  \label{lem:listmap_potential}
  $\LB\|\T{map f xs}\|_p\RB = ((f\ i)_p,n)$ where $f$ is the
  interpretation of the translation of \T{f} and $(i,n)$ is the interpretation
  of the translation of \T{xs}.
\end{lemma}
%
Lemma \ref{lem:listmap_cost} shows that the cost of \T{map f xs} is linear in
the size of the list but also depends on the cost of applying \T{f} to the
elements of \T{xs}. Lemma \ref{lem:listmap_potential} shows the cost of future
uses of \T{map f xs} depends on the length of \T{xs} and the size of the result
of applying \T{f} to the elements of \T{xs}.

\section{Sequential Tree Map}

This example is presented for comparison with the parallel tree map given in
chapter 4,

\chapter{Parallel Functional Program Analysis}

We demonstrate the flexibility of the method developed in \citet{Danner2015} by
extending it to parallel cost semantics. We introduce a parallel operational
cost semantics for the source language, alter the translation into the
complexity language to produce a new notion of cost, and then prove the
bounding theorem for the new complexity translation function.  Finally we give
two examples of the recurrence extraction and interpretation.


To analyze the complexity of a sequential program, we are only interested in a
measure of the steps required to run the program. To analyze the complexity of
a parallel program, we need a measure which takes into account the extent to
which a computation may be run on multiple processors.  The cost semantics we
use are work and span \citet{Harper2012PFPL}.  We give a brief overview of work
and span.


\section{Work and span}
Work and span is a method of predicting the running time of programs that may
be run on arbitrary number of processors. Instead of producing an approximation
of the number of steps required to execute a program, work and span instead
produces a cost graph; a specification of the dependencies between
subcomputations of the program. The cost graph can be compiled into two
measures, work and span. The work of a program corresponds to the total number
of steps required to execute the program.  The span is the number of steps
in the critical path.  The critical path is the longest number of steps that
must be executed sequentially.  The length of the critical path determines the
extent to which a program may be parallelized.  If the span is equal to the
work, than every step in the computation depends on the previous step, and the
subcomputations of the program cannot be run independently, so the program
cannot be parallelized. If the span is smaller than the work, then there are
subcomputations which may be run independently and the program may be
parallelized. The upper bound on the running time of a program is summarized by
theorem \ref{thm:ws_brents_theorem}.

\begin{theorem}[Brent's Theorem]
  \label{thm:ws_brents_theorem}
  A program with work $w$ and span $s$ may be evaluated on $p$ processors in $O(max(w/p,s))$ steps.
\end{theorem}

A cost graph is defined as follows.

\[ \mathcal{C} ::= 0\ |\  1\ |\  \mathcal{C} \oplus \mathcal{C}\ |\  \mathcal{C} \otimes \mathcal{C} \]

The operator $\oplus$ connects to cost graphs who must be combined
sequentially.  The operator $\otimes$ connects cost graphs which may be
combined in parallel.

The work of a cost graph is defined as
%
\begin{equation*}
  work(c) = \begin{cases}
    0 &\text{if } c = 0 \\
    1 &\text{if } c = 1 \\
    work(c_0) + work(c_1) &\text{if } c = c_0 \otimes c_1 \\
    work(c_0) + work(c_1) &\text{if } c = c_0 \oplus c_1
  \end{cases}
\end{equation*}
%
Since the work of a program is the total number of steps required to run the
program, we add the work of subgraphs regardless whether the may be run
independently or not.


The span of a cost graph is defined as
%
\begin{equation*}
  span(c) = \begin{cases}
    0 &\text{if } c = 0 \\
    1 &\text{if } c = 1 \\
    max(span(c_0), span(c_1)) &\text{if } c = c_0 \otimes c_1 \\
    span(c_0) + span(c_1) &\text{if } c = c_0 \oplus c_1
  \end{cases}
\end{equation*}
%
Cost graphs connected by $\oplus$ must be run sequentially, so their span is the
sum of the spans of the subgraphs. Cost graphs connected by $\otimes$ may be
run independently, so their span is the maximum of the spans of the subgraphs.


\subsection{Operational Cost Semantics}
%
We alter the operational cost semantics of the source language to produce a cost
graph instead of a natural number. Figure \ref{fig:ws_srclang_oper_sem} shows
the new operational cost semantics. For tuples, the subexpressions may be evaluated
in parallel, so the cost of evaluating a tuple is the cost graphs of the
subexpressions connected by $\otimes$.  For \T{split}, the second subexpression
depends on the result of the first subexpression, so the cost of evaluating the
\T{split} is the cost graphs of the subexpression connected with $\oplus$. In
every rule except for tuples and function application, we replace $+$ with
$\oplus$. Because tuples and function application are the two syntactic
forms which consist of multiple subexpressions whose evaluation does not depend
on each other.
%
\begin{figure}
\label{fig:ws_srclang_oper_sem}
\caption{Source language operational semantics}

\bigskip

\AxiomC{$e_0 \downarrow^{n_0} v_0$}
\AxiomC{$e_1 \downarrow^{n_1} v_1$}
\BinaryInfC{$\LP e_0, e_1 \RP \downarrow^{n_0 \otimes n_1} \LP v_0, v_1 \RP$}
\DisplayProof
%
\quad
%
\AxiomC{$e_0 \downarrow^{n_0} \LP v_0, v_1 \RP$}
\AxiomC{$e_1[v_0/x_0, v_1/x_1] \downarrow^{n_1} v$}
\BinaryInfC{$split(e_0, x_0.x_1.e_1) \downarrow^{n_0 \oplus n_1} v$}
\DisplayProof

\bigskip

\AxiomC{$e_0 \downarrow^{n_0} \lambda x.e_0'$}
\AxiomC{$e_1 \downarrow^{n_1} v_1$}
\AxiomC{$e_0'[v_1/x] \downarrow^n v$}
\TrinaryInfC{$e_0\ e_1 \downarrow^{(n_0 \otimes n_1) \oplus n \oplus 1} v$}
\DisplayProof
%
\quad
%
\AxiomC{}
\UnaryInfC{$delay(e) \downarrow^0 delay(e)$}
\DisplayProof

\bigskip

\AxiomC{$e \downarrow^{n_0} delay(e_0)$}
\AxiomC{$e_0 \downarrow^{n_1} v$}
\BinaryInfC{$force(e) \downarrow^{n_0 \oplus n_1} v$}
\DisplayProof
%
\quad
%
\AxiomC{$e \downarrow^n v$}
\UnaryInfC{$C e \downarrow^n C v$}
\DisplayProof

\bigskip

\AxiomC{$e \downarrow^{n_0} C v_0$}
\AxiomC{$map^{\phi_C}(y.\LP y, delay(rec(y, \overline{C \mapsto x.e_C}))\RP, v_0) \downarrow^{n_1} v_1$}
\AxiomC{$e_C[v_1/x] \downarrow^{n_2} v$}
\TrinaryInfC{$rec(e, \overline{C \mapsto x.e_C}) \downarrow^{1 \oplus n_0 \oplus n_1 \oplus n_2} v$}
\DisplayProof

\bigskip

\AxiomC{}
\UnaryInfC{$map^t(x.v, v_0) \downarrow^0 v[v_0/x]$}
\DisplayProof
%
\quad
%
\AxiomC{}
\UnaryInfC{$map^\tau(x.v, v_0) \downarrow^0 v_0$}
\DisplayProof

\bigskip

\AxiomC{$map^{\phi_0}(x.v, v_0) \downarrow^{n_0} v_0'$}
\AxiomC{$map^{\phi_1}(x.v, v_1) \downarrow^{n_1} v_1'$}
\BinaryInfC{$map^{\phi_0 \times \phi_1}(x.v, \LP v_0, v_1 \RP) \downarrow^{n_0 \otimes n_1} \LP v_0', v_1'\RP$}
\DisplayProof

\bigskip

\AxiomC{}
\UnaryInfC{$map^{\tau \to \phi}(x.v, \lambda y.e) \downarrow^0 \lambda y.let(e, z.map^\phi(x.v, z))$}
\DisplayProof
%
\quad
%
\AxiomC{$e_0 \downarrow^{n_0} v_0$}
\AxiomC{$e_1[v_0/x] \downarrow^{n_1} v$}
\BinaryInfC{$let(e_0, x.e_1) \downarrow^{n_0 \oplus n_1} v$}
\DisplayProof
\end{figure}
%
The complexity translation is given in Figure
\ref{fig:ws_complexity_translation}.  The operator $E_0 \oplus_c E_1$ is
syntactic sugar for $\LP E_0 \oplus E_{1c}, E_{1p} \RP$.  The
translation is similar to the original translation except we replace the use of
$+$ and $+_c$ with $\oplus$ and $\oplus_c$.  In the tuple case and function
application case, the subexpressions may be computed in parallel so the cost is
the costs of the subgraphs connected with $\otimes$.
%
\begin{figure}
  \label{fig:ws_complexity_translation}
  \caption{Work and span translation from source language to complexity language}
  \begin{align*}
    \|x\| &= \LP 0, x \RP \\
    \|\LP\RP\| &= \LP 0, \LP \RP \RP \\
    \|\LP e_0, e_1 \RP \| &= \LP \|e_0\|_c \otimes \|e_1\|_c, \LP \|e_0\|_p, \|e_1\|_p\RP\RP \\
    \|split(e_0, x_0.x_1.e_1)\| &= \|e_0\|_c \oplus_c \|e_1\|[\pi_0\|e_0\|_p/x_0, \pi_1\|e_1\|_p/x_1] \\
    \|\lambda x.e\| &= \LP 0, \lambda x.\|e\| \RP \\
    \|e_0\ e_1\| &= 1 \oplus (\|e_0\|_c \otimes \|e_1\|_c) \oplus_c \|e_0\|_p\ \|e_1\|_p \\
    \|delay(e)\| &= \LP 0, \|e\|\RP \\
    \|force(e)\| &= \|e\|_c \oplus_c \|e\|_p \\
    \|C_i^\delta e\| &= \LP \|e\|_c, C_i^\delta \|e\|_p \RP \\
    \|rec^\delta(e, \overline{C \mapsto x.e_C})\| &= \|e\|_c \oplus_c rec^\delta(\|e\|_p, \overline{C \mapsto x.1 \oplus_c \|e_C\|}) \\
    \|map^\phi(x.v_0, v_1)\| &= \LP 0, map^{\LP\LP \phi \RP \RP} (x. \|v_0\|_p, \|v_1\|_p)\RP \\
    \|let(e_0, x.e_1)\| &= \|e_0\|_c \oplus_c \|e_1\|[\|e_0\|_p/x]
  \end{align*}
\end{figure}
%
\section{Bounding Relation}
%
We verify that the the translation of a well-typed source language is bounded
by its translation into the complexity language. We mutually define the
following bounding relations:
%
\begin{enumerate}
  \item $e \sqsubseteq_\tau E, \emptyset \vdash_\phi e : \tau$ and $\emptyset \vdash_{\|\psi\|} E : \|\tau\|$
  \item $v \sqsubseteq_\tau^{val} E, $ where $\emptyset \vdash v : \tau$ and $\emptyset \vdash_{\|\psi\|} E : \llangle \tau \rrangle$
  \item $v \sqsubseteq_{\phi,R}^{val} E, $ where $ \emptyset \vdash_\psi v : \phi[\gamma]$ and $\emptyset \vdash_{\|\psi\|} E : \llangle \phi \rrangle[\delta]$
  \item $e \sqsubseteq_{\phi,R} E$, where $\emptyset \vdash_\psi e : \phi[\delta]$ and $\emptyset \vdash_{\|\psi\|} E : \|\phi\|[\delta]$
\end{enumerate}
%
The bounding relation $e \sqsubseteq_\tau$ defines the notion of an source
language expression to be bounded by a complexity language expression. We will
refer to this as "bounding". The second bounding relation $e
\sqsubseteq_\tau^{val} E$ defines a notion of a complexity language potential
bounding a source language value. We will refer to this relation as "value
bounding". Relations three and four are parameterized by any relation $R$ and
interpret strictly positive functors as relation transformers. The mutual
definitions of the relations are given below.
%
\begin{defn}[Bounding Relation]\leavevmode
  \label{def:ws_bounding_relations}
\begin{enumerate}
  \item We define $e \sqsubseteq_\tau E$ as if $e \downarrow^n v$, then \label{ws:bounding_rel_defn}
    \begin{itemize}
      \item $n \leq E_c$
      \item $v \sqsubseteq_\tau^{val} E_p$.
    \end{itemize}
  \item We define $v \sqsubseteq_\tau^{val} E$ as
    \begin{itemize}
      \item $v \sqsubseteq_{unit}^{val} E$ always.
      \item $\LP v_0,v_1 \RP \sqsubseteq_{\tau_0 \times \tau_1}^{val} E$ iff $v_0 \sqsubseteq_{\tau_0}^{val} \pi_0 E$ and $v_1 \sqsubseteq_{\tau_2}^{val} \pi_1 E$.
      \item $\T{delay}(e) \sqsubseteq_{\T{susp }\tau}^{val}$ if $e \sqsubseteq_\tau E$.
      \item $v \sqsubseteq_\delta^{val} E$ is inductively defined by
        \begin{prooftree}
          \AxiomC{$\textbf{C} : \phi \to \delta \in \psi$}
          \AxiomC{$v \sqsubseteq_{\phi, -\sqsubseteq_\delta^{val}-}^{val} E'$}
          \AxiomC{$\textbf{C}\ E' \leq_\delta E$}
          \TrinaryInfC{$\textbf{C}\ v \sqsubseteq_\delta^{val} E$}
        \end{prooftree}
      \item $\lambda x.e \sqsubseteq^{val}_{\tau \to \phi,R} E$ if for all $v$ and $E_0$, if $v \sqsubseteq_\tau^{val} E_0$, then $e[v/x] \sqsubseteq_{\phi,R}(E\ E_0)$
    \end{itemize}
  \item We define $v \sqsubseteq_{\phi,R}^{val} E_p$ as
    \begin{itemize}
      \item $v \sqsubseteq_{t,R}^{val} E $ if $R(v,E)$
      \item $v \sqsubseteq_{\tau,R}^{val} E$ if $v \sqsubseteq_\tau^{val} E$.
      \item $\LP v_0,v_1\RP \sqsubseteq_{\phi_0\times\phi_1,R}^{val} E$ if $v_0 \sqsubseteq_{\phi_0,R}^{val} \pi_0 E$ and $v_1 \sqsubseteq_{\phi_0,R}^{val} \pi_1 E$
    \end{itemize}
  \item We define as $e \sqsubseteq_{\phi,R} E$ as if $e \downarrow^n v$, then
    \begin{itemize}
      \item $n \leq E_c$
      \item $v \sqsubseteq_{\phi,R}^{val} E_p$
    \end{itemize}
\end{enumerate}
\end{defn}
%
The definition \ref{ws:bounding_rel_defn} of $e \sqsubseteq_\tau E$ states that
if a source language expression $e$ steps to a value $v$ with cost $n$, then
the cost $n$ is less than or equal to the cost of the complexity language
expression $E$ and the value $v$ is value bounded by potential of the
complexity language expression $E$. In the sequential cost semantics, the cost
is a natural number and $\leq$ is the ordering $a \leq b$ if there exists a
natural number $c$ such that $a+c=b$. We need to impose a partial ordering on
cost graphs. We use the containment ordering. A cost graph $a$ is less than or
equal to a cost graph $b$ if $a$ is a subset of $b$. The subset relation is
transitive and antisymmetric, so the ordering is transitive and antisymmetric.
The ordering must satisfy these two properties in order for the proof of the
bounding relation to fall through.


The bounding theorem states well-typed source language expressions are bounded
by their translations into the complexity language.
%The following lemmas are required.
% %Weakening consists of two parts. The first states if a source language
%expression is bounded by a complexity language expression and the complexity
%language expression is less than or equal to another complexity language
%expression, then the source language expression is bounded by the second
%complexity language expression. The second states if a source language value is
%value bounded by a complexity language expression and the complexity language
%expression is less than or equal to another complexity language expression,
%then the source language value is value bounded by the second complexity
%language expression.
%%
%\begin{lemma}[Weakening]
%  \label{lem:ws_weakening}
%  \begin{align}
%    &e \sqsubseteq_\tau E \ and\ E \leq_{\|\tau\|} E' \implies e \sqsubseteq_\tau E' \\
%    &v \sqsubseteq_\tau^{val} E \ and\ E \leq_{\llangle \tau \rrangle} E' \implies  v \sqsubseteq_\tau^{val} E'
%  \end{align}
%\end{lemma}
%\begin{proof}
%  The proof is by induction on type $\tau$.
%\end{proof}
%%
%The compositionality
%
\begin{theorem}[Bounding Theorem]
  If $\gamma \vdash e : \tau, $ then $e \sqsubseteq_\tau \|e\|$.
\end{theorem}
\begin{proof}
  The proof proceeds by induction on the typing derivation and is identical to
  the proof of the bounding theorem \citet{Danner2015}. We will not reproduce
  it here.
\end{proof}

%
% PARALLEL LIST MAP
%
\section{Parallel List Map}
We will analyze the cost of the \T{map} function from section
\ref{sec:sequential_list_map}. We translate the source language program using
the new translation function and interpret the resulting complexity language
expression in a similar denotational semantics.

We use the same data type \T{list} and \T{map} definition.
%
\begin{equation*}
  \T{datatype list} = \T{Nil of Unit | Cons of int $\times$ list}
\end{equation*}
%
\begin{equation*}
  \T{map} = \lambda f. \lambda xs . \T{rec}(xs, \T{Nil} \mapsto \T{Nil}, \T{Cons} \mapsto \LP y \LP ys, y \RP\RP. \T{Cons}\LP f\ y, \T{force}(r)\RP)
\end{equation*}
%
The derivation of the complexity expression is given in Figure
\ref{fig:ws_map_complexity_translation}.
%
\begin{figure}
  \label{fig:ws_map_complexity_translation}
  \caption{Work and span complexity translation of \T{map f xs}.}
  \[\begin{split}
    &\|\T{map f xs}\| = \\
    &  (2 \oplus f_c \otimes xs_c) \oplus_c \T{rec}(xs_p, \T{Nil}\mapsto 1 \oplus_c \|\T{Nil}\|, \\
    &\quadten\qquad \T{Cons} \mapsto \LP y, \LP ys, r \RP \RP. 1 \oplus_c \|\T{Cons}\LP f\ y, \T{force}(r)\RP \|) \\
    %Nil branch
    &\quad 1 \oplus_c \|\T{Nil}\| = 1 \oplus_c \LP 0, Nil \RP = \LP 1, Nil \RP \\
    %Cons branch
    &\quad\|\T{Cons}\LP f\ y, \T{force}(r)\RP \| = \LP \|\LP f\ y, \T{force}(r)\RP\|_c, \T{Cons} \| \LP f\ y, \T{force}(r)\RP \|_p\RP \\
    &\qquad \|\LP f\ y, \T{force}(r)\RP\| = \LP \|f\ y\|_c \otimes \|\T{force}(r)\|_c, \LP \|f\ y\|_p, \|\T{force}(r)\|_p\RP\RP \\
    %f y
    &\quadthree \|f\ y\| = (1 \oplus \|f\|_c \otimes \|y\|_c) \oplus_c \|f\|_p \|y\|_p \\
    &\quadfour = (1 \oplus \LP 0, f_p \RP_c \otimes \LP 0, y \RP_c) \oplus_c \LP 0, f_p \RP_p \LP 0, y \RP_p \\
    &\quadfour = 1 \oplus_c f_p\ y \\
    &\quadfour = \LP 1 \oplus (f_p y)_c, (f_p\ y)_p \RP \\
    %force r
    &\quadthree \|\T{force}(r)\| = \|r\|_c \oplus_c \|r\|_p \\
    &\quadfour = \LP 0, r \RP_c \oplus_c \LP 0, r \RP_p \\
    &\quadfour = r \\
    %<...>
    &\qquad \|\LP f\ y, \T{force}(r)\RP\| = \LP (1 \oplus (f_p\ y)_c) \otimes r_c, \LP (f_p\ y)_p, r_p \RP \RP \\
    %Cons<..>
    &\quad\|\T{Cons}\LP f\ y, \T{force}(r)\RP \| = \LP (1 \oplus (f_p\ y)_c) \otimes r_c, \T{Cons} \LP (f_p\ y)_p, r_p \RP \RP \\
    %rec
    &\|\T{map f xs}\| = (2 \oplus f_c \otimes xs_c) \oplus_c \\
    &\quad \T{rec}(xs_p, \T{Nil}\mapsto \LP 1, \T{Nil}\RP, \\
    &\quadfour \T{Cons} \mapsto \LP y, \LP ys, r \RP \RP. \LP 1 \oplus ((1 \oplus (f_p\ y)_c) \otimes r_c), \T{Cons} \LP (f_p\ y)_p, r_p \RP \RP)
  \end{split}\]
\end{figure}
%
The complexity language translation is
%
\begin{equation*}
\begin{split}
    &\|\T{map f xs}\| = (2 \oplus f_c \otimes xs_c) \oplus_c \\
    &\quad \T{rec}(xs_p, \T{Nil}\mapsto \LP 1, \T{Nil}\RP, \\
    &\quadfour \T{Cons} \mapsto \LP y, \LP ys, r \RP \RP. \LP 1 \oplus ((1 \oplus (f_p\ y)_c) \otimes r_c), \T{Cons} \LP (f_p\ y)_p, r_p \RP \RP)
\end{split}
\end{equation*}
%
We interpret lists as a pair of their largest element and length.
%
\begin{align*}
  \LB \T{list} \RB &= \mathbb{Z} \times \mathbb{N} \\
  D^\T{list} &= \{*\} + (\LB \mathbb{Z} \RB \times \LB \T{list} \RB) \\
  size_\T{list}(*) &= (0, 0) \\
  size_\T{list}((x, (m, n))) &= (max(x, m), 1 + n)
\end{align*}
%
The interpretation of the recursor is given in Figure \ref{fig:ws_map_interpretation}.
%
\begin{figure}
  \label{fig:ws_map_interpretation}
  \caption{Interpretation of recursor in \T{map}}
  \[\begin{split}
      &\quad \text{Let } \eta = \{ xs \mapsto (0, (m, n)), f \mapsto (0, f)\}\\
      &\quad g(f, (m, n)) = \LB \T{rec}(xs_p, \T{Nil} \mapsto \LP 1, \T{Nil}\RP,\\
      &\quadten \T{Cons} \mapsto \LP y, \LP ys, r \RP \RP . \LP 1 \oplus (1 \oplus (f_p\ y)_c \otimes r_c), \T{Cons}\LP (f_p\ y)_p, r_p\RP\RP) \RB \eta\\
      &\qquad = \bigvee\limits_{(m_1, n_1) \leq (m, n)} case((m_1, n_1), \T{Nil} \mapsto \LB \LP 1, \T{Nil} \RP \RB \eta, \\
      &\quadten \T{N} \mapsto \LP y, \LP ys, r \RP \RP . \LB \LP 1 \oplus ((1 \oplus (f_p\ y)_c) \otimes r_c), \T{Cons} \LP (f_p\ y)_p, r_p \RP \RP \RB \eta_c) \\
      &\qquad \text{where } \eta_c = \{xs \mapsto (0, (m, n)), f \mapsto (0, f), y \mapsto m, ys \mapsto (0, (m, n)), \\
      &\quadeight r \mapsto g(f, (m, n-1)))\} \\
      %Nil branch
      &\quadthree \text{\T{Nil} branch} \\
      &\quadthree \LB \LP 1, \T{Nil} \RP \RB \eta = (\LB 1 \RB \eta , \LB \T{Nil} \RB \eta) = (1, (0, 0)) \\
      %Cons branch
      &\quadthree \text{\T{Cons} branch} \\
      &\quadthree  \LB \LP 1 \oplus ((1 \oplus (f_p\ m)_c) \otimes r_c), \T{Cons} \LP (f_p\ m)_p, r_p \RP \RP \RB \eta_c \\
      &\quadfour = (1 \oplus ((1 \oplus (f\ m)_c) \otimes g_c(f, (m, n-1))), ((f\ m)_p, \pi_1 g_p(f, (m, n-1)))) \\
      %putting branches together
      &\quad g(f, (m, n)) = \\
      &\qquad \bigvee\limits_{(m_1, n_1) \leq (m, n)} case((m_1, n_1), \T{Nil} \mapsto (1, (0, 0)), \\
      &\quadfour \T{Cons} \mapsto (1 \oplus ((1 \oplus (f\ m)_c) \otimes g_c(f, (m, n-1))), ((f\ m)_p, \pi_1 g_p(f, (m, n-1)))))
  \end{split}\]
\end{figure}
%
The result is
%
\begin{equation*}
  \begin{split}
  &g(f, (m, n)) = \\
  &\quad \bigvee\limits_{(m_1, n_1) \leq (m, n)} case((m_1, n_1), \T{Nil} \mapsto (1, (0, 0)), \\
  &\quadfour \T{Cons} \mapsto (1 \oplus ((1 \oplus (f\ y)_c) \otimes g_c(f, (m, n-1))), ((f\ y)_p, \pi_1 g_p(f, (m, n-1)))))
  \end{split}
\end{equation*}
%
%In figure \ref{fig:ws_map_massage}, we massage the recurrence into a simpler form by eliminating the big max and the $case$.
%\begin{figure}
%  \label{fig:ws_map_massage}
%  \caption{Simplification of the interpretation of \T{map}}
%  \[\begin{split}
%  &g(f, (m, n)) = \\
%  &\quad \bigvee\limits_{(m_1, n_1) \leq (m, n)} case((m_1, n_1), \T{Nil} \mapsto (1, (0, 0)), \\
%  &\quadfive \T{Cons} \mapsto (1 \oplus ((1 \oplus (f\ y)_c) \otimes g_c(f, (m, n-1))), ((f\ y)_p, \pi_1 g_p(f, (m, n-1)))))  \\
%      %base case
%  &\text{For } n = 0 \\
%  &\quad g(f, (m, 0) = \\
%  &\qquad (1, (0, 0)) \vee \bigvee\limits_{1 + n_1 } \\
%  &\quadsix \T{Cons} \mapsto (1 \oplus ((1 \oplus (f\ y)_c) \otimes g_c(f, (m, -1))), ((f\ y)_p, \pi_1 g_p(f, (m, -1)))))  \\
%  &\qquad = (1, (0, 0)) \\
%      %inductive case
%  &\text{For } n > 0 \\
%  &fubar
%  \end{split} \]
%\end{figure}
%
%The result is
%\begin{equation*}
%  g(f, (m, n)) = \begin{cases}
%    (1, (0, 0)) &n \equiv 0 \\
%    (1 \oplus (1 \oplus (f\ m)_c \otimes g(f, (m, n-1))_c), ((f\ m)_p, g(f, (m, n-1))_p)) &n > 0
%  \end{cases}
%\end{equation*}
We compile the recurrence down to the work and the span to make it easier to
manipulate.
%
\begin{equation*}
  \begin{split}
  &g(f, (m, n)) = \\
  &\quad \bigvee\limits_{(m_1, n_1) \leq (m, n)} case((m_1, n_1), \\
  &\quadfive\quadfour \T{Nil} \mapsto ((1, 1), (0, 0)), \\
  &\quadfive\quadfour \T{Cons} \mapsto ((2 + \pi_0(f\ y)_c + \pi_0 g_c(f, (m, n-1)), \\
  &\quadfive\quadten                     1 + max(1 + \pi_1(f\ y)_c, \pi_1 g_c(f, (m, n-1)))), \\
  &\quadfive\quadeight                  ((f\ y)_p, \pi_1 g_p(f, (m, n-1)))))
  \end{split}
\end{equation*}
%
We prove by induction bounds on the work and span of the cost of $g$.
%
\begin{theorem}
$\pi_0 g_c(f, (m, n)) \leq 1 + (2 + \pi_0(f\ m)_c)n$
\end{theorem}
%
\begin{proof}
   The proof is by induction on $n$.
  \begin{description}
    \item[case $n=0$]\mbox{}\\[-1.5\baselineskip]
      \begin{align*}
      \pi_0 g_c(f, (m, 0)) = \pi_0((1, 1), (0, 0))_c = \pi_0(1, 1) = 1
      \end{align*}
    \item[case $n>0$]\mbox{}\\[-1.5\baselineskip]
      \[\begin{split}
        &\pi_0(g_c(f, (m, n))) = \\
        &\quad \pi_0(\bigvee\limits_{(m_1, n_1) \leq (m, n)} case((m_1, n_1), \\
        &\quadfive\quadfour \T{Nil} \mapsto ((1, 1), (0, 0)), \\
        &\quadfive\quadfour \T{Cons} \mapsto ((2 + \pi_0(f\ m_1)_c + \pi_0g_c(f, (m_1, n_1-1)), \\
        &\quadfive\quadten                     1 + max(1 + \pi_1(f\ m_1)_c, \pi_1g_c(f, (m_1, n_1-1)))), \\
        &\quadfive\quadeight                  ((f\ m_1)_p, \pi_1 g_p(f, (m_1, n_1-1))))) )_c \\
        &\quad = 1 \vee \pi_0(\bigvee\limits_{(m_1, n_1) \leq (m, n)} ((2 + \pi_0(f\ m_1)_c + \pi_0g_c(f, (m_1, n_1-1)), \\
        &\quadfour\quadten                     1 + max(1 + \pi_1(f\ m_1)_c, \pi_1g_c(f, (m_1, n_1-1)))), \\
        &\quadfour\quadeight                  ((f\ m_1)_p, \pi_1 g_p(f, (m_1, n_1-1))))) )_c \\
        &\quad = 1 \vee \bigvee\limits_{(m_1, n_1) \leq (m, n)} 2 + \pi_0 (f\ m_1)_c + \pi_0 g_c(f, (m_1, n_1-1)) \\
        &\quad \leq \bigvee\limits_{(m_1, n_1) \leq (m, n)} 2 + \pi_0 (f\ m_1)_c + (1 + (2 + \pi_0 (f\ m_1)_c)(n_1-1)) \\
        &\quad \leq 2 + \pi_0 (f\ m)_c + 1 + (2 + \pi_0 (f\ m)_c)(n - 1) \\
        &\quad \leq 1 + (2 + \pi_0 (f\ m)_c)n \\
      \end{split}\]
  \end{description}
\end{proof}
%
\begin{theorem}
  $\pi_1 g_c(f, (m, n)) \leq 1 + \pi_1 (f\ m)_c + n$
\end{theorem}
%
\begin{proof}
  \begin{description}
    \item[case $n=0$]\mbox{}\\[-1.5\baselineskip]
      \begin{align*}
      \pi_1 g_c(f, (m, 0)) = \pi_1((1, 1), (0, 0))_c = \pi_1(1, 1) = 1
      \end{align*}
    \item[case $n>0$]\mbox{}\\[-1.5\baselineskip]
      \[\begin{split}
        &\pi_1(g_c(f, (m, n))) = \\
        &\quad \pi_1(\bigvee\limits_{(m_1, n_1) \leq (m, n)} case((m_1, n_1), \\
        &\quadfive\quadfour \T{Nil} \mapsto ((1, 1), (0, 0)), \\
        &\quadfive\quadfour \T{Cons} \mapsto ((2 + \pi_0(f\ m_1)_c + \pi_0g_c(f, (m_1, n_1-1)), \\
        &\quadfive\quadten                     1 + max(1 + \pi_1(f\ m_1)_c, \pi_1g_c(f, (m_1, n_1-1)))), \\
        &\quadfive\quadeight                  ((f\ m_1)_p, \pi_1 g_p(f, (m_1, n_1-1))))) )_c \\
        &\quad = 1 \vee \pi_1(\bigvee\limits_{(m_1, n_1) \leq (m, n)} ((2 + \pi_0(f\ m_1)_c + \pi_0g_c(f, (m_1, n_1-1)), \\
        &\quadfour\quadten                     1 + max(1 + \pi_1(f\ m_1)_c, \pi_1g_c(f, (m_1, n_1-1)))), \\
        &\quadfour\quadeight                  ((f\ m_1)_p, \pi_1 g_p(f, (m_1, n_1-1))))) )_c \\
        &\quad = 1 \vee \bigvee\limits_{(m_1, n_1) \leq (m, n)} 1 + max(1 + \pi_1 (f\ m_1)_c, \pi_1 g_c(f, (m_1, n_1-1))) \\
        &\quad \leq \bigvee\limits_{(m_1, n_1) \leq (m, n)} 1 + max(1 + \pi_1 (f\ m_1)_c, 1 + \pi_1 (f\ m_1)_c + n_1 - 1) \\
        &\quad \leq 1 + max(1 + \pi_1 (f\ m)_c, 1 + \pi_1 (f\ m)_c + n - 1) \\
        &\quad \leq 1 + 1 + \pi_1 (f\ m)_c + n - 1 \\
        &\quad \leq 1 + \pi_1 (f\ m)_c + n
      \end{split}\]
  \end{description}
\end{proof}
%
Compare these results with sequential map.


\section{Parallel Tree Map}
A program which is embarrasingly parallel is tree map.  When a function $f$ is
mapped over a tree $t$, each application of $f$ to the label at each node can
be done independently.  Furthermore, the tree data structure itself is
dividable by construction.  Dividing the work requires only destruction of the
node constructor to yield the left and right subtrees.

We will use \T{int} labelled binary trees.
%
\begin{equation*}
  \T{datatype tree} = \T{E of Unit | N of int$\times$tree$\times$tree}
\end{equation*}

%
\T{map} simply deconstructs each node, applies the function to the label,
recuses on the children, and reconstructs a node using the results.
%
\begin{equation*}
  \T{map} = \lambda f.\lambda t.\T{rec}(t, \T{E} \mapsto \T{E}, \T{N} \mapsto \LP x, \LP t_0, r_0 \RP, \LP t_1, r_1 \RP\RP.\T{N} \LP f\ x, \T{force}(r_0), \T{force}(r_1)\RP)
\end{equation*}
%
\begin{figure}
  \label{fig:ws_treemap_complexity_translation}
  \caption{Complexity translation of \T{map f t}}
  \begin{equation*}
    \begin{split}
      &2 \oplus (f_c \otimes t_c) \oplus_c \T{rec}(t_p, \T{E} \mapsto 1 \oplus_c \|\T{E}\|, \\
      &\quadten \T{N} \mapsto \LP y, \LP t_0, r_0 \RP \LP t_1, r_1 \RP \RP 1 \oplus_c \|\T{N}\LP f\ x, \T{force}(r_0) \T{force}(r_1)\RP \| ) \\
      %E branch
      &\quad 1 \oplus_c \|E\| = 1 \oplus \LP 0, E \RP = \LP 1, E \RP \\
      %N branch
      &\quad \|\T{N}\LP f\ x, \T{force}(r_0) \T{force}(r_1)\RP \| = \\
      &\qquad \LP \|\LP f\ x, \T{force}(r_0) \T{force}(r_1)\RP \|_c, \T{N} \|\LP f\ x, \T{force}(r_0) \T{force}(r_1)\RP\|_p\RP \\
      &\quadthree \|\LP f\ x, \T{force}(r_0) \T{force}(r_1)\RP \| = \\
      &\quadfour \LP \|f\ x\|_c \otimes \|\T{force}(r_0)\|_c \otimes \|\T{force}(r_1)\|_c, \LP \|f\ x\|_p, \|\T{force}(r_0)\|_p, \|\T{force}(r_1)\|_p \RP \RP \\
      % \|f x\|
      &\quadfive  \|f\ x\| = 1 \oplus \|f\|_c \otimes \|x\|_c \oplus_c \|f\|_p \|x\|_p \\
      &\quadeight = 1 \oplus \LP 0, f \RP_c \otimes \LP 0, x \RP_c \oplus_c \LP 0, f \RP_p \LP 0, x \RP_p \\
      &\quadeight = 1 \oplus_c (f_p\ x) \\
      &\quadeight = \LP 1 \oplus (f_p\ x)_c, (f_p\ x)_p \RP \\
      % \|force(r0)\|
      &\quadfive \|\T{force}(r_0)\| = \|r_0\|_c \oplus_c \|r_0\|_p \\
      &\quadeight = \LP 0, r_0 \RP \oplus_c \LP 0, r_0 \RP_p \\
      &\quadeight = \LP 0 + r_{0c}, r_{0p} \RP \\
      &\quadeight = r_0 \\
      % \|force(r1)\|
      &\quadfive \|\T{force}(r_1)\| = \|r_1\|_c \oplus_c \|r_1\|_p \\
      &\quadeight = \LP 0, r_1 \RP \oplus_c \LP 0, r_1 \RP_p \\
      &\quadeight = \LP 0 + r_{1c}, r_{1p} \RP \\
      &\quadeight = r_1  \\
      %\|<...>\|
      &\quadthree \|\LP f\ x, \T{force}(r_0) \T{force}(r_1)\RP \| = \\
      &\quadsix = \LP 1 \oplus (f_p\ x)_c \otimes r_{0c} \otimes r_{1c}, \LP (f_p\ x)_p, r_{0p}, r_{1p}\RP\RP \\
      %\|N<...>\|
      &\qquad \|\T{N}\LP f\ x, \T{force}(r_0) \T{force}(r_1)\RP \| = \\
      &\quadfive = \LP 1 \oplus (f_p\ x)_c \otimes r_{0c} \otimes r_{1c}, \T{N} \LP (f_p\ x)_p, r_{0p}, r_{1p}\RP\RP \\
      %rec
      &\|\T{map f t}\| = \\
      &2 \oplus (f_c \otimes t_c) \oplus 1 \oplus_c \T{rec}(t_p, \T{E} \mapsto \LP 1, \T{E}\RP, \\
      &\quadeight \T{N} \mapsto \LP y, \LP t_0, r_0 \RP \LP t_1, r_1 \RP \RP.  \LP 2 \oplus (f_p\ x)_c \otimes r_{0c} \otimes r_{1c}, \T{N} \LP (f_p\ x)_p, r_{0p}, r_{1p}\RP\RP
    \end{split}
  \end{equation*}
\end{figure}

The translation of \T{map f t} is given in figure \ref{fig:ws_treemap_complexity_translation}.
\begin{equation}
  \label{eq:ws_treemap_complexity}
  \begin{split}
    &\|\T{map f t}\| = 2 \oplus (f_c \otimes t_c) \oplus 1 \oplus_c \\
    &\quadfour \T{rec}(t_p, \T{E} \mapsto \LP 1, \T{E}\RP, \\
    &\quadseven \T{N} \mapsto \LP y, \LP t_0, r_0 \RP \LP t_1, r_1 \RP \RP.  \LP 2 \oplus (f_p\ x)_c \otimes r_{0c} \otimes r_{1c}, \T{N} \LP (f_p\ x)_p, r_{0p}, r_{1p}\RP\RP
\end{split}
\end{equation}

The result is shown in equation \ref{eq:ws_treemap_complexity}.

We interpret trees as the number of \T{N} constructors and the maximum label.
\begin{align*}
  \LB tree \RB &= \mathbb{Z} \times \mathbb{Z} \\
  D_\T{tree} &= \{\ast\} + \mathbb{Z} \times \LB \T{tree} \RB \times \LB \T{tree} \RB \\
  size_\T{tree}(\ast) &= (0, 0) \\
  size_\T{tree}(x, (m_0, n_0), (m_1, n_1)) &= (max(x, m_0, m_1), 1 + n_0 + n_1)
\end{align*}

\begin{figure}
  \label{fig:ws_map_interpretation_derivation}
  \caption{Deriviation of interpretation of \T{map f t}}
\[ \begin{split}
\end{split} \]
\end{figure}

\chapter{Mutual Recurrence}

\section{Motivation}
The interpretation of a recursive function can be seperated into a recurrence
for the cost and a recurrence for the potential. The recurrence for the cost
depends on the recurrence for the potential. However, the recurrence for the
potential does not depend on the cost. We prove this by designing a pure
potential translation. The pure potential translation is identical to the
complexity translation except that it does not keep track of the cost.

We then show by logical relations that the potential of the complexity
translation is related to the pure potential relation.

\section{Pure Potential Translation}
Our pure potential translation is defined below. The translation of an
expression is essientially the expression itself, without suspensions.
%
\begin{align*}
  |x| &= x                                                                                     \\
  |\langle\rangle| &= \langle\rangle                                                           \\
  |\langle e_0, e_1 \rangle | &= \langle |e_0|, |e_1| \rangle                                  \\
  |\texttt{split}(e_0, x_0. x_1. e_1)| &= |e_1|[\pi_0|e_0|/x_0, \pi_0|e_0|/x_1]                \\
  |\lambda x.e | &= \lambda x.|e|                                                              \\
  |e_0\ e_1| &= |e_0|\ |e_1|                                                                   \\
  |delay(e)| &= |e|                                                                            \\
  |force(e)| &= |e|                                                                            \\
  |C_i^\delta e| &= C_i^\delta |e|                                                             \\
  |rec^\delta(e, \overline{C \mapsto x.e_C})| &= rec^\delta(|e|, \overline{C \mapsto x.|e_C|}) \\
  |map^\phi(x.v_0, v_1)| &= map^{|\phi|}(x.|v_0|, |v_1|)                                       \\
  |let(e_0, x.e_1)| &= |e_1|[|e_0|/x]
\end{align*}
%
\section{Logical Relation}
We define our logical relation below.
%
\begin{align*}
  E &\sim_{\texttt{\tiny{Unit}}} E' \text{always}  \\
  E &\sim_{\tau_0 \times \tau_1} E' \Leftrightarrow \forall k. \langle k, \pi_0 E_p\rangle \sim_{\tau_0} \pi_0 E', \forall k. \langle k, \pi_1 E_p\rangle \sim_{\tau_1} \pi_1 E' \\
  E &\sim_{\texttt{\tiny{susp }} \tau} E' \Leftrightarrow E_p \sim_\tau E' \\
  E &\sim_{\sigma \to \tau} E' \Leftrightarrow \forall E_0 \sim_\sigma E'_0. E_p E_{0p} \sim_\tau E' E'_0 \\
  E &\sim_\delta E' \Leftrightarrow \exists k, k', C, V, V'. V \sim_{\phi[\delta]} V', E \downarrow \langle k, C V_p \rangle, E' \downarrow C V'
\end{align*}
%
The relation is defined on closed terms, but we extend it to open terms.
Let $\Theta$ and $\Theta'$ be any substitutions such that $\forall x : \|\tau\|, \forall k, \langle k, \Theta(x) \rangle \sim_\tau \Theta'(x)$.
If $E\ \Theta \sim_\tau E'\ \Theta'$, then $E \sim_\tau E'$.

\section{Proof}

We require some lemmas.

The first states we can always ignore the cost of related terms.
%
\begin{lemma}[Ignore Cost]
  \label{lem:ignorecost}
\[
  E \sim_\tau E' \Leftrightarrow \forall k, \langle k, E_p \rangle \sim_\tau E'
\]
\end{lemma}
%
\begin{proof}
  We proceed by induction on type.

  Case $E \sim_{\texttt{Unit}} E'$. 
  Then $\forall k, \langle k, E_p \rangle \sim_{\texttt{\tiny{Unit}}} E'$ by definition.

  Case $E \sim_{\tau_0 \times \tau_1} E'$.
  By definition for $i\in{0,1}, \forall k_i, \langle k_i, \pi_i E_p \rangle \sim_{\tau_i} \pi_i E'$.
  Let $k$ be some cost.
  Then $\langle k, E_p \rangle \sim_{\tau_0 \times \tau_1} E'$ by definition.

  Case $E \sim_{\texttt{\tiny{susp }} \tau} E'$.
  By definition $E_p \sim_\tau E'$.
  Let $k$ be some cost.
  Then $\langle k, E_p \rangle \sim_{\texttt{\tiny{susp }} \tau} E'$.

  Case $E \sim_{\sigma \to \tau} E'$.
  Let $E_0, E_0'$ by some complexity language terms such that $E_0 \sim_\sigma E_0'$.
  Let $k$ be some cost.
  Then, $E_p\ E_0 \sim_\tau E'\ E_0'$.
  So $\langle k, E_p \rangle \sim_{\sigma \to \tau} E'$.

  Case $E \sim_\delta E'$.
  Then by definition there exists costs $k$ and $k'$, a constructor $C$, and complexity language values $V$ and $V'$ such that $V \sim_{\Phi[\delta]} V', E \downarrow \langle k, C V_p \rangle$, and $E' \downarrow C V'$.
  Since $E \downarrow \langle k, C V_p \rangle$, we know $\forall k_0, \exists k_0'. \langle k_0, E_p \rangle \downarrow \langle k_0', C V_p \rangle$.
  So by definition we have $\forall k_0, \langle k_0, E_p \rangle \sim_\Phi E'$.
\end{proof}
%
The next lemma states that if two terms step to related terms, then those terms are related.
%
\begin{lemma}[Related Step Back]
  \label{lem:relatedstepback}
  \[
    E \to F, E' \to F', F \sim_\sigma F' \implies E \sim_\sigma E'
  \]
\end{lemma}
%
\begin{proof}
  The proof proceeds by induction on type.
  
  Case \texttt{Unit}. Trivial since $E \sim_{\texttt{\tiny{Unit}}} E'$ always.

  Case $\delta$.
  By definition $\exists C, U, U', k, k'$ such that $F \downarrow \langle k, C U_p \rangle, F' \downarrow C U', U \sim_{\phi[\delta]} U'$.
  Since $E \to F$ and $E' \to F'$, $E \downarrow \langle k, C U_p \rangle$ and $E' \downarrow C U'$.
  Therefore since $U \sim_{\phi[\delta]} U'$, we have $E \sim_\delta E'$.

  Case $\sigma \to \tau$.
  Let $E_0 \sim_\sigma E'_0$.
  By definition, $F\ E_0 \sim_\tau F'\ E'_0$.
  Since $E \to F$ and $E' \to F'$, $E\ E_0 \to F\ E_0$ and $E'\ E'_0 \to F'\ E'_0$.
  So by the induction hypothesis, $E\ E_0 \sim_\tau E'\ E_0'$.
  So by definition, $E \sim_{\sigma \to \tau} E'$.

  Case $\tau_0 \times \tau_1$.
  Since $F \sim_{\tau_0 \times \tau_1} F'$, for $i\in\{0, 1\}$,  $\forall k_i, \langle k_i, \pi_i F_p \rangle \sim_{\tau_i} \pi_i F'$, by definition.
  From $E \to F$, we get $\langle k_i, \pi_i E_p \rangle \to \langle k_i', \pi_i F_p \rangle$.
  From $E' \to F'$, we get $\pi_i E' \to \pi_i F'$.
  We can apply our induction hypothesis to get $\langle k_i, \pi_i E_p \rangle \sim_{\tau_i} \pi_i E'$.
  By \ref{lem:ignorecost}, $\forall k_i, \langle k_i, \pi_i E_p \rangle \sim_{\tau_i} \pi_i E$.
  So by definition $E \sim_{\tau_0 \times \tau_1} E'$.

  Case $\texttt{susp }\tau$.
  Since $F \sim_{\texttt{\tiny{susp }}\tau} F'$, by definition $F_p \sim_\tau F'$.
  Since $E \to F$, $E_p \to F_p$.
  So by the induction hypothesis, since $E_p \to F_p, E' \to F', F_p \sim_\tau F'$, $E_p \sim_\tau E'$.
  So by definition $E \sim_{\texttt{\tiny{susp }}\tau} E'$.

\end{proof}
%
The next lemma states that related terms step to related terms
%
\begin{lemma}
  \label{lem:relatedstep}[Related Step]
  \[ E \to F, E' \to F', E \sim_\sigma E' \implies F \sim_\sigma F' \]
\end{lemma}
%
\begin{proof}
  The proof is by induction on type.

  Case \texttt{Unit}.
  $F \sim_{\texttt{\tiny{Unit}}} F'$ always.

  Case $\delta$.
  By definition, $E \sim_\delta E'$ implies $\exists C, V, V', k$ such that $E \downarrow \langle k, C V_p \rangle, E' \downarrow C V', V \sim_{\phi[\delta]} V'$.
  Since $E \to F$, $F \downarrow \langle k, C V_p \rangle$; and since $E \to F'$, $F' \downarrow C V'$.
  By \ref{lem:ignorecost}, $\langle k, V_p \rangle \sim_{\phi[\delta]} V'$.
  So because $F \downarrow \langle k, C V_p \rangle, F' \downarrow C V', \langle k, V_p \rangle \sim_{\phi[\delta]} V'$, we can apply our induction hypothesis to get $F \sim_\delta F'$.

  Case $\tau_0 \times \tau_1$.
  By definition $E \sim_{\tau_0 \times \tau_1} \implies \forall i \in \{0, 1\}, \forall k, \langle k_i, \pi_i E_p \rangle \sim_{\tau_i} \pi_i E'$.
  Fix some $k_i$.
  Since $E \to F$, $\langle k_i, \pi_i E_p \rangle \to \langle k_i, \pi_i F_p \rangle$.
  Since $E' \to F'$, $\pi_i E' \to \pi_i F'$.
  From $\langle k_i, \pi_i E_p \rangle \to \langle k_i, \pi_i F_p \rangle, \langle k_i, \pi_i E_p \rangle \sim_{\tau_i} \pi_i E'$, the induction hypothesis tells us $\langle k_i, \pi_i F_p \rangle \sim_{\tau_i} \pi_i F'$.
  So by definition $F \sim_{\tau_0 \times \tau_1} F'$.

  Case $\texttt{susp } \tau$.
  By definition $E \sim_{\texttt{susp }\tau} E' \implies E_p \sim_\tau E'$.
  Since $E \to F$, $E_p \to F_p$.
  From $E_p \to F_p, E' \to F', E_p \sim_\tau E'$, the induction hypothesis gives us $F_p \sim_\tau F'$.
  So by definition $F \sim_{\texttt{susp }\tau} F'$.

  Case $\sigma \to \tau$.
  Let $E_0 \sim_\sigma E_0'$.
  By definition, $E\ E_0 \sim_\tau E'\ E_0'$.
  Since $E \to F$, $E\ E_0 \to F\ E_0$.
  Since $E' \to F'$, $E'\ E_0' \to F'\ E_0'$.
  From $E\ E_0 \to F\ E_0, E'\ E_0' \to F'\ E_0', E\ E_0 \sim_\tau E'\ E_0'$, the induction hypothesis tells us $F\ E_0 \sim_\tau F'\ E_0'$.
  So by definition $F \sim_{\sigma \to \tau} F'$.
\end{proof}
%
The next lemma states that if the arguments to $map$ are related, then $map$ preserves the relatedness.
%
\begin{lemma}
  \label{lem:relatedmap}[Related Map]
  \[ E \sim_{\tau_1} E', E_0 \sim_{\tau_0} E_0' \implies \forall k. \langle k, map^\Phi(x, E_p, E_{0p})\rangle \sim_{\Phi[\tau_1]} map^\Phi(x, E', E_0') \]
\end{lemma}
%
\begin{proof}
  The proof proceeds by induction on type.

  Recall the definition of the $map$ macro.
  \begin{align*}
    map^t(x.E, E_0) &= E[E_0/x]                                                                                       \\
    map^T(x.E, E_0) &= E_0                                                                                            \\ 
    map^{\Phi_0 \times \Phi_1}(x.E, E_0) &= \langle map^{\Phi_0}(x.E, \pi_0 E_0), map^{\Phi_1}(x.E, \pi_1 E_0 \rangle \\
    map^{T \to \Phi}(x.E, E_0) &= \lambda y.map^\Phi(x.E, E_0\ y)
  \end{align*}

  Case $\Phi = t$.
  Then $map^t(x.E_p, E_{0p}) = E_p[E_{0p}/x]$ and $map^t(x.E', E_0') = E'[E_0'/x]$.
  Let $k$ be some cost.
  By \ref{lem:ignorecost}, $E \sim_{\tau_1} E'$ implies $\langle k, E_p \rangle \sim_{\tau_1} E'$.
  Since $\langle k, E_p \rangle \sim_{\tau_1} E'$ and $E_0 \sim_{\tau_0} E_0'$, $\langle k, E_p \rangle [E_{0p}/x] \sim_{\phi[\tau_0]} E'[E_0'/x]$.
  So $\forall k, \langle k, map^t(x.E_p, E_{0p}) \rangle \sim_{\Phi[\tau_1]} map^t(x.E', E_0')$.

  Case $\Phi = T$.
  Then $map^T(x.E_p, E_{0p}) = E_{0p}$ and $map^T(x.E', E_0') = E_0'$.
  By \ref{lem:ignorecost} $\forall k, \langle k, E_{0p} \rangle \sim_{\tau_0} E_0'$.
  So $\forall k, \langle k, map^T(x.E_p, E_{0p}) \rangle \sim_{\Phi[\tau_1]} map^T(x.E', E_0')$.

  Case $\Phi = \Phi_0 \times \Phi_1$.
  Then \\
  $map^{\Phi_0 \times \Phi_1}(x. E_p, E_{0p}) = \langle map^{\Phi_0}(x. E_p, \pi_0 E_{0p}), map^{\Phi_1}(x. E_p, \pi_1 E_{0p}) \rangle$.\\
  Similarly $map^{\Phi_0 \times \Phi_1}(x. E', E_0') = \langle map^{\Phi_0}(x. E', \pi_0 E_0'), map^{\Phi_1}(x. E', \pi_1 E_0') \rangle$.\\
  By definition, $\forall k, \langle k, \pi_0 E_{0p} \rangle \sim_{\Phi_0[\tau_0]} \pi_0 E_0'$.\\
  By the induction hypothesis, $\forall k, \langle k, map^{\Phi_0}(x. E_p, \pi_0 E_{0p}) \sim_{\Phi_0[\tau_1]} map^{\Phi_0[\tau_1]}(x. E', E_0')$.\\
  By definition, $\forall k, \langle k, \pi_1 E_{0p} \rangle \sim_{\Phi_1[\tau_0]} \pi_1 E_0'$.\\
  By the induction hypothesis, $\forall k, \langle k, map^{\Phi_1}(x. E_p, \pi_1 E_{0p}) \sim_{\Phi_1[\tau_1]} map^{\Phi_1[\tau_1]}(x. E', E_0')$.\\
  So by definition,
  \begin{align*}
    &\forall k, \langle k, \langle map^{\Phi_0}(x. E_p, \pi_0 E_{0p}), map^{\Phi_1}(x. E_p, \pi_1 E_{0p}) \rangle \rangle \sim_{\Phi[\tau_1]}  \\
    &\langle \langle map^{\Phi_0[\tau_1]}(x. E', E_0'), map^{\Phi_1[\tau_1]}(x. E', E_0') \rangle \rangle
  \end{align*}

  Case $T \to \Phi$.
  Then $map^{T \to \Phi}(x. E_p, E_{0p}) = \lambda y.map^\Phi(x.E_p, E_{0p}\ y)$ \\
  and $map^{T \to \Phi}(x. E', E_0') = \lambda y.map^\Phi(x.E', E_0'\ y)$.
  Let $E_1 : T$.
  Then \\
  $\lambda y.map^\Phi(x.E_p, E_{0p}\ y)\ E_1 \to map^\Phi(x.E_p, E_{0p}\ E_1)$.
  Similarly, $\lambda y.map^\Phi(x.E', E_0'\ y)\ E_1' \to map^\Phi(x. E', E_0'\ E_1')$.
  Since $E_0 \sim E_0'$ and $E_1 \sim E_1'$, we have $E_{0p}\ E_1 \sim E_0'\ E_1'$.
  So by our induction hypothesis, $map^\Phi(x.E_p, E_{0p}\ E_1) \sim map^\Phi(x. E', E_0'\ E_1')$.
  So by \ref{lem:relatedstepback}, $\lambda y.map^\Phi(x.E_p, E_{0p}\ y)\ E_1 \sim \lambda y.map^\Phi(x.E', E_0'\ y)\ E_1'$.
  So by definition, \\
  $\lambda y.map^\Phi(x.E_p, E_{0p}\ y) \sim \lambda y.map^\Phi(x.E', E_0'\ y)$.\\
  So $map^{T \to \Phi}(x. E_p, E_{0p}) \sim map^{T \to \Phi}(x. E', E_0')$.
\end{proof}
%
Our last lemma is about the relatedness of $rec$ terms.
%
\begin{lemma}[Related Rec]
  \label{lem:relatedrec}
  \[ E \sim_\delta E', \forall C, E_C \sim_\tau E_C' \implies rec(E_p, \overline{C \mapsto x.E_c}) \sim_\tau rec(E', \overline{C \mapsto x.E_c'}) \]
\end{lemma}
%
\begin{proof}
  Recall the rule for evaluating $rec$ in the complexity language:
  \begin{prooftree}
    \AxiomC{$E \downarrow C V_0$}
    \AxiomC{$map^\Phi(y,\langle y, rec(y, \overline{C \mapsto x.E_C})\rangle, V_0) \downarrow V_1$}
    \AxiomC{$E_C[V_1/x] \downarrow V$}
    \TrinaryInfC{$rec(E, \overline{C \mapsto x.E_C}) \downarrow V$}
  \end{prooftree}
  By definition of $\sim_\delta$, $\exists k, C, V_0, V_0'$ such that $E \downarrow \langle k, C V_{0p} \rangle, E' \downarrow C V_0'$, and $V_0 \sim_\delta V_0'$.
  Our proof proceeds by induction on the number of constructors in $C V_{0p}$.
  If $\Phi = T$, then $map^\Phi(y, \langle y, rec(y, \overline{C \mapsto x.E_C})\rangle, V_{0p}) = \langle y, rec(y, \overline{C \mapsto x.E_C})\rangle[V_{0p}/y] = \langle V_{0p}, rec(V_{0p}, \overline{C \mapsto x.E_C}) \rangle$.
  Similarly for the pure potential, $map^\Phi(y, \langle y, rec(y, \overline{C \mapsto x.E_C'})\rangle, V_{0p}') = \langle y, rec(y, \overline{C \mapsto x.E_C'})\rangle [V_0'/y] = \langle V_0', rec(V_0', \overline{C \mapsto x.E_C'}) \rangle$.
  By the induction hypothesis, $rec(V_{0p}, \overline{C \mapsto x.E_C}) \sim_\tau rec(V_0', \overline{C \mapsto x.E_C'})$.
  By definition of $\sim_{\texttt{susp }\tau}$, for any $k$, $\langle k, rec(V_{0p}, \overline{C \mapsto x.E_C}) \rangle \sim_{\texttt{susp }\tau} rec(V_0', \overline{C \mapsto x.E_C'})$.
  So by definition of $\sim_{\tau_0 \times \tau_1}$, $\langle 0, \langle V_{0p}, rec(V_{0p}, C \mapsto x.E_C) \rangle\rangle \sim_{\phi[\delta \times \texttt{susp }\tau]} \langle V_0', rec(V_0', \overline{C \mapsto x.E_C'})\rangle$.
  So by \ref{lem:relatedmap}, $\forall k. \langle k, map^\Phi(y, \langle y, rec(y, \overline{C \mapsto x.E_C})\rangle, V_{0p}) \sim_{\phi[\delta \times \texttt{susp }\tau]} map^\Phi(y, \langle y, rec(y, \overline{C \mapsto x.E_C'}) \rangle, V_0')$.
  Let $\langle 0, map^\Phi(y, \langle y, rec(y, \overline{C \mapsto x.E_C})\rangle, V_{0p}) \downarrow V_1$.
  Let $map^\Phi(y, \langle y, rec(y, \overline{C \mapsto x.E_C'}) \rangle, V_0') \downarrow V_1'$.
  By \ref{lem:relatedstep}, $V_1 \sim_{\phi[\delta \times \texttt{susp }\tau]} V_1'$.

  If $\Phi = t$, then $map^\Phi(y, \langle y, rec(y, \overline{C \mapsto x.E_C})\rangle, V_{0p}) = V_{0p}$.
  Similarly, $map^\Phi(y, \langle y, rec(y, \overline{C \mapsto x.E_C'})\rangle V_0') = V_0'$.
  So in this case $V_0 = V_1$ and $V_0' = V_1'$.
  We have already established $V_0 \sim_\tau V_0'$.

  So in both cases $V_1 \sim_{\phi[\delta \times \texttt{susp }\tau]} V_1'$.

  By definition of the relation $E_C[V_{1p}/x] \sim_\tau E_C'[V_1'/x]$.
  Let $E_C[V_{1p}/x] \downarrow V_2$ and $E_C'[V_1'/x] \downarrow V_2'$.
  By \ref{lem:relatedstep}, $V_2 \sim_\tau V_2'$.
  So by \ref{lem:relatedstepback}, $rec(E_p, \overline{C \mapsto x.E_C}) \sim_\tau rec(E', \overline{C \mapsto x.E_C'})$.
\end{proof}
%
Our theorem is that for all well-typed terms in the source language, the
complexity translation of the term is related to the pure potential translation
of that term.
%
\begin{theorem}[Distinct Recurrence]
  \[ \gamma \vdash e : \tau \implies \|e\| \sim_\tau |e| \]
\end{theorem}
%
\begin{proof}
  Our proof is by induction on the typing derivation $\gamma \vdash e : \tau$.

  Case \AxiomC{}\UnaryInfC{$\gamma, x : \sigma \vdash x : \sigma$}\DisplayProof.
  Then by definition of the logical relation, $\forall k, \langle k, \Theta(x) \rangle \sim_\sigma \Theta'(x)$.
  Since $\|x\| = \langle 0, x \rangle$ and $|x| = x$, we have $\langle 0, x \rangle \sim_\sigma x$.

  Case \AxiomC{}\UnaryInfC{$\gamma \vdash e : Unit$}\DisplayProof.
  By definition, $\|e\| \sim_{\texttt{Unit}} |e|$ always.

  Case \AxiomC{$\gamma \vdash e_0 : \tau_0$}\AxiomC{$\gamma \vdash e_1 : \tau_1$}\BinaryInfC{$\gamma \vdash \langle e_0, e_1 \rangle : \tau_0 \times \tau_1$}\DisplayProof
  By the induction hypothesis, $\|e_0\| \sim_{\tau_0} |e_0|$ and $\|e_1\| \sim_{\tau_1} |e_1|$.
  By \ref{lem:ignorecost}, $\forall k, \langle k, \|e_0\|_p \rangle \sim_{\tau_0} |e_0|$
  and $\forall k, \langle k, \|e_1\|_p \rangle \sim_{\tau_1} |e_1|$.
  So by definition, $\|\langle e_0, e_1 \rangle \| \sim_{\tau_0 \times \tau_1} |\langle e_0, e_1 \rangle |$.

  Case \AxiomC{$\gamma \vdash e_0 : \tau_0 \times \tau_1$}\AxiomC{$\gamma, x_0 : \tau_0, x_1 : \tau_1 \vdash e_1 : \tau$}\BinaryInfC{$\gamma \vdash split(e_0, x_0.x_1.e_1) : \tau$}\DisplayProof
  By the induction hypothesis, $\|e_0\| \sim_{\tau_0 \times \tau_1} |e_0|$ and $\|e_1\| \sim_\tau |e_1|$.
  From $\|e_0\| \sim_{\tau_0 \times \tau_1} |e_0|$ it follows by definition that 
    $\forall k, \langle k, \pi_0 \|e_0\|_p \rangle \sim_{\tau_0} \pi_0 |e_0|$ and
    $\forall k, \langle k, \pi_1 \|e_1\|_p \rangle \sim_{\tau_1} \pi_1 |e_1|$.
  The complixity translation is $\|split(e_0, x_0.x_1.e_1)\| = \|e_0\|_c +_c \|e_1\|[\pi_0\|e_0\|_p/x_0, \pi_1\|e_1\|_p/x_1]$.
  The pure potential translation is $|split(e_0, x_0.x_1.e_1)| = |e_1|[\pi_0|e_0|/x_0, \pi_1|e_0|/x_1]$.
  By \ref{lem:ignorecost}, it suffices to show $\|e_1\|[\pi_0\|e_0\|_p/x_0, \pi_1\|e_1\|_p/x_1] \sim_\tau |e_1|[\pi_0|e_0|/x_0, \pi_1|e_0|/x_1]$
  By definition of the relation, it suffices to show $\|e_1\| \sim_\tau |e_1|$,
    $\forall k, \langle k, \pi_0 \|e_0\|_p \rangle \sim_{\tau_0} \pi_0 |e_0|$,
    and $\forall k, \langle k, \pi_1 \|e_0\|_p \rangle \sim_{\tau_1} \pi_1 |e_0|$.
  Since we have already establed all three conditions, we have $\|split(e_0, x_0. x_1.e_1)\| \sim_\tau |split(e_0,x_0.x_1.e_1)|$.


  Case \AxiomC{$\gamma, x : \sigma \vdash e : \tau$}\UnaryInfC{$\gamma \vdash \lambda x.e : \sigma \to \tau$}\DisplayProof
  By the induction hypothesis $\|e\|\sim_\tau|e|$.
  The complexity translation is $\|\lambda x.e\| = \langle 0, \lambda x.\|e\|\rangle$.
  The pure potential translation is $|\lambda x.e| = \lambda x.|e|$.
  Let $E_0 : \|\sigma\|$ and $E_0' : |\sigma|$ be complexity language terms such that $E_0 \sim_\sigma E_0'$.
  Then $\langle 0, \lambda x.\|e\|\rangle\ E_0 \to \langle 0 + E_{0c}, \|e\|[x \mapsto E_0]\rangle$
    and $\lambda x.|e|\ E_0' \to |e|[x \mapsto E_0']$.
  Since $\|e\| \sim_\tau |e|$ and $E_0 \sim_\sigma E_0'$, $\|e\|[x \mapsto E_0] \sim_\tau |e|[x \mapsto E_0']$.
  By \ref{lem:relatedstepback}, $\langle 0, \lambda x. \|e\| \rangle\ E_0 \sim_\tau (\lambda x.|e|)\ E_0'$.
  So by definition $\langle 0, \lambda x. \|e\| \rangle \sim_{\sigma \to \tau} \lambda x. |e|$.
  So $\|\lambda x.e\| \sim_{\sigma \to \tau} |\lambda x.e|$.

  Case \AxiomC{$\gamma \vdash e_0 : \sigma \to \tau$}\AxiomC{$\gamma \vdash e_1 : \sigma$}\BinaryInfC{$\gamma \vdash e_0\ e_1 : \tau$}\DisplayProof
  The complexity translation is $\|e_0\ e_1\| = (1 + \|e_0\|_c + \|e_1\|_c) +_c \|e_0\|_p \|e_1\|_p$.
  The pure potential translation is $|e_0\ e_1| = |e_0| |e_1|$.
  By \ref{lem:ignorecost}, it suffices to show $\|e_0\|_p \|e_1\|_p \sim_\tau |e_0||e_1|$.
  By the induction hypothesis, $\|e_0\| \sim_{\sigma \to \tau} |e_0|$ and $\|e_1\| \sim_\sigma |e_1|$.
  By definition, $\|e_0\|_p \|e_1\|_p \sim_{\tau} |e_0| |e_1|$.

  Case \AxiomC{$\gamma \vdash e : \tau$}\UnaryInfC{$\gamma \vdash delay(e) : susp\ \tau$}\DisplayProof
  By the induction hypothesis $\|e\| \sim_\tau |e|$.
  So $\langle 0, \|e\|\rangle \sim_{\texttt{susp }\tau} |e|$.
  The complexity translation is $\|delay(e)\| = \langle 0, \|e\|\rangle$.
  The pure potential translatio is $|delay(e)| = |e|$.
  So $\|delay(e)\| \sim_{\texttt{susp }\tau} |delay(e)|$.

  Case \AxiomC{$\gamma \vdash e : susp\ \tau$}\UnaryInfC{$\gamma \vdash force(e) : \tau$}\DisplayProof
  By the induction hypothesis $\|e\| \sim_{\texttt{susp }\tau} |e|$.
  So by definition of the relation at \texttt{susp} type, $\|e\|_p \sim_\tau |e|$.
  By \ref{lem:ignorecost}, $\forall k, \langle k, \|e\|_{pp} \rangle \sim_\tau |e|$.
  The complexity translation is $\|force(e)\| = \|e\|_c +_c \|e\|_p$.
  The pure potential translation is $|force(e)| = |e|$.
  So $\|e\|_c +_c \|e\|_p \sim_\tau |e|$.
  So $\|force(e)\| \sim_\tau |force(e)|$.

  Case \AxiomC{$\gamma \vdash e_0 : \sigma$}\AxiomC{$\gamma, x : \sigma \vdash e_1 : \tau$}\BinaryInfC{$\gamma \vdash let(e_0, x.e_1) : \tau$}\DisplayProof
  By the induction hypothesis $\|e_0\| \sim_\sigma |e_0|$ and $\|e_1\| \sim_\tau |e_1|$.
  So $\|e_1\|[\|e_0\|_p/x] \sim_\tau |e_1|[|e_0|/x]$.
  By \ref{lem:ignorecost}, $\forall k, \langle k, \|e_1\|_p[\|e_0\|_p/x] \rangle \sim_\tau |e_1|[|e_0|/x]$.
  The complexity translation is $\|let(e_0, x.e_1)\| = \|e_0\|_c +_c \|e_1\|[\|e_0\|_p/x]$.
  The pure potential translation is $|let(e_0, x.e_1)| = |e_1|[|e_0|/x]$.
  So $\|let(e_0, x.e_1)\|  \sim_\tau |let(e_0, x.e_1)|$.

  Case \AxiomC{$\gamma, x : \tau_0 \vdash v_1 : \tau_1$}\AxiomC{$\gamma \vdash v_0 : \phi[\tau_0]$}\BinaryInfC{$\gamma \vdash map^\phi(x.v_1, v_0) : \phi[\tau_1]$}\DisplayProof
  By the induction hypothesis $\|v_1\| \sim_{\tau_1} |v_1|$ and $\|v_0\| \sim_{\phi[\tau_0]} |v_0|$.
  By \ref{lem:relatedmap}, $\forall k, \langle k, map^\Phi(x.\|v_1\|_p, \|v_0\|_p)\rangle \sim_{\phi[\tau_1]} map^\Phi(x.|v_1|, |v_0|)$.
  The complexity translation is $\|map^\phi(x.v_1, v_0)\| = \langle 0, map^\Phi(x.\|v_0\|_p, \|v_1\|_p)\rangle$.
  The pure potential translation is $|map^\phi(x.v_1, v_0| = map^\Phi(x, |v_0|, |v_1|)$.
  So we have $\|map^\phi(x.v_1, v_0)\| \sim_{\phi[\tau_1]} |map^\phi(x.v_1, v_0|$.

  Case \AxiomC{$\gamma \vdash e_0 : \delta$}\AxiomC{$\forall C (\gamma, x: \phi_C[\delta \times susp\ \tau] \vdash e_c : \tau$}\BinaryInfC{$\gamma \vdash rec^\delta(e_0, \overline{C \mapsto x.e_C}) : \tau$}\DisplayProof
  By the induction hypothesis $\|e_0\| \sim_\delta |e_0|$ and $\forall C, \|e_c\| \sim_\tau |e_c|$.
  By \ref{lem:ignorecost}, $\forall k, \langle k \|e_C\| \sim_\tau |e_c$, so $1 +_c \|e_C\| \sim_\tau |e_c|$.
  So by \ref{lem:relatedrec}, $rec(\|e_0\|_p, \overline{C \mapsto x.1 +_c \|e_C\|}) \sim_\tau rec(|e_0|, \overline{C \mapsto x.1 +_c |e_C|})$.
  So by \ref{lem:ignorecost}, $\|e_0\|_c +_c rec(\|e_0\|_p, \overline{C \mapsto x.1 +_c \|e_C\|}) \sim_\tau rec(|e_0|, \overline{C \mapsto x.1 +_c |e_C|})$

  Case \AxiomC{$\gamma \vdash e : \phi[\delta]$}\UnaryInfC{$\gamma \vdash C e : \delta$}\DisplayProof
  By the induction hypothesis, $\|e\| \sim_{\phi[\delta]} |e|$.
  There exists $V, V'$ such that $\|e\| \downarrow V$ and $|e| \downarrow V'$.
  By \ref{lem:relatedstep} $V \sim_{\phi[\delta]} V'$.
  Since $\|e\| \downarrow V$, $\langle k, C\ \|e\| \rangle \downarrow \langle k, C\ V_p$.
  Similarly, since $|e| \downarrow V'$, $C |e| \downarrow C V'$.
  So by definition we have $\langle k, C\ \|e\|\rangle \sim_\delta C |e|$.
  The complexity translation is $\|C e\| = \langle \|e\|, C\|e\|_p\rangle$.
  The pure potential translation is $|C e| = C |e|$.
  Therefore by \ref{lem:ignorecost}, $\|C e\| \sim_\delta |C e|$.
\end{proof}


\chapter{Conclusions and Future Work}

This is the conclusion.

\section{Future Work}
One of the criticisms leveed against traditional complexity analysis is there
is no formal connection between the source language program and the recurrence
for its cost. Although the method from \citet{Danner2015} does provide the
formal connection, the translation from source language to complexity language
and subsequent interpretation is tedious and prone to errors. Automation of the
source to complexity translation and interpretation could easily be automated.
Both transformations are simply forms of cross compilation between different
languages. This leaves only the process of obtaining closed form solutions of
recurrences initially obtained from the interpretation. Solving the recurrences
requires a significant amount of cleverness. For simpler recurrences such as
\T{list map}, the recurrences are straightforward to solve. However for more
complex functions, the recurrences may be very difficult to solve. Consider a
function that calculates the $n$th term of the Fibonacci sequence. Under an
interpretation of natural numbers as the number of successor constructors, the
potential of this function is the $n$th Fibonacci number. The closed form
solution to the potential recurrence is
\[
  \frac{(1 + \sqrt{5})^n + (1 - \sqrt{5})^n}{2^n\sqrt{5}}
\]
Obtaining this solution is not at all obvious.


\bibliography{bibliography}
\bibliographystyle{plainnat}
\end{document}
