\section{Reverse}

\subsection*{Source Language}

\begin{lstlisting}
datatype intlist = Nil of unit | Cons of int $\times$ intlist
\end{lstlisting}
The following function reverses a list on $\Theta(n^2)$ time.
It walks down a list, appending the head of the list to the end of the result of recursively calling itself on the tail of the list.
\begin{lstlisting}[frame=single]
rev = $\lambda$xs.rec(xs, Nil $\mapsto$ Nil
                , Cons $\mapsto \langle$x,$\langle$xs',r$\rangle \rangle$.
                       rec(force(r), Nil $\mapsto$ Cons$\langle$x, Nil$\rangle$
                                   , Cons $\mapsto \langle$y,$\langle$ys,rys$\rangle \rangle$. Cons$\langle$y,force(rys)$\rangle$))
\end{lstlisting}


\subsection*{Complexity Language}
The translation into the complexity language is
\begin{lstlisting}
rev = $\langle$0,$\lambda$xs.rec(xs, Nil $\mapsto$ Nil
                  , Cons $\mapsto \langle$x,$\langle$xs',r$\rangle \rangle$.
                       rec(r, Nil $\mapsto$ Cons$\langle$x, Nil$\rangle$
                            , Cons $\mapsto \langle$y,$\langle$ys,rys$\rangle \rangle$. Cons$\langle$y,rys$\rangle$))$\rangle$
\end{lstlisting}

It is more interesting if we consider the translation of \texttt{rev} applied to some \texttt{xs:intlist}.
The translation of this function into the complexity language proceeds as follows.
We begin with translating the  outer \texttt{rec} construct.

\begin{lstlisting}
rev xs  = 1 $+$ $\|$xs$\|_c$ $+_c$ rec($\|$xs$\|_p$
                         , Nil $\mapsto$ 1 $+_c$ $\|$Nil$\|$
                         , Cons $\mapsto \langle$x,$\langle$xs',r$\rangle \rangle$. 1 $+_c$ $\|$rec($\dots$)$\|$)
\end{lstlisting}

The cost of the translation of an \texttt{intlist} is zero, and the potential of the translation of an \texttt{intlist} is the list itself.
\begin{lstlisting}
rev xs = 1 $+_c$ rec(xs, Nil $\mapsto$ $\langle$1,Nil$\rangle$ , Cons $\mapsto \langle$x,$\langle$xs',r$\rangle \rangle$. 1 $+_c$ $\|$rec($\dots$)$\|$)
\end{lstlisting}

Next we translate the inner \texttt{rec}.
\begin{lstlisting}
$\|$rec(force(r), Nil $\mapsto$ Cons($\langle$x, Nil$\rangle$)
                , Cons $\mapsto \langle$y,$\langle$ys,rys$\rangle \rangle$. Cons$\langle$y,force(rys)$\rangle$))$\|$
\end{lstlisting}

Since \texttt{x}, \texttt{xs'}, \texttt{r} are terms in the complexity language, they do not need to be translated.
First we apply the rules for \texttt{rec} and \texttt{force}.
\begin{lstlisting}
r$_c$ $+_c$ rec(r$_p$, Nil $\mapsto$ 1 $+_c$ $\|$Cons($\langle$x, Nil$\rangle$)$\|$
          , Cons $\mapsto \langle$y,$\langle$ys,rys$\rangle \rangle$. 1 $+_c$ $\|$Cons$\langle$y,force(rys)$\rangle\|$)
\end{lstlisting}

The cost of the \texttt{Cons} constructor is zero, and the translation of \texttt{force} is just the translation of its argument.
\begin{lstlisting}
r$_c$ $+_c$ rec(r$_p$, Nil $\mapsto$ $\langle$1,Cons$\langle$x$_p$, Nil$\rangle\rangle$
           , Cons $\mapsto \langle$y,$\langle$ys,rys$\rangle \rangle$. $\langle$1 $+$ rys$_c$,Cons$\langle$y$_p$,rys$_p$$\rangle\rangle$)
\end{lstlisting}

Putting the pieces together, we get

\begin{lstlisting}[frame=single]
$\|$rev xs$\|$ = 1 $+_c$ rec(xs$_p$
                   , Nil $\mapsto$ $\langle$1,Nil$\rangle$
                   , Cons $\mapsto \langle$x,$\langle$xs',r$\rangle \rangle$. 1 $+$ r$_c$ $+_c$
                        rec(r$_p$, Nil $\mapsto$ $\langle$1,Cons$\langle$x$_p$, Nil$\rangle\rangle$
                             , Cons $\mapsto \langle$y,$\langle$ys,rys$\rangle \rangle$. $\langle$1 $+$ rys$_c$,Cons$\langle$y$_p$,rys$_p$$\rangle\rangle$))
\end{lstlisting}


\subsection*{Interpretation}
We intepret the size of an \texttt{intlist} to be the number of constructors.

$\llbracket$ \texttt{intlist} $\rrbracket$ = $\mathbb{N}^\infty$\\
$D^{intlist} = \{\ast\} + \{1\} \times \mathbb{N}^\infty$\\
$size_{intlist}(\texttt{Nil}) = 0$\\
$size_{intlist}(\texttt{Cons(1,n)}) = 1 + n$\\

Then $\llbracket \| \texttt{rev xs} \|_c \rrbracket = 1 + g(\|xs\|_p)$, where
\[g(n) = \llbracket rec(z, Nil \mapsto 1, Cons \mapsto \langle x, \langle xs',r\rangle \rangle.1 + r_c + h(r_p))\rrbracket \{z \mapsto n\}\]
\[h(n) = \llbracket rec(z, Nil \mapsto 1, Cons \mapsto \langle y, \langle ys',r\rangle \rangle.1 + r_c \rrbracket \{z \mapsto n\}\]

We calculate that $h(0)=1$ and for $n > 0$, $h(n) = 1 + h(n-1)$.
$g(0) = 1$ and for $n > 0$, $g(n) = 1 + g(n-1) + h(n-1)$
