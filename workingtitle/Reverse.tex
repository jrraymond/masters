\chapter{Reverse}
%
Here we present the naive implementation of list reverse. The naive
implementation reverses a list in quadratic time as opposed to linear time.
%
\begin{align*}
  \T{datatype list} &= \T{Nil of unit}\ |\ \T{Cons of int} \times \T{list}
\end{align*}
%
The implementation walks down a list, appending the head of the list to the end
of the result of recursively calling itself on the tail of the list. We use the
syntactic sugar introduced earlier. \T{rev} uses the auxilary function
\T{snoc}.  \T{snoc} appends an item to the end of a list.
%
\begin{flalign*}
  \T{snoc} &= \lambda xs.\lambda x.\T{rec}(xs, \T{Nil} \mapsto \T{Cons} \langle x, \T{Nil} \rangle, \\
           &\quadsix \T{Cons}  \mapsto \langle y, \langle ys,r \rangle \rangle . \T{Cons} \langle y,\T{force}(r) \rangle)
\end{flalign*}
%
The quadratic time implementation of reverse recurses on the list, appending
the head of the list to the recursively reversed tail of the list.
%
\begin{flalign*}
  rev &= \lambda xs.\T{rec}(xs, \T{Nil} \mapsto \T{Nil}, \\
      &\quadfive \T{Cons} \mapsto \langle x, \langle xs',r\rangle \rangle. \T{snoc}\ \T{force}(r)\ x)
\end{flalign*}
%
%
\section{Translation}
%
% BEGIN SNOC TRANSLATION ========================================
%
\subsection{\T{snoc} Translation}
%
First we translate the function \T{snoc}. To do so we apply the rule for
translating an abstraction two times. Recall the rule is
$\|\lambda x.e\| = \langle 0, \lambda x.\|e\|\rangle$.
%
\begin{flalign*}
  \|\T{snoc}\| &= \|\lambda xs.\lambda x.\T{rec}(xs, \T{Nil} \mapsto \T{Cons} \langle x, \T{Nil} \rangle, \\
               &\quadsix \T{Cons}  \mapsto \langle y, \langle ys,r \rangle \rangle . \T{Cons} \langle y,\T{force}(r) \rangle) \| \\
               &= \langle 0, \lambda xs.\|\lambda x.\T{rec}(xs, \T{Nil} \mapsto \T{Cons} \langle x, \T{Nil} \rangle, \\
               &\quadsix \T{Cons}  \mapsto \langle y, \langle ys,r \rangle \rangle . \T{Cons} \langle y,\T{force}(r) \rangle)\|\rangle \\
               &= \langle 0, \lambda xs.\langle 0, \lambda x.\|\T{rec}(xs, \T{Nil} \mapsto \T{Cons} \langle x, \T{Nil} \rangle, \\
               &\quadsix \T{Cons}  \mapsto \langle y, \langle ys,r \rangle \rangle . \T{Cons} \langle y,\T{force}(r) \rangle)\|\rangle\rangle \\
               %
               &\text{Next we apply the rule for translating a \T{rec}.} \\
               %
               &= \langle 0, \lambda xs.\langle 0, \lambda x.\|xs\|_c +_c \T{rec}(\|xs\|_p, \T{Nil} \mapsto 1 +_c \|\T{Cons} \langle x, \T{Nil} \rangle\|, \\
               &\quadsix \T{Cons}  \mapsto \langle y, \langle ys,r \rangle \rangle . 1 +_c \|\T{Cons} \langle y,\T{force}(r) \rangle\|)\rangle\rangle \\
               %
               &\text{$xs$ is a variable, so its translation is $\langle 0,xs \rangle$.} \\
               %
               &= \langle 0, \lambda xs.\langle 0, \lambda x.\langle 0,xs\rangle_c +_c \T{rec}(\langle 0,xs\rangle_p, \T{Nil} \mapsto 1 +_c \|\T{Cons} \langle x, \T{Nil} \rangle\|, \\
               &\quadsix \T{Cons}  \mapsto \langle y, \langle ys,r \rangle \rangle . 1 +_c \|\T{Cons} \langle y,\T{force}(r) \rangle\|)\rangle\rangle \\
               %
               &\text{We take the cost and potential projections of the translated term.} \\
               %
               &= \langle 0, \lambda xs.\langle 0, \lambda x.\T{rec}(xs, \T{Nil} \mapsto 1 +_c \|\T{Cons} \langle x, \T{Nil} \rangle\|, \\
               &\quadsix \T{Cons}  \mapsto \langle y, \langle ys,r \rangle \rangle . 1 +_c \|\T{Cons} \langle y,\T{force}(r) \rangle\|)\rangle\rangle \\
\end{flalign*}
%
We will translate $\T{Cons}\langle x,\T{Nil}\rangle$. In order to do so we will
first translate $\langle x,\T{Nil}\rangle$.
%
\begin{flalign*}
  \|\langle x,\T{Nil}\rangle\| &= \langle \|x\|_c + \|\T{Nil}\|_c, \langle\|x\|_p,\|\T{Nil}\|_p\rangle \\
       &\text{$x$ is a variable, so its translation is $\langle 0,x\rangle$.} \\
       &\text{The translation of \T{Nil} is $\langle 0,\T{Nil}\rangle$.} \\
       &= \langle \langle 0,x\rangle_c + \langle 0,\T{Nil}\rangle_c, \langle \langle 0,x\rangle_p,\langle 0,\T{Nil}\rangle_p\rangle\rangle \\
       &= \langle 0, \langle x,\T{Nil}\rangle\rangle
\end{flalign*}
%
We use the result in translation of $\T{Cons}\langle x,\T{Nil}\rangle$.
%
\begin{flalign*}
  \quad \|\T{Cons}\langle x,\T{Nil}\rangle\| &= \langle \|\langle x,\T{Nil}\rangle\|_c, \T{Cons} \|\langle x,\T{Nil}\rangle\|_p\rangle \\
     &= \langle \langle 0, \langle x,\T{Nil}\rangle\rangle_c, \T{Cons} \langle 0, \langle x,\T{Nil}\rangle\rangle_p\rangle \\
     &= \langle 0, \T{Cons} \langle x,\T{Nil}\rangle\rangle
\end{flalign*}
%
Now that we have translated $\T{Cons}\langle x,\T{Nil}\rangle$ we return can
substitute it in to the translation of \T{snoc} to complete the translation of
the \T{Nil} branch of the \T{rec}.
%
\begin{flalign*}
   &= \langle 0, \lambda xs.\langle 0, \lambda x.\T{rec}(xs, \T{Nil} \mapsto 1 +_c \langle 0, \T{Cons} \langle x,\T{Nil}\rangle\rangle \\
   &\quadsix \T{Cons}  \mapsto \langle y, \langle ys,r \rangle \rangle . 1 +_c \|\T{Cons} \langle y,\T{force}(r) \rangle\|)\rangle\rangle \\
   &\text{we can expand the $+_c$ macro to simplify the \T{Nil} branch.} \\
   &= \langle 0, \lambda xs.\langle 0, \lambda x.\T{rec}(xs, \T{Nil} \mapsto \langle 1, \T{Cons} \langle x,\T{Nil}\rangle\rangle \\
   &\quadsix \T{Cons}  \mapsto \langle y, \langle ys,r \rangle \rangle . 1 +_c \|\T{Cons} \langle y,\T{force}(r) \rangle\|)\rangle\rangle \\
\end{flalign*}
%
To complete the translation of \T{snoc} we must translate
$\T{Cons} \langle y,\T{force}(r) \rangle$. To do so we first translate
$\langle y,\T{force}(r)\rangle$.
%
\begin{flalign*}
  \|\langle y,\T{force}(r)\rangle\| &= \langle \|y\|_c + \|\T{force}(r)\|_c, \langle \|y\|_p,\|\T{force}(r)\|_p\rangle\rangle \\
                                    &\text{$y$ is a variable, so } \\
                                    &\quad \|y\| = \langle 0,y\rangle \\
                                    &\quad \T{force}(r) = \|r\|_c +_c \|r\|_p \\
                                    &\quad\text{$r$ is also a variable.} \\
                                    &\quad\quadthree = \langle 0,r\rangle_c +_c \langle 0,r\rangle_p \\
                                    &\quad\quadthree = 0 +_c r = r \\
                                    &= \langle \langle 0,y\rangle_c + r_c,\langle \langle 0,y\rangle_p,r_p\rangle\rangle \\
                                    &= \langle r_c,\langle y,r_p\rangle\rangle
\end{flalign*}
%
We use this in our translation of $\T{Cons} \langle y,\T{force}(r) \rangle$.
%
\begin{flalign*}
  \T{Cons} \langle y,\T{force}(r) \rangle &= \langle \|\langle y,\T{force}(r)\rangle\|_c, \T{Cons} \|\langle y,\T{force}(r)\rangle\|_p\rangle \\
                                          &= \langle \langle r_c,\langle y,r_p\rangle\rangle_c, \langle r_c,\langle y,r_p\rangle\rangle_p\rangle \\
                                          &= \langle r_c, \T{Cons} \langle y,r_p\rangle\rangle
\end{flalign*}
%
We substitute this result into our translation of \T{rev}
\begin{flalign*}
  \|\T{snoc}\| &= \langle 0, \lambda xs.\langle 0, \lambda x.\T{rec}(xs, \T{Nil} \mapsto \langle 1,\T{Cons} \langle x, \T{Nil} \rangle\rangle, \\
               &\quadsix \T{Cons}  \mapsto \langle y, \langle ys,r \rangle \rangle . 1 +_c \|\T{Cons} \langle y,\T{force}(r) \rangle\|)\rangle\rangle \\
               &= \langle 0, \lambda xs.\langle 0, \lambda x.\T{rec}(xs, \T{Nil} \mapsto \langle 1,\T{Cons} \langle x, \T{Nil} \rangle\rangle, \\
               &\quadsix \T{Cons}  \mapsto \langle y, \langle ys,r \rangle \rangle . 1 +_c \langle r_c, \T{Cons} \langle y,r_p\rangle\rangle)\rangle\rangle \\
               &= \langle 0, \lambda xs.\langle 0, \lambda x.\T{rec}(xs, \T{Nil} \mapsto \langle 1,\T{Cons} \langle x, \T{Nil} \rangle\rangle, \\
               &\quadsix \T{Cons}  \mapsto \langle y, \langle ys,r \rangle \rangle .\langle 1 + r_c, \T{Cons} \langle y,r_p\rangle\rangle)\rangle\rangle
\end{flalign*}
%
% END SNOC TRANSLATION
%
% ========================================
%
% BEGIN REV TRANSLATION
%
\subsection{\T{rev} Translation}
%
The translation into the complexity language follows
%
First we apply the abstraction translation rule:
$\|\lambda x.e\| = \langle 0, \lambda x.\|e\|\rangle$.
%
\begin{flalign*}
  \|rev\| &= \|\lambda xs.\T{rec}(xs, \T{Nil} \mapsto \T{Nil}, \\
          &\quadfive \T{Cons} \mapsto \langle x, \langle xs',r\rangle \rangle. \T{snoc}\ \T{force}(r)\ x)\\
          &= \langle 0,\lambda xs.\|\T{rec}(xs, \T{Nil} \mapsto \T{Nil}, \\
          &\quadfive \T{Cons} \mapsto \langle x, \langle xs',r\rangle \rangle. \T{snoc}\ \T{force}(r)\ x)\|\rangle\\
          &\text{Next we apply the \T{rec} translation rule.} \\
          &= \langle 0,\lambda xs.\|xs\|_c +_c \T{rec}(\|xs\|_p, \T{Nil} \mapsto 1 +_c \|\T{Nil}\|, \\
          &\quadfive \T{Cons} \mapsto \langle x, \langle xs',r\rangle \rangle. 1 +_c \|\T{snoc}\ \T{force}(r)\ x\|)\rangle\\
          &\text{As before, the translation of the variable $xs$ is $\langle 0, xs\rangle$,}\\
          &\text{and the translation of \T{Nil} is $\langle 0,\T{Nil}\rangle$.} \\
          &= \langle 0,\lambda xs.\langle 0,xs\rangle_c +_c \T{rec}(\langle 0, xs\rangle_p, \T{Nil} \mapsto 1 +_c \langle 0,\T{Nil}\rangle, \\
          &\quadfive \T{Cons} \mapsto \langle x, \langle xs',r\rangle \rangle. 1 +_c \|\T{snoc}\ xs\ x\|)\rangle\\
          &\text{We take the projections of the translated expressions and expand the $+_c$ macro.} \\
          &= \langle 0,\lambda xs.\T{rec}(xs, \T{Nil} \mapsto \langle 1,\T{Nil}\rangle, \\
          &\quadfive \T{Cons} \mapsto \langle x, \langle xs',r\rangle \rangle. 1 +_c \|\T{snoc}\ xs\ x\|)\rangle\\
          %
          &\text{Next we translate the application \T{snoc force(r) x}.}\\
          &\quad \|\T{snoc}\ \T{force}(r)\ x\| = (1 + \|\T{snoc force}(r)\|_c + \|x\|_c) +_c \|\T{snoc}\ \T{force}(r)\|_p \|x\|_p \\
          &\quad \|\T{snoc}\ \T{force}(r)\| = (1 + \|\T{snoc}\|_c + \|\T{force}(r)\|_c) +_c \|\T{snoc}\|_p \|\T{force}(r)\|_p \\
          %
          &\quad\text{Next we translate the \T{force}.} \\
          &\quadthree \|\T{force}(r)\| = \|r\|_c +_c \|r\|_p \\
          &\quadthree \text{$r$ is also a variable, so its translation is $\langle 0,xs\rangle$. The cost of $\|\T{snoc}\|$ is 0.} \\
          &\quadsix = \langle 0,r \rangle_c +_c \langle 0,r\rangle_p = r \\
          %
          &\quad \|\T{snoc}\ \T{force}(r)\| = (1 + 0 + r_c) +_c \|\T{snoc}\|_p\ r_p \\
          &\quad\text{$x$ is a variable so its translation is $\langle 0,x\rangle$.} \\
          &\quad \|\T{snoc force}(r)\ x\| = (1 + \|\T{snoc force}(r)\|_c + \|x\|_c) +_c \|\T{snoc}\ r\|_p \|x\|_pi \\
          &\quadfour = (1 + 1 + r_c + (\|\T{snoc}\|_p\ r_p)_c) +_c (\|\T{snoc}\|_p\ r_p)_p\ x \\
          &\quad\text{The cost of the partially applied function is 0.} \\
          &\quadfour = (2 + r_c) +_c (\|\T{snoc}\|_p\ r_p)_p\ x \\
          %
          &\text{We can use this to complete the translation of the \T{Cons} branch.}\\
          &= \langle 0,\lambda xs.\T{rec}(xs, \T{Nil} \mapsto \langle 1,\T{Nil}\rangle, \\
          &\quadfive \T{Cons} \mapsto \langle x, \langle xs',r\rangle \rangle. 1 +_c ((2 + r_c) +_c (\|\T{snoc}\|_p\ r_p)_p\ x)\rangle\\
          &= \langle 0,\lambda xs.\T{rec}(xs, \T{Nil} \mapsto \langle 1,\T{Nil}\rangle, \\
          &\quadfive \T{Cons} \mapsto \langle x, \langle xs',r\rangle \rangle. (3 + r_c) +_c (\|\T{snoc}\|_p\ r_p)_p\ x)\rangle\\
\end{flalign*}
%
%
It is more interesting if we consider the translation of \T{rev} applied to
some list \T{xs}. The translation of this function into the complexity
language proceeds as follows. First we apply the rule for translating an
application.
%
\begin{flalign*}\
  \|\T{rev}\ xs\| &= (1 + \|\T{rev}\|_c + \|xs\|_c) +_c \|\T{rev}\|_p \|xs\|_p \\
                  &= (1 + \|xs\|_c) +_c \|\T{rev}\|_p\ \|xs\|_p \\
                  &= (1 + \|xs\|_c) +_c (\lambda xs.\T{rec}(xs, \T{Nil} \mapsto \langle 1,\T{Nil}\rangle, \\
                  &\quadfive \T{Cons} \mapsto \langle x, \langle xs',r\rangle \rangle. (3 + r_c) +_c (\|\T{snoc}\|_p\ r_p)_p\ x))\ \|xs\|_p
\end{flalign*}
%
%
%
\section{Interpretation}
%
We intepret the size of an \T{list} to be the number of \T{Cons} constructors.
%
\begin{align*}
  \llbracket \T{list} \rrbracket &= \mathbb{N}^\infty \\
  D^{list} &= \{\ast\} + \{1\} \times \mathbb{N}^\infty \\
  size_{list}(\T{Nil}) &= 0 \\
  size_{list}(\T{Cons(1,n)}) &= 1 + n
\end{align*}
%
%
\subsection{\T{snoc} Interpretation}
%
We interpret $\|\T{snoc xs x}\|$. Recall the translation.
%
\begin{flalign*}
  \|\T{snoc}\ xs\ x\| &= (2 + \|xs\|_c + \|x\|_c) +_c (\|\T{snoc}\|_p\ \|xs\|_p)_p\ \|x\|_p
\end{flalign*}
%
The cost of \T{snoc} is driven by the recursion. We interpret the cost of the
\T{rec} by defining a recurrence $g(n)$. We add $x \mapsto x$ to the
environment, where $x$ is the interpretation of $x$.
%
\begin{flalign*}
  g(n) &= \llbracket \T{rec}(xs, \T{Nil} \mapsto \langle 1,\T{Cons} \langle x, \T{Nil} \rangle\rangle, \\
       &\quadsix \T{Cons}  \mapsto \langle y, \langle ys,r \rangle \rangle .\langle 1 + r_c, \T{Cons} \langle y,r_p\rangle\rangle)\rrbracket \{xs \mapsto n, x \mapsto x\}\\
       &= \bigvee\limits_{size\ z \leq n} case(z, f_{Nil}, f_{Cons}) \\
  \text{where}& \\
  f_{Nil}(\ast) &= \llbracket \langle 1,\T{Cons} \langle x, \T{Nil} \rangle\rangle \rrbracket \{xs \mapsto n, x \mapsto x\}\\
             &= (1, 1) \\
  f_{Cons}(1, m) &= \llbracket \langle 1 + r_c, \T{Cons} \langle y,r_p\rangle\rangle \rrbracket \{xs \mapsto n, x \mapsto x, y \mapsto 1, ys \mapsto m, r \mapsto g(m) \} \\
                 &= (1 + g_c(m), 1 + g_p(m))
\end{flalign*}
%
To eliminate the big maximum operator, we use the same technique as in fast
reverse, by splitting the big maximum into two cases: $size\ z < n$ and $size\
z = n$.
%
\begin{description}
  \item[case $n=0$]\hfill \\
    The only $z$ such that $size\ z \leq 0$ is $\ast$. So $g(0) = f_{Nil}(0) = (1,1)$.
  \item[case $n>0$]\hfill \\
    \begin{align*}
      g(n) &= \bigvee\limits_{size z < n} case(z,f_{Nil},f_{Cons}) \vee \bigvee\limits_{size\ z = n} case(z,f_{Nil},f_{Cons}) \\
           &= g(n-1) \vee (1 + g_c(n-1), 1 + g_p(n-1)) \\
           &\text{Since $\leq$ is symmetric, $g(n-1)\leq (g_c(n-1), g_p(n-1))$, and} \\
           &\text{$(g_c(n-1),g_p(n-1)) < (1 + g_c(n-1),g_p(n-1))$.}\\
           &\leq (1 + g_c(n-1), 1 + g_p(n-1))
    \end{align*}
\end{description}
%
The solution to this recurrence is given in lemma \ref{lem:rev_interp}.
%
\begin{lemma}
  \label{lem:rev_interp}
  $g(n) = (1 + n, 1 + n)$.
\end{lemma}
\begin{proof}
  We prove this by induction on $n$.
  \begin{description}
    \item[case $n=0$]\hfill \\
      $g(0) = (1, 1)$.
    \item[case $n>0$]\hfill \\
      \begin{align*}
        g(n) &= (1 + g_c(n-1), 1 + g_p(n-1)) \\
             &= (1 + (n,n)_c, 1 + (n,n)_p) \\
             &= (1 + n, 1 + n)
      \end{align*}
  \end{description}
\end{proof}
%
This is a closed form solution for the recurrence describing the complexity of
the \T{rec} expression in the body of \T{snoc}. We use this to produce the
equation describing the complexity of $\|\T{snoc}\|$.
%
\begin{flalign*}
  snoc(n, x) &= g(n) = (1 + n, 1 + n)
\end{flalign*}
%
%\begin{flalign*}
%  \llbracket\|\T{snoc}\ xs\ x\|\rrbracket &= \llbracket(2 + \|xs\|_c + \|x\|_c) +_c (\|\T{snoc}\|_p\ \|xs\|_p)_p\ \|x\|_p\rrbracket \\
%                                          &= (2 + \llbracket\|xs\|_c\rrbracket + \llbracket\|x\|_c\rrbracket) \pplus_c \llbracket(\|\T{snoc}\|_p\ \|xs\|_p)_p \|x\|_p\rrbracket\\
%                                          &= (2 + \llbracket\|xs\|
%\end{flalign*}
%
%
\subsection{\T{rev} Interpretation}
%
Recall the translation of \T{rev xs}.
\begin{flalign*}
  \T{rev}\ xs &= (1 + \|xs\|_c) +_c (\lambda xs.\T{rec}(xs, \T{Nil} \mapsto \langle 1,\T{Nil}\rangle, \\
              &\quadfive \T{Cons} \mapsto \langle x, \langle xs',r\rangle \rangle. (3 + r_c) +_c (\|\T{snoc}\|_p\ r_p)_p\ x))\ \|xs\|_p
\end{flalign*}
%
We will interpret the \T{rec} construct first.
%
\begin{flalign*}
  g(n) &= \llbracket \T{rec}(xs, \T{Nil} \mapsto \langle 1,\T{Nil}\rangle, \\
       &\quadfive \T{Cons} \mapsto \langle x, \langle xs',r\rangle \rangle. (3 + r_c) +_c (\|\T{snoc}\|_p\ r_p)_p\ x)\rrbracket \{xs \mapsto n\} \\
       &= \bigvee\limits_{size\ z \leq n} case(z, f_{Nil}, f_{Cons}) \\
       &\text{where}\\
  f_{Nil}(\ast) &= \llbracket \langle 1,\T{Nil}\rangle\rrbracket \{xs \mapsto n\} \\
                &= (1, 0) \\
  f_{Cons}((1, m)) &= \llbracket (3 + r_c) +_c (\|\T{snoc}\|_p\ r_p\ x)\rrbracket \{xs \mapsto n, x \mapsto 1, r \mapsto g(m)\} \\
                   &= (3 + g_c(m)) \pplus_c (\llbracket\|\T{snoc}\|_p\rrbracket \{xs \mapsto n, x \mapsto 1, r \mapsto g(m)\}\ g_p(m)\ 1) \\
                   &= (3 + g_c(m)) \pplus_c snoc(g_p(m), 1) \\
                   &= (3 + g_c(m) + snoc_c(g_p(m), 1), snoc_p(g_p(m), 1))
\end{flalign*}
%
To obtain a solution to this recurrence, we apply the same technique as in the
interpretation of \T{snoc}. We break the recurrence into the case where the
argument is $0$ and when the argument is greater than $0$. Then we eliminate
the big maximum operator by breaking the maximum into cases where $size\ z < n$
and when $size\ z = n$.
%
\begin{description}
  \item[case $n=0$]\hfill \\
    The only $z$ such that $size\ z \leq 0$ is $\ast$.
    \begin{align*}
      g(0) = (1, 0)
    \end{align*}
  \item[case $n>0$]\hfill \\
    \begin{align*}
      g(n) &= \bigvee\limits_{size\ z \leq n} case(z, f_{Nil}, f_{Cons}) \\
           &= \bigvee\limits_{size\ z < n} case(z, f_{Nil}, f_{Cons}) \vee \bigvee\limits_{size\ z = n} case(z, f_{Nil}, f_{Cons}) \\
           &= g(n-1) \vee \bigvee\limits_{size\ z = n} case(z, f_{Nil}, f_{Cons}) \\
           &= g(n-1) \vee (3 + g_c(n-1) + snoc_c(g_p(n-1), 1), snoc_p(g_p(n-1), 1)) \\
           &\text{We substitute the definition of $snoc(n, x)$.} \\
           &= g(n-1) \vee (3 + g_c(n-1) + g_p(n-1) + 1, g_p(n-1) + 1) \\
           &\text{$g_p(n-1)$ is nonnegative, so we can eliminate the max.} \\
           &= (4 + g_c(n-1) + g_p(n-1), 1 + g_p(n-1))
    \end{align*}
\end{description}
%
We use the substitution method to solve the recurrence.
\begin{lemma}
  $g(n) = (4n^2 + 1, n)$
\end{lemma}
\begin{proof}
  We prove this by induction on $n$.
  \begin{description}
    \item[case $n=0$]\hfill \\
      $g(0) = (1, 0) = (4\dot0^2 + 1, 0)$.
    \item[case $n>0$]\hfill \\
      \begin{align*}
        g(n) &= (4 + g_c(n-1) + g_p(n-1), 1 + g_p(n-1)) \\
             &= (4 + (4(n-1)^2 + 1) + n - 1, 1 + n - 1) \\
             &= (4 + 4n^2 - 8n + 4 + n, n) \\
             &=TODO: this was wrong
      \end{align*}
  \end{description}
\end{proof}
