\section{Insertion Sort}
%
Insertion sort is a quadratic time sorting algorithm which sorts a list by
inserting an element from an unsorted segment of a container into a sorted
segment of the container.  Although the asymptotic complexity of insertion sort
is less than the optimal $\mathcal{O}(nlog_2n)$, insertion sort does have
redeeming attributes.  Insertion sort has small constant factors, making it
more efficient on small datasets. Insertion sort may be done in-place
(\citet{Cormen2001}).  The running time of insertion sort is
$\mathcal{O}(n^2)$.
%
\begin{flalign*}
  \T{data list} &= \T{Nil of unit | Cons of int}\times\T{list}
\end{flalign*}
%
\begin{flalign*}
  \T{insert} &= \lambda x.\lambda xs.\T{rec}(xs, \T{Nil} \mapsto \T{Cons}\langle x,\T{Nil}\rangle, \\
             &\quadthree \T{Cons} \mapsto \langle y,\langle ys,r \rangle\rangle.\T{rec}(x \leq y, \T{True} \mapsto \T{Cons} \langle x,\T{Cons}\langle y,ys\rangle\rangle, \\
             &\quadsix \T{False} \mapsto \T{Cons} \langle y,\T{force}(r)\rangle))
\end{flalign*}
%
\subsection{Translation}

The translation of the function \T{insert} is shown below.
%
\begin{flalign*}
  \T{sort} &= \lambda xs.\T{rec}(xs,\T{Nil} \mapsto \T{Nil}, \T{Cons} \mapsto \langle y,\langle ys,r\rangle\rangle.\T{insert}\ y\ \T{force}(r))
\end{flalign*}
%
The translation of the function \T{sort} is shown below.
%
\begin{flalign*}
  \|\T{sort}\| &= \langle 0,\lambda xs.\T{rec}(xs, \T{Nil} \mapsto \T{Nil}, \\
               &\quadthree \T{Cons} \mapsto \langle y,\langle ys,r\rangle\rangle.(3 + r_c) +_c (\|\T{insert}\|_p\ y)_p r_p)
\end{flalign*}
%

\subsection{Interpretation}

We will interpret lists as their lengths.
%
\begin{align*}
    \llbracket list \rrbracket &= \mathbb{N}^\infty \\
    D^{list} &= \{\ast\} + \{1\} \times \mathbb{N}^\infty \\
    size_{list} (\ast) &= 0 \\
    size_{list} ((1,n)) &= 1 + n
\end{align*}
%
The interpretation of the \T{rec} that drives the cost of insert is given below.
%
\begin{align}
  \label{eq:insert_initial_recurrence}
  g(n) &= \bigvee_{size(z) \leq n} case(z, (f_{Nil},f_{Cons})) \\
  &\text{where} \\
  f_{Nil}() &= (1,1) \\
  f_{Cons}((1,m)) &= (3 + (1 \leq\ 1)_c + g_c(m), (2+m) \vee (1+g_p(m)))
\end{align}
%
We can extract the recurrence for the cost and eliminate the big maximum operator.
\begin{equation*}
\label{eq:insert_cost}
c(n) = \begin{cases}
  1 & n = 0 \\
  4 + \pi_0(\pi_1(f\ 1)\ 1) + c(n-1) & n > 0
\end{cases}
\end{equation*}
%
The closed form solution to this recurrence is:
\begin{lemma}
\label{lem:insert_cost}
  $c(n) = (3 + (1 \leq 1)_c)n + 1$
\end{lemma}
%
We also extract a recurrence for the potential and eliminate the big maximum
operator.
%
\begin{equation*}
  \label{eq:insert_potential}
  p(n) = \begin{cases}
    1 & n = 0 \\
    1 + p(n-1) & n > 0
  \end{cases}
\end{equation*}
%
The closed form solution to this recurrence is:
\begin{equation*}
  p(n) = 1 + n
\end{equation*}
%
The interpretation of \T{sort} is:
%
\begin{equation}
  \label{eq:sort_interp0_init}
  g(n) = \bigvee_{size(z)\leq n} case(z,(\lambda(\langle\rangle).(1,0),\lambda(1,m).3 + g_c(m)) +_c \llbracket\|\T{insert}\|\rrbracket\ 1\ g_p(m))
\end{equation}
%
The recurrence with the big maximum operator eliminated is:
%
\begin{equation}
  \label{eq:sort_rec_final}
  g(n) = \begin{cases}
    (1,0) & n=0 \\
    (3 + g_c(n-1) + (\llbracket\|\T{insert}\|\rrbracket\ 1\ g_p(n-1))_c, (\llbracket\|\T{insert}\|\rrbracket\ 1\ g_p(n-1))) & n > 0
  \end{cases}
\end{equation}
%
We extract the recurrence for the potential.
Let $p = \pi_1 \circ g$.
%
\begin{equation}
  \label{eq:sort_rec_potential}
  p(n) = \begin{cases}
    0 & n=0 \\
  (\llbracket\|\T{insert}\|\rrbracket\ 1\ g_p(n-1))) & n > 0
  \end{cases}
\end{equation}
%
The closed form solution to this recurrence is given below.
%
\begin{lemma}
  \label{lem:sort_potential}
  $p(n) = n$
\end{lemma}
%
We extract the recurrence for the cost.
Let $c = \pi_0 \circ g$.
%
\begin{equation}
  \label{eq:sort_rec_cost}
  c(n) = \begin{cases}
    1 & n=0 \\
    3 + g_c(n-1) + (\llbracket\|\T{insert}\|\rrbracket\ 1\ g_p(n-1) & n > 0
  \end{cases}
\end{equation}
%

% ================================================================================
%This recurrence is difficult to work with.
%Specifically, we cannot apply traditional methods of solving it.
%We will manipulate it into a more usable form by eliminating the arbitrary maximum.
%Observe that for $n=0$, $g(n) = f_{Nil}(\langle\rangle) = (1,1)$.
%For $n>0$,
%\begin{align*}
%  g(n) &= \bigvee_{size(z) \leq n} case(z, (f_{Nil},f_{Cons})) &&\\
%  &= g(n-1) \vee \bigvee_{size(z) = n} case(z, (f_{Nil},f_{Cons})) &&\\
%  &= g(n-1) \vee f_{Cons}(n) &&\\
%  &= g(n-1) \vee (4 + \pi_0(\pi_1(f\ 1)\ 1) + \pi_0g(n-1), (1+n) \vee (1+\pi_1g(n-1))) &&\text{$m=n-1$}\\
%  &= (4 + \pi_0(\pi_1(f\ 1)\ 1) + \pi_0g(n-1), (1+n) \vee (1+\pi_1g(n-1))) &&\text{lemma \ref{lem:insert_g_monotonicity}}\\
%  &= (4 + \pi_0(\pi_1(f\ 1)\ 1) + \pi_0g(n-1), 1+\pi_1g(n-1)) &&\text{lemma \ref{lem:insert_potential_inc}}
%\end{align*}
%\begin{lemma}
%  \label{lem:insert_g_monotonicity}
%  $g(n) > g(n-1)$
%\end{lemma}
%\begin{proof}
%  TODO
%\end{proof}
%
%\begin{lemma}
%\label{lem:insert_potential_inc}
%$\pi_1 g(n) > n$
%\end{lemma}
%\begin{proof}
%We prove this by induction on $n$.
%\begin{description}
%  \item[case $n=0$:] $\pi_1g(0) = 1$
%  \item[case $n>0$:]\hfill
%    \begin{align*}
%      \pi_1g(n) &= \pi_1(g(n-1) \vee (4 + \pi_0(\pi_1(f\ 1)\ 1) + \pi_0g(n-1),(1+n) \vee (1 + \pi_1 g(n-1)))) \\
%      &= \pi_1g(n-1) \vee (1+n) \vee (1 + \pi_1 g(n-1)) \\
%      &\geq n-1 \vee (1+n) \vee (1 + n - 1) \\
%      &\geq 1+n \\
%      &> n
%    \end{align*}
%\end{description}
%\end{proof}
%
%Equation \ref{eq:insert_recurrence} shows the extracted recurrence.
%Without the arbitrary maximum, it is much more obvious how to find a solution to the recurrence.
%The recurrence is from a potential to a complexity, consequently we can extract a recurrence for the cost,
%equation \ref{eq:insert_cost}, and a recurrence for the potential, equation \ref{eq:insert_potential}, simply by taking the projections of equation \ref{eq:insert_recurrence}.
%The extracted recurrences for the cost and potential can then be solved by the substitution method.
%\begin{equation}
%  \label{eq:insert_recurrence}
%  g(n) = \begin{cases}
%    (1,1) & n = 0 \\
%    (4 + \pi_0(\pi_1(f\ 1)\ 1) + \pi_0g(n-1), 1+\pi_1g(n-1)) & n > 0
%  \end{cases}
%\end{equation}
%
%The cost recurrence is given by $\pi_0 \circ g$.
%\begin{equation}
%\label{eq:insert_cost}
%c(n) = \begin{cases}
%  1 & n = 0 \\
%  4 + \pi_0(\pi_1(f\ 1)\ 1) + c(n-1) & n > 0
%\end{cases}
%\end{equation}
%
%This recurrence is quite simple to solve.
%The solution and proof of the solution are given in theorem \ref{thm:insert_cost}.
%\begin{theorem}
%\label{thm:insert_cost}
%  $c(n) = (4 + \pi_0(\pi_1(f\ 1)\ 1))n + 1$
%\end{theorem}
%\begin{proof}
%  We prove this by induction on $n$.
%  \begin{description}
%    \item[case $n=0$] $c(0) = \pi_0g(0) = 1$
%    \item[case $n>0$]
%      \begin{align*}
%        c(n) &= 4 + \pi_0(\pi_1(f\ 1)\ 1) + c(n-1) \\
%        &= 4 + \pi_0(\pi_1(f\ 1)\ 1) + (4 + \pi_0(\pi_1(f\ 1)\ 1))(n-1) + 1 \\
%        &= (4 + \pi_0(\pi_1(f\ 1)\ 1)) n + 1
%      \end{align*}
%  \end{description}
%\end{proof}
%The solution tells us the cost of the \T{rec} construct in insert is linear in the size of the list.
%The constant factor cannot be determined because we do not know the cost of applying $f$ to its arguments.
%
%The potential recurrence is given by $\pi_1 \circ g$.
%\begin{equation}
%  \label{eq:insert_potential}
%  p(n) = \begin{cases}
%    1 & n = 0 \\
%    1 + p(n-1) & n > 0
%  \end{cases}
%\end{equation}
%
%\begin{theorem}
%  $p(n) = 1 + n$
%\end{theorem}
%\begin{proof}
%  We prove this by induction on $n$.
%  \begin{description}
%    \item[case $n=0$] $p(0) = 1$
%    \item[case $n>0$] $p(n) = 1 + p(n - 1) = 1 + n$
%  \end{description}
%\end{proof}
%
%
%\subsubsection{Interpretation}
%The \T{rec} construct again drives the cost and potential of \T{sort}.
%The walkthrough of the interpretation of the \T{rec} is given in figure \ref{fig:sort_rec_interp}.
%Equation \ref{eq:sort_interp0_init} shows the initial recurrence extracted.
%\begin{equation}
%  \label{eq:sort_interp0_init}
%  g(n) = \bigvee_{size(z)\leq n} case(z,(\lambda(\langle\rangle).(1,0),\lambda(1,m).4 + \pi_0 g(m)) +_c insert\ f\ 1\ \pi_1g(m))
%\end{equation}
%
%We work the recurrence into a more recognisable form using some manipulation of the big max operator and some facts about $insert$.
%Observe for $n=0$, $g(n) = (1,0)$ and for $n>0$
%\begin{align*}
%  g(n) &= \bigvee_{size(z)\leq n} case(z,(\lambda(\langle\rangle).(1,0),\lambda(1,m).4 + \pi_0 g(m)) +_c insert\ f\ 1\ \pi_1g(m)) \\
%  &= g(n-1) \vee \bigvee_{size(z) = n} case(z,(\lambda(\langle\rangle).(1,0),\lambda(1,m).4 + \pi_0 g(m)) +_c insert\ f\ 1\ \pi_1g(m)) \\
%  &= g(n-1) \vee (4 + \pi_0 g(n-1)) +_c insert\ f\ 1\ \pi_1g(n-1) \\
%  &= g(n-1) \vee (4 + \pi_0 g(n-1) + \pi_0 (insert\ f\ 1\ \pi_1g(n-1)), \pi_1 (insert\ f\ 1\ \pi_1g(n-1))) \\
%  &\text{since $\pi_0 (insert\ f\ 1\ m) > 0$ and $\pi_1 (insert\ f\ 1\ m) = 1 + m$} \\
%  &= (4 + \pi_0 g(n-1) + \pi_0 (insert\ f\ 1\ \pi_1g(n-1)), \pi_1 (insert\ f\ 1\ \pi_1g(n-1)))
%\end{align*}
%
%So our simplified recurrence is
%\begin{equation}
%  \label{eq:sort_rec_final}
%  g(n) = \begin{cases}
%    (1,0) & n=0 \\
%    (4 + \pi_0 g(n-1) + \pi_0 (insert\ f\ 1\ \pi_1g(n-1)), \pi_1 (insert\ f\ 1\ \pi_1g(n-1))) & n > 0
%  \end{cases}
%\end{equation}
%
%From this we can extract recurrences for the cost and the potential simply by taking projections from $g$.
%We begin with the potential because we will require the solution to the potential recurrence to solve the cost recurrence.
%
%Let $p = \pi_1 \circ g$.
%\begin{equation}
%  \label{eq:sort_rec_potential}
%  p(n) = \begin{cases}
%    0 & n=0 \\
%    \pi_1 (insert\ f\ 1\ \pi_1g(n-1))) & n > 0
%  \end{cases}
%\end{equation}
%
%We prove the size of the potential of the output is same as the size of the input.
%In other words, \T{sort} does not change the size of the list.
%\begin{theorem}
%  \label{thm:sort_potential}
%  $p(n) = n$
%\end{theorem}
%\begin{proof}
%  We prove this by straightforward induction on $n$.
%  \begin{description}
%    \item[case $n=0$] $p(0) = 0$\hfill \\
%    \item[case $n>0$] $p(n) = \pi_1(insert\ f\ 1\ pi_g(n-1)) = \pi_1(insert\ f\ 1\ (n-1)) = n$
%  \end{description}
%\end{proof}
%
%Let $c = \pi_0 \circ g$.
%\begin{equation}
%  \label{eq:sort_rec_cost}
%  c(n) = \begin{cases}
%    1 & n=0 \\
%    4 + \pi_0 g(n-1) + \pi_0 (insert\ f\ 1\ \pi_1g(n-1) & n > 0
%  \end{cases}
%\end{equation}
%
%\begin{theorem}
%  \label{thm:sort_cost}
%\end{theorem}
%\begin{proof}
%  We prove this by straightforward induction on $n$.
%  \begin{description}
%    \item[case $n=0$:] $c(0) = 1$
%    \item[case $n>0$:]
%      \begin{align*}
%        c(n) &= 4 + \pi_0 g(n-1) + \pi_0(insert\ f\ 1\ \pi_1g(n-1)) \\
%        &= 4 + \pi_0 g(n-1) + \pi_0(insert\ f\ 1\ (n-1) \\
%        &= 4 + (4 + \pi_0(\pi_1(f\ 1)\ 1))(n-1) + 1) + \pi_0 g(n-1)
%      \end{align*}
%      TODO COMPLETE
%  \end{description}
%\end{proof}
%
%
