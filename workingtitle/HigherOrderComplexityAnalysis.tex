\chapter{Higher Order Complexity Analysis}

Programs are written in the {source language. Then the program is translated
to a complexity language. The semantic interpretation of the complexity
language program may be used to analyse the complexity of the original
program.

\section{Source Language}
The source language is the simply typed lambda calculus with \T{Unit},
products, suspensions, user-defined inductive datatypes and a recursion
construct.  Valid signatuers, types, and constructor arguments are given in
figure \ref{fig:source_lang_sigs_types_constructors}. The types, expressions,
and typing judgments of the source language are given in figure
\ref{fig:source_lang_syntax_types}.  Evaluation is call-by-value and the rules
for evaluation are given in figure \ref{fig:source_lang_oper_sem}.

\T{unit} is a singleton type with only one inhabitant, the value
$\LP\RP$, also called unit.

Product types are a compound types consisting of an ordered pair of types.
Products are introduced using $\LP e_0, e_1 \RP$ Since evaluation is
call-by-value, products are strict.  So both expressions of in product must be
evaluated before the product may be destructured.  Products are eliminated
using \T{split}.

A suspension is an unevaluated computation.  A suspension has type \T{susp}
$\tau$ where $\tau$ is the type of the suspended computation.  Suspensions are
introduced using the \T{delay}$(e)$ operator.  Suspensions are eliminated using
the \T{force}$(e)$ operator, which evaluates the suspended computation.


A program using datatypes must have a top-level signature $\psi$ consisting of
datatype declarations of the form
%
\[ \T{datatype} \delta = C^\delta_0 \T{of} \phi_{C_0}[\delta] \ |\ ...\ |\ C^\delta_{n-1} \T{of} \phi_{C_{n-1}}[\delta] \]
%
Each datatype may only refer to datatypes declared earlier in the signature.
This prevents general recursive datatypes.  The argument to each constructor is
given by a strictly positive functor $\phi$, which is one of $t$, $\tau$,
$\phi_0 \times \phi_1$, and $\tau \rightarrow \phi$.  The identity functor $t$
represents recursive occurence of the datatype.  The constant functor $\tau$
represents a non-recursive type.  The product functor $\phi_0 \times \phi_1$
represents a pair of arguments.  The constant exponential $\tau \rightarrow \phi$
represents a function type.  The introduction forms for datatypes are the
constructors.  The elimination form for a datatype is the \T{rec} construct.


The \T{rec} construct allows for structural recursion.  \T{rec} is given an
argument to recurse on and a sequence of statements corresponding to each
constructor for the datatype of the first argument.  The first argument to
\T{rec} is evaluated to a value, and then depending on the outermost
constructor of the value, \T{rec} evaluates to the appropriate branch.


\T{map} is used to lift functions from $\sigma \rightarrow \tau$ to
$\phi[\sigma] \rightarrow \phi[\tau]$. \T{map} is restricted to syntactic
values and is used in the operational semantics to insert recursive calls in
their places. For example, if recursing on a value that does not contain a
recursive occurence of a datatype, such as a boolean or a tree leaf, then
\T{map} does not insert a recursive call anywhere.


\begin{figure}
  \label{fig:source_lang_syntax_types}
  \caption{Source language syntax and types}
  Types
  \begin{align*}
    \tau &::= \T{unit}\ |\ \tau \times \tau\ |\ \tau \rightarrow \tau\ |\ \T{susp}\ \tau\ |\ \delta \\
    \phi &::= t\ |\ \tau\ |\ \phi \times \phi\ |\ \tau \rightarrow \phi \\
    \T{datatype}\ \delta &= C^\delta_0 \T{of} \phi_{C_0}[\delta]\ |\ ...\ |\ C^\delta_{n-1} \T{of} \phi_{C_{n-1}}[\delta]
  \end{align*}

  Expressions
  \begin{align*}
    v &::= x\ |\ \LP\RP\ |\ \LP v, v \RP\ |\ \lambda x.e\ |\ \T{delay}(e)\ |\ C\ v \\
    e &::= x\ |\ \LP\RP\ |\ \LP e, e \RP\ |\ \T{split}(e, x.x.e)\ |\ \lambda x.e\ |\ e\ e \\
      &\quad\ |\ \T{delay}(e)\ |\ \T{force}(e)\ |\ C^\delta\ e\ |\ \T{rec}^\delta(e, \overline{C \mapsto x.e_C}) \\
      &\quad\ |\ \T{map}^\phi(x.v, v)\ |\ \T{let}(e, x.e) \\
    n &::= 0\ |\ 1\ |\ n + n
  \end{align*}

  Typing Judgments

  \AxiomC{}
  \UnaryInfC{$\gamma, x : \sigma \vdash x : \sigma$}
  \DisplayProof
  \AxiomC{}
  \UnaryInfC{$\gamma \vdash \LP \RP : \T{unit}$}
  \DisplayProof

  \bigskip

  \AxiomC{$\gamma \vdash e_0 : \tau_0$}
  \AxiomC{$\gamma \vdash e_1 : \tau_1$}
  \BinaryInfC{$\LP e_0, e_1 \RP : \tau_0 \times \tau_1$}
  \DisplayProof
  \AxiomC{$\gamma \vdash e_0 : \tau_0 \times \tau_1$}
  \AxiomC{$\gamma, x_0 : \tau_0, x_1 : \tau_1 \vdash e_1 : \tau$}
  \BinaryInfC{$\gamma \vdash \T{split}(e_0, x_0.x_1.e_1) : \tau$}
  \DisplayProof

  \bigskip

  \AxiomC{$\gamma, x : \sigma \vdash e : \tau$}
  \UnaryInfC{$\gamma \vdash \lambda x.e : \sigma \rightarrow \tau$}
  \DisplayProof
  \AxiomC{$\gamma \vdash e_0 : \sigma \rightarrow \tau$}
  \AxiomC{$\gamma \vdash e_1 : \sigma$}
  \BinaryInfC{$\gamma \vdash e_0\ e_1 : \tau$}
  \DisplayProof

  \bigskip

  \AxiomC{$\gamma \vdash e : \tau$}
  \UnaryInfC{$\gamma \vdash \T{delay}(e) : \T{susp}\ \tau$}
  \DisplayProof
  \AxiomC{$\gamma \vdash e : \T{susp}\ \tau$}
  \UnaryInfC{$\gamma \vdash \T{force}(e) : \tau$}
  \DisplayProof

  \bigskip

  \AxiomC{$\gamma \vdash e : \phi_C[\delta]$}
  \UnaryInfC{$\gamma \vdash C^\delta\ e : \delta$}
  \DisplayProof
  \AxiomC{$\gamma \vdash e : \delta$}
  \AxiomC{$\forall C . \gamma, x : \phi_C[\delta \times \T{susp}\ \tau] \vdash e_C : \tau$}
  \BinaryInfC{$\gamma \vdash \T{rec}^\delta(e, \overline{C \mapsto x.e_C}) : \tau$}
  \DisplayProof

  \bigskip

  \AxiomC{$\gamma, x : \tau_0 \vdash v_1 : \tau_1$}
  \AxiomC{$\gamma \vdash v_0 : \phi[\tau_0]$}
  \BinaryInfC{$\T{map}^\phi(x.v_1, v_0) : \phi[\tau_1]$}
  \DisplayProof
  \AxiomC{$\gamma \vdash e_0 : \sigma$}
  \AxiomC{$\gamma, x : \sigma \vdash e_1 : \tau$}
  \BinaryInfC{$\T{let}(e_0, x.e_1) : \tau$}
  \DisplayProof
\end{figure}


\begin{figure}
  \label{fig:source_lang_sigs_types_constructors}
  \caption{Source language valid signatures, types, and constructor arguments}
  Signatures: $\psi$ \T{sig}

  \bigskip

  \AxiomC{}
  \UnaryInfC{$\LP\RP$ \T{sig}}
  \DisplayProof
  \quadfive
  \AxiomC{$\delta \notin \forall\text{C}(\psi \vdash \phi_C \T{ok})$}
  \UnaryInfC{$\psi, \T{ datatype } \delta = \overline{C \T{ of } \phi_C[\delta]} \T{ sig}$}
  \DisplayProof

  \bigskip \bigskip

  Types : $\psi \vdash \tau\ \T{type}$

  \bigskip

  \AxiomC{}
  \UnaryInfC{$\psi \vdash \T{unit type}$}
  \DisplayProof
  \quad
  \AxiomC{$\psi \vdash \tau_0\ \T{type}$}
  \AxiomC{$\psi \vdash \tau_1\ \T{type}$}
  \BinaryInfC{$\psi \vdash \tau_0 \times \tau_1\ \T{type}$}
  \DisplayProof

  \bigskip

  \AxiomC{$\psi \vdash \tau_0\ \T{type}$}
  \AxiomC{$\psi \vdash \tau_1\ \T{type}$}
  \BinaryInfC{$\psi \vdash \tau_0 \rightarrow \tau_1\ \T{type}$}
  \DisplayProof
  \quad
  \AxiomC{$\psi \vdash \tau\ \T{type}$}
  \UnaryInfC{$\psi \vdash \T{susp}\ \tau\ \T{type}$}
  \DisplayProof
  \quad
  \AxiomC{$\delta \in \psi$}
  \UnaryInfC{$\psi \vdash \delta\ \T{type}$}
  \DisplayProof

  \bigskip

  Constructor arguments: $\psi \vdash \phi \T{ ok}$

  \bigskip

  \AxiomC{}
  \UnaryInfC{$\psi \vdash t\ \T{ ok}$}
  \DisplayProof
  \quadseven
  \AxiomC{$\psi \vdash \tau \T{ type}$}
  \UnaryInfC{$\psi \vdash \tau \T{ ok}$}
  \DisplayProof

  \bigskip

  \AxiomC{$\psi \vdash \phi_0\ \T{ok}$}
  \AxiomC{$\psi \vdash \phi_1\ \T{ok}$}
  \BinaryInfC{$\psi \vdash \phi_0 \times \phi_1\ \T{ok}$}
  \DisplayProof
  \quadfive
  \AxiomC{$\psi \vdash \tau\ \T{type}$}
  \AxiomC{$\psi \vdash \phi\ \T{ok}$}
  \BinaryInfC{$\psi \vdash \tau \rightarrow \phi\ \T{ok}$}
  \DisplayProof
\end{figure}


\begin{figure}
  \label{fig:source_lang_oper_sem}
  \caption{Source language operational semantics}

  \bigskip

  \AxiomC{$e_0 \downarrow^{n_0} v_0$}
  \AxiomC{$e_1 \downarrow^{n_1} v_1$}
  \BinaryInfC{$\LP e_0, e_1 \RP \downarrow^{n_0 + n_1} \LP v_0, v_1 \RP$}
  \DisplayProof
  \quad
  \AxiomC{$e_0 \downarrow^{n_0} \LP v_0, v_1 \RP$}
  \AxiomC{$e_1[v_0/x_0, v_1/x_1] \downarrow^{n_1} v$}
  \BinaryInfC{$\T{split}(e_0, x_0.x_1.e_1) \downarrow^{n_0 + n_1} v$}
  \DisplayProof

  \bigskip

  \AxiomC{$e_0 \downarrow^{n_0} \lambda x.e_0'$}
  \AxiomC{$e_1 \downarrow^{n_1} v_1$}
  \AxiomC{$e_0'[v_1/x] \downarrow^n v$}
  \TrinaryInfC{$e_0\ e_1 \downarrow^{1 + n_0 + n_1 + n} v$}
  \DisplayProof
  \quad
  \AxiomC{}
  \UnaryInfC{$\T{delay}(e) \downarrow^0 \T{delay}(e)$}
  \DisplayProof

  \bigskip

  \AxiomC{$e \downarrow^{n_0} \T{delay}(e_0)$}
  \AxiomC{$e_0 \downarrow^{n_1} v$}
  \BinaryInfC{$\T{force}(e) \downarrow^{n_0 + n_1} v$}
  \DisplayProof
  \quad
  \AxiomC{$e \downarrow^n v$}
  \UnaryInfC{$C e \downarrow^n C v$}
  \DisplayProof

  \bigskip

  \AxiomC{$e \downarrow^{n_0} C v_0$}
  \AxiomC{$\T{map}^{\phi_C}(y.\LP y, \T{delay}(rec(y, \overline{C \mapsto x.e_C}))\RP, v_0) \downarrow^{n_1} v_1$}
  \AxiomC{$e_C[v_1/x] \downarrow^{n_2} v$}
  \TrinaryInfC{$rec(e, \overline{C \mapsto x.e_C}) \downarrow^{1 + n_0 + n_1 + n_2} v$}
  \DisplayProof

  \bigskip

  \AxiomC{}
  \UnaryInfC{$\T{map}^t(x.v, v_0) \downarrow^0 v[v_0/x]$}
  \DisplayProof
  \quad
  \AxiomC{}
  \UnaryInfC{$\T{map}^\tau(x.v, v_0) \downarrow^0 v_0$}
  \DisplayProof

  \bigskip

  \AxiomC{$\T{map}^{\phi_0}(x.v, v_0) \downarrow^{n_0} v_0'$}
  \AxiomC{$\T{map}^{\phi_1}(x.v, v_1) \downarrow^{n_1} v_1'$}
  \BinaryInfC{$\T{map}^{\phi_0 \times \phi_1}(x.v, \LP v_0, v_1 \RP) \downarrow^{n_0 + n_1} \LP v_0', v_1'\RP$}
  \DisplayProof

  \AxiomC{}
  \UnaryInfC{$\T{map}^{\tau \to \phi}(x.v, \lambda y.e) \downarrow^0 \lambda y.\T{let}(e, z.\T{map}^\phi(x.v, z))$}
  \DisplayProof
  \quad
  \AxiomC{$e_0 \downarrow^{n_0} v_0$}
  \AxiomC{$e_1[v_0/x] \downarrow^{n_1} v$}
  \BinaryInfC{$\T{let}(e_0, x.e_1) \downarrow^{n_0 + n_1} v$}
  \DisplayProof
\end{figure}

\section{Complexity Language}

The types, expressions, and typing judgments of the complexity language are given in figure \ref{fig:complexity_lang}.
The complexity language is similar to the source language with a few exceptions.

Suspensions are no longer present.

Tuples are destructured using projections instead of \T{split}.

\begin{figure}
  \label{fig:complexity_lang}
  \caption{Complexity language types, expressions, and typing judgments}

  Types
  \begin{align*}
    T &::= \textbf{C} \ |\ \T{unit} \ |\ \Delta \ |\ T \times T \ |\ T \rightarrow T \\
    \Phi &::= t \ |\ T \ |\ \Phi \times \Phi \ |\ T \rightarrow \Phi \\
    \textbf{C} &::= 0\ |\ 1\ |\ 2\ |\ ... \\
    \T{datatype}\Delta &= C^\Delta_0 \T{of} \Phi_{C_0}[\Delta] \ |\ ... \ |\ C^\Delta_{n-1} \T{of} \Phi_{C_{n-1}}[\Delta]
  \end{align*}

  Expressions
  \begin{align*}
    E &::= x | 0 | 1 | E + E | \LP\RP | \LP E,E \RP | \\
      &\quad \pi_0 E | \pi_1 E | \lambda x.E | E\ E | C^\delta\ E | \text{rec}^\Delta(E, \overline{C \mapsto x.E_C})
  \end{align*}

  Typing Judgments

  \bigskip

  \AxiomC{}
  \UnaryInfC{$\Gamma, x : T \vdash x : T$}
  \DisplayProof
  \quad
  \AxiomC{}
  \UnaryInfC{$\Gamma \vdash 0 : \textbf{C}$}
  \DisplayProof
  \quad
  \AxiomC{}
  \UnaryInfC{$\Gamma \vdash 1 : \textbf{C}$}
  \DisplayProof
  \quad
  \AxiomC{}
  \UnaryInfC{$\Gamma \vdash \LP\RP : \textbf{unit}$}
  \DisplayProof

  \bigskip

  \AxiomC{$\Gamma \vdash E_0 : \textbf{C}$}
  \AxiomC{$\Gamma \vdash E_1 : \textbf{C}$}
  \BinaryInfC{$\Gamma \vdash E_0 + E_1 : \textbf{C}$}
  \DisplayProof
  \quad
  \AxiomC{$\Gamma \vdash E_0 : T_0$}
  \AxiomC{$\Gamma \vdash E_1 : T_1$}
  \BinaryInfC{$\Gamma \vdash \LP E_0, E_1 \RP : T_0 \times T_1$}
  \DisplayProof

  \bigskip

  \AxiomC{$\Gamma \vdash E : T_0 \times T_1$}
  \UnaryInfC{$\Gamma \vdash \pi_i E : T_i$}
  \DisplayProof
  \quad
  \AxiomC{$\Gamma, x : T_0 \vdash E : T_1$}
  \UnaryInfC{$\Gamma \vdash \lambda x.E : T_0 \rightarrow T_1$}
  \DisplayProof

  \bigskip

  \AxiomC{$\Gamma \vdash E_0 : T_0 \rightarrow T_1$}
  \AxiomC{$\Gamma \vdash E_1 : T_0$}
  \BinaryInfC{$\Gamma \vdash E_0\ E_1 : T_1$}
  \DisplayProof
  \quad
  \AxiomC{$\Gamma \vdash E : \Phi_C[\Delta]$}
  \UnaryInfC{$\Gamma \vdash C^\Delta E : \Delta$}
  \DisplayProof

  \bigskip

  \AxiomC{$\Gamma \vdash E : \Delta$}
  \AxiomC{$\forall C . \Gamma, x : \Phi_C[\Delta \times T] \vdash E_C : T$}
  \BinaryInfC{$\Gamma \vdash \text{rec}^\Delta(E, \overline{C \mapsto x.E_C}) : T$}
  \DisplayProof

\end{figure}

The translation from the source language to the complexity language is given in figure \ref{fig:complexity_translation}.
\begin{figure}
  \label{fig:complexity_translation}
  \caption{Translation from source language to complexity language}
  %
  \begin{align*}
    \|\tau\| &= \textbf{C} \times \llangle \tau \rrangle \\
    \llangle\T{unit}\rrangle &= \T{unit} \\
    \llangle \sigma \times \tau \rrangle &= \llangle \sigma \rrangle \times \llangle \tau \rrangle \\
    \llangle \sigma \rightarrow \tau \rrangle &= \llangle \sigma \rrangle \rightarrow \|\tau\| \\
    \llangle \T{susp}\ \tau \rrangle &= \|\tau\| \\
    \llangle \delta \rrangle &= \delta \\
  \end{align*}
  %
  \begin{align*}
    \|\phi\| &= \textbf{C} \times \llangle \phi \rrangle \\
    \llangle t \rrangle &= t \\
    \llangle \tau \rrangle &= \llangle \tau \rrangle \\
    \llangle \phi_0 \times \phi_1 \rrangle &= \llangle \phi_0 \rrangle \times \llangle phi_1 \rrangle \\
    \llangle \tau \rightarrow \phi \rrangle &= \llangle \phi \rrangle \rightarrow \|\phi\| \\
  \end{align*}
  \begin{align*}
    \llangle \psi \rrangle &= \text{ for each }\delta \in \psi, \delta = C_0^\delta \T{ of } \llangle \phi_{C_0}\rrangle[\delta], . . . , C_{n-1}^\delta \T{ of } \llangle \phi_{n-1}\rrangle[\delta] \\
  \end{align*}
  %
  \begin{align*}
    \|x\| &= \LP 0, x \RP \\
    \|\LP\RP\| &= \LP 0, \LP\RP\RP \\
    \|\LP e_0,e_1 \RP\| &= \LP \|e_0\|_c + \|e_1\|_c, \LP \|e_0\|_p, \|e_1\|_p\RP\RP \\
    \|\T{split}(e_0, x_0.x_1.e_1)\| &= \|e_0\|_c +_c \|e_1\|[\pi_0\|e_0\|_p/x_0, \pi_1\|e_0\|_p/x_1] \\
    \|\lambda x.e\| &= \LP 0, \lambda x.\|e\| \RP \\
    \|e_0\ e_1\| &= (1 + \|e_0\|_c + \|e_1\|_c) +_c \|e_0\|_p \|e_1\|_p \\
    \|\T{delay}(e)\| &= \LP 0, \|e\|\RP \\
    \|\T{force}(e)\| &= \|e\|_c +_c \|e\|_p \\
    \|\text{C}^\delta_i\ e\| = \LP \|e\|_c, \text{C}^\delta_i \|e\|_p \RP \\
    \|\T{rec}^\delta(e, \overline{C \mapsto x.e_C})\| &= \|e\|_c +_c \T{rec}^\delta(\|e\|_p, \overline{C \mapsto x.1 +_c \|e_C\|}) \\
    \|\T{map}^\phi(x.v_0, v_1)\| &= \LP 0, \T{map}^{\llangle \phi \rrangle}(x.\|v_0\|_p, \|v_1\|_p)\RP \\
    \|\T{let}(e_0, x.e_1)\rrangle &= \|e_0\|_c +_c \|e_1\|[\|e_0\|_p/x]
  \end{align*}
  %
\end{figure}
