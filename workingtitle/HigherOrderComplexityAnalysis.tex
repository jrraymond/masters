\chapter{Higher Order Complexity Analysis}
\paragraph{}
Programs are written in the source language. Then the program is translated
to a complexity language. The semantic interpretation of the complexity
language program may be used to analyse the complexity of the original
program.

\section{Source Language}

\begin{figure}
  \caption{Source language syntax and types}
  \label{fig:source_lang_syntax_types}
  Types
  \begin{align*}
    \tau &::= \T{unit}\ |\ \tau \times \tau\ |\ \tau \rightarrow \tau\ |\ \T{susp}\ \tau\ |\ \delta \\
    \phi &::= t\ |\ \tau\ |\ \phi \times \phi\ |\ \tau \rightarrow \phi \\
    \T{datatype}\ \delta &= C^\delta_0 \T{of} \phi_{C_0}[\delta]\ |\ ...\ |\ C^\delta_{n-1} \T{of} \phi_{C_{n-1}}[\delta]
  \end{align*}

  Expressions
  \begin{align*}
    v &::= x\ |\ \LP\RP\ |\ \LP v, v \RP\ |\ \lambda x.e\ |\ \T{delay}(e)\ |\ C\ v \\
    e &::= x\ |\ \LP\RP\ |\ \LP e, e \RP\ |\ \T{split}(e, x.x.e)\ |\ \lambda x.e\ |\ e\ e \\
      &\quad\ |\ \T{delay}(e)\ |\ \T{force}(e)\ |\ C^\delta\ e\ |\ \T{rec}^\delta(e, \overline{C \mapsto x.e_C}) \\
      &\quad\ |\ \T{map}^\phi(x.v, v)\ |\ \T{let}(e, x.e) \\
    n &::= 0\ |\ 1\ |\ n + n
  \end{align*}

  Typing Judgments

  \AxiomC{}
  \UnaryInfC{$\gamma, x : \sigma \vdash x : \sigma$}
  \DisplayProof
  \AxiomC{}
  \UnaryInfC{$\gamma \vdash \LP \RP : \T{unit}$}
  \DisplayProof

  \bigskip

  \AxiomC{$\gamma \vdash e_0 : \tau_0$}
  \AxiomC{$\gamma \vdash e_1 : \tau_1$}
  \BinaryInfC{$\LP e_0, e_1 \RP : \tau_0 \times \tau_1$}
  \DisplayProof
  \AxiomC{$\gamma \vdash e_0 : \tau_0 \times \tau_1$}
  \AxiomC{$\gamma, x_0 : \tau_0, x_1 : \tau_1 \vdash e_1 : \tau$}
  \BinaryInfC{$\gamma \vdash \T{split}(e_0, x_0.x_1.e_1) : \tau$}
  \DisplayProof

  \bigskip

  \AxiomC{$\gamma, x : \sigma \vdash e : \tau$}
  \UnaryInfC{$\gamma \vdash \lambda x.e : \sigma \rightarrow \tau$}
  \DisplayProof
  \AxiomC{$\gamma \vdash e_0 : \sigma \rightarrow \tau$}
  \AxiomC{$\gamma \vdash e_1 : \sigma$}
  \BinaryInfC{$\gamma \vdash e_0\ e_1 : \tau$}
  \DisplayProof

  \bigskip

  \AxiomC{$\gamma \vdash e : \tau$}
  \UnaryInfC{$\gamma \vdash \T{delay}(e) : \T{susp}\ \tau$}
  \DisplayProof
  \AxiomC{$\gamma \vdash e : \T{susp}\ \tau$}
  \UnaryInfC{$\gamma \vdash \T{force}(e) : \tau$}
  \DisplayProof

  \bigskip

  \AxiomC{$\gamma \vdash e : \phi_C[\delta]$}
  \UnaryInfC{$\gamma \vdash C^\delta\ e : \delta$}
  \DisplayProof
  \AxiomC{$\gamma \vdash e : \delta$}
  \AxiomC{$\forall C . \gamma, x : \phi_C[\delta \times \T{susp}\ \tau] \vdash e_C : \tau$}
  \BinaryInfC{$\gamma \vdash \T{rec}^\delta(e, \overline{C \mapsto x.e_C}) : \tau$}
  \DisplayProof

  \bigskip

  \AxiomC{$\gamma, x : \tau_0 \vdash v_1 : \tau_1$}
  \AxiomC{$\gamma \vdash v_0 : \phi[\tau_0]$}
  \BinaryInfC{$\T{map}^\phi(x.v_1, v_0) : \phi[\tau_1]$}
  \DisplayProof
  \AxiomC{$\gamma \vdash e_0 : \sigma$}
  \AxiomC{$\gamma, x : \sigma \vdash e_1 : \tau$}
  \BinaryInfC{$\T{let}(e_0, x.e_1) : \tau$}
  \DisplayProof
\end{figure}


\begin{figure}
  \caption{Source language valid signatures, types, and constructor arguments}
  \label{fig:source_lang_sigs_types_constructors}

  Signatures: $\psi$ \T{sig}

  \bigskip

  \AxiomC{}
  \UnaryInfC{$\LP\RP$ \T{sig}}
  \DisplayProof
  \quadfive
  \AxiomC{$\delta \notin \forall\text{C}(\psi \vdash \phi_C \T{ok})$}
  \UnaryInfC{$\psi, \T{ datatype } \delta = \overline{C \T{ of } \phi_C[\delta]} \T{ sig}$}
  \DisplayProof

  \bigskip \bigskip

  Types : $\psi \vdash \tau\ \T{type}$

  \bigskip

  \AxiomC{}
  \UnaryInfC{$\psi \vdash \T{unit type}$}
  \DisplayProof
  \quad
  \AxiomC{$\psi \vdash \tau_0\ \T{type}$}
  \AxiomC{$\psi \vdash \tau_1\ \T{type}$}
  \BinaryInfC{$\psi \vdash \tau_0 \times \tau_1\ \T{type}$}
  \DisplayProof

  \bigskip

  \AxiomC{$\psi \vdash \tau_0\ \T{type}$}
  \AxiomC{$\psi \vdash \tau_1\ \T{type}$}
  \BinaryInfC{$\psi \vdash \tau_0 \rightarrow \tau_1\ \T{type}$}
  \DisplayProof
  \quad
  \AxiomC{$\psi \vdash \tau\ \T{type}$}
  \UnaryInfC{$\psi \vdash \T{susp}\ \tau\ \T{type}$}
  \DisplayProof
  \quad
  \AxiomC{$\delta \in \psi$}
  \UnaryInfC{$\psi \vdash \delta\ \T{type}$}
  \DisplayProof

  \bigskip

  Constructor arguments: $\psi \vdash \phi \T{ ok}$

  \bigskip

  \AxiomC{}
  \UnaryInfC{$\psi \vdash t\ \T{ ok}$}
  \DisplayProof
  \quadseven
  \AxiomC{$\psi \vdash \tau \T{ type}$}
  \UnaryInfC{$\psi \vdash \tau \T{ ok}$}
  \DisplayProof

  \bigskip

  \AxiomC{$\psi \vdash \phi_0\ \T{ok}$}
  \AxiomC{$\psi \vdash \phi_1\ \T{ok}$}
  \BinaryInfC{$\psi \vdash \phi_0 \times \phi_1\ \T{ok}$}
  \DisplayProof
  \quadfive
  \AxiomC{$\psi \vdash \tau\ \T{type}$}
  \AxiomC{$\psi \vdash \phi\ \T{ok}$}
  \BinaryInfC{$\psi \vdash \tau \rightarrow \phi\ \T{ok}$}
  \DisplayProof
\end{figure}


\begin{figure}
  \caption{Source language operational semantics}
  \label{fig:source_lang_oper_sem}

  \bigskip

  \AxiomC{$e_0 \downarrow^{n_0} v_0$}
  \AxiomC{$e_1 \downarrow^{n_1} v_1$}
  \BinaryInfC{$\LP e_0, e_1 \RP \downarrow^{n_0 + n_1} \LP v_0, v_1 \RP$}
  \DisplayProof
  \quad
  \AxiomC{$e_0 \downarrow^{n_0} \LP v_0, v_1 \RP$}
  \AxiomC{$e_1[v_0/x_0, v_1/x_1] \downarrow^{n_1} v$}
  \BinaryInfC{$\T{split}(e_0, x_0.x_1.e_1) \downarrow^{n_0 + n_1} v$}
  \DisplayProof

  \bigskip

  \AxiomC{$e_0 \downarrow^{n_0} \lambda x.e_0'$}
  \AxiomC{$e_1 \downarrow^{n_1} v_1$}
  \AxiomC{$e_0'[v_1/x] \downarrow^n v$}
  \TrinaryInfC{$e_0\ e_1 \downarrow^{1 + n_0 + n_1 + n} v$}
  \DisplayProof
  \quad
  \AxiomC{}
  \UnaryInfC{$\T{delay}(e) \downarrow^0 \T{delay}(e)$}
  \DisplayProof

  \bigskip

  \AxiomC{$e \downarrow^{n_0} \T{delay}(e_0)$}
  \AxiomC{$e_0 \downarrow^{n_1} v$}
  \BinaryInfC{$\T{force}(e) \downarrow^{n_0 + n_1} v$}
  \DisplayProof
  \quad
  \AxiomC{$e \downarrow^n v$}
  \UnaryInfC{$C e \downarrow^n C v$}
  \DisplayProof

  \bigskip

  \AxiomC{$e \downarrow^{n_0} C v_0$}
  \AxiomC{$\T{map}^{\phi_C}(y.\LP y, \T{delay}(rec(y, \overline{C \mapsto x.e_C}))\RP, v_0) \downarrow^{n_1} v_1$}
  \AxiomC{$e_C[v_1/x] \downarrow^{n_2} v$}
  \TrinaryInfC{$rec(e, \overline{C \mapsto x.e_C}) \downarrow^{1 + n_0 + n_1 + n_2} v$}
  \DisplayProof

  \bigskip

  \AxiomC{}
  \UnaryInfC{$\T{map}^t(x.v, v_0) \downarrow^0 v[v_0/x]$}
  \DisplayProof
  \quad
  \AxiomC{}
  \UnaryInfC{$\T{map}^\tau(x.v, v_0) \downarrow^0 v_0$}
  \DisplayProof

  \bigskip

  \AxiomC{$\T{map}^{\phi_0}(x.v, v_0) \downarrow^{n_0} v_0'$}
  \AxiomC{$\T{map}^{\phi_1}(x.v, v_1) \downarrow^{n_1} v_1'$}
  \BinaryInfC{$\T{map}^{\phi_0 \times \phi_1}(x.v, \LP v_0, v_1 \RP) \downarrow^{n_0 + n_1} \LP v_0', v_1'\RP$}
  \DisplayProof

  \AxiomC{}
  \UnaryInfC{$\T{map}^{\tau \to \phi}(x.v, \lambda y.e) \downarrow^0 \lambda y.\T{let}(e, z.\T{map}^\phi(x.v, z))$}
  \DisplayProof
  \quad
  \AxiomC{$e_0 \downarrow^{n_0} v_0$}
  \AxiomC{$e_1[v_0/x] \downarrow^{n_1} v$}
  \BinaryInfC{$\T{let}(e_0, x.e_1) \downarrow^{n_0 + n_1} v$}
  \DisplayProof
\end{figure}


\paragraph{}
The source language is the simply typed lambda calculus with \T{Unit},
products, suspensions, programmer-defined inductive datatypes and a recursion
construct. Valid signatures, types, and constructor arguments are given in
Figure \ref{fig:source_lang_sigs_types_constructors}. The types, expressions,
and typing judgments of the source language are given in Figure
\ref{fig:source_lang_syntax_types}. Evaluation is call-by-value and the rules
for evaluation are given in Figure \ref{fig:source_lang_oper_sem}.

\paragraph{}
We use big-step operational cost semantics. Small-step operational semantics
provide an indirect notion of number of steps required to evaluate a program to
a value. Big-step operational semantics do not allow this since intermediate
evaluation steps are suppressed. Big-step operational semantics introduce a
notion of cost by using evaluation judgements of the form $e \downarrow^n v$.
For example the evaluation judgement for a tuple is
%
\begin{prooftree}
  \AxiomC{$e_0 \downarrow^{n_0} v_0$}
  \AxiomC{$e_1 \downarrow^{n_1} v_1$}
  \BinaryInfC{$\LP e_0,e_1 \RP \downarrow^{n_0 + n_1} \LP v_0,v_1\RP$}
\end{prooftree}
%
This judgement reads if $e_0$ evaluates to a value $v_0$ in $n_0$ steps and
$e_1$ evaluates to a value $v_1$ in $n_1$ steps, then the tuple $\LP
e_0,v_0\RP$ evaluates to the value $\LP v_0,v_1\RP$ in $n_0+n_1$ steps.

\paragraph{}
A program using datatypes must have a top-level signature $\psi$ consisting of
datatype declarations of the form
%
\[ \T{datatype} \delta = C^\delta_0 \T{of} \phi_{C_0}[\delta] \ |\ ...\ |\ C^\delta_{n-1} \T{of} \phi_{C_{n-1}}[\delta] \]
%
Each datatype may only refer to datatypes declared earlier in the signature.
This prevents general recursive datatypes.  The argument to each constructor is
given by a strictly positive functor $\phi$, which is one of $t$, $\tau$,
$\phi_0 \times \phi_1$, and $\tau \rightarrow \phi$.  The identity functor $t$
represents recursive occurrence of the datatype.  The constant functor $\tau$
represents a non-recursive type.  The product functor $\phi_0 \times \phi_1$
represents a pair of arguments.  The constant exponential $\tau \rightarrow \phi$
represents a function type.  The introduction forms for datatypes are the
constructors.  The elimination form for a datatype is the \T{rec} construct.

\paragraph{}
To give the reader a better understanding of the source language, we will
implement a small program, explaining the syntax and semantics we need as we
go.  We define an \T{list} datatype in the source language below.
%
\[
  \T{datatype list = Nil of unit | Cons of int$\times$list}
\]
%
\T{unit} is a singleton type with only one inhabitant, the value
$\LP\RP$, also called unit.


The \T{listmap} function applies a function to each element in a list.
%
\[
  \T{listmap f xs} = \T{rec}(xs, \T{Nil} \mapsto z.\T{Nil}, \T{Cons} \mapsto z.\T{Cons}\LP\pi_0 z, \T{force}(\pi_1\pi_1 z)\RP)
\]
%
This function uses the \T{rec} construct, which is how we do structural
recursion on datatypes.
%
\begin{prooftree}
\AxiomC{$\gamma \vdash e : \delta$}
\AxiomC{$\forall C . \gamma, x : \phi_C[\delta \times \T{susp}\ \tau] \vdash e_C : \tau$}
\BinaryInfC{$\gamma \vdash \T{rec}^\delta(e, \overline{C \mapsto x.e_C}) : \tau$}
\end{prooftree}
%
The \T{rec} is a branch on an expression. The expression is evaluated to a
value, and the branch of the \T{rec} matching the outermost constructor of the
value is taken. Inside the each branch of the \T{rec}, the variable $x$ is a
value of type $\phi_C[\delta \times \T{susp } \tau]$. A suspension is an
unevaluated computation.  A suspension has type \T{susp} $\tau$ where $\tau$ is
the type of the suspended computation.

\paragraph{}
Suspensions are introduced using the \T{delay}$(e)$ operator.  Suspensions are
eliminated using the \T{force}$(e)$ operator, which evaluates the suspended
computation. The \T{rec} construct makes available all recursive calls.
Suspensions are necessary to avoid charging for recursive calls that are not
actually used.

\paragraph{}
The operational semantics for \T{rec} are
%
\begin{prooftree}
  \AxiomC{$e \downarrow^{n_0} C v_0$}
  \AxiomC{$\T{map}^{\phi_C}(y.\LP y, \T{delay}(\T{rec}(y, \overline{C \mapsto x.e_C}))\RP, v_0) \downarrow^{n_1} v_1$}
  \AxiomC{$e_C[v_1/x] \downarrow^{n_2} v$}
  \TrinaryInfC{$rec(e, \overline{C \mapsto x.e_C}) \downarrow^{1 + n_0 + n_1 + n_2} v$}
\end{prooftree}
%
\T{map} is used to lift functions from $\sigma \rightarrow \tau$ to
$\phi[\sigma] \rightarrow \phi[\tau]$. To understand the role of \T{map} in
\T{rec}, let us consider the two branches of \T{rec} in \T{listmap}.
%
\[
  \text{Let } E = \T{rec}(y, \T{Nil} \mapsto \T{Nil}, \T{Cons} \mapsto z.\T{Cons}\LP\pi_0 z, \T{force}(\pi_1\pi_1 z)\RP)
\]
%

\paragraph{}
The first case, $xs$ is \T{Nil}, so according to the operational semantics,
$e \downarrow \T{Nil} \LP\RP$ in $0$ steps. Next
%
\[
  \T{map}^{\phi_\T{Nil}}(y,\LP y, \T{delay}(E)\RP, \LP\RP)
\]
%
is evaluated to a value $v_1$. We substitute $v_1$ for $z$ in the body of the
\T{Nil} branch and evaluate the body to a value to get our result.
%
\begin{prooftree}
  \AxiomC{}
  \UnaryInfC{$\T{map}^\tau(x.v, v_0) \downarrow^0 v_0$}
\end{prooftree}
%
So the \T{map} evaluates to $\LP\RP$.


\paragraph{}
In the second case, the outermost constructor of $xs$ is \T{Cons}. Let the
argument to this constructor be the tuple $\LP x,xs'\RP$. So the map expression
is
%
\[
  \T{map}^{\phi_\T{Cons}}(y,\LP y,\T{delay}(E)\RP, (x, xs') \RP)
\]
%
To evaluate the \T{map} expression, we use the rule for mapping over a pair.
%
\begin{prooftree}
  \AxiomC{$\T{map}^{\phi_0}(x.v, v_0) \downarrow^{n_0} v_0'$}
  \AxiomC{$\T{map}^{\phi_1}(x.v, v_1) \downarrow^{n_1} v_1'$}
  \BinaryInfC{$\T{map}^{\phi_0 \times \phi_1}(x.v, \LP v_0, v_1 \RP) \downarrow^{n_0 + n_1} \LP v_0', v_1'\RP$}
\end{prooftree}
%
We apply the rule for mapping over a pair.
%
\[
  \LP \T{map}^\T{int}(y,\LP y,\T{delay}(E)\RP, x), \T{map}^{\phi_\T{susp list}}(y,\LP y,\T{delay}(E)\RP, xs')\RP
\]
%
The first \T{map} is over a non-recursive argument of a constructor. Recall the
rule for evaluating this \T{map}.
%
\begin{prooftree}
  \AxiomC{}
  \UnaryInfC{$\T{map}^\tau(x.v, v_0) \downarrow^0 v_0$}
\end{prooftree}
%
So the first \T{map} evaluates to $x$.


The second map is over a recursive argument of a constructor. The rule to
evaluate this \T{map} is
%
\begin{prooftree}
  \AxiomC{}
  \UnaryInfC{$\T{map}^t(x.v, v_0) \downarrow^0 v[v_0/x]$}
\end{prooftree}
%
The result of the \T{map} over the second element of the tuple is $\LP y,
\T{delay}(E) \RP[xs'/y] = \LP xs',\T{delay}(E[xs'/y])\RP$. So the result of the
\T{map} over the tuple is $\LP x, \LP xs', \T{delay}(E[xs'/y])\RP\RP$. Recall
the body of the \T{Cons} branch of the \T{rec} is $\T{Cons}\LP f \pi_0 z,
\T{force}(\pi_1\pi_1 z)\RP$. We have just shown how the \T{map} expression
results in the term $\LP x, \LP xs',\T{delay}(E[xs'/z])\RP\RP$. This is the
term $z$ is bound to inside the body of the \T{Cons} branch. So $\pi_0 z$ is
the head of the list and $\pi_1\pi_1 z$ is a suspended computation representing
the recursive call on the tail of the list. Since it is  suspended, we need to
use the \T{force} function to evaluate it.


\paragraph{}
The $\T{let}(e_0,x.e_1)$ syntactic construct allows us to do function
application in \T{map} without charging for cost. It also serves the purpose
avoiding recomputation of values. If $e_0$ is an expensive computation that
occurs more than once in $e_1$, we can use \T{let} to compute $e_0$ and use the
result inside $e_1$ multiple times without paying cost multiple times.



\section{Complexity Language}

\paragraph{}
The types, expressions, and typing judgments of the complexity language are
given in Figure \ref{fig:complexity_lang}.  The complexity language is similar
to the source language with a few exceptions.

\paragraph{}
Suspensions are no longer present in the complexity language. Recall
suspensions served the purpose of avoiding charging costs in unused recursive
calls during the translation into the complexity language. Since the complexity
language program has already been translated, the complexity language does not
need suspensions.

\paragraph{}
Another difference is tuples are deconstructed using projection functions
instead of \T{split}. In the source language, to add two elements of a tuple
together we write
%
\[
  \lambda p.\T{split}(p,x_0.x_1.x_0 + x_1)
\]
%
In the complexity language we write
%
\[
  \lambda p.\pi_0 p + \pi_1 p
\]
%

\paragraph{}
The \T{map} function is treated as a macro $\T{map}^\Phi$ in the complexity
language. The macro is defined by $\Phi$ and the definition mirrors the
semantics of \T{map} in the source language. The definition is given in Figure
\ref{fig:complexity_language_map}.
%
\begin{figure}
\caption{Complexity language \T{map} macro}
\label{fig:complexity_language_map}
\begin{align*}
  \T{map}^t(x.E,E_0) &= E[E_0/x] \\
  \T{map}^T(x.E,E_0) &= E_0 \\
  \T{map}^{\Phi_0\times\Phi_1}(x.E,E_0) &= \LP\T{map}^{\Phi_0}(x.E,\pi_0 E_0), \T{map}^{\Phi_1}(x.E,\phi_1 E_1)\RP \\
  \T{map}^{T \to \Phi}(x.E,E_0) &= \lambda y.\T{map}^\Phi(x.E,E_0\ y)
\end{align*}
\end{figure}
%


\begin{figure}
  \caption{Complexity language types, expressions, and typing judgments}
  \label{fig:complexity_lang}

  Types
  \begin{align*}
    T &::= \textbf{C} \ |\ \T{unit} \ |\ \Delta \ |\ T \times T \ |\ T \rightarrow T \\
    \Phi &::= t \ |\ T \ |\ \Phi \times \Phi \ |\ T \rightarrow \Phi \\
    \textbf{C} &::= 0\ |\ 1\ |\ 2\ |\ ... \\
    \T{datatype}\Delta &= C^\Delta_0 \T{of} \Phi_{C_0}[\Delta] \ |\ ... \ |\ C^\Delta_{n-1} \T{of} \Phi_{C_{n-1}}[\Delta]
  \end{align*}

  Expressions
  \begin{align*}
    E &::= x | 0 | 1 | E + E | \LP\RP | \LP E,E \RP | \\
      &\quad \pi_0 E | \pi_1 E | \lambda x.E | E\ E | C^\delta\ E | \text{rec}^\Delta(E, \overline{C \mapsto x.E_C})
  \end{align*}

  Typing Judgments

  \bigskip

  \AxiomC{}
  \UnaryInfC{$\Gamma, x : T \vdash x : T$}
  \DisplayProof
  \quad
  \AxiomC{}
  \UnaryInfC{$\Gamma \vdash 0 : \textbf{C}$}
  \DisplayProof
  \quad
  \AxiomC{}
  \UnaryInfC{$\Gamma \vdash 1 : \textbf{C}$}
  \DisplayProof
  \quad
  \AxiomC{}
  \UnaryInfC{$\Gamma \vdash \LP\RP : \textbf{unit}$}
  \DisplayProof

  \bigskip

  \AxiomC{$\Gamma \vdash E_0 : \textbf{C}$}
  \AxiomC{$\Gamma \vdash E_1 : \textbf{C}$}
  \BinaryInfC{$\Gamma \vdash E_0 + E_1 : \textbf{C}$}
  \DisplayProof
  \quad
  \AxiomC{$\Gamma \vdash E_0 : T_0$}
  \AxiomC{$\Gamma \vdash E_1 : T_1$}
  \BinaryInfC{$\Gamma \vdash \LP E_0, E_1 \RP : T_0 \times T_1$}
  \DisplayProof

  \bigskip

  \AxiomC{$\Gamma \vdash E : T_0 \times T_1$}
  \UnaryInfC{$\Gamma \vdash \pi_i E : T_i$}
  \DisplayProof
  \quad \AxiomC{$\Gamma, x : T_0 \vdash E : T_1$} \UnaryInfC{$\Gamma \vdash \lambda x.E : T_0 \rightarrow T_1$}
  \DisplayProof

  \bigskip

  \AxiomC{$\Gamma \vdash E_0 : T_0 \rightarrow T_1$}
  \AxiomC{$\Gamma \vdash E_1 : T_0$}
  \BinaryInfC{$\Gamma \vdash E_0\ E_1 : T_1$}
  \DisplayProof
  \quad
  \AxiomC{$\Gamma \vdash E : \Phi_C[\Delta]$}
  \UnaryInfC{$\Gamma \vdash C^\Delta E : \Delta$}
  \DisplayProof

  \bigskip

  \AxiomC{$\Gamma \vdash E : \Delta$}
  \AxiomC{$\forall C . \Gamma, x : \Phi_C[\Delta \times T] \vdash E_C : T$}
  \BinaryInfC{$\Gamma \vdash \text{rec}^\Delta(E, \overline{C \mapsto x.E_C}) : T$}
  \DisplayProof

\end{figure}

The translation from the source language to the complexity language is given in
Figure \ref{fig:complexity_translation_types} and Figure
\ref{fig:complexity_translation_expressions}.
%
\begin{figure}
  \caption{Translation from source language to complexity language types.}
  \label{fig:complexity_translation_types}
  %
  \begin{align*}
    \|\tau\| &= \textbf{C} \times \llangle \tau \rrangle \\
    \llangle\T{unit}\rrangle &= \T{unit} \\
    \llangle \sigma \times \tau \rrangle &= \llangle \sigma \rrangle \times \llangle \tau \rrangle \\
    \llangle \sigma \rightarrow \tau \rrangle &= \llangle \sigma \rrangle \rightarrow \|\tau\| \\
    \llangle \T{susp}\ \tau \rrangle &= \|\tau\| \\
    \llangle \delta \rrangle &= \delta \\
  \end{align*}
  %
  \begin{align*}
    \|\phi\| &= \textbf{C} \times \llangle \phi \rrangle \\
    \llangle t \rrangle &= t \\
    \llangle \tau \rrangle &= \llangle \tau \rrangle \\
    \llangle \phi_0 \times \phi_1 \rrangle &= \llangle \phi_0 \rrangle \times \llangle phi_1 \rrangle \\
    \llangle \tau \rightarrow \phi \rrangle &= \llangle \phi \rrangle \rightarrow \|\phi\| \\
  \end{align*}
  \begin{align*}
    \llangle \psi \rrangle &= \text{ for each }\delta \in \psi, \delta = C_0^\delta \T{ of } \llangle \phi_{C_0}\rrangle[\delta], . . . , C_{n-1}^\delta \T{ of } \llangle \phi_{n-1}\rrangle[\delta] \\
  \end{align*}
  %
\end{figure}
%
We denote the complexity translation of a source language expression $e$ as
$\|e\|$. We refer to complexity language expressions of type \textbf{C} as
\textit{costs}, complexity language expressions of type $\llangle \tau \rrangle$ as
\textit{potentials}, and complexity language expressions of type
$\textbf{C}\times \llangle \tau \rrangle$ as \textit{complexities}.
\paragraph{}
Examining the translation of source language types to complexity language types
in Figure \ref{fig:complexity_translation_types}, we see that the translation
of a source language expression of type $\tau$ is
$\textit{C}\times\llangle\tau\rrangle$.  The first element is the cost, a bound
on the cost of evaluating the expression, and the second element is the
potential, an expression for the size of the value.  The potential translation
of types \T{unit} and $\delta$ is the corresponding complexity language types
\T{unit} and $\delta$. The potential translation of a product type is the
complexity language type of a product of the potential translations of the
components of the product type.
%
\begin{figure}
  \caption{Translation from source language to complexity language expressions.}
  \label{fig:complexity_translation_expressions}
  \begin{align*}
    \|x\| &= \LP 0, x \RP \\
    \|\LP\RP\| &= \LP 0, \LP\RP\RP \\
    \|\LP e_0,e_1 \RP\| &= \LP \|e_0\|_c + \|e_1\|_c, \LP \|e_0\|_p, \|e_1\|_p\RP\RP \\
    \|\T{split}(e_0, x_0.x_1.e_1)\| &= \|e_0\|_c +_c \|e_1\|[\pi_0\|e_0\|_p/x_0, \pi_1\|e_0\|_p/x_1] \\
    \|\lambda x.e\| &= \LP 0, \lambda x.\|e\| \RP \\
    \|e_0\ e_1\| &= (1 + \|e_0\|_c + \|e_1\|_c) +_c \|e_0\|_p \|e_1\|_p \\
    \|\T{delay}(e)\| &= \LP 0, \|e\|\RP \\
    \|\T{force}(e)\| &= \|e\|_c +_c \|e\|_p \\
    \|\text{C}^\delta_i\ e\| &= \LP \|e\|_c, \text{C}^\delta_i \|e\|_p \RP \\
    \|\T{rec}^\delta(e, \overline{C \mapsto x.e_C})\| &= \|e\|_c +_c \T{rec}^\delta(\|e\|_p, \overline{C \mapsto x.1 +_c \|e_C\|}) \\
    \|\T{map}^\phi(x.v_0, v_1)\| &= \LP 0, \T{map}^{\llangle \phi \rrangle}(x.\|v_0\|_p, \|v_1\|_p)\RP \\
    \|\T{let}(e_0, x.e_1)\rrangle &= \|e_0\|_c +_c \|e_1\|[\|e_0\|_p/x]
  \end{align*}
  %
\end{figure}



\section{Denotational Semantics}

\paragraph{}
The recurrences in the complexity language do not look like the recurrences one
would expect in complexity analysis. This is because the complexity language
recurrences contain as much information about the size of an expression as the
source language does. In order to get recognizable recurrences, we must
abstract values to sizes by interpreting the complexity language in a
denotational semantics.



The denotational interpretation of the complexity types are standard. We
interpret numbers as elements of $\mathbb{Z}$, tuples of type $\tau \times
\sigma$ as elements of the cross-product of the set of values of type $\tau$
and the set of values of type $\sigma$, lambda expressions of type $\tau \to
\sigma$ as mathematical functions from the set of values of type $\tau$ to the
set of values of type $\sigma$, and application as mathematical function
application. The nonstandard interpretations are those of datatype constructors
and \T{rec}. Since datatypes are programmer-defined and there are multiple
interpretations for a single datatype, the programmer must provide their own
interpretation. For example we may decide to interpret \T{list} as their length.
%
\[
  \LB \T{list} \RB = \mathbb{N}
\]
%
\paragraph{}
Semantically, we need to distinguish between the constructors of a datatype, so
we also define a semantic value $D^\T{list}$. $D^\T{list}$ is a sum type of the
arguments to the \T{list} constructors. \T{list} has two constructors, \T{Nil},
which has argument of type \T{unit}, and \T{Cons}, which has argument of type
$\T{int} \times \T{list}$.
%
\[
  D^\T{list} = \ast + \mathbb{Z} \times \mathbb{N}
\]
%
We will write $C_i$ for the $i^{th}$ injection into $D^\T{list}$. $C_i$ takes
us from the interpretation of the argument of a constructor to a value of type
$D^\T{list}$. We also need a function $size_{list}$ which takes us from
$D^\T{list}$ back to $\llbracket \T{list} \rrbracket$. $size_\Delta$ is the
programmers notion of size for programmer-defined datatypes. In this case, we want
the size of a list to be its length. So our $size$ function is defined as
follows.
%
\begin{align*}
  size_{list} (\ast) &= 0 \\
  size_{list}(i,n) &= 1 + n
\end{align*}
%
There is a restriction on the definition of the $size$ function. The size of a
value must be strictly greater than the size of any of its substructures of the
same type. In the case of \T{list}, the restriction means the size of a $(1,
n)$ must be strictly greater than the size of $(j, n-1)$.

\paragraph{}
\sloppypar
In general, when we interpret a source language with programmer-defined
datatypes, for each datatype $\Delta$ we must define an interpretation $\LB
\Delta \RB$ and a function $size : D^\Delta \to \LB \Delta \RB$.


\paragraph{}
The interpretation of a datatype with constructor $C$ under the environment
$\xi$ is
%
\[
  \llbracket C\ e\rrbracket \xi = size(C (\llbracket e \rrbracket \xi))
\]
%
In the \T{list} case, if $e$ is \T{Nil}, then the interpretation $\llbracket \T{Nil}
\rrbracket$ is $size_{list}(C_0 \LB\LP\RP\RB)$, since the argument to the \T{Nil}
constructor is $\LP\RP$. The interpretation of \T{unit} is $0$. So
$size_{list}(C_0 \LB\LP\RP\RB = size_{list}(C_0\ 0)$. $C_0$ is the $0^{th}$
injection from $\LB \Phi[\T{list}\ \RB$ to $D^\T{list}$.  So
$size_{list}(C_0\ 0) = size_{list}(\ast)$. By our definition of
$size_{list}$, $size_{list}(\ast) = 0$.


\paragraph{}
The interpretation of \T{rec} is also nonstandard. To interpret \T{rec}, we
introduce a semantic $case$ function.
%
\[
  case^\delta : D^\delta \times \Uppi_{C} (\LB T\RB^{\LB \Phi_C[\delta]\RB} \to \LB T \RB^\tau) \to \LB T \RB^\tau
\]
%
The interpretation of a \T{rec} is
%
\[
  \LB rec^\delta(E^\delta, \overline{C \mapsto x^{\phi_C[\delta \times \tau]}.E_C^\tau}) \RB \xi = \bigvee\limits_{size(z) \leq \LB E \RB \xi} case(z, \overline{f_C})
\]
%
where for each constructor $C$,
%
\[
  f_C(x) = \LB E_C \RB \xi \{ x \mapsto map^{\Phi_C}(a.(a, \LB rec(w, \overline{C \mapsto x.E_C})\RB \xi \{ w \mapsto a \}), x) \}
\]
%
Since we cannot predict which branch the \T{rec} will take, we must take the
maximum over all possible branches to obtain an upper bound. Recall our
restriction on the $size$ function that the size of a value must be strictly
greater than the size of any of its substructure of the same type. This ensures
the recursion used to interpret the \T{rec} expressions is well-defined.
Continuing with the \T{list} example, the interpretation of \T{rec} on a \T{list} is
%
\begin{align*}
  \LB\T{rec}(E_0, \T{Nil} \mapsto E_\T{Nil}, \T{Cons} \mapsto x.E_\T{Cons})\RB &= \bigvee\limits_{size(z) \leq \LB E_0 \RB} case(z, f_{Nil}, f_{Cons}) \\
  \text{where} \\
  f_{Nil}(\ast) &= \LB E_\T{Nil} \RB \\
  f_{Cons}(i,n) &= \LB E_\T{Cons} \RB
\end{align*}
