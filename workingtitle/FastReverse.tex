% ---------------------------------------- FAST REVERSE ----------------------------------------
\chapter{Fast Reverse}

Fast reverse is an implementation reverse in linear time complexity. 
A naive implementation of reverse appends the head of the list to recursively
reversed tail of  the list. Fast reverse instead uses an abstraction to delay
the consing. As this is the first example, we will walk through the translation
and interpretation in gory detail. In following examples we will relegate the
walk-through of the translation to the appendices, where the reader can peruse
them, perhaps over a glass of carbenet sauvignon, as a relaxing end to a
stressful day.

The definition of the list datatype holds no suprises.
\[ \T{datatype list} = \T{Nil of unit}\ |\ \T{Cons of int} \times \T{list} \]

The implementation of fast reverse is not obvious. We write a function \T{rev}
that applies an auxilary function to an empty list to produce the result.  The
specification of reverse is \T{rev [$x_0,\dots,x_{n-1}$] =
[$x_{n-1},\dots,x_0$]}. The specification of the auxilary function
\T{rec(xs,$\dots$)} is \T{rec($[x_0,\dots,x_{n-1}],\dots$)
[$y_0,\dots,y_{m-1}$] = [$x_{n-1},\dots,x_0,y_0,\dots,y_{m-1}$]}.

\begin{lstlisting}
rev xs = $\lambda$xs.rec(xs,
               Nil $\mapsto$ $\lambda$a.a,
               Cons$\mapsto$b.split(b,x.c.split(c,xs'.r.
                        $\lambda$a.force(r) Cons$\langle$x,a$\rangle$))) Nil
\end{lstlisting}

Notice that the implementation of \T{rev} would be much cleaner if we where
able to pattern match on cases of the \T{rec}. Below is \T{rev} written with
this syntactic sugar.

\begin{lstlisting}
rev = $\lambda$xs.rec(xs, Nil $\mapsto \lambda$a.a,
              Cons$\mapsto\langle$y$\langle$ys,r$\rangle\rangle$.$\lambda$b.force(r) Cons$\langle$x,b$\rangle$) Nil
\end{lstlisting}

Each recursive call creates an abstraction that applies the recursive call on
the tail of the list to the list created by consing the head of the list onto
the abstraction argument. The recursive calls builds nested abstractions as
deep as the length of the list which is collapsed by application of the
outermost abastraction to \T{Nil}. Below we show the evaluation of \T{rev}
applied to a small list of just two elements.

\begin{lstlisting}
rev (Cons$\langle$0,Cons$\langle$1, Nil$\rangle\rangle$) $\to$ 
  rec(Cons$\langle$0,Cons$\langle$1,Nil$\rangle\rangle$,
      Nil $\mapsto\lambda$a.a
      Cons$\mapsto$b.split(b,x.c.split(c,xs'.r.
               $\lambda$a.force(r) Cons$\langle$x,a$\rangle$))) Nil
  $\to^* (\lambda$a0.($\lambda$a1.($\lambda$a2.a2) Cons$\langle$1,a1$\rangle$) Cons$\langle$0,a0$\rangle$) Nil
  $\to_\beta$ ($\lambda$a1.($\lambda$a2.a2) Cons$\langle$1,a1$\rangle$) Cons$\langle$0,Nil$\rangle$
  $\to_\beta$ ($\lambda$a2.a2) Cons$\langle$1,Cons$\langle$0,Nil$\rangle\rangle$
  $\to_\beta$ Cons$\langle$1,Cons$\langle$0,Nil$\rangle\rangle$
\end{lstlisting}

% -------------------- BEGIN REV TRANSLATION SPLIT --------------------   
We will walk through the translation from the source language to the complexity
language.
%
\begin{flalign*}
  \|\T{rev}\| &= \|(\lambda\T{xs.rec(xs, Nil}\mapsto\lambda\T{a.a,} \\
              &\quad \T{Cons}\mapsto\T{b.split(b,x.c.split(c,xs'.r.}\lambda\T{a.force(r) Cons}\langle\T{x,a}\rangle\T{)))) Nil}\|
\end{flalign*}
%
% BEGIN APP
%
First we apply the rule for translating an application. The rule is
$\|e_0\ e_1\| = (1 + \|e_0\|_c + \|e_1\|_c) +_c (\|e_0\|_p\ \|e_1\|_p)$.
In this case, $\lambda\T{xs.rec(...)}$ is $e_0$ and \T{Nil} is $e_1$.
%
% APP ARGUMENT
%
The translation of a constructor applied to an expression is a tuple of the
cost of the translated expression and the corresponding complexity language
constructor applied to the potential of the translated expression. Since the
expression inside \T{Nil} is $\langle\rangle$, and
$\|\langle\rangle\| = \langle 0,\langle\rangle\rangle$, we have
%
\begin{flalign*}
  |\T{Nil}\| &= \langle\langle 0, \langle\rangle\rangle_c, \T{Nil}\langle 0,\langle\rangle\rangle_p\rangle \\
             &= \langle 0, \T{Nil}\langle\rangle\rangle
\end{flalign*}
%
% BEGIN APP FUNCTION
%
To translate $\lambda\T{xs.rec(...)}$ we apply the rule
$\|\lambda x.e\| = \langle 0, \lambda x.\|e\|\rangle$. So
%
\begin{flalign*}
  &\|\lambda\T{xs.rec(xs, Nil}\mapsto\lambda\T{a.a,} \\
  &\qquad \T{Cons}\mapsto\T{b.split(b,x.c.split(c,xs'.r.}\lambda\T{a.force(r) Cons}\langle\T{x,a}\rangle\T{)))}\| \\
  &\quad= \langle 0, \lambda xs.\|\T{rec(xs, Nil}\mapsto\lambda\T{a.a,} \\
  &\quadthree \T{Cons}\mapsto\T{b.split(b,x.c.split(c,xs'.r.}\lambda\T{a.force(r) Cons}\langle\T{x,a}\rangle\T{)))}\|\rangle
\end{flalign*}
%
% BEGIN REC
%
The rule for translating a \T{rec} expression is
\[
  \|\T{rec}(e,\overline{C \mapsto x.e_C})\| = (1 + \|e\|_c) +_c \T{rec}(\|e\|_p, \overline{C \mapsto x.\|e_C\|})
\]
%
\begin{flalign*}
  &\|\T{rec}(xs, \T{Nil}\mapsto\lambda a.a, \\
  &\qquad \T{Cons}\mapsto b.\T{split}(b,x.c.\T{split}(c,xs'.r.\lambda a.\T{force}(r)\ \T{Cons}\langle x,a\rangle)))\| \\
  &= \|xs\|_c +_c \T{rec}(\|xs\|_p, \T{Nil} \mapsto 1 +_c \|\lambda a.a\| \\
  &\quadthree \T{Cons}\mapsto b. 1 +_c \|\T{split}(b,x.c.\T{split}(c,xs'.r.\lambda a.\T{force}(r)\ \T{Cons}\langle x,a\rangle))\|) \\
  &= \langle 0, xs \rangle_c +_c \T{rec}(\langle 0, xs\rangle_p, \T{Nil} \mapsto 1 +_c \|\lambda a.a\| \\
  &\quadthree \T{Cons}\mapsto b. 1 +_c \|\T{split}(b,x.c.\T{split}(c,xs'.r.\lambda a.\T{force}(r)\ \T{Cons}\langle x,a\rangle))\|) \\
  &\text{The term $xs$ is a variable and the rule for translating variables is $\|xs\| = \langle 0, xs\rangle$.} \\
  &= \T{rec}(xs, \T{Nil} \mapsto 1 +_c \|\lambda a.a\| \\
  &\quadthree \T{Cons}\mapsto b. 1 +_c \|\T{split}(b,x.c.\T{split}(c,xs'.r.\lambda a.\T{force}(r)\ \T{Cons}\langle x,a\rangle))\|)
\end{flalign*}
%
% NIL BRANCH
%
The translation of the \T{Nil} branch is
simple application of the $\|\lambda x.e\| = \langle 0, \lambda
x.\|e\|\rangle$ and the variable translation rule.
%
\begin{flalign*}
  &1 +_c \|\lambda a.a\| \\
  &= 1  +_c   \langle 0, \lambda a. \| a \|\rangle \\
  &=  \langle 1, \lambda a. \langle 0,a \rangle\rangle 
\end{flalign*}
%
% BEGIN CONS BRANCH
%
The translation of the \T{Cons} branch is a slightly more involved. The rule
for translating \T{split} is
%
\[ \|\T{split}(e_0,x_0.x_1.e_1)\| = \|e_0\|_c +_c \|e_1\|[\pi_0\|e_0\|_p/x_0, \pi_1\|e_0\|_p/x_1] \]
%
After applying the rule to the \T{Cons} branch we get
%
\begin{flalign*}
  &1 +_c \|\T{split}(b,x.c.\T{split}(c,xs'.r.\lambda a.\T{force}(r)\ \T{Cons} \langle x,a \rangle )) \| \\
  &= 1 +_c \|b\|_c +_c \|\T{split}(c,xs'.r.\lambda a.\T{force}(r)\ \T{Cons}\langle x,a \rangle) \|[\pi_0 \| b \|_p/x,\pi_1 \| b \|_p/c] 
\end{flalign*}
%
Remember that $b$ is a variable and has type
$\phi_\T{Cons}[\T{list} \times \T{susp (list} \to \T{list)}]$.
The translation of this type is 
$\textbf{C} \times \llangle \phi_\T{Cons} \rrangle [\T{list} \times \langle \T{list} \to \langle \textbf{C} \times \T{list} \rangle\rangle]$.
We can say that \T{$\pi_0\|$b$\|_p$} is the head of the list \T{xs},
\T{$\pi_0\pi_1\|$b$\|_p$} is the tail of the list \T{xs}, and
\T{$\pi_1\pi_1\|$b$\|_p$} is the result of the recursive call.
The translation of $b$ is $\langle 0, \|b\|_p\rangle$.
%
\begin{flalign*}
  &1 +_c \|b\|_c +_c \|\T{split}(c,xs'.r.\lambda a.\T{force}(r)\ \T{Cons}\langle x,a \rangle) \|[\pi_0 \| b \|_p/x,\pi_1 \| b \|_p/c] \\
  &=1 +_c \|\T{split}(c,xs'.r.\lambda a.\T{force}(r)\ \T{Cons}\langle x,a \rangle) \|[\pi_0 \| b \|_p/x,\pi_1 \| b \|_p/c] \\
  %
  &\qquad \text{We apply the rule for \T{split} again.} \\
  %
  &=1 +_c (\|c\|_c +_c \|\lambda a.\T{force}(r)\ \T{Cons}\langle x,a \rangle\|[\pi_0 \|c\|_p/xs', \pi_1\|c\|_p/r][\pi_0 \| b \|_p/x,\pi_1 \| b \|_p/c] \\
  %
  &\qquad \text{$c$ is a variable, so its translation is $\langle 0, c \rangle$.} \\
  %
  &=1 +_c \|\lambda a.\T{force}(r)\ \T{Cons}\langle x,a \rangle\|[\pi_0 \|c\|_p/xs', \pi_1\|c\|_p/r][\pi_0 \| b \|_p/x,\pi_1 \| b \|_p/c] \\
  %
  &\qquad \text{We apply the rule for abstraction.} \\
  %
  &=1 +_c \langle 0, \lambda a.\|\T{force}(r)\ \T{Cons}\langle x,a \rangle\|[\pi_0 \|c\|_p/xs', \pi_1\|c\|_p/r][\pi_0 \| b \|_p/x,\pi_1 \| b \|_p/c] \\
  %
  &\qquad \text{Recall $C +_c E$ is a macro for $\langle C + E_c, E_p\rangle$. We use this to eliminate the $+_c$.} \\
  &\qquad \text{We also apply the translation rule for application.} \\
  %
  &=\langle 1, \lambda a.(1 + \|\T{force}(r)\|_c + \|\T{Cons}\langle x,a\rangle\|_c) \\
  &\quadfive +_c \|\T{force}(r)\|_p \|\T{Cons}\langle x,a \rangle\|_p\rangle[\pi_0 \|c\|_p/xs', \pi_1\|c\|_p/r][\pi_0 \| b \|_p/x,\pi_1 \| b \|_p/c] \\
\end{flalign*}
%
% COMPOSE SUBSTITUTIONS
%
\begin{flalign*}
  &\text{We will translate $\T{force}(r)$ and $\T{Cons}\langle x,a\rangle$ individually.} \\
  &\text{First we compose the two substitutions.} \\
  %
  &\text{let } \Theta = [\pi_0 \|c\|_p/xs', \pi_1\|c\|_p/r][\pi_0 \| b \|_p/x,\pi_1 \| b \|_p/c] \\
  &\quadthree = [\pi_0\pi_1 \|b\|_p/xs', \pi_1\pi_1\|b\|_p/r, \pi_0 \| b \|_p/x] \\
  &\quad \text{Since $b$ is a variable, the potential  of its translation is $b$.} \\
  &\Theta = [\pi_0\pi_1 b/xs', \pi_1\pi_1 b/r, \pi_0 b/x] \\
\end{flalign*}
%
% FORCE
%
\begin{flalign*}
  &\quad \text{In translation of $\T{force}(r)$ we apply the rule $\|\T{force}(e)\| = \|e\|_c +_c \|e\|_p$.}\\
  &\quadthree \|\T{force}(r)\|\Theta = \|r\|_c\Theta +_c \|r\|_p\Theta \\
  %
  &\quad \text{We apply the variable translation rule to $r$, then apply the substitution $\Theta$.}\\
  %
  &\quadfive = \langle 0, r \rangle_c\Theta +_c \langle 0, r \rangle_p \Theta \\
  &\quadfive = r\Theta = \pi_1\pi_1 b \\
\end{flalign*}
%
% CONS
%
\begin{flalign*}
  &\quad \text{Next we do the translation of $\T{Cons}\langle x,a \rangle$.} \\
  &\quadthree \|\T{Cons}\langle x, a\rangle\| = \langle \|\langle x,a \rangle\|_c, \T{Cons} \|\langle x,a \rangle\|_p\rangle \\
  %
  &\quad \text{Notice the translation of $\langle x,a \rangle$ appears twice, so we will do this seperately.} \\
  %
  &\quadfour \|\langle x,a \rangle\| = \langle \|x\|_c + \|a\|_c, \langle \|x\|_p, \|a\|_p \rangle \rangle \Theta\\
  %
  &\quad \text{Both $x$ and $a$ are variables, so they have $0$ cost.}\\
  %
  &\quadsix = \langle 0, \langle x, a \rangle\rangle \Theta\\
  %
  &\quad \text{We apply the substitution $\Theta$.}\\
  %
  &\quadsix = \langle 0, \langle \pi_1 b, \pi_1\pi_1 b \rangle\rangle \\
  %
  &\quadsix = \langle 0, \langle \pi_1 b, \pi_1\pi_1 b \rangle\rangle \\
  %
  &\quad \text{We complete the translation of $\T{Cons}\langle x, a\rangle$ using $\langle x, a \rangle$.} \\
  %
  &\quadthree \|\T{Cons}\langle x, a\rangle\| = \langle \|\langle x,a \rangle\|_c, \T{Cons} \|\langle x,a \rangle\|_p\rangle \\
  &\quadfive = \langle 0, \T{Cons} \langle \pi_1 b, \pi_1\pi_1 b \rangle\rangle \\
\end{flalign*}
%
% END CONS BRANCH
%
\begin{flalign*}
  &\quad \text{We use substitute in the translations of $\T{force}(r)$ and $\T{Cons}\langle x, a\rangle$.}\\
  &\quad \text{$\|\T{force}(r)\|$ has cost $(\pi_1\pi_1 b)_c$ and $\|\T{Cons}\langle x,a\rangle\|$ has cost $0$.}\\
  &\langle 1, \lambda a.(1 + \|\T{force}(r)\|_c + \|\T{Cons}\langle x,a\rangle\|_c) +_c \|\T{force}(r)\|_p \|\T{Cons}\langle x,a \rangle\|_p\rangle \rangle\Theta \\
  %
  &= \langle 1, \lambda a.(1 + (\pi_1\pi_1 b)_c) +_c (\pi_1\pi_1 b)_p\ \T{Cons}\langle \pi_1 b, a \rangle\rangle \\
\end{flalign*}
%
% END REC
%
\begin{flalign*}
  &\text{We can now complete the translation of the \T{rec} expression.} \\
  &\|\lambda\T{xs.rec(xs, Nil}\mapsto\lambda\T{a.a,} \\
  &\qquad \T{Cons}\mapsto\T{b.split(b,x.c.split(c,xs'.r.}\lambda\T{a.force(r) Cons}\langle\T{x,a}\rangle\T{)))}\| \\
  &= \T{rec}(xs, \T{Nil} \mapsto 1 +_c \|\lambda a.a\| \\
  &\quadthree \T{Cons}\mapsto b. 1 +_c \|\T{split}(b,x.c.\T{split}(c,xs'.r.\lambda a.\T{force}(r)\ \T{Cons}\langle x,a\rangle))\|) \\
  &= \T{rec}(xs, \T{Nil} \mapsto \langle 1, \lambda a. \langle 0,a \rangle\rangle \\
  &\quadthree \T{Cons}\mapsto b.\langle 1, \lambda a.(1 + (\pi_1\pi_1 b)_c) +_c (\pi_1\pi_1 b)_p\ \T{Cons}\langle \pi_1 b, a \rangle\rangle) \\
\end{flalign*}
%
% END APP FUNCTION
%
\begin{flalign*}
  &\text{We substitute the translation of \T{rec} into the translation of the abstraction.}\\
  &\|\lambda\T{xs.rec(xs, Nil}\mapsto\lambda\T{a.a,} \\
  &\qquad \T{Cons}\mapsto\T{b.split(b,x.c.split(c,xs'.r.}\lambda\T{a.force(r) Cons}\langle\T{x,a}\rangle\T{)))}\| \\
  &\quad= \langle 0, \lambda xs.\|\T{rec(xs, Nil}\mapsto\lambda\T{a.a,} \\
  &\quadthree \T{Cons}\mapsto\T{b.split(b,x.c.split(c,xs'.r.}\lambda\T{a.force(r) Cons}\langle\T{x,a}\rangle\T{)))}\|\rangle \\
  &= \langle 0, \T{rec}(xs, \T{Nil} \mapsto \langle 1, \lambda a. \langle 0,a \rangle\rangle \\
  &\quadthree \T{Cons}\mapsto b.\langle 1, \lambda a.(1 + (\pi_1\pi_1 b)_c) +_c (\pi_1\pi_1 b)_p\ \T{Cons}\langle \pi_1 b,a \rangle\rangle)\rangle \\
  &= \langle 0, \T{rec}(xs, \T{Nil} \mapsto \langle 1, \lambda a. \langle 0,a \rangle\rangle \\
  &\quadthree \T{Cons}\mapsto b.\langle 1, \lambda a.(1 + (\pi_1\pi_1 b)_c) +_c (\pi_1\pi_1 b)_p\ \T{Cons}\langle \pi_1 b,a \rangle\rangle)\rangle \\
\end{flalign*}
%
% END APP
%
\begin{flalign*}
  &\text{Finally, we substitute this into the translation of \T{rev}.} \\
  \|\T{rev}\| &= \|(\lambda\T{xs.rec(xs, Nil}\mapsto\lambda\T{a.a,} \\
              &\quad \T{Cons}\mapsto\T{b.split(b,x.c.split(c,xs'.r.}\lambda\T{a.force(r) Cons}\langle\T{x,a}\rangle\T{)))) Nil}\| \\
              &= 1 +_c \T{rec}(xs, \T{Nil} \mapsto \langle 1, \lambda a. \langle 0,a \rangle\rangle \\
  &\quadthree \T{Cons}\mapsto b.\langle 1, \lambda a.(1 + (\pi_1\pi_1 b)_c) +_c (\pi_1\pi_1 b)_p\ \T{Cons}\langle \pi_1 b,a \rangle\rangle)\rangle \\
\end{flalign*}
%
% END REV TRANSLATION SPLIT
%
Observe that $\|\T{rev}\|$ admits the same syntactic sugar as \T{rev}. In the
complexity language, instead of taking projections of $b$, we can use the same
pattern matching syntactic sugar as in the source language.

\begin{flalign*}
  &\|\T{rev}\| = 1 +_c \T{rec}(xs, \T{Nil} \mapsto \langle 1, \lambda a. \langle 0,a \rangle\rangle \\
  &\quadthree \T{Cons}\mapsto \langle x, \langle xs',r \rangle\rangle.\langle 1, \lambda a.(1 + r_c) +_c r_p\ \T{Cons}\langle x,a \rangle\rangle)\ \T{Nil}\\
\end{flalign*}


%
% FAST REVESE TRANSLATION USING SYNTACTIC SUGAR
%
We walk through the same translation of fast reverse, but we use the syntactic
sugar for matching introducted earlier. Recall the implementation of fast using
syntactic sugar. The translation is identicaly to the translation of \T{rev}
written without syntactic sugar until we translate the \T{Cons} branch of the
\T{rec}.
%
\begin{flalign*}
  \|\T{rev}\| &= \|(\lambda\T{xs.rec(xs, Nil}\mapsto\lambda\T{a.a,} \\
              &\quad \T{Cons}\mapsto\langle x, \langle xs', r\rangle\rangle.\lambda a.\T{force}(r)\ \T{Cons}\langle x, a\rangle))\ \T{Nil}\|
\end{flalign*}
%
% BEGIN APP
%
First we apply the rule for translating an application. The rule is
$\|e_0\ e_1\| = (1 + \|e_0\|_c + \|e_1\|_c) +_c (\|e_0\|_p\ \|e_1\|_p)$.
In this case, $\lambda\T{xs.rec(...)}$ is $e_0$ and \T{Nil} is $e_1$.
%
% APP ARGUMENT
%
The translation of a constructor applied to an expression is a tuple of the
cost of the translated expression and the corresponding complexity language
constructor applied to the potential of the translated expression. Since the
expression inside \T{Nil} is $\langle\rangle$, and
$\|\langle\rangle\| = \langle 0,\langle\rangle\rangle$, we have
%
\begin{flalign*}
  |\T{Nil}\| &= \langle\langle 0, \langle\rangle\rangle_c, \T{Nil}\langle 0,\langle\rangle\rangle_p\rangle \\
             &= \langle 0, \T{Nil}\langle\rangle\rangle
\end{flalign*}
%
% BEGIN APP FUNCTION
%
To translate $\lambda\T{xs.rec(...)}$ we apply the rule
$\|\lambda x.e\| = \langle 0, \lambda x.\|e\|\rangle$. So
%
\begin{flalign*}
  &\|\lambda\T{xs.rec(xs, Nil}\mapsto\lambda\T{a.a,} \\
  &\qquad \T{Cons}\mapsto \langle x, \langle xs', r\rangle\rangle.\lambda a.\T{force}(r)\ \T{Cons}\langle x, a\rangle)\| \\
  &\quad= \langle 0, \lambda xs.\|\T{rec(xs, Nil}\mapsto\lambda\T{a.a,} \\
  &\quadthree \T{Cons}\mapsto \langle x, \langle xs', r\rangle\rangle.\lambda a.\T{force}(r)\ \T{Cons}\langle x, a\rangle)\|\rangle
\end{flalign*}
%
% BEGIN REC
%
Next we apply the translation rule for \T{rec}.
%
\begin{flalign*}
  &\|\T{rec}(xs, \T{Nil}\mapsto\lambda a.a, \\
  &\qquad \T{Cons}\mapsto  \langle x, \langle xs', r\rangle\rangle.\lambda a.\T{force}(r)\ \T{Cons}\langle x, a\rangle)\| \\
  &= \|xs\|_c +_c \T{rec}(\|xs\|_p, \T{Nil} \mapsto 1 +_c \|\lambda a.a\| \\
  &\quadthree \T{Cons}\mapsto \langle x, \langle xs', r\rangle\rangle.\lambda a.\T{force}(r)\ \T{Cons}\langle x, a\rangle)\|) \\
  &= \langle 0, xs \rangle_c +_c \T{rec}(\langle 0, xs\rangle_p, \T{Nil} \mapsto 1 +_c \|\lambda a.a\| \\
  &\quadthree \T{Cons}\mapsto \langle x, \langle xs', r\rangle\rangle.\lambda a.\T{force}(r)\ \T{Cons}\langle x, a\rangle)\|) \\
  &\text{The term $xs$ is a variable and the rule for translating variables is $\|xs\| = \langle 0, xs\rangle$.} \\
  &= \T{rec}(xs, \T{Nil} \mapsto 1 +_c \|\lambda a.a\| \\
  &\quadthree \T{Cons}\mapsto \langle x, \langle xs', r\rangle\rangle.\lambda a.\T{force}(r)\ \T{Cons}\langle x, a\rangle)\|)
\end{flalign*}
%
% NIL BRANCH
%
The translation of the \T{Nil} branch is the same as before.
%
\begin{flalign*}
  &1 +_c \|\lambda a.a\| =  \langle 1, \lambda a. \langle 0,a \rangle\rangle 
\end{flalign*}
%
% BEGIN CONS BRANCH
%
The translation of the \T{Cons} branch where the difference lies.
%
\begin{flalign*}
  &1 +_c \|\lambda a.\T{force}(r)\ \T{Cons}\langle x, a\rangle)\| \\
  &\quad = 1 +_c \langle 0, \lambda a.\|\T{force}(r)\ \T{Cons}\langle x, a\rangle)\| \\
  &\quad = \langle 1, \lambda a.(1 + \|\T{force}(r)\|_c + \|\T{Cons}\langle x, a\rangle)\|_c) +_c \|\T{force}(r)\|_p\ \|\T{Cons}\langle x, r \rangle\|_p \rangle \\
\end{flalign*}
%
The translation of $\T{force}(r)$ and $\T{Cons}\langle x, a \rangle$
are the same as before, except we do not have a substitution to apply.
%
\begin{flalign*}
  &\|\T{force}(r)\| = \|r\|_c +_c \|r\|_p = \langle 0, r \rangle_c +_c \langle 0, r \rangle_p = 0 +_c r = r
\end{flalign*}
%
\begin{flalign*}
  &\|\T{Cons}\langle x, a\rangle\| =  \langle 0, \T{Cons} \langle x, a \rangle\rangle \\
\end{flalign*}
%
So the complete translation of the \T{Cons} branch is
%
\begin{flalign*}
  &1 +_c \|\lambda a.\T{force}(r)\ \T{Cons}\langle x, a\rangle)\| \\
  &\quad = 1 +_c \langle 0, \lambda a.\|\T{force}(r)\ \T{Cons}\langle x, a\rangle)\| \\
  &\quad = \langle 1, \lambda a.(1 + \|\T{force}(r)\|_c + \|\T{Cons}\langle x, a\rangle)\|_c) +_c \|\T{force}(r)\|_p\ \|\T{Cons}\langle x, r \rangle\|_p \rangle \\
  &\quad = \langle 1, \lambda a.(1 + r_c + 0) +_c r_p\ \T{Cons}\langle x, r \rangle \rangle \\
  &\quad = \langle 1, \lambda a.(1 + r_c) +_c r_p\ \T{Cons}\langle x, r \rangle \rangle \\
\end{flalign*}
%
The complete translation of the \T{rec} becomes
%
\begin{flalign*}
  &\|\T{rec}(xs, \T{Nil}\mapsto\lambda a.a, \\
  &\qquad \T{Cons}\mapsto  \langle x, \langle xs', r\rangle\rangle.\lambda a.\T{force}(r)\ \T{Cons}\langle x, a\rangle)\| \\
  &= \T{rec}(xs, \T{Nil} \mapsto 1 +_c \|\lambda a.a\| \\
  &\quadthree \T{Cons}\mapsto \langle x, \langle xs', r\rangle\rangle.\lambda a.\T{force}(r)\ \T{Cons}\langle x, a\rangle)\|) \\
  &= \T{rec}(xs, \T{Nil} \mapsto \langle 0, \lambda a. \langle 0, a \rangle \rangle \\
  &\quadthree \T{Cons}\mapsto \langle x, \langle xs', r\rangle\rangle. \langle 1, \lambda a.(1 + r_c) +_c r_p\ \T{Cons}\langle x, r \rangle \rangle \\
\end{flalign*}
%
The translation of the $\lambda$ is
%
\begin{flalign*}
  &\|\lambda\T{xs.rec(xs, Nil}\mapsto\lambda\T{a.a,} \\
  &\qquad \T{Cons}\mapsto \langle x, \langle xs', r\rangle\rangle.\lambda a.\T{force}(r)\ \T{Cons}\langle x, a\rangle)\| \\
  &= \langle 0, \T{rec}(xs, \T{Nil} \mapsto \langle 0, \lambda a. \langle 0, a \rangle \rangle \\
  &\quadthree \T{Cons}\mapsto \langle x, \langle xs', r\rangle\rangle. \langle 1, \lambda a.(1 + r_c) +_c r_p\ \T{Cons}\langle x, r \rangle \rangle \rangle\\
\end{flalign*}
%
And our complete translation of \T{rev} is
%
\begin{flalign*}
  \|\T{rev}\| &= \|(\lambda\T{xs.rec(xs, Nil}\mapsto\lambda\T{a.a,} \\
              &\quad \T{Cons}\mapsto\langle x, \langle xs', r\rangle\rangle.\lambda a.\T{force}(r)\ \T{Cons}\langle x, a\rangle))\ \T{Nil}\| \\
              &= (1 + \|(\lambda\T{xs.rec(xs, Nil}\mapsto\lambda\T{a.a,} \\
              &\quad \T{Cons}\mapsto\langle x, \langle xs', r\rangle\rangle.\lambda a.\T{force}(r)\ \T{Cons}\langle x, a\rangle))\|_c + \|\T{Nil}\|_c) \\
              &\quad +_c \|(\lambda\T{xs.rec(xs, Nil}\mapsto\lambda\T{a.a,} \\
              &\quad \T{Cons}\mapsto\langle x, \langle xs', r\rangle\rangle.\lambda a.\T{force}(r)\ \T{Cons}\langle x, a\rangle))\|_p\ \|\T{Nil}\|_p \\
              &= 1 +_c \T{rec}(xs, \T{Nil} \mapsto \langle 0, \lambda a. \langle 0, a \rangle \rangle \\
              &\quadthree \T{Cons}\mapsto \langle x, \langle xs', r\rangle\rangle. \langle 1, \lambda a.(1 + r_c) +_c r_p\ \T{Cons}\langle x, r \rangle \rangle)\ \T{Nil} \\
\end{flalign*}
%
This is the same as the translation of \T{rev} without the syntactic sugar.
%
% END FAST REVERSE SYNTACTIC SUGAR TRANSLATION
%
%
% FAST REVERSE INTERPRETATION
%

The interpretation of \T{rev} is rather dull as the cost of \T{rev} is always null.
Instead of interpreting \T{rev}, we will interpret \T{rev xs}.
In preparation we will translate \T{rev xs}.
\begin{lstlisting}
$\|$rev xs$\|$ = $(1 + \|$rev$\|_c + \|$xs$\|_c) +_c \|$rev$\|_p\|$xs$\|_p$
          = $(1 + 0 + \langle$0,xs$\rangle_c) +_c \|$rev$\|_p\langle$0,xs$\rangle_p$
          = $(1 + 0 + 0) +_c \|$rev$\|_p$xs
          = $1 +_c \|$rev$\|_p$xs
          = $1 +_c (\lambda$xs.rec(xs, Nil$\mapsto \langle$1,$\lambda$a.$\langle$0,a$\rangle\rangle$,
                      Cons$\mapsto\langle$x,$\langle$xs',r$\rangle\rangle$.$\langle 1,\lambda$a.$2 +_c $ r$_p $Cons$\langle$x,a$\rangle\rangle$)$_p $Nil$)$ xs
          = $1 +_c $rec(xs, Nil$\mapsto \langle$1,$\lambda$a.$\langle$0,a$\rangle\rangle$,
                      Cons$\mapsto\langle$x,$\langle$xs',r$\rangle\rangle$.$\langle 1,\lambda$a.$2 +_c $ r$_p $Cons$\langle$x,a$\rangle\rangle$)$_p $Nil$)$
          = $1 +_c$rec(xs, Nil$\mapsto \langle$1,$\lambda$a.$\langle$0,a$\rangle\rangle$,
                  Cons$\mapsto\langle$x,$\langle$xs',r$\rangle\rangle$.$\langle 1,\lambda$a.$2 +_c $ r$_p $Cons$\langle$x,a$\rangle\rangle$)$_p $Nil
\end{lstlisting}
\begin{framed}
\begin{small}
\begin{lstlisting}
$\|$rev xs$\|$ = $1 +_c$rec(xs, Nil$\mapsto \langle$1,$\lambda$a.$\langle$0,a$\rangle\rangle$,
                  Cons$\mapsto\langle$x,$\langle$xs',r$\rangle\rangle$.$\langle 1,\lambda$a.$2 +_c $ r$_p $Cons$\langle$x,a$\rangle\rangle$)$_p $Nil
\end{lstlisting}
\end{small}
\end{framed}

\subsubsection*{Interpretation}

We intepret the size of an \T{list} to be the number of list constructors.
\begin{framed}
$\llbracket$ \T{list} $\rrbracket$ = $\mathbb{N}^\infty$\\
$D^{list} = \{\ast\} + \{1\} \times \mathbb{N}^\infty$\\
$size_{list}(\T{Nil}) = 1$\\
$size_{list}(\T{Cons(1,n)}) = 1 + n$\\
\end{framed}

The interpretation of \T{rev xs} proceeds as follows.
\begin{lstlisting}
  $\llbracket \|$rev xs$\| \rrbracket = \llbracket 1 +_c $rec(xs,$\dots$)$_p$ Nil$ \rrbracket$
  = $\llbracket \langle 1 + ($rec(xs,$\dots$)$_p$ Nil$)_c, ($rec(xs,$\dots$)$_p$ Nil$)_p \rangle \rrbracket$
  = $\langle 1 + \llbracket ($rec(xs,$\dots$)$_p$ Nil$)_c \rrbracket, \llbracket ($rec(xs,$\dots$)$_p$ Nil$)_p \rrbracket \rangle$
\end{lstlisting}

We will focus on the interpretation of the auxilary function \T{rec(xs,$\dots$)}.

Let $g(n) = \llbracket \T{rec(xs,$\dots$)} \rrbracket \{xs \mapsto n\}$

\[g(n) = \bigvee_{size\ ys \leq n} case(ys, Nil \mapsto \langle1,\lambda a.\langle 0,a\rangle\rangle, Cons \mapsto \langle 1,m \rangle.\langle 1, \lambda a. 2 +_c \pi_1g(m) (a+1)\rangle)\]

For $n=0$, $g(0) = \langle 1,\lambda a.\langle 0,a\rangle\rangle$.

For $n>0$,
\[g(n+1) = \bigvee_{size\ ys \leq n+1} case(ys, Nil \mapsto \langle1,\lambda a.\langle 0,a\rangle\rangle, Cons \mapsto \langle 1,m \rangle.\langle 1, \lambda a. 2 +_c \pi_1g(m) (a+1)\rangle)\]

\[ g(n+1) = \bigvee_{size\ ys \leq n} case(ys, Nil \mapsto \langle1,\lambda a.\langle 0,a\rangle\rangle, Cons \mapsto \langle 1,m \rangle.\langle 1, \lambda a. 2 +_c \pi_1g(m) (a+1)\rangle) \]
\[ \vee \bigvee_{size\ ys = n+1} case(ys, Nil \mapsto \langle1,\lambda a.\langle 0,a\rangle\rangle, Cons \mapsto \langle 1,m \rangle.\langle 1, \lambda a. 2 +_c \pi_1g(m) (a+1)\rangle) \]
 
\[ g(n+1) = g(n) \vee \bigvee_{size\ ys = n+1} case(ys, Nil \mapsto \langle1,\lambda a.\langle 0,a\rangle\rangle, Cons \mapsto \langle 1,m \rangle.\langle 1, \lambda a. 2 +_c \pi_1g(m) (a+1)\rangle) \]

\[ g(n+1) = g(n) \vee \langle 1, \lambda a. 2 +_c \pi_1g(n) (a+1)\rangle)\]

We want to show that $g$ is monotonically increasing; $\forall n.g(n) \leq g(n+1)$.
By definition of $\leq$, $g(n) \leq g(n+1) \Leftrightarrow \pi_0 g(n) \leq \pi_0 g(n+1) \land \pi_1 g(n) \leq \pi_1 g(n+1)$.
First we will show $\forall n. \pi_0 g(n) = 1$, the immediate corollary of which is $\forall n. \pi_0 g(n) \leq \pi_0 g(n+1)$.
\begin{proof}
We prove this by induction on $n$.
  \begin{description}
    \item[Base case: $n=0$]\hfill \\
      By definition, $\pi_0 g(0) = 1$.
    \item[Induction step: $n>0$]\hfill \\
      By definition $\pi_0 g(n+1) = \pi_0 (g(n) \vee \langle 1, \lambda a. 2 +_c \pi_1g(n) (a+1)\rangle)$.
      We distribute the projection over the max: $\pi_0 g(n+1) = \pi_0 g(n) \vee 1$.
      By the induction hypothesis, $\pi_0 g(n) = 1$, so $\pi_0 g(n+1) = 1$.
  \end{description}
\end{proof}
Now we argue that $\pi_1g(n) \leq \pi_1 g(n+1)$.
First we prove the lemma $\forall n.\pi_1 g(n) a \leq \pi_1 g(n) (a+1)$.
\begin{proof}
  We prove this by induction on $n$.
  \begin{description}
    \item[$n=0$]\hfill \\
      $\pi_1 g(0) a = \langle 0,a\rangle \leq \pi_1 g(0) (a+1) = \langle 0,a+1 \rangle$.
    \item[$n>0$]\hfill \\
      We assume $\pi_1 g(n) a \leq \pi_1 g(n) (a+1)$.
      \[ \pi_1 g(n) a \leq \pi_1 g(n) (a+1) \]
      \[ \pi_1 g(n) a \vee 2 +_c g(n) a \leq \pi_1 g(n) (a+1) \vee 2 +_c g(n) (a+1) \]
      \[ \pi_1 g(n+1) a \leq \pi_1 g(n+1) (a+1) \]
  \end{description}
\end{proof}

Now we show $\pi_1 g(n) \leq \pi_1 g(n+1)$.
\begin{proof}
  By reflexivity, $\pi_1 g(n) \leq \pi_1 g(n)$.
  By the lemma we just proved:
  \[ \pi_1 g(n) a \leq \pi_1 g(n) (a+1) \]
  \[ \pi_1 g(n) a \leq 2 +_c \pi_1 g(n) (a+1) \]
  %\[ \pi_1 g(n) a \leq \langle 2 + \pi_0 (\pi_1 g(n) (a+1)), \pi_1 (\pi_1 g(n) (a+1)) \rangle \]
  \[ \lambda a.\pi_1 g(n) a \leq \lambda a. 2 +_c \pi_1 g(n) (a+1) \]
\end{proof}

So since for all $n$, $\pi_0 g(n) = 1$ and $\pi_1 g(n) \leq \lambda a. 2 +_c \pi_1 g(n) (a+1)$, we can say

\[ g(n) \leq \langle 1, \lambda a. 2 +_c \pi_1g(n) (a+1)\rangle) \]

So 
\[ g(n+1) = \langle 1, \lambda a. 2 +_c \pi_1g(n) (a+1)\rangle\]


To extract a recurrence from $g$, we apply $g$ to the interpretation of a list $a$.

Let $h(n,a) = \pi_1 g(n) a$

For $n=0$
\begin{align*}
h(0,a) &= \pi_1 g(0) a \\
&= (\lambda a.\langle 0,a\rangle) a \\
&= \langle 0, a\rangle
\end{align*}
For $n>0$
\begin{align*}
h(n,a) &= \pi_1 g(n) a \\ 
&= (\lambda a. 2 +_c \pi_1g(n-1) (a+1)) a \\
&= 2 +_c \pi_1 g(n-1) (a+1)) \\
&= 2 +_c h(n-1,a+1) \\
&= \langle 2 + \pi_0 h(n-1,a+1), \pi_1 h(n-1,a+1)\rangle
\end{align*}

From this recurrence, we can extract a recurrence for the cost. Let $h_c = \pi_0 \circ h$.

For $n=0$
\begin{align*}
h_c(0,a) &= \pi_0 h(0,a)\\
&= \pi_0 \langle 0, a\rangle\\
&= 0
\end{align*}
For $n>0$
\begin{align*}
h_c(n,a) &= \pi_0 \langle 2 + \pi_0 h(n-1,a+1), \pi_1 h(n-1,a+1)\rangle\\
&= 2 + \pi_0 h(n-1,a+1)\\
&= 2 + h_c(n-1,a+1)
\end{align*}

We now have a recurrence for the cost of the auxilary function \T{rec(xs,$\dots$)}:
\begin{framed}
  \begin{equation}
    h_c(n,a) = \begin{cases}
      0 & n = 0 \\
      2 + h_c(n-1,a+1) & n > 0
    \end{cases}
  \end{equation}
\end{framed}

\textbf{Theroem: $h_c(n,a) = 2n$}
\begin{proof}
  We prove this by induction on $n$.
  \begin{description}
    \item{Base case: $n=0$}\hfill \\
      \[ h_c(0,a) = 0 = 2\cdot0 \]
    \item{Induction case:}\hfill \\
      We assume $h_c(n,a+1) = 2n$.\[h_c(n+1,a) = 2 + h_c(n,a+1) = 2 + 2n = 2(n+1)\]
  \end{description}
\end{proof}  

The solution to the recurrence for the cost of the auxilary function \T{rec(xs,$\dots$)} is:
\begin{framed}
  \[h_c(n,a) = 2n \]
\end{framed}


We can also extract a recurrence for the potential. Let $h_p = \pi_1 \circ h$.

For $n=0$
\begin{align*}
h_p(0,a) &= \pi_1 h(0,a)\\
&= \pi_1 \langle 0, a\rangle\\
&= a
\end{align*}
For $n>0$
\begin{align*}
h_p(n,a) &= \pi_1 \langle 2 + \pi_0 h(n-1,a+1), \pi_1 h(n-1,a+1)\rangle\\
&= \pi_1 h(n-1,a+1)\\
&= h_p (n-1,a+1)
\end{align*}

We now have a recurrence for the potential of the auxilary function in \T{rev xs}:
\begin{framed}
  \begin{equation}
    h_p(n,a) = \begin{cases}
      a & n = 0 \\
      h_p(n-1,a+1) & n > 0
    \end{cases}
  \end{equation}
\end{framed}

\textbf{Theroem: $h_p(n,a) = n + a$}
\begin{proof}
  We prove this by induction on $n$.
  \begin{description}
    \item{Base case: $n=0$}\hfill \\
      \[ h_p(0,a) = a \]
    \item{Induction case:}\hfill \\
      \[h_p(n,a) = h_p(n-1,a+1) = n - 1 + a + 1 = n + a\]
  \end{description}
\end{proof}  

So the solution to the recurrence for the potential of the auxilary function.
\begin{framed}
  \[h_p(n,a) = n + a \]
\end{framed}


We return to our interpretation of \T{rev xs}.
\begin{lstlisting}
  $\llbracket \|$rev xs$\| \rrbracket = \langle 1 + \llbracket ($rec(xs,$\dots$)$_p$ Nil$)_c \rrbracket, \llbracket ($rec(xs,$\dots$)$_p$ Nil$)_p \rrbracket \rangle$
  $= \langle 1 + \pi_0 (\llbracket ($rec(xs,$\dots$)$_p\rrbracket 0) , \pi_1 (\llbracket $rec(xs,$\dots$)$_p\rrbracket 0)\rangle$
  $= \langle 1 + \pi_0 (\pi_1 g(n)\ 0) , \pi_1 (\pi_1g(n)\ 0)\rangle \text{ where }n\text{ is the length of}$ xs
  $= \langle 1 + \pi_0 h(n,0) , \pi_1 h(n,0)\rangle$
  $= \langle 1 + h_c(n,0) , h_p(n,0)\rangle$
  $= \langle 1 + 2n , n\rangle$
\end{lstlisting}

This result tells us the cost of applying \T{rev} to a list \T{xs} of length $n$ is $1+2n$, and the resulting list has size $n$.
So \T{rev} = $\Theta(n)$.




\subsection*{Translation of \T{rev} using \T{split}}

\subsection*{Interpretation}

We intepret the size of an \T{list} to be the number of list constructors.
\begin{framed}
$\llbracket$ \T{list} $\rrbracket$ = $\mathbb{N}^\infty$\\
$D^{list} = \{\ast\} + \{1\} \times \mathbb{N}^\infty$\\
$size_{list}(\T{Nil}) = 1$\\
$size_{list}(\T{Cons(1,n)}) = 1 + n$\\
\end{framed}

The interpretation of \T{rev xs} proceeds as follows.
\begin{lstlisting}
  $\llbracket \|$rev xs$\| \rrbracket = \llbracket 1 +_c $rec(xs,$\dots$)$_p$ Nil$ \rrbracket$
  = $\llbracket \langle 1 + ($rec(xs,$\dots$)$_p$ Nil$)_c, ($rec(xs,$\dots$)$_p$ Nil$)_p \rangle \rrbracket$
  = $\langle 1 + \llbracket ($rec(xs,$\dots$)$_p$ Nil$)_c \rrbracket, \llbracket ($rec(xs,$\dots$)$_p$ Nil$)_p \rrbracket \rangle$
\end{lstlisting}

We will focus on the interpretation of the auxilary function \T{rec(xs,$\dots$)}.

Let $g(n) = \llbracket \T{rec(xs,$\dots$)} \rrbracket \{xs \mapsto n\}$

\[g(n) = \bigvee_{size\ ys \leq n} case(ys, Nil \mapsto \langle1,\lambda a.\langle 0,a\rangle\rangle, Cons \mapsto \langle 1,m \rangle.\langle 1, \lambda a. 2 +_c \pi_1g(m) (a+1)\rangle)\]

For $n=0$, $g(0) = \langle 1,\lambda a.\langle 0,a\rangle\rangle$.

For $n>0$,
\[g(n+1) = \bigvee_{size\ ys \leq n+1} case(ys, Nil \mapsto \langle1,\lambda a.\langle 0,a\rangle\rangle, Cons \mapsto \langle 1,m \rangle.\langle 1, \lambda a. 2 +_c \pi_1g(m) (a+1)\rangle)\]

\[ g(n+1) = \bigvee_{size\ ys \leq n} case(ys, Nil \mapsto \langle1,\lambda a.\langle 0,a\rangle\rangle, Cons \mapsto \langle 1,m \rangle.\langle 1, \lambda a. 2 +_c \pi_1g(m) (a+1)\rangle) \]
\[ \vee \bigvee_{size\ ys = n+1} case(ys, Nil \mapsto \langle1,\lambda a.\langle 0,a\rangle\rangle, Cons \mapsto \langle 1,m \rangle.\langle 1, \lambda a. 2 +_c \pi_1g(m) (a+1)\rangle) \]
 
\[ g(n+1) = g(n) \vee \bigvee_{size\ ys = n+1} case(ys, Nil \mapsto \langle1,\lambda a.\langle 0,a\rangle\rangle, Cons \mapsto \langle 1,m \rangle.\langle 1, \lambda a. 2 +_c \pi_1g(m) (a+1)\rangle) \]

\[ g(n+1) = g(n) \vee \langle 1, \lambda a. 2 +_c \pi_1g(n) (a+1)\rangle)\]

We want to show that $g$ is monotonically increasing; $\forall n.g(n) \leq g(n+1)$.
By definition of $\leq$, $g(n) \leq g(n+1) \Leftrightarrow \pi_0 g(n) \leq \pi_0 g(n+1) \land \pi_1 g(n) \leq \pi_1 g(n+1)$.
First we will show $\forall n. \pi_0 g(n) = 1$, the immediate corollary of which is $\forall n. \pi_0 g(n) \leq \pi_0 g(n+1)$.
\begin{proof}
We prove this by induction on $n$.
  \begin{description}
    \item[Base case: $n=0$]\hfill \\
      By definition, $\pi_0 g(0) = 1$.
    \item[Induction step: $n>0$]\hfill \\
      By definition $\pi_0 g(n+1) = \pi_0 (g(n) \vee \langle 1, \lambda a. 2 +_c \pi_1g(n) (a+1)\rangle)$.
      We distribute the projection over the max: $\pi_0 g(n+1) = \pi_0 g(n) \vee 1$.
      By the induction hypothesis, $\pi_0 g(n) = 1$, so $\pi_0 g(n+1) = 1$.
  \end{description}
\end{proof}
Now we argue that $\pi_1g(n) \leq \pi_1 g(n+1)$.
First we prove the lemma $\forall n.\pi_1 g(n) a \leq \pi_1 g(n) (a+1)$.
\begin{proof}
  We prove this by induction on $n$.
  \begin{description}
    \item[$n=0$]\hfill \\
      $\pi_1 g(0) a = \langle 0,a\rangle \leq \pi_1 g(0) (a+1) = \langle 0,a+1 \rangle$.
    \item[$n>0$]\hfill \\
      We assume $\pi_1 g(n) a \leq \pi_1 g(n) (a+1)$.
      \[ \pi_1 g(n) a \leq \pi_1 g(n) (a+1) \]
      \[ \pi_1 g(n) a \vee 2 +_c g(n) a \leq \pi_1 g(n) (a+1) \vee 2 +_c g(n) (a+1) \]
      \[ \pi_1 g(n+1) a \leq \pi_1 g(n+1) (a+1) \]
  \end{description}
\end{proof}

Now we show $\pi_1 g(n) \leq \pi_1 g(n+1)$.
\begin{proof}
  By reflexivity, $\pi_1 g(n) \leq \pi_1 g(n)$.
  By the lemma we just proved:
  \[ \pi_1 g(n) a \leq \pi_1 g(n) (a+1) \]
  \[ \pi_1 g(n) a \leq 2 +_c \pi_1 g(n) (a+1) \]
  %\[ \pi_1 g(n) a \leq \langle 2 + \pi_0 (\pi_1 g(n) (a+1)), \pi_1 (\pi_1 g(n) (a+1)) \rangle \]
  \[ \lambda a.\pi_1 g(n) a \leq \lambda a. 2 +_c \pi_1 g(n) (a+1) \]
\end{proof}

So since for all $n$, $\pi_0 g(n) = 1$ and $\pi_1 g(n) \leq \lambda a. 2 +_c \pi_1 g(n) (a+1)$, we can say

\[ g(n) \leq \langle 1, \lambda a. 2 +_c \pi_1g(n) (a+1)\rangle) \]

So 
\[ g(n+1) = \langle 1, \lambda a. 2 +_c \pi_1g(n) (a+1)\rangle\]


To extract a recurrence from $g$, we apply $g$ to the interpretation of a list $a$.

Let $h(n,a) = \pi_1 g(n) a$

For $n=0$
\begin{align*}
h(0,a) &= \pi_1 g(0) a \\
&= (\lambda a.\langle 0,a\rangle) a \\
&= \langle 0, a\rangle
\end{align*}
For $n>0$
\begin{align*}
h(n,a) &= \pi_1 g(n) a \\ 
&= (\lambda a. 2 +_c \pi_1g(n-1) (a+1)) a \\
&= 2 +_c \pi_1 g(n-1) (a+1)) \\
&= 2 +_c h(n-1,a+1) \\
&= \langle 2 + \pi_0 h(n-1,a+1), \pi_1 h(n-1,a+1)\rangle
\end{align*}

From this recurrence, we can extract a recurrence for the cost. Let $h_c = \pi_0 \circ h$.

For $n=0$
\begin{align*}
h_c(0,a) &= \pi_0 h(0,a)\\
&= \pi_0 \langle 0, a\rangle\\
&= 0
\end{align*}
For $n>0$
\begin{align*}
h_c(n,a) &= \pi_0 \langle 2 + \pi_0 h(n-1,a+1), \pi_1 h(n-1,a+1)\rangle\\
&= 2 + \pi_0 h(n-1,a+1)\\
&= 2 + h_c(n-1,a+1)
\end{align*}

We now have a recurrence for the cost of the auxilary function \T{rec(xs,$\dots$)}:
\begin{framed}
  \begin{equation}
    h_c(n,a) = \begin{cases}
      0 & n = 0 \\
      2 + h_c(n-1,a+1) & n > 0
    \end{cases}
  \end{equation}
\end{framed}

\textbf{Theroem: $h_c(n,a) = 2n$}
\begin{proof}
  We prove this by induction on $n$.
  \begin{description}
    \item{Base case: $n=0$}\hfill \\
      \[ h_c(0,a) = 0 = 2\cdot0 \]
    \item{Induction case:}\hfill \\
      We assume $h_c(n,a+1) = 2n$.\[h_c(n+1,a) = 2 + h_c(n,a+1) = 2 + 2n = 2(n+1)\]
  \end{description}
\end{proof}  

The solution to the recurrence for the cost of the auxilary function \T{rec(xs,$\dots$)} is:
\begin{framed}
  \[h_c(n,a) = 2n \]
\end{framed}


We can also extract a recurrence for the potential. Let $h_p = \pi_1 \circ h$.

For $n=0$
\begin{align*}
h_p(0,a) &= \pi_1 h(0,a)\\
&= \pi_1 \langle 0, a\rangle\\
&= a
\end{align*}
For $n>0$
\begin{align*}
h_p(n,a) &= \pi_1 \langle 2 + \pi_0 h(n-1,a+1), \pi_1 h(n-1,a+1)\rangle\\
&= \pi_1 h(n-1,a+1)\\
&= h_p (n-1,a+1)
\end{align*}

We now have a recurrence for the potential of the auxilary function in \T{rev xs}:
\begin{framed}
  \begin{equation}
    h_p(n,a) = \begin{cases}
      a & n = 0 \\
      h_p(n-1,a+1) & n > 0
    \end{cases}
  \end{equation}
\end{framed}

\textbf{Theroem: $h_p(n,a) = n + a$}
\begin{proof}
  We prove this by induction on $n$.
  \begin{description}
    \item{Base case: $n=0$}\hfill \\
      \[ h_p(0,a) = a \]
    \item{Induction case:}\hfill \\
      \[h_p(n,a) = h_p(n-1,a+1) = n - 1 + a + 1 = n + a\]
  \end{description}
\end{proof}  

So the solution to the recurrence for the potential of the auxilary function.
\begin{framed}
  \[h_p(n,a) = n + a \]
\end{framed}


We return to our interpretation of \T{rev xs}.
\begin{lstlisting}
  $\llbracket \|$rev xs$\| \rrbracket = \langle 1 + \llbracket ($rec(xs,$\dots$)$_p$ Nil$)_c \rrbracket, \llbracket ($rec(xs,$\dots$)$_p$ Nil$)_p \rrbracket \rangle$
  $= \langle 1 + \pi_0 (\llbracket ($rec(xs,$\dots$)$_p\rrbracket 0) , \pi_1 (\llbracket $rec(xs,$\dots$)$_p\rrbracket 0)\rangle$
  $= \langle 1 + \pi_0 (\pi_1 g(n)\ 0) , \pi_1 (\pi_1g(n)\ 0)\rangle \text{ where }n\text{ is the length of}$ xs
  $= \langle 1 + \pi_0 h(n,0) , \pi_1 h(n,0)\rangle$
  $= \langle 1 + h_c(n,0) , h_p(n,0)\rangle$
  $= \langle 1 + 2n , n\rangle$
\end{lstlisting}


This result tells us the cost of applying \T{rev} to a list \T{xs} of length $n$ is $1+2n$, and the resulting list has size $n$.
So \T{rev} = $\Theta(n)$.
