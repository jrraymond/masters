\chapter{Mutual Recurrence}

\section{Motivation}
The interpretation of a recursive function can be seperated into a recurrence
for the cost and a recurrence for the potential. The recurrence for the cost
depends on the recurrence for the potential. However, the recurrence for the
potential does not depend on the cost. We prove this by designing a pure
potential translation. The pure potential translation is identical to the
complexity translation except that it does not keep track of the cost.

We then show by logical relations that the potential of the complexity
translation is related to the pure potential relation.

\section{Pure Potential Translation}
Our pure potential translation is defined below. The translation of an
expression is essientially the expression itself, without suspensions.
%
\begin{align*}
  |x| &= x                                                                                     \\
  |\langle\rangle| &= \langle\rangle                                                           \\
  |\langle e_0, e_1 \rangle | &= \langle |e_0|, |e_1| \rangle                                  \\
  |\texttt{split}(e_0, x_0. x_1. e_1)| &= |e_1|[\pi_0|e_0|/x_0, \pi_0|e_0|/x_1]                \\
  |\lambda x.e | &= \lambda x.|e|                                                              \\
  |e_0\ e_1| &= |e_0|\ |e_1|                                                                   \\
  |delay(e)| &= |e|                                                                            \\
  |force(e)| &= |e|                                                                            \\
  |C_i^\delta e| &= C_i^\delta |e|                                                             \\
  |rec^\delta(e, \overline{C \mapsto x.e_C})| &= rec^\delta(|e|, \overline{C \mapsto x.|e_C|}) \\
  |map^\phi(x.v_0, v_1)| &= map^{|\phi|}(x.|v_0|, |v_1|)                                       \\
  |let(e_0, x.e_1)| &= |e_1|[|e_0|/x]
\end{align*}
%
\section{Logical Relation}
We define our logical relation below.
%
\begin{align*}
  E &\sim_{\texttt{\tiny{Unit}}} E' \text{always}  \\
  E &\sim_{\tau_0 \times \tau_1} E' \Leftrightarrow \forall k. \langle k, \pi_0 E_p\rangle \sim_{\tau_0} \pi_0 E', \forall k. \langle k, \pi_1 E_p\rangle \sim_{\tau_1} \pi_1 E' \\
  E &\sim_{\texttt{\tiny{susp }} \tau} E' \Leftrightarrow E_p \sim_\tau E' \\
  E &\sim_{\sigma \to \tau} E' \Leftrightarrow \forall E_0 \sim_\sigma E'_0. E_p E_{0p} \sim_\tau E' E'_0 \\
  E &\sim_\delta E' \Leftrightarrow \exists k, k', C, V, V'. V \sim_{\phi[\delta]} V', E \downarrow \langle k, C V_p \rangle, E' \downarrow C V'
\end{align*}
%
The relation is defined on closed terms, but we extend it to open terms.
Let $\Theta$ and $\Theta'$ be any substitutions such that $\forall x : \|\tau\|, \forall k, \langle k, \Theta(x) \rangle \sim_\tau \Theta'(x)$.
If $E\ \Theta \sim_\tau E'\ \Theta'$, then $E \sim_\tau E'$.

\section{Proof}

We require some lemmas.

The first states we can always ignore the cost of related terms.
%
\begin{lemma}[Ignore Cost]
  \label{lem:ignorecost}
\[
  E \sim_\tau E' \Leftrightarrow \forall k, \langle k, E_p \rangle \sim_\tau E'
\]
\end{lemma}
%
\begin{proof}
  We proceed by induction on type.

  Case $E \sim_{\texttt{Unit}} E'$. 
  Then $\forall k, \langle k, E_p \rangle \sim_{\texttt{\tiny{Unit}}} E'$ by definition.

  Case $E \sim_{\tau_0 \times \tau_1} E'$.
  By definition for $i\in{0,1}, \forall k_i, \langle k_i, \pi_i E_p \rangle \sim_{\tau_i} \pi_i E'$.
  Let $k$ be some cost.
  Then $\langle k, E_p \rangle \sim_{\tau_0 \times \tau_1} E'$ by definition.

  Case $E \sim_{\texttt{\tiny{susp }} \tau} E'$.
  By definition $E_p \sim_\tau E'$.
  Let $k$ be some cost.
  Then $\langle k, E_p \rangle \sim_{\texttt{\tiny{susp }} \tau} E'$.

  Case $E \sim_{\sigma \to \tau} E'$.
  Let $E_0, E_0'$ by some complexity language terms such that $E_0 \sim_\sigma E_0'$.
  Let $k$ be some cost.
  Then, $E_p\ E_0 \sim_\tau E'\ E_0'$.
  So $\langle k, E_p \rangle \sim_{\sigma \to \tau} E'$.

  Case $E \sim_\delta E'$.
  Then by definition there exists costs $k$ and $k'$, a constructor $C$, and complexity language values $V$ and $V'$ such that $V \sim_{\Phi[\delta]} V', E \downarrow \langle k, C V_p \rangle$, and $E' \downarrow C V'$.
  Since $E \downarrow \langle k, C V_p \rangle$, we know $\forall k_0, \exists k_0'. \langle k_0, E_p \rangle \downarrow \langle k_0', C V_p \rangle$.
  So by definition we have $\forall k_0, \langle k_0, E_p \rangle \sim_\Phi E'$.
\end{proof}
%
The next lemma states that if two terms step to related terms, then those terms are related.
%
\begin{lemma}[Related Step Back]
  \label{lem:relatedstepback}
  \[
    E \to F, E' \to F', F \sim_\sigma F' \implies E \sim_\sigma E'
  \]
\end{lemma}
%
\begin{proof}
  The proof proceeds by induction on type.
  
  Case \texttt{Unit}. Trivial since $E \sim_{\texttt{\tiny{Unit}}} E'$ always.

  Case $\delta$.
  By definition $\exists C, U, U', k, k'$ such that $F \downarrow \langle k, C U_p \rangle, F' \downarrow C U', U \sim_{\phi[\delta]} U'$.
  Since $E \to F$ and $E' \to F'$, $E \downarrow \langle k, C U_p \rangle$ and $E' \downarrow C U'$.
  Therefore since $U \sim_{\phi[\delta]} U'$, we have $E \sim_\delta E'$.

  Case $\sigma \to \tau$.
  Let $E_0 \sim_\sigma E'_0$.
  By definition, $F\ E_0 \sim_\tau F'\ E'_0$.
  Since $E \to F$ and $E' \to F'$, $E\ E_0 \to F\ E_0$ and $E'\ E'_0 \to F'\ E'_0$.
  So by the induction hypothesis, $E\ E_0 \sim_\tau E'\ E_0'$.
  So by definition, $E \sim_{\sigma \to \tau} E'$.

  Case $\tau_0 \times \tau_1$.
  Since $F \sim_{\tau_0 \times \tau_1} F'$, for $i\in\{0, 1\}$,  $\forall k_i, \langle k_i, \pi_i F_p \rangle \sim_{\tau_i} \pi_i F'$, by definition.
  From $E \to F$, we get $\langle k_i, \pi_i E_p \rangle \to \langle k_i', \pi_i F_p \rangle$.
  From $E' \to F'$, we get $\pi_i E' \to \pi_i F'$.
  We can apply our induction hypothesis to get $\langle k_i, \pi_i E_p \rangle \sim_{\tau_i} \pi_i E'$.
  By \ref{lem:ignorecost}, $\forall k_i, \langle k_i, \pi_i E_p \rangle \sim_{\tau_i} \pi_i E$.
  So by definition $E \sim_{\tau_0 \times \tau_1} E'$.

  Case $\texttt{susp }\tau$.
  Since $F \sim_{\texttt{\tiny{susp }}\tau} F'$, by definition $F_p \sim_\tau F'$.
  Since $E \to F$, $E_p \to F_p$.
  So by the induction hypothesis, since $E_p \to F_p, E' \to F', F_p \sim_\tau F'$, $E_p \sim_\tau E'$.
  So by definition $E \sim_{\texttt{\tiny{susp }}\tau} E'$.

\end{proof}
%
The next lemma states that related terms step to related terms
%
\begin{lemma}
  \label{lem:relatedstep}[Related Step]
  \[ E \to F, E' \to F', E \sim_\sigma E' \implies F \sim_\sigma F' \]
\end{lemma}
%
\begin{proof}
  The proof is by induction on type.

  Case \texttt{Unit}.
  $F \sim_{\texttt{\tiny{Unit}}} F'$ always.

  Case $\delta$.
  By definition, $E \sim_\delta E'$ implies $\exists C, V, V', k$ such that $E \downarrow \langle k, C V_p \rangle, E' \downarrow C V', V \sim_{\phi[\delta]} V'$.
  Since $E \to F$, $F \downarrow \langle k, C V_p \rangle$; and since $E \to F'$, $F' \downarrow C V'$.
  By \ref{lem:ignorecost}, $\langle k, V_p \rangle \sim_{\phi[\delta]} V'$.
  So because $F \downarrow \langle k, C V_p \rangle, F' \downarrow C V', \langle k, V_p \rangle \sim_{\phi[\delta]} V'$, we can apply our induction hypothesis to get $F \sim_\delta F'$.

  Case $\tau_0 \times \tau_1$.
  By definition $E \sim_{\tau_0 \times \tau_1} \implies \forall i \in \{0, 1\}, \forall k, \langle k_i, \pi_i E_p \rangle \sim_{\tau_i} \pi_i E'$.
  Fix some $k_i$.
  Since $E \to F$, $\langle k_i, \pi_i E_p \rangle \to \langle k_i, \pi_i F_p \rangle$.
  Since $E' \to F'$, $\pi_i E' \to \pi_i F'$.
  From $\langle k_i, \pi_i E_p \rangle \to \langle k_i, \pi_i F_p \rangle, \langle k_i, \pi_i E_p \rangle \sim_{\tau_i} \pi_i E'$, the induction hypothesis tells us $\langle k_i, \pi_i F_p \rangle \sim_{\tau_i} \pi_i F'$.
  So by definition $F \sim_{\tau_0 \times \tau_1} F'$.

  Case $\texttt{susp } \tau$.
  By definition $E \sim_{\texttt{susp }\tau} E' \implies E_p \sim_\tau E'$.
  Since $E \to F$, $E_p \to F_p$.
  From $E_p \to F_p, E' \to F', E_p \sim_\tau E'$, the induction hypothesis gives us $F_p \sim_\tau F'$.
  So by definition $F \sim_{\texttt{susp }\tau} F'$.

  Case $\sigma \to \tau$.
  Let $E_0 \sim_\sigma E_0'$.
  By definition, $E\ E_0 \sim_\tau E'\ E_0'$.
  Since $E \to F$, $E\ E_0 \to F\ E_0$.
  Since $E' \to F'$, $E'\ E_0' \to F'\ E_0'$.
  From $E\ E_0 \to F\ E_0, E'\ E_0' \to F'\ E_0', E\ E_0 \sim_\tau E'\ E_0'$, the induction hypothesis tells us $F\ E_0 \sim_\tau F'\ E_0'$.
  So by definition $F \sim_{\sigma \to \tau} F'$.
\end{proof}
%
The next lemma states that if the arguments to $map$ are related, then $map$ preserves the relatedness.
%
\begin{lemma}
  \label{lem:relatedmap}[Related Map]
  \[ E \sim_{\tau_1} E', E_0 \sim_{\tau_0} E_0' \implies \forall k. \langle k, map^\Phi(x, E_p, E_{0p})\rangle \sim_{\Phi[\tau_1]} map^\Phi(x, E', E_0') \]
\end{lemma}
%
\begin{proof}
  The proof proceeds by induction on type.

  Recall the definition of the $map$ macro.
  \begin{align*}
    map^t(x.E, E_0) &= E[E_0/x]                                                                                       \\
    map^T(x.E, E_0) &= E_0                                                                                            \\ 
    map^{\Phi_0 \times \Phi_1}(x.E, E_0) &= \langle map^{\Phi_0}(x.E, \pi_0 E_0), map^{\Phi_1}(x.E, \pi_1 E_0 \rangle \\
    map^{T \to \Phi}(x.E, E_0) &= \lambda y.map^\Phi(x.E, E_0\ y)
  \end{align*}

  Case $\Phi = t$.
  Then $map^t(x.E_p, E_{0p}) = E_p[E_{0p}/x]$ and $map^t(x.E', E_0') = E'[E_0'/x]$.
  Let $k$ be some cost.
  By \ref{lem:ignorecost}, $E \sim_{\tau_1} E'$ implies $\langle k, E_p \rangle \sim_{\tau_1} E'$.
  Since $\langle k, E_p \rangle \sim_{\tau_1} E'$ and $E_0 \sim_{\tau_0} E_0'$, $\langle k, E_p \rangle [E_{0p}/x] \sim_{\phi[\tau_0]} E'[E_0'/x]$.
  So $\forall k, \langle k, map^t(x.E_p, E_{0p}) \rangle \sim_{\Phi[\tau_1]} map^t(x.E', E_0')$.

  Case $\Phi = T$.
  Then $map^T(x.E_p, E_{0p}) = E_{0p}$ and $map^T(x.E', E_0') = E_0'$.
  By \ref{lem:ignorecost} $\forall k, \langle k, E_{0p} \rangle \sim_{\tau_0} E_0'$.
  So $\forall k, \langle k, map^T(x.E_p, E_{0p}) \rangle \sim_{\Phi[\tau_1]} map^T(x.E', E_0')$.

  Case $\Phi = \Phi_0 \times \Phi_1$.
  Then \\
  $map^{\Phi_0 \times \Phi_1}(x. E_p, E_{0p}) = \langle map^{\Phi_0}(x. E_p, \pi_0 E_{0p}), map^{\Phi_1}(x. E_p, \pi_1 E_{0p}) \rangle$.\\
  Similarly $map^{\Phi_0 \times \Phi_1}(x. E', E_0') = \langle map^{\Phi_0}(x. E', \pi_0 E_0'), map^{\Phi_1}(x. E', \pi_1 E_0') \rangle$.\\
  By definition, $\forall k, \langle k, \pi_0 E_{0p} \rangle \sim_{\Phi_0[\tau_0]} \pi_0 E_0'$.\\
  By the induction hypothesis, $\forall k, \langle k, map^{\Phi_0}(x. E_p, \pi_0 E_{0p}) \sim_{\Phi_0[\tau_1]} map^{\Phi_0[\tau_1]}(x. E', E_0')$.\\
  By definition, $\forall k, \langle k, \pi_1 E_{0p} \rangle \sim_{\Phi_1[\tau_0]} \pi_1 E_0'$.\\
  By the induction hypothesis, $\forall k, \langle k, map^{\Phi_1}(x. E_p, \pi_1 E_{0p}) \sim_{\Phi_1[\tau_1]} map^{\Phi_1[\tau_1]}(x. E', E_0')$.\\
  So by definition,
  \begin{align*}
    &\forall k, \langle k, \langle map^{\Phi_0}(x. E_p, \pi_0 E_{0p}), map^{\Phi_1}(x. E_p, \pi_1 E_{0p}) \rangle \rangle \sim_{\Phi[\tau_1]}  \\
    &\langle \langle map^{\Phi_0[\tau_1]}(x. E', E_0'), map^{\Phi_1[\tau_1]}(x. E', E_0') \rangle \rangle
  \end{align*}

  Case $T \to \Phi$.
  Then $map^{T \to \Phi}(x. E_p, E_{0p}) = \lambda y.map^\Phi(x.E_p, E_{0p}\ y)$ \\
  and $map^{T \to \Phi}(x. E', E_0') = \lambda y.map^\Phi(x.E', E_0'\ y)$.
  Let $E_1 : T$.
  Then \\
  $\lambda y.map^\Phi(x.E_p, E_{0p}\ y)\ E_1 \to map^\Phi(x.E_p, E_{0p}\ E_1)$.
  Similarly, $\lambda y.map^\Phi(x.E', E_0'\ y)\ E_1' \to map^\Phi(x. E', E_0'\ E_1')$.
  Since $E_0 \sim E_0'$ and $E_1 \sim E_1'$, we have $E_{0p}\ E_1 \sim E_0'\ E_1'$.
  So by our induction hypothesis, $map^\Phi(x.E_p, E_{0p}\ E_1) \sim map^\Phi(x. E', E_0'\ E_1')$.
  So by \ref{lem:relatedstepback}, $\lambda y.map^\Phi(x.E_p, E_{0p}\ y)\ E_1 \sim \lambda y.map^\Phi(x.E', E_0'\ y)\ E_1'$.
  So by definition, \\
  $\lambda y.map^\Phi(x.E_p, E_{0p}\ y) \sim \lambda y.map^\Phi(x.E', E_0'\ y)$.\\
  So $map^{T \to \Phi}(x. E_p, E_{0p}) \sim map^{T \to \Phi}(x. E', E_0')$.
\end{proof}
%
Our last lemma is about the relatedness of $rec$ terms.
%
\begin{lemma}[Related Rec]
  \label{lem:relatedrec}
  \[ E \sim_\delta E', \forall C, E_C \sim_\tau E_C' \implies rec(E_p, \overline{C \mapsto x.E_c}) \sim_\tau rec(E', \overline{C \mapsto x.E_c'}) \]
\end{lemma}
%
\begin{proof}
  Recall the rule for evaluating $rec$ in the complexity language:
  \begin{prooftree}
    \AxiomC{$E \downarrow C V_0$}
    \AxiomC{$map^\Phi(y,\langle y, rec(y, \overline{C \mapsto x.E_C})\rangle, V_0) \downarrow V_1$}
    \AxiomC{$E_C[V_1/x] \downarrow V$}
    \TrinaryInfC{$rec(E, \overline{C \mapsto x.E_C}) \downarrow V$}
  \end{prooftree}
  By definition of $\sim_\delta$, $\exists k, C, V_0, V_0'$ such that $E \downarrow \langle k, C V_{0p} \rangle, E' \downarrow C V_0'$, and $V_0 \sim_\delta V_0'$.
  Our proof proceeds by induction on the number of constructors in $C V_{0p}$.
  If $\Phi = T$, then $map^\Phi(y, \langle y, rec(y, \overline{C \mapsto x.E_C})\rangle, V_{0p}) = \langle y, rec(y, \overline{C \mapsto x.E_C})\rangle[V_{0p}/y] = \langle V_{0p}, rec(V_{0p}, \overline{C \mapsto x.E_C}) \rangle$.
  Similarly for the pure potential, $map^\Phi(y, \langle y, rec(y, \overline{C \mapsto x.E_C'})\rangle, V_{0p}') = \langle y, rec(y, \overline{C \mapsto x.E_C'})\rangle [V_0'/y] = \langle V_0', rec(V_0', \overline{C \mapsto x.E_C'}) \rangle$.
  By the induction hypothesis, $rec(V_{0p}, \overline{C \mapsto x.E_C}) \sim_\tau rec(V_0', \overline{C \mapsto x.E_C'})$.
  By definition of $\sim_{\texttt{susp }\tau}$, for any $k$, $\langle k, rec(V_{0p}, \overline{C \mapsto x.E_C}) \rangle \sim_{\texttt{susp }\tau} rec(V_0', \overline{C \mapsto x.E_C'})$.
  So by definition of $\sim_{\tau_0 \times \tau_1}$, $\langle 0, \langle V_{0p}, rec(V_{0p}, C \mapsto x.E_C) \rangle\rangle \sim_{\phi[\delta \times \texttt{susp }\tau]} \langle V_0', rec(V_0', \overline{C \mapsto x.E_C'})\rangle$.
  So by \ref{lem:relatedmap}, $\forall k. \langle k, map^\Phi(y, \langle y, rec(y, \overline{C \mapsto x.E_C})\rangle, V_{0p}) \sim_{\phi[\delta \times \texttt{susp }\tau]} map^\Phi(y, \langle y, rec(y, \overline{C \mapsto x.E_C'}) \rangle, V_0')$.
  Let $\langle 0, map^\Phi(y, \langle y, rec(y, \overline{C \mapsto x.E_C})\rangle, V_{0p}) \downarrow V_1$.
  Let $map^\Phi(y, \langle y, rec(y, \overline{C \mapsto x.E_C'}) \rangle, V_0') \downarrow V_1'$.
  By \ref{lem:relatedstep}, $V_1 \sim_{\phi[\delta \times \texttt{susp }\tau]} V_1'$.

  If $\Phi = t$, then $map^\Phi(y, \langle y, rec(y, \overline{C \mapsto x.E_C})\rangle, V_{0p}) = V_{0p}$.
  Similarly, $map^\Phi(y, \langle y, rec(y, \overline{C \mapsto x.E_C'})\rangle V_0') = V_0'$.
  So in this case $V_0 = V_1$ and $V_0' = V_1'$.
  We have already established $V_0 \sim_\tau V_0'$.

  So in both cases $V_1 \sim_{\phi[\delta \times \texttt{susp }\tau]} V_1'$.

  By definition of the relation $E_C[V_{1p}/x] \sim_\tau E_C'[V_1'/x]$.
  Let $E_C[V_{1p}/x] \downarrow V_2$ and $E_C'[V_1'/x] \downarrow V_2'$.
  By \ref{lem:relatedstep}, $V_2 \sim_\tau V_2'$.
  So by \ref{lem:relatedstepback}, $rec(E_p, \overline{C \mapsto x.E_C}) \sim_\tau rec(E', \overline{C \mapsto x.E_C'})$.
\end{proof}
%
Our theorem is that for all well-typed terms in the source language, the
complexity translation of the term is related to the pure potential translation
of that term.
%
\begin{theorem}[Distinct Recurrence]
  \[ \gamma \vdash e : \tau \implies \|e\| \sim_\tau |e| \]
\end{theorem}
%
\begin{proof}
  Our proof is by induction on the typing derivation $\gamma \vdash e : \tau$.

  Case \AxiomC{}\UnaryInfC{$\gamma, x : \sigma \vdash x : \sigma$}\DisplayProof.
  Then by definition of the logical relation, $\forall k, \langle k, \Theta(x) \rangle \sim_\sigma \Theta'(x)$.
  Since $\|x\| = \langle 0, x \rangle$ and $|x| = x$, we have $\langle 0, x \rangle \sim_\sigma x$.

  Case \AxiomC{}\UnaryInfC{$\gamma \vdash e : Unit$}\DisplayProof.
  By definition, $\|e\| \sim_{\texttt{Unit}} |e|$ always.

  Case \AxiomC{$\gamma \vdash e_0 : \tau_0$}\AxiomC{$\gamma \vdash e_1 : \tau_1$}\BinaryInfC{$\gamma \vdash \langle e_0, e_1 \rangle : \tau_0 \times \tau_1$}\DisplayProof
  By the induction hypothesis, $\|e_0\| \sim_{\tau_0} |e_0|$ and $\|e_1\| \sim_{\tau_1} |e_1|$.
  By \ref{lem:ignorecost}, $\forall k, \langle k, \|e_0\|_p \rangle \sim_{\tau_0} |e_0|$
  and $\forall k, \langle k, \|e_1\|_p \rangle \sim_{\tau_1} |e_1|$.
  So by definition, $\|\langle e_0, e_1 \rangle \| \sim_{\tau_0 \times \tau_1} |\langle e_0, e_1 \rangle |$.

  Case \AxiomC{$\gamma \vdash e_0 : \tau_0 \times \tau_1$}\AxiomC{$\gamma, x_0 : \tau_0, x_1 : \tau_1 \vdash e_1 : \tau$}\BinaryInfC{$\gamma \vdash split(e_0, x_0.x_1.e_1) : \tau$}\DisplayProof
  By the induction hypothesis, $\|e_0\| \sim_{\tau_0 \times \tau_1} |e_0|$ and $\|e_1\| \sim_\tau |e_1|$.
  From $\|e_0\| \sim_{\tau_0 \times \tau_1} |e_0|$ it follows by definition that 
    $\forall k, \langle k, \pi_0 \|e_0\|_p \rangle \sim_{\tau_0} \pi_0 |e_0|$ and
    $\forall k, \langle k, \pi_1 \|e_1\|_p \rangle \sim_{\tau_1} \pi_1 |e_1|$.
  The complixity translation is $\|split(e_0, x_0.x_1.e_1)\| = \|e_0\|_c +_c \|e_1\|[\pi_0\|e_0\|_p/x_0, \pi_1\|e_1\|_p/x_1]$.
  The pure potential translation is $|split(e_0, x_0.x_1.e_1)| = |e_1|[\pi_0|e_0|/x_0, \pi_1|e_0|/x_1]$.
  By \ref{lem:ignorecost}, it suffices to show $\|e_1\|[\pi_0\|e_0\|_p/x_0, \pi_1\|e_1\|_p/x_1] \sim_\tau |e_1|[\pi_0|e_0|/x_0, \pi_1|e_0|/x_1]$
  By definition of the relation, it suffices to show $\|e_1\| \sim_\tau |e_1|$,
    $\forall k, \langle k, \pi_0 \|e_0\|_p \rangle \sim_{\tau_0} \pi_0 |e_0|$,
    and $\forall k, \langle k, \pi_1 \|e_0\|_p \rangle \sim_{\tau_1} \pi_1 |e_0|$.
  Since we have already establed all three conditions, we have $\|split(e_0, x_0. x_1.e_1)\| \sim_\tau |split(e_0,x_0.x_1.e_1)|$.


  Case \AxiomC{$\gamma, x : \sigma \vdash e : \tau$}\UnaryInfC{$\gamma \vdash \lambda x.e : \sigma \to \tau$}\DisplayProof
  By the induction hypothesis $\|e\|\sim_\tau|e|$.
  The complexity translation is $\|\lambda x.e\| = \langle 0, \lambda x.\|e\|\rangle$.
  The pure potential translation is $|\lambda x.e| = \lambda x.|e|$.
  Let $E_0 : \|\sigma\|$ and $E_0' : |\sigma|$ be complexity language terms such that $E_0 \sim_\sigma E_0'$.
  Then $\langle 0, \lambda x.\|e\|\rangle\ E_0 \to \langle 0 + E_{0c}, \|e\|[x \mapsto E_0]\rangle$
    and $\lambda x.|e|\ E_0' \to |e|[x \mapsto E_0']$.
  Since $\|e\| \sim_\tau |e|$ and $E_0 \sim_\sigma E_0'$, $\|e\|[x \mapsto E_0] \sim_\tau |e|[x \mapsto E_0']$.
  By \ref{lem:relatedstepback}, $\langle 0, \lambda x. \|e\| \rangle\ E_0 \sim_\tau (\lambda x.|e|)\ E_0'$.
  So by definition $\langle 0, \lambda x. \|e\| \rangle \sim_{\sigma \to \tau} \lambda x. |e|$.
  So $\|\lambda x.e\| \sim_{\sigma \to \tau} |\lambda x.e|$.

  Case \AxiomC{$\gamma \vdash e_0 : \sigma \to \tau$}\AxiomC{$\gamma \vdash e_1 : \sigma$}\BinaryInfC{$\gamma \vdash e_0\ e_1 : \tau$}\DisplayProof
  The complexity translation is $\|e_0\ e_1\| = (1 + \|e_0\|_c + \|e_1\|_c) +_c \|e_0\|_p \|e_1\|_p$.
  The pure potential translation is $|e_0\ e_1| = |e_0| |e_1|$.
  By \ref{lem:ignorecost}, it suffices to show $\|e_0\|_p \|e_1\|_p \sim_\tau |e_0||e_1|$.
  By the induction hypothesis, $\|e_0\| \sim_{\sigma \to \tau} |e_0|$ and $\|e_1\| \sim_\sigma |e_1|$.
  By definition, $\|e_0\|_p \|e_1\|_p \sim_{\tau} |e_0| |e_1|$.

  Case \AxiomC{$\gamma \vdash e : \tau$}\UnaryInfC{$\gamma \vdash delay(e) : susp\ \tau$}\DisplayProof
  By the induction hypothesis $\|e\| \sim_\tau |e|$.
  So $\langle 0, \|e\|\rangle \sim_{\texttt{susp }\tau} |e|$.
  The complexity translation is $\|delay(e)\| = \langle 0, \|e\|\rangle$.
  The pure potential translatio is $|delay(e)| = |e|$.
  So $\|delay(e)\| \sim_{\texttt{susp }\tau} |delay(e)|$.

  Case \AxiomC{$\gamma \vdash e : susp\ \tau$}\UnaryInfC{$\gamma \vdash force(e) : \tau$}\DisplayProof
  By the induction hypothesis $\|e\| \sim_{\texttt{susp }\tau} |e|$.
  So by definition of the relation at \texttt{susp} type, $\|e\|_p \sim_\tau |e|$.
  By \ref{lem:ignorecost}, $\forall k, \langle k, \|e\|_{pp} \rangle \sim_\tau |e|$.
  The complexity translation is $\|force(e)\| = \|e\|_c +_c \|e\|_p$.
  The pure potential translation is $|force(e)| = |e|$.
  So $\|e\|_c +_c \|e\|_p \sim_\tau |e|$.
  So $\|force(e)\| \sim_\tau |force(e)|$.

  Case \AxiomC{$\gamma \vdash e_0 : \sigma$}\AxiomC{$\gamma, x : \sigma \vdash e_1 : \tau$}\BinaryInfC{$\gamma \vdash let(e_0, x.e_1) : \tau$}\DisplayProof
  By the induction hypothesis $\|e_0\| \sim_\sigma |e_0|$ and $\|e_1\| \sim_\tau |e_1|$.
  So $\|e_1\|[\|e_0\|_p/x] \sim_\tau |e_1|[|e_0|/x]$.
  By \ref{lem:ignorecost}, $\forall k, \langle k, \|e_1\|_p[\|e_0\|_p/x] \rangle \sim_\tau |e_1|[|e_0|/x]$.
  The complexity translation is $\|let(e_0, x.e_1)\| = \|e_0\|_c +_c \|e_1\|[\|e_0\|_p/x]$.
  The pure potential translation is $|let(e_0, x.e_1)| = |e_1|[|e_0|/x]$.
  So $\|let(e_0, x.e_1)\|  \sim_\tau |let(e_0, x.e_1)|$.

  Case \AxiomC{$\gamma, x : \tau_0 \vdash v_1 : \tau_1$}\AxiomC{$\gamma \vdash v_0 : \phi[\tau_0]$}\BinaryInfC{$\gamma \vdash map^\phi(x.v_1, v_0) : \phi[\tau_1]$}\DisplayProof
  By the induction hypothesis $\|v_1\| \sim_{\tau_1} |v_1|$ and $\|v_0\| \sim_{\phi[\tau_0]} |v_0|$.
  By \ref{lem:relatedmap}, $\forall k, \langle k, map^\Phi(x.\|v_1\|_p, \|v_0\|_p)\rangle \sim_{\phi[\tau_1]} map^\Phi(x.|v_1|, |v_0|)$.
  The complexity translation is $\|map^\phi(x.v_1, v_0)\| = \langle 0, map^\Phi(x.\|v_0\|_p, \|v_1\|_p)\rangle$.
  The pure potential translation is $|map^\phi(x.v_1, v_0| = map^\Phi(x, |v_0|, |v_1|)$.
  So we have $\|map^\phi(x.v_1, v_0)\| \sim_{\phi[\tau_1]} |map^\phi(x.v_1, v_0|$.

  Case \AxiomC{$\gamma \vdash e_0 : \delta$}\AxiomC{$\forall C (\gamma, x: \phi_C[\delta \times susp\ \tau] \vdash e_c : \tau$}\BinaryInfC{$\gamma \vdash rec^\delta(e_0, \overline{C \mapsto x.e_C}) : \tau$}\DisplayProof
  By the induction hypothesis $\|e_0\| \sim_\delta |e_0|$ and $\forall C, \|e_c\| \sim_\tau |e_c|$.
  By \ref{lem:ignorecost}, $\forall k, \langle k \|e_C\| \sim_\tau |e_c$, so $1 +_c \|e_C\| \sim_\tau |e_c|$.
  So by \ref{lem:relatedrec}, $rec(\|e_0\|_p, \overline{C \mapsto x.1 +_c \|e_C\|}) \sim_\tau rec(|e_0|, \overline{C \mapsto x.1 +_c |e_C|})$.
  So by \ref{lem:ignorecost}, $\|e_0\|_c +_c rec(\|e_0\|_p, \overline{C \mapsto x.1 +_c \|e_C\|}) \sim_\tau rec(|e_0|, \overline{C \mapsto x.1 +_c |e_C|})$

  Case \AxiomC{$\gamma \vdash e : \phi[\delta]$}\UnaryInfC{$\gamma \vdash C e : \delta$}\DisplayProof
  By the induction hypothesis, $\|e\| \sim_{\phi[\delta]} |e|$.
  There exists $V, V'$ such that $\|e\| \downarrow V$ and $|e| \downarrow V'$.
  By \ref{lem:relatedstep} $V \sim_{\phi[\delta]} V'$.
  Since $\|e\| \downarrow V$, $\langle k, C\ \|e\| \rangle \downarrow \langle k, C\ V_p$.
  Similarly, since $|e| \downarrow V'$, $C |e| \downarrow C V'$.
  So by definition we have $\langle k, C\ \|e\|\rangle \sim_\delta C |e|$.
  The complexity translation is $\|C e\| = \langle \|e\|, C\|e\|_p\rangle$.
  The pure potential translation is $|C e| = C |e|$.
  Therefore by \ref{lem:ignorecost}, $\|C e\| \sim_\delta |C e|$.
\end{proof}

