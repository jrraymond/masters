\chapter{Parallel Functional Program Analysis}

We demonstrate the flexibility of the method developed in \citet{Danner2015} by
extending it to parallel cost semantics.  To analyze the complexity of a
sequential program, we are only interested in a measure of how many steps
required to run the program. To analyze the complexity of a parallel program,
we need a measure which takes into account the extent to which a computation
may be run on multiple processors.  The cost semantics we use are work and span
\citet{Harper2012PFPL}.  We give a brief overview of work and span, then
demonstrate how to alter the semantics and translation function. We state and
sketch a proof of a bounding theorem stating the translation of a program is an
upper bound on the cost of executing the program. Finally we give two examples
of the recurrence extraction and interpretation.

\section{Work and span}
Work and span is a method of predicting the running time of programs that may
be run on arbitrary number of processors. Instead of an approximation of the
number of steps required to execute a program, work and span instead results in
a cost graph, a specification of the dependencies between subcomputations of
the program. The cost graph can be compiled into The work of a program
corresponds to the total number of steps needed to run.  The span of a program
is the steps in the critical path.  The critical path is the largest number of
steps that must be executed sequentially.  The length of the critical path
determines how much a program can be parallelized.  If the span is equal to the
work, than every step in the computation depends on the previous step, and the
program cannot be parallelized.


A cost graph is defined as follows.

\[ \mathcal{C} ::= 0\ |\  1\ |\  \mathcal{C} \oplus \mathcal{C}\ |\  \mathcal{C} \otimes \mathcal{C} \]

The operator $\oplus$ connects to cost graphs who must be combined
sequentially.  The operator $\otimes$ connects cost graphs which may be
combined in parallel.

The work of a cost graph is defined as
%
\begin{equation*}
  work(c) = \begin{cases}
    0 &\text{if } c = 0 \\
    1 &\text{if } c = 1 \\
    work(c_0) + work(c_1) &\text{if } c = c_0 \otimes c_1 \\
    work(c_0) + work(c_1) &\text{if } c = c_0 \oplus c_1
  \end{cases}
\end{equation*}
%
The span of a cost graph is defined as
%
\begin{equation*}
  span(c) = \begin{cases}
    0 &\text{if } c = 0 \\
    1 &\text{if } c = 1 \\
    max(span(c_0), span(c_1)) &\text{if } c = c_0 \otimes c_1 \\
    span(c_0) + span(c_1) &\text{if } c = c_0 \oplus c_1
  \end{cases}
\end{equation*}
%
We alter the operational cost semantics of the source language to produce a cost
graph instead of a natural number. Figure \ref{fig:ws_srclang_oper_sem} shows
the new operational cost semantics. For tuples, the subexpressions may be evaluated
in parallel, so the cost of evaluating a tuple is the cost graphs of the
subexpressions connected by $\otimes$.  For \T{split}, the second subexpression
depends on the result of the first subexpression, so the cost of evaluating the
\T{split} is the cost graphs of the subexpression connected by $\oplus$. In
every rule except for tuples and function application, $+$ gets replacted with
$\oplus$. This is because tuples and function application are the two syntactic
forms which consist of multiple subexpressions whose evaluation does not depend
on each other.
%
\begin{figure}
\label{fig:ws_srclang_oper_sem}
\caption{Source language operational semantics}

\bigskip

\AxiomC{$e_0 \downarrow^{n_0} v_0$}
\AxiomC{$e_1 \downarrow^{n_1} v_1$}
\BinaryInfC{$\langle e_0, e_1 \rangle \downarrow^{n_0 \otimes n_1} \langle v_0, v_1 \rangle$}
\DisplayProof
%
\quad
%
\AxiomC{$e_0 \downarrow^{n_0} \langle v_0, v_1 \rangle$}
\AxiomC{$e_1[v_0/x_0, v_1/x_1] \downarrow^{n_1} v$}
\BinaryInfC{$split(e_0, x_0.x_1.e_1) \downarrow^{n_0 \oplus n_1} v$}
\DisplayProof

\bigskip

\AxiomC{$e_0 \downarrow^{n_0} \lambda x.e_0'$}
\AxiomC{$e_1 \downarrow^{n_1} v_1$}
\AxiomC{$e_0'[v_1/x] \downarrow^n v$}
\TrinaryInfC{$e_0\ e_1 \downarrow^{(n_0 \otimes n_1) \oplus n \oplus 1} v$}
\DisplayProof
%
\quad
%
\AxiomC{}
\UnaryInfC{$delay(e) \downarrow^0 delay(e)$}
\DisplayProof

\bigskip

\AxiomC{$e \downarrow^{n_0} delay(e_0)$}
\AxiomC{$e_0 \downarrow^{n_1} v$}
\BinaryInfC{$force(e) \downarrow^{n_0 \oplus n_1} v$}
\DisplayProof
%
\quad
%
\AxiomC{$e \downarrow^n v$}
\UnaryInfC{$C e \downarrow^n C v$}
\DisplayProof

\bigskip

\AxiomC{$e \downarrow^{n_0} C v_0$}
\AxiomC{$map^{\phi_C}(y.\langle y, delay(rec(y, \overline{C \mapsto x.e_C}))\rangle, v_0) \downarrow^{n_1} v_1$}
\AxiomC{$e_C[v_1/x] \downarrow^{n_2} v$}
\TrinaryInfC{$rec(e, \overline{C \mapsto x.e_C}) \downarrow^{1 \oplus n_0 \oplus n_1 \oplus n_2} v$}
\DisplayProof

\bigskip

\AxiomC{}
\UnaryInfC{$map^t(x.v, v_0) \downarrow^0 v[v_0/x]$}
\DisplayProof
%
\quad
%
\AxiomC{}
\UnaryInfC{$map^\tau(x.v, v_0) \downarrow^0 v_0$}
\DisplayProof

\bigskip

\AxiomC{$map^{\phi_0}(x.v, v_0) \downarrow^{n_0} v_0'$}
\AxiomC{$map^{\phi_1}(x.v, v_1) \downarrow^{n_1} v_1'$}
\BinaryInfC{$map^{\phi_0 \times \phi_1}(x.v, \langle v_0, v_1 \rangle) \downarrow^{n_0 \otimes n_1} \langle v_0', v_1'\rangle$}
\DisplayProof

\bigskip

\AxiomC{}
\UnaryInfC{$map^{\tau \to \phi}(x.v, \lambda y.e) \downarrow^0 \lambda y.let(e, z.map^\phi(x.v, z))$}
\DisplayProof
%
\quad
%
\AxiomC{$e_0 \downarrow^{n_0} v_0$}
\AxiomC{$e_1[v_0/x] \downarrow^{n_1} v$}
\BinaryInfC{$let(e_0, x.e_1) \downarrow^{n_0 \oplus n_1} v$}
\DisplayProof
\end{figure}
%
The complexity translation is given in Figure
\ref{fig:ws_complexity_translation}.  The operator $E_0 \oplus_c E_1$ is
syntactic sugar for $\langle E_0 \oplus E_{1c}, E_{1p} \rangle$.  The
translation is similar to the original translation except we replace the use of
$+$ and $+_c$ with $\oplus$ and $\oplus_c$.  In the tuple case and function
application case we use $\otimes$ since the arguments do not depend on each
other and may be computed in parallel.
%
\begin{figure}
  \label{fig:ws_complexity_translation}
  \caption{Work and span translation from source language to compleity language}
  \begin{align*}
    \|x\| &= \langle 0, x \rangle \\
    \|\langle\rangle\| &= \langle 0, \langle \rangle \rangle \\
    \|\langle e_0, e_1 \rangle \| &= \langle \|e_0\|_c \otimes \|e_1\|_c, \langle \|e_0\|_p, \|e_1\|_p\rangle\rangle \\
    \|split(e_0, x_0.x_1.e_1)\| &= \|e_0\|_c \oplus_c \|e_1\|[\pi_0\|e_0\|_p/x_0, \pi_1\|e_1\|_p/x_1] \\
    \|\lambda x.e\| &= \langle 0, \lambda x.\|e\| \rangle \\
    \|e_0\ e_1\| &= 1 \oplus (\|e_0\|_c \otimes \|e_1\|_c) \oplus_c \|e_0\|_p\ \|e_1\|_p \\
    \|delay(e)\| &= \langle 0, \|e\|\rangle \\
    \|force(e)\| &= \|e\|_c \oplus_c \|e\|_p \\
    \|C_i^\delta e\| &= \langle \|e\|_c, C_i^\delta \|e\|_p \rangle \\
    \|rec^\delta(e, \overline{C \mapsto x.e_C})\| &= \|e\|_c \oplus_c rec^\delta(\|e\|_p, \overline{C \mapsto x.1 \oplus_c \|e_C\|}) \\
    \|map^\phi(x.v_0, v_1)\| &= \langle 0, map^{\langle\langle \phi \rangle \rangle} (x. \|v_0\|_p, \|v_1\|_p)\rangle \\
    \|let(e_0, x.e_1)\| &= \|e_0\|_c \oplus_c \|e_1\|[\|e_0\|_p/x]
  \end{align*}
\end{figure}
%
\section{Bounding Relation}
%
We mutually define the following relations.
\begin{enumerate}
  \item $e \sqsubseteq_\tau E, \emptyset \vdash_\phi e : \tau$ and $\emptyset \vdash_{\|\psi\|} E : \|\tau\|$
  \item $v \sqsubseteq_\tau^{val} E, $ where $\emptyset \vdash v : \tau$ and $\emptyset \vdash_{\|\psi\|} E : \llangle \tau \rrangle$
  \item $v \sqsubseteq_{\phi,R}^{val} E, $ where $ \emptyset \vdash_\psi v : \phi[\gamma]$ and $\emptyset \vdash_{\|\psi\|} E : \llangle \phi \rrangle[\delta]$
  \item $e \sqsubseteq_{\phi,R} E$, where $\emptyset \vdash_\psi e : \phi[\delta]$ and $\emptyset \vdash_{\|\psi\|} E : \|\phi\|[\delta]$
\end{enumerate}
%
The $R$ parameterizes the relation. The mutual definitions are given below.
%
\begin{enumerate}
  \item We define $e \sqsubseteq_\tau E$ as if $e \downarrow^n v$, then
    \begin{itemize}
      \item $n \leq E_c$
      \item $v \sqsubseteq_\tau^{val} E_p$.
    \end{itemize}
  \item We define $v \sqsubseteq_\tau^{val} E$ as
    \begin{itemize}
      \item $v \sqsubseteq_{unit}^{val} E$ always.
      \item $\langle v_0,v_1 \rangle \sqsubseteq_{\tau_0 \times \tau_1}^{val} E$ iff $v_0 \sqsubseteq_{\tau_0}^{val} \pi_0 E$ and $v_1 \sqsubseteq_{\tau_2}^{val} \pi_1 E$.
      \item $\T{delay}(e) \sqsubseteq_{\T{susp }\tau}^{val}$ if $e \sqsubseteq_\tau E$.
      \item $v \sqsubseteq_\delta^{val} E$ is inductively defined by
        \begin{prooftree}
          \AxiomC{$\textbf{C} : \phi \to \delta \in \psi$}
          \AxiomC{$v \sqsubseteq_{\phi, -\sqsubseteq_\delta^{val}-}^{val} E'$}
          \AxiomC{$\textbf{C}\ E' \leq_\delta E$}
          \TrinaryInfC{$\textbf{C}\ v \sqsubseteq_\delta^{val} E$}
        \end{prooftree}
      \item $\lambda x.e \sqsubseteq^{val}_{\tau \to \phi,R} E$ if for all $v$ and $E_0$, if $v \sqsubseteq_\tau^{val} E_0$, then $e[v/x] \sqsubseteq_{\phi,R}(E\ E_0)$
    \end{itemize}
  \item We define $v \sqsubseteq_{\phi,R}^{val} E_p$ as
    \begin{itemize}
      \item $v \sqsubseteq_{t,R}^{val} E $ if $R(v,E)$
      \item $v \sqsubseteq_{\tau,R}^{val} E$ if $v \sqsubseteq_\tau^{val} E$.
      \item $\langle v_0,v_1\rangle \sqsubseteq_{\phi_0\times\phi_1,R}^{val} E$ if $v_0 \sqsubseteq_{\phi_0,R}^{val} \pi_0 E$ and $v_1 \sqsubseteq_{\phi_0,R}^{val} \pi_1 E$
    \end{itemize}
  \item We define as $e \sqsubseteq_{\phi,R} E$ as if $e \downarrow^n v$, then
    \begin{itemize}
      \item $n \leq E_c$
      \item $v \sqsubseteq_{\phi,R}^{val} E_p$
    \end{itemize}
\end{enumerate}
%
The statement of the bounding theorem is the same.
%
\begin{theorem}
  If $\gamma \vdash e : \tau, $ then $e \sqsubseteq_\tau \|e\|$.
\end{theorem}
\begin{proof}
  The proof is by induction on the derivation of $\gamma \vdash e : \tau$ and
  is identical to the proof given in \citet{Danner2015} so is not included
  here.
\end{proof}

\section{Parallel List Map}
If we revisit \T{map} using the work and span translation, we will get a different result.


We use the same data type \T{list} and \T{map} function as in sequential list map.
\begin{equation*}
  \T{datatype list} = \T{Nil of Unit | Cons of int $\times$ list}
\end{equation*}

\begin{equation*}
  \T{map} = \lambda f. \lambda xs . \T{rec}(xs, \T{Nil} \mapsto \T{Nil}, \T{Cons} \mapsto \langle y \langle ys, y \rangle\rangle. \T{Cons}\langle f\ y, \T{force}(r)\rangle)
\end{equation*}

The derivation of the complexity expression is given in Figure \ref{fig:ws_map_complexity_translation}.
\begin{figure}
  \label{fig:ws_map_complexity_translation}
  \caption{Work and span complexity translation of \T{map f xs}.}
  \[\begin{split}
    &\|\T{map f xs}\| = \\
    &  (3 \oplus f_c \otimes xs_c) \oplus_c \T{rec}(xs_p, \T{Nil}\mapsto 1 \oplus_c \|\T{Nil}\|, \\
    &\quadten\qquad \T{Cons} \mapsto \langle y, \langle ys, r \rangle \rangle. 1 \oplus_c \|\T{Cons}\langle f\ y, \T{force}(r)\rangle \|) \\
    %Nil branch
    &\quad 1 \oplus_c \|\T{Nil}\| = 1 \oplus_c \langle 0, Nil \rangle = \langle 1, Nil \rangle \\
    %Cons branch
    &\quad\|\T{Cons}\langle f\ y, \T{force}(r)\rangle \| = \langle \|\langle f\ y, \T{force}(r)\rangle\|_c, \T{Cons} \| \langle f\ y, \T{force}(r)\rangle \|_p\rangle \\
    &\qquad \|\langle f\ y, \T{force}(r)\rangle\| = \langle \|f\ y\|_c \otimes \|\T{force}(r)\|_c, \langle \|f\ y\|_p, \|\T{force}(r)\|_p\rangle\rangle \\
    %f y
    &\quadthree \|f\ y\| = (1 \oplus \|f\|_c \otimes \|y\|_c) \oplus_c \|f\|_p \|y\|_p \\
    &\quadfour = (1 \oplus \langle 0, f_p \rangle_c \otimes \langle 0, y \rangle_c) \oplus_c \langle 0, f_p \rangle_p \langle 0, y \rangle_p \\
    &\quadfour = 1 \oplus_c f_p\ y \\
    &\quadfour = \langle 1 \oplus (f_p y)_c, (f_p\ y)_p \rangle \\
    %force r
    &\quadthree \|\T{force}(r)\| = \|r\|_c \oplus_c \|r\|_p \\
    &\quadfour = \langle 0, r \rangle_c \oplus_c \langle 0, r \rangle_p \\
    &\quadfour = r \\
    %<...>
    &\qquad \|\langle f\ y, \T{force}(r)\rangle\| = \langle (1 \oplus (f_p\ y)_c) \otimes r_c, \langle (f_p\ y)_p, r_p \rangle \rangle \\
    %Cons<..>
    &\quad\|\T{Cons}\langle f\ y, \T{force}(r)\rangle \| = \langle (1 \oplus (f_p\ y)_c) \otimes r_c, \T{Cons} \langle (f_p\ y)_p, r_p \rangle \rangle \\
    %rec
    &\|\T{map f xs}\| = (3 \oplus f_c \otimes xs_c) \oplus_c \\
    &\quad \T{rec}(xs_p, \T{Nil}\mapsto \langle 1, \T{Nil}\rangle, \\
    &\quadfour \T{Cons} \mapsto \langle y, \langle ys, r \rangle \rangle. \langle 1 \oplus ((1 \oplus (f_p\ y)_c) \otimes r_c), \T{Cons} \langle (f_p\ y)_p, r_p \rangle \rangle)
  \end{split}\]
\end{figure}

The complexity language translation is
\begin{equation*}
\begin{split}
    &\|\T{map f xs}\| = (3 \oplus f_c \otimes xs_c) \oplus_c \\
    &\quad \T{rec}(xs_p, \T{Nil}\mapsto \langle 1, \T{Nil}\rangle, \\
    &\quadfour \T{Cons} \mapsto \langle y, \langle ys, r \rangle \rangle. \langle 1 \oplus ((1 \oplus (f_p\ y)_c) \otimes r_c), \T{Cons} \langle (f_p\ y)_p, r_p \rangle \rangle)
\end{split}
\end{equation*}


We interpret lists as a pair of their largest element and length.
\begin{align*}
  \llbracket \T{list} \rrbracket &= \mathbb{Z} \times \mathbb{N} \\
  D^\T{list} &= \{*\} + (\llbracket \mathbb{Z} \rrbracket \times \llbracket \T{list} \rrbracket) \\
  size_\T{list}(*) &= (0, 0) \\
  size_\T{list}((x, (m, n))) &= (max(x, m), 1 + n)
\end{align*}

The interpretation of the recurser is given in Figure \ref{fig:ws_map_interpretation}.

\begin{figure}
  \label{fig:ws_map_interpretation}
  \caption{Interpretation of recurser in \T{map}}
  \[\begin{split}
      &\quad \text{Let } \eta = \{ xs \mapsto (0, (m, n)), f \mapsto (0, f)\}\\
      &\quad g(f, (m, n)) = \llbracket \T{rec}(xs_p, \T{Nil} \mapsto \langle 1, \T{Nil}\rangle,\\
      &\quadten \T{Cons} \mapsto \langle y, \langle ys, r \rangle \rangle . \langle 1 \oplus (1 \oplus (f_p\ y)_c \otimes r_c), \T{Cons}\langle (f_p\ y)_p, r_p\rangle\rangle) \rrbracket \eta\\
      &\qquad = \bigvee\limits_{(m_1, n_1) \leq (m, n)} case((m_1, n_1), \T{Nil} \mapsto \llbracket \langle 1, \T{Nil} \rangle \rrbracket \eta, \\
      &\quadten \T{N} \mapsto \langle y, \langle ys, r \rangle \rangle . \llbracket \langle 1 \oplus ((1 \oplus (f_p\ y)_c) \otimes r_c), \T{Cons} \langle (f_p\ y)_p, r_p \rangle \rangle \rrbracket \eta_c) \\
      &\qquad \text{where } \eta_c = \{xs \mapsto (0, (m, n)), f \mapsto (0, f), y \mapsto m, ys \mapsto (0, (m, n)), \\
      &\quadeight r \mapsto g(f, (m, n-1)))\} \\
      %Nil branch
      &\quadthree \text{\T{Nil} branch} \\
      &\quadthree \llbracket \langle 1, \T{Nil} \rangle \rrbracket \eta = (\llbracket 1 \rrbracket \eta , \llbracket \T{Nil} \rrbracket \eta) = (1, (0, 0)) \\
      %Cons branch
      &\quadthree \text{\T{Cons} branch} \\
      &\quadthree  \llbracket \langle 1 \oplus ((1 \oplus (f_p\ m)_c) \otimes r_c), \T{Cons} \langle (f_p\ m)_p, r_p \rangle \rangle \rrbracket \eta_c \\
      &\quadfour = (1 \oplus ((1 \oplus (f\ m)_c) \otimes g_c(f, (m, n-1))), ((f\ m)_p, \pi_1 g_p(f, (m, n-1)))) \\
      %putting branches together
      &\quad g(f, (m, n)) = \\
      &\qquad \bigvee\limits_{(m_1, n_1) \leq (m, n)} case((m_1, n_1), \T{Nil} \mapsto (1, (0, 0)), \\
      &\quadfour \T{Cons} \mapsto (1 \oplus ((1 \oplus (f\ m)_c) \otimes g_c(f, (m, n-1))), ((f\ m)_p, \pi_1 g_p(f, (m, n-1)))))
  \end{split}\]
\end{figure}

The result is
\begin{equation*}
  \begin{split}
  &g(f, (m, n)) = \\
  &\quad \bigvee\limits_{(m_1, n_1) \leq (m, n)} case((m_1, n_1), \T{Nil} \mapsto (1, (0, 0)), \\
  &\quadfour \T{Cons} \mapsto (1 \oplus ((1 \oplus (f\ y)_c) \otimes g_c(f, (m, n-1))), ((f\ y)_p, \pi_1 g_p(f, (m, n-1)))))
  \end{split}
\end{equation*}

%In figure \ref{fig:ws_map_massage}, we massage the recurrence into a simpler form by eliminating the big max and the $case$.
%\begin{figure}
%  \label{fig:ws_map_massage}
%  \caption{Simplification of the interpretation of \T{map}}
%  \[\begin{split}
%  &g(f, (m, n)) = \\
%  &\quad \bigvee\limits_{(m_1, n_1) \leq (m, n)} case((m_1, n_1), \T{Nil} \mapsto (1, (0, 0)), \\
%  &\quadfive \T{Cons} \mapsto (1 \oplus ((1 \oplus (f\ y)_c) \otimes g_c(f, (m, n-1))), ((f\ y)_p, \pi_1 g_p(f, (m, n-1)))))  \\
%      %base case
%  &\text{For } n = 0 \\
%  &\quad g(f, (m, 0) = \\
%  &\qquad (1, (0, 0)) \vee \bigvee\limits_{1 + n_1 } \\
%  &\quadsix \T{Cons} \mapsto (1 \oplus ((1 \oplus (f\ y)_c) \otimes g_c(f, (m, -1))), ((f\ y)_p, \pi_1 g_p(f, (m, -1)))))  \\
%  &\qquad = (1, (0, 0)) \\
%      %inductive case
%  &\text{For } n > 0 \\
%  &fubar
%  \end{split} \]
%\end{figure}
%
%The result is
%\begin{equation*}
%  g(f, (m, n)) = \begin{cases}
%    (1, (0, 0)) &n \equiv 0 \\
%    (1 \oplus (1 \oplus (f\ m)_c \otimes g(f, (m, n-1))_c), ((f\ m)_p, g(f, (m, n-1))_p)) &n > 0
%  \end{cases}
%\end{equation*}

We compile the recurrence down to the work and the span to make it easier to manipulate.
The costs change from cost graphs to a tuple of the work and the span.
\begin{equation*}
  \begin{split}
  &g(f, (m, n)) = \\
  &\quad \bigvee\limits_{(m_1, n_1) \leq (m, n)} case((m_1, n_1), \\
  &\quadfive\quadfour \T{Nil} \mapsto ((1, 1), (0, 0)), \\
  &\quadfive\quadfour \T{Cons} \mapsto ((2 + \pi_0(f\ y)_c + \pi_0 g_c(f, (m, n-1)), \\
  &\quadfive\quadten                     1 + max(1 + \pi_1(f\ y)_c, \pi_1 g_c(f, (m, n-1)))), \\
  &\quadfive\quadeight                  ((f\ y)_p, \pi_1 g_p(f, (m, n-1)))))
  \end{split}
\end{equation*}

We prove by induction bounds on the work and span of the cost of $g$.

\begin{theorem}
$\pi_0 g_c(f, (m, n)) \leq 1 + (2 + \pi_0(f\ m)_c)n$
\end{theorem}

\begin{proof}
   The proof is by induction on $n$.
  \begin{description}
    \item[case $n=0$]\mbox{}\\[-1.5\baselineskip]
      \begin{align*}
      \pi_0 g_c(f, (m, 0)) = \pi_0((1, 1), (0, 0))_c = \pi_0(1, 1) = 1
      \end{align*}
    \item[case $n>0$]\mbox{}\\[-1.5\baselineskip]
      \[\begin{split}
        &\pi_0(g_c(f, (m, n))) = \\
        &\quad \pi_0(\bigvee\limits_{(m_1, n_1) \leq (m, n)} case((m_1, n_1), \\
        &\quadfive\quadfour \T{Nil} \mapsto ((1, 1), (0, 0)), \\
        &\quadfive\quadfour \T{Cons} \mapsto ((2 + \pi_0(f\ m_1)_c + \pi_0g_c(f, (m_1, n_1-1)), \\
        &\quadfive\quadten                     1 + max(1 + \pi_1(f\ m_1)_c, \pi_1g_c(f, (m_1, n_1-1)))), \\
        &\quadfive\quadeight                  ((f\ m_1)_p, \pi_1 g_p(f, (m_1, n_1-1))))) )_c \\
        &\quad = 1 \vee \pi_0(\bigvee\limits_{(m_1, n_1) \leq (m, n)} ((2 + \pi_0(f\ m_1)_c + \pi_0g_c(f, (m_1, n_1-1)), \\
        &\quadfour\quadten                     1 + max(1 + \pi_1(f\ m_1)_c, \pi_1g_c(f, (m_1, n_1-1)))), \\
        &\quadfour\quadeight                  ((f\ m_1)_p, \pi_1 g_p(f, (m_1, n_1-1))))) )_c \\
        &\quad = 1 \vee \bigvee\limits_{(m_1, n_1) \leq (m, n)} 2 + \pi_0 (f\ m_1)_c + \pi_0 g_c(f, (m_1, n_1-1)) \\
        &\quad \leq \bigvee\limits_{(m_1, n_1) \leq (m, n)} 2 + \pi_0 (f\ m_1)_c + (1 + (2 + \pi_0 (f\ m_1)_c)(n_1-1)) \\
        &\quad \leq 2 + \pi_0 (f\ m)_c + 1 + (2 + \pi_0 (f\ m)_c)(n - 1) \\
        &\quad \leq 1 + (2 + \pi_0 (f\ m)_c)n \\
      \end{split}\]
  \end{description}
\end{proof}

\begin{theorem}
  $\pi_1 g_c(f, (m, n)) \leq 1 + \pi_1 (f\ m)_c + n$
\end{theorem}
\begin{proof}
  \begin{description}
    \item[case $n=0$]\mbox{}\\[-1.5\baselineskip]
      \begin{align*}
      \pi_1 g_c(f, (m, 0)) = \pi_1((1, 1), (0, 0))_c = \pi_1(1, 1) = 1
      \end{align*}
    \item[case $n>0$]\mbox{}\\[-1.5\baselineskip]
      \[\begin{split}
        &\pi_1(g_c(f, (m, n))) = \\
        &\quad \pi_1(\bigvee\limits_{(m_1, n_1) \leq (m, n)} case((m_1, n_1), \\
        &\quadfive\quadfour \T{Nil} \mapsto ((1, 1), (0, 0)), \\
        &\quadfive\quadfour \T{Cons} \mapsto ((2 + \pi_0(f\ m_1)_c + \pi_0g_c(f, (m_1, n_1-1)), \\
        &\quadfive\quadten                     1 + max(1 + \pi_1(f\ m_1)_c, \pi_1g_c(f, (m_1, n_1-1)))), \\
        &\quadfive\quadeight                  ((f\ m_1)_p, \pi_1 g_p(f, (m_1, n_1-1))))) )_c \\
        &\quad = 1 \vee \pi_1(\bigvee\limits_{(m_1, n_1) \leq (m, n)} ((2 + \pi_0(f\ m_1)_c + \pi_0g_c(f, (m_1, n_1-1)), \\
        &\quadfour\quadten                     1 + max(1 + \pi_1(f\ m_1)_c, \pi_1g_c(f, (m_1, n_1-1)))), \\
        &\quadfour\quadeight                  ((f\ m_1)_p, \pi_1 g_p(f, (m_1, n_1-1))))) )_c \\
        &\quad = 1 \vee \bigvee\limits_{(m_1, n_1) \leq (m, n)} 1 + max(1 + \pi_1 (f\ m_1)_c, \pi_1 g_c(f, (m_1, n_1-1))) \\
        &\quad \leq \bigvee\limits_{(m_1, n_1) \leq (m, n)} 1 + max(1 + \pi_1 (f\ m_1)_c, 1 + \pi_1 (f\ m_1)_c + n_1 - 1) \\
        &\quad \leq 1 + max(1 + \pi_1 (f\ m)_c, 1 + \pi_1 (f\ m)_c + n - 1) \\
        &\quad \leq 1 + 1 + \pi_1 (f\ m)_c + n - 1 \\
        &\quad \leq 1 + \pi_1 (f\ m)_c + n
      \end{split}\]
  \end{description}
\end{proof}

Compare these results with sequential map.


\section{Parallel Tree Map}
A program which is embarrasingly parallel is tree map.
When a function $f$ is mapped over a tree $t$, each application of $f$ to the label at each node can be done independently.
Furthermore, the tree data structure itself is dividable by construction.
Dividing the work requires only destruction of the node constructor to yield the left and right subtrees.

We will use \T{int} labelled binary trees.
\begin{equation*}
  \T{datatype tree} = \T{E of Unit | N of int$\times$tree$\times$tree}
\end{equation*}

\T{map} simply deconstructs each node, applies the function to the label, recuses on the children, and reconstructs a node using the results.
\begin{equation*}
  \T{map} = \lambda f.\lambda t.\T{rec}(t, \T{E} \mapsto \T{E}, \T{N} \mapsto \langle x, \langle t_0, r_0 \rangle, \langle t_1, r_1 \rangle\rangle.\T{N} \langle f\ x, \T{force}(r_0), \T{force}(r_1)\rangle)
\end{equation*}

\begin{figure}
  \label{fig:ws_treemap_complexity_translation}
  \caption{Complexity translation of \T{map f t}}
  \begin{equation*}
    \begin{split}
      &3 \oplus (f_c \otimes t_c) \oplus_c \T{rec}(t_p, \T{E} \mapsto 1 \oplus_c \|\T{E}\|, \\
      &\quadten \T{N} \mapsto \langle y, \langle t_0, r_0 \rangle \langle t_1, r_1 \rangle \rangle 1 \oplus_c \|\T{N}\langle f\ x, \T{force}(r_0) \T{force}(r_1)\rangle \| ) \\
      %E branch
      &\quad 1 \oplus_c \|E\| = 1 \oplus \langle 0, E \rangle = \langle 1, E \rangle \\
      %N branch
      &\quad \|\T{N}\langle f\ x, \T{force}(r_0) \T{force}(r_1)\rangle \| = \\
      &\qquad \langle \|\langle f\ x, \T{force}(r_0) \T{force}(r_1)\rangle \|_c, \T{N} \|\langle f\ x, \T{force}(r_0) \T{force}(r_1)\rangle\|_p\rangle \\
      &\quadthree \|\langle f\ x, \T{force}(r_0) \T{force}(r_1)\rangle \| = \\
      &\quadfour \langle \|f\ x\|_c \otimes \|\T{force}(r_0)\|_c \otimes \|\T{force}(r_1)\|_c, \langle \|f\ x\|_p, \|\T{force}(r_0)\|_p, \|\T{force}(r_1)\|_p \rangle \rangle \\
      % \|f x\|
      &\quadfive  \|f\ x\| = 1 \oplus \|f\|_c \otimes \|x\|_c \oplus_c \|f\|_p \|x\|_p \\
      &\quadeight = 1 \oplus \langle 0, f \rangle_c \otimes \langle 0, x \rangle_c \oplus_c \langle 0, f \rangle_p \langle 0, x \rangle_p \\
      &\quadeight = 1 \oplus_c (f_p\ x) \\
      &\quadeight = \langle 1 \oplus (f_p\ x)_c, (f_p\ x)_p \rangle \\
      % \|force(r0)\|
      &\quadfive \|\T{force}(r_0)\| = \|r_0\|_c \oplus_c \|r_0\|_p \\
      &\quadeight = \langle 0, r_0 \rangle \oplus_c \langle 0, r_0 \rangle_p \\
      &\quadeight = \langle 0 + r_{0c}, r_{0p} \rangle \\
      &\quadeight = r_0 \\
      % \|force(r1)\|
      &\quadfive \|\T{force}(r_1)\| = \|r_1\|_c \oplus_c \|r_1\|_p \\
      &\quadeight = \langle 0, r_1 \rangle \oplus_c \langle 0, r_1 \rangle_p \\
      &\quadeight = \langle 0 + r_{1c}, r_{1p} \rangle \\
      &\quadeight = r_1  \\
      %\|<...>\|
      &\quadthree \|\langle f\ x, \T{force}(r_0) \T{force}(r_1)\rangle \| = \\
      &\quadsix = \langle 1 \oplus (f_p\ x)_c \otimes r_{0c} \otimes r_{1c}, \langle (f_p\ x)_p, r_{0p}, r_{1p}\rangle\rangle \\
      %\|N<...>\|
      &\qquad \|\T{N}\langle f\ x, \T{force}(r_0) \T{force}(r_1)\rangle \| = \\
      &\quadfive = \langle 1 \oplus (f_p\ x)_c \otimes r_{0c} \otimes r_{1c}, \T{N} \langle (f_p\ x)_p, r_{0p}, r_{1p}\rangle\rangle \\
      %rec
      &\|\T{map f t}\| = \\
      &2 \oplus (f_c \otimes t_c) \oplus 1 \oplus_c \T{rec}(t_p, \T{E} \mapsto \langle 1, \T{E}\rangle, \\
      &\quadeight \T{N} \mapsto \langle y, \langle t_0, r_0 \rangle \langle t_1, r_1 \rangle \rangle.  \langle 2 \oplus (f_p\ x)_c \otimes r_{0c} \otimes r_{1c}, \T{N} \langle (f_p\ x)_p, r_{0p}, r_{1p}\rangle\rangle
    \end{split}
  \end{equation*}
\end{figure}

The translation of \T{map f t} is given in figure \ref{fig:ws_treemap_complexity_translation}.
\begin{equation}
  \label{eq:ws_treemap_complexity}
  \begin{split}
    &\|\T{map f t}\| = 2 \oplus (f_c \otimes t_c) \oplus 1 \oplus_c \\
    &\quadfour \T{rec}(t_p, \T{E} \mapsto \langle 1, \T{E}\rangle, \\
    &\quadseven \T{N} \mapsto \langle y, \langle t_0, r_0 \rangle \langle t_1, r_1 \rangle \rangle.  \langle 2 \oplus (f_p\ x)_c \otimes r_{0c} \otimes r_{1c}, \T{N} \langle (f_p\ x)_p, r_{0p}, r_{1p}\rangle\rangle
\end{split}
\end{equation}

The result is shown in equation \ref{eq:ws_treemap_complexity}.

We interpret trees as the number of \T{N} constructors and the maximum label.
\begin{align*}
  \llbracket tree \rrbracket &= \mathbb{Z} \times \mathbb{Z} \\
  D_\T{tree} &= \{\ast\} + \mathbb{Z} \times \llbracket \T{tree} \rrbracket \times \llbracket \T{tree} \rrbracket \\
  size_\T{tree}(\ast) &= (0, 0) \\
  size_\T{tree}(x, (m_0, n_0), (m_1, n_1)) &= (max(x, m_0, m_1), 1 + n_0 + n_1)
\end{align*}

\begin{figure}
  \label{fig:ws_map_interpretation_derivation}
  \caption{Deriviation of interpretation of \T{map f t}}
\[ \begin{split}
\end{split} \]
\end{figure}
