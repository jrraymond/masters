\section{Sequential Tree Map}

This example is presented for comparison with the parallel tree map given in
chapter 4.  We use \T{int} labelled binary trees.
%
\begin{equation*}
  \T{datatype tree} = \T{E of Unit | N of int$\times$tree$\times$tree}
\end{equation*}
%
Tree \T{map f t} applies the function \T{f} to every label in the tree \T{t}.
%
\begin{align*}
  \T{map} &= \lambda f.\lambda t.\T{rec}(t, \T{E} \mapsto \T{E}, \T{N} \mapsto \LP x,\LP t_0,r_0\RP,\LP t_1,r_1\RP\RP. \T{N}\LP f\ x, \T{force}(r_0), \T{force}(r_1)\RP)
\end{align*}

\subsection{Translation}
%
\begin{align*}
  &\text{We apply the abstraction rule twice.} \\
  \|\T{map}\| &= \LP 0,\lambda f.\LP 0,\lambda t.\|\T{rec}(t, \T{E} \mapsto \T{E},\\
              &\quadsix\T{N} \mapsto \LP x,\LP t_0,r_0\RP,\LP t_1,r_1\RP\RP.\T{N}\LP f\ x,\T{force}(r_0),\T{force}(r_1)\RP)\|\RP\RP \\
              &\text{We apply the recursor translation rule.} \\
              &= \LP 0,\lambda f.\LP 0,\lambda t.\|t\|_c +_c \T{rec}(\|t\|_p, \T{E} \mapsto 1 +_c \|\T{E}\|,\\
              &\quadsix \T{N} \mapsto \LP x,\LP t_0,r_0\RP,\LP t_1,r_1\RP\RP.1 +_c \|\T{N}\LP f\ x,\T{force}(r_0),\T{force}(r_1)\RP\|)\RP\RP \\
  %
              &\text{The translation of the variable $t$ is $\LP 0,t\RP$.} \\
              &\text{The translation of the constructor \T{E} with $\LP\RP$ as its argument is $\LP 0,\T{E}\RP$.} \\
              &\text{The translation of the constructor \T{N}$\LP e\RP$ is $\LP \|e\|_c,\T{N}\LP\|e\|_p\RP\RP$.} \\
  %
              &\text{$f$ and $x$ are variables, so their translations are $\LP 0,f\RP$ and $\LP 0,x\RP$ respectively.}\\
              &\quadthree \|f\ x\| = \LP 1 + \|f\|_c + \|x\|_c +_c \|f\|_p\|x\|_p = \LP 1 + (f\ x)_c,(f\ x)_c\RP \\
              &\text{$r_0$ and $r_1$ are also variables.} \\
              &\quadthree \|\T{force}(r_i)\| = \|r_i\|_c +_c \|r_i\|_p = \LP 0,r_i\RP +_c \LP 0,r_i\RP = r_i \\
  %
              &\text{We use this result to translate the argument to the \T{N} constructor.} \\
              &\quadthree \|\LP f\ x,\T{force}(r_0),\T{force}(r_1)\RP\| = \\
              &\quadsix \LP \|f\ x\|_c + \|\T{force}(r_0)\|_c + \|\T{force}(r_1)\|_c, \\
              &\quadseven \LP \|f\ x\|_p,\|\T{force}(r_0)\|_p\RP,\|\T{force}(r_1)\|_p\RP\RP \\
              &\quadthree = \LP 1 + (f\ x)_c + r_{0c} + r_{1c},\LP (f\ x)_p, r_{0p}, r_{1p}\RP\RP\\
  %
              &\text{We use this to translate the \T{N} constructor.} \\
              &\T{N}\LP f\ x,\T{force}(r_0),\T{force}(r_1)\RP = \LP 1 + (f\ x)_c + r_{0c} + r_{1c}, \T{N}\LP(f\ x)_p,r_{0p},r_{1p}\RP\RP \\
  %
              &\text{We use this to complete the translation of \T{map}.} \\
              &= \LP 0,\lambda f.\LP 0,\lambda t.\T{rec}(t, \T{E} \mapsto \LP 1,\T{E}\RP,\\
              &\quadfive \T{N} \mapsto \LP x,\LP t_0,r_0\RP,\LP t_1,r_1\RP\RP.\LP 2 + (f\ x)_c + r_{0c} + r_{1c}, \T{N}\LP (f\ x)_p,r_{0p},r_{1p}\RP\RP)\RP\RP \\
\end{align*}
%
We are interested in the cost of applying \T{map} to a function and a tree.
The translation of \T{map} applied to a function \T{f} and a tree \T{t} is:
%
\begin{align*}
  \|\T{map f t}\| &= (1 + \|\T{map f}\|_c + \|t\|_c) +_c \|\T{map f}\|_p \|t\|_p \\
                  &\qquad \|\T{map f}\| = (1 + \|\T{map}\|_c + \|f\|_c) +_c \|\T{map}\|_p \|f\|_p \\
                  &\text{Since the cost of \T{map} and \T{map f} are $0$, we can simplify this.} \\
                  &\quadfive = \LP 1 + \|f\|_c, (\|\T{map}\|_p \|f\|_p)_p \RP \\
                  &= (2 + \|f\|_c + \|t\|_c) +_c (\|\T{map}\|_p \|f\|_p)_p \|t\|_p
\end{align*}
%

\subsection{Interpretation}
%
We interpret trees as a three-tuple of their largest element and number of
\T{N} constructors.
%
\begin{align*}
  \LB tree \RB &= \mathbb{Z} \times \mathbb{Z} \\
  D_\T{tree} &= \{\ast\} + \mathbb{Z} \times \LB \T{tree} \RB \times \LB \T{tree} \RB \\
  size_\T{tree}(\ast) &= (0, 0) \\
  size_\T{tree}(x, (i_0, n_0), (i_1, n_1)) &= (max(x, i_0, i_1), 1 + n_0 + n_1)
\end{align*}
%
We will interpret the \T{rec} construct first, since it drives the cost of
\T{map}.
%
\begin{align*}
  g(i,n) &= \LB\T{rec}(t, \T{E} \mapsto \LP 1,\T{E}\RP, \\
         &\quad \T{N} \mapsto \LP x,\LP t_0,r_0\RP,\LP t_1,r_1\RP\RP. \\
         &\quadfive \LP 2 + (f\ x)_c + r_{0c} + r_{1c}, \T{N}\LP (f\ x)_p,r_{0p},r_{1p}\RP\RP)\RP\RP\RB \xi\\
         &\text{where } \xi = \{t \mapsto (i,n), f \mapsto f \} \\
         &= \bigvee\limits_{size(z) \leq n} case(z, f_E, f_N) \\
         &\text{where} \\
  f_E(\ast) &= \LB \LP 1,\T{E}\RP\RB\xi  \\
            &= (1,(0,0)) \\
  f_N(j,(j_0, n_0),(j_1,n_1)) &= \LB \LP 2 + (f\ x)_c + r_{0c} + r_{1c}, \T{N}\LP (f\ x)_p,r_{0p},r_{1p}\RP\RP \RB \xi' \\
         &\text{where } \xi' = \xi \{ x \mapsto j, r_0 \mapsto g(j_0,n_0), r_1 \mapsto g(j_1,n_1) \} \\
         &= (2 + (f\ j)_c + g_c(j_0,n_0) + g_c(j_1,n_1), \\
         &\qquad (max((f\ j)_p, \pi_0g_p(j_0,n_0), \pi_0g_p(j_1,n_1)), \\
         &\qquad 1 + \pi_1g_p(j_0,n_0) + \pi_1g_p(j_1,n_1)))
\end{align*}
%
To obtain a closed form solution for this recurrence, we use the substitution
method to prove a bound on the cost.
%
\begin{lemma}
  $g_c(i,n) = (3 + (f i)_c)n + 1$
\end{lemma}
%
\begin{proof}
\begin{align*}
  g_c(i,n) &= (\bigvee\limits_{size(z) \leq (i,n)} case(z,f_E,f_N))_c \\
           &= 1 \vee (\bigvee\limits_{size(j,(j_0,n_0),(j_1,n_1)) \leq (i,n)} f_N(j,(j_0,n_0),(j_1,n_1)))_c \\
           &= \bigvee\limits_{\substack{1 + n_0 + n_1 \leq n\\max(j,j_0,j_1)\leq i}} 2 + (f\ j)_c + g_c(j_0,n_0) + g_c(j_1,n_1) \\
           &\leq \bigvee\limits_{\substack{1 + n_0 + n_1 \leq n\\max(j,j_0,j_1)\leq i}} 2 + (f\ j)_c + (3 + (f\ j_0)_c)n_0 + 1 + (3 + (f\ j_1))n_1 + 1\\
           &\leq \bigvee\limits_{\substack{1 + n_0 + n_1 \leq n\\max(j,j_0,j_1)\leq i}} (3 + (f\ j)_c)(1 + n_0 + n_1) + 1\\
           &\leq (3 + (f\ i)_c)n + 1
\end{align*}
\end{proof}
%
The result is the cost of the recursor of the \T{map} function is linear in the
number of nodes in the tree but also depends on the cost of applying the
function to the largest label in the tree.

\paragraph{}
Recall the translation of \T{map f t}.
%
\begin{align*}
  \|\T{map f t}\| &= (2 + \|f\|_c + \|t\|_c) +_c (\|\T{map}\|_p \|f\|_p)_p \|t\|_p
\end{align*}
%
The interpretation of the application of \T{map} to a function \T{f} and a tree \T{t} is
%
\begin{lemma}
  \label{lem:treemap_cost}
  $\LB\|\T{map f t}\|\RB = 3 + (3 + (f\ i)_c)n \text{ where } \LB\|t\|\RB = (i,n) \text{ and } \LB\|f\|\RB = f$
\end{lemma}
\begin{proof}
\begin{align*}
  \LB\|\T{map f t}\|\RB &= \LB(2 + \|f\|_c + \|t\|_c) +_c (\|\T{map}\|_p \|f\|_p)_p \|t\|_p\RB \\
                        &= 2 + \LB\|f\|_c\RB + \LB\|t\|\RB_c) +_c (\LB\|\T{map}\|_p \LB\|f\|_p\|\RB)_p \LB\|t\|_p\RB \\
                        &= 3 + (3 + (f\ i)_c)n \text{ where } \LB\|t\|\RB = (i,n) \text{ and } \LB\|f\|\RB = f
\end{align*}
\end{proof}
%
