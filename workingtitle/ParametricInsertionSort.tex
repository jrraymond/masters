\chapter{Parametric Insertion Sort}
%
Parametric insertion sort is a higher order algorithm which sorts a list using
a comparison function which is passed to it as an argument.  The running time
of insertion sort is $\mathcal{O}(n^2)$.  This characterization of the
complexity of parametric insertion sort does not capture role of the comparison
function in the running time.  When sorting a list of integers, where
comparison between any two integers takes constant time, this does not matter.
However, when sorting a list of strings, where the complexity of comparison is
order the length of the string, the length of the strings may influence the
running time more than the length of the list when sorting small lists of large
strings.

We use the familiar \T{list} datatype.
\begin{align*}
  \T{data list} &= \T{Nil of unit | Cons of int $\times$ list}
\end{align*}
%
The function \T{sort} relies on the function \T{insert}. \T{insert} inserts an
element into a sorted list.
%
\begin{align*}
  \T{insert} &= \lambda f.\lambda x.\lambda xs.\T{rec}(xs, \T{Nil} \mapsto \T{Cons} \langle x, \T{Nil}\rangle,\\
             &\quadeight \T{Cons}\mapsto \langle y, \langle ys,r \rangle\rangle.\T{rec}(f\ x\ y, \T{True}\mapsto \T{Cons}\langle x,\T{Cons}\langle y,ys \rangle\rangle, \\
             &\quadten\quadten\quad \T{False}\mapsto \T{Cons}\langle y,\T{force}(r)\rangle))
\end{align*}
%
The \T{sort} function recurses on the list, using the \T{insert} function to
insert the head of the list into the recursively sorted tail of the list.
%
\begin{align*}
  \T{sort} &= \lambda f.\lambda xs.\T{rec}(xs, \T{Nil} \mapsto \T{Nil}, \T{Cons} \mapsto \langle y,\langle ys,r \rangle\rangle.\T{insert}\ f\ y\ \T{force}(r))
\end{align*}
%
\section{Translation}
%
\subsection{\T{insert} Translation}
%
The translation of \T{insert} is broken into chunks to make it more manageable.
Figure \ref{eq:fxy} steps through the translation of the comparison function
\T{f} applied to variables \T{x} and \T{y}.
%
% BEGIN INSERT TRANSLATION
%
To translate $f\ x\ y$ we apply the function application rule twice. First we
apply the rule to $(f\ x)\ y$. Then we apply the rule to $f\ x$. Then we expand
the $+_c$ macro to simplify the result.
%
\begin{align*}
\numberthis \label{eq:fxy}
\|f\ x\ y\| &= (1 + \|f\ x\|_c + \|y\|_c) +_c \|f\ x\|_p \|y\|_p \\
            &\qquad \|f\ x\| = (1 + \|f\|_c + \|x\|_c) +_c \|f\|_p \|x\|_p \\
            &\quadfour = (1 + \|f\|_c + \|x\|_c + (\|f\|_p \|x\|_p)_c, \|f\|_p \|x\|_p) \\
            &= (1 + (1 + \|f\|_c + \|x\|_c +_c (\|f\|_p \|x\|_p)_c) + \|y\|_c) +_c (\|f\|_p \|x\|_p)_p \|y\|_p \\
            &= (2 + \|f\|_c + \|x\|_c + \|y\|_c + (\|f\|_p \|x\|_p)_c) +_c (\|f\|_p \|x\|_p)_p \|y\|_p
\end{align*}
%
%\begin{lstlisting}
%$\|$f x y$\| = (1 + \|$f x$\|_c + \|$y$\|_c) +_c \|$f x$\|_p \|$y$\|_p$
%
%  $= (1 + 1 + \|$f$\|_c + \|$x$\|_c + (\|$f$\|_p \|$x$\|_p)_c + \|$y$\|_c) +_c \|$f$\|_p \|$x$\|_p \|$y$\|_p$
%
%  $= (2 + \langle$0,f$\rangle_c + \langle$0,x$\rangle_c + \langle$0,y$\rangle_c) +_c \langle$0,f$\rangle_p \langle$0,x$\rangle_p \langle$0,y$\rangle_p$
%
%  $= (2 + 0 + 0 + 0) +_c $(f x y)
%
%  $= \langle 2 + (($f x$)_p$ y$)_c, (($f x$)_p$ y$)_p\rangle$
%\end{lstlisting}
$= (1 + \langle 1 + \|$f$\|_c + \|$x$\|_c + (\|$f$\|_p\|$x$\|_p)_c,(\|$f$\|_p\|$x$\|_p)_p \rangle_c + \langle$0,y$\rangle_c) +_c$
     $(\langle 1 + \|$f$\|_c + \|$x$\|_c + (\|$f$\|_p\|$x$\|_p)_c,(\|$f$\|_p\|$x$\|_p)_p \rangle_p  \langle$0,y$\rangle_p)$
%

% BEGIN TRUE BRANCH
The translation of a datatype is the cost of translating its argument, and
complexity language constructor applied to the potential of the translated
argument.
%
\begin{align*}
  \|\T{Cons}\langle x,\T{Cons}\langle y, rs \rangle\rangle\| &= \langle \|\langle x, \T{Cons} \langle y,ys\rangle\rangle\|_c, \T{Cons}\|\langle x, \T{Cons}\langle y,ys\rangle\|_p\rangle
\end{align*}
%
The argument to the \T{Cons} constructor is a tuple. The cost of the
translation of a tuple is the cost of the translation of each element and the
potential is the tuple of the potentials of the translations of each element.
%
\begin{align*}
  \|\langle x, \T{Cons}\langle y,ys\rangle\rangle\| &= \langle \|x\|_c + \|\T{Cons}\langle y,ys\rangle\|_c, \langle \|x\|_p,\T{Cons}\langle y,ys\rangle\|_p\rangle\rangle
\end{align*}
%
The first element of the tuple is a variable, but the second element is another
\T{list}. So we translate the second element first. To do so we apply the rule
for translating a datatype.
%
\begin{align*}
  \|\T{Cons}\langle y,ys\rangle\| &= \langle \|y,ys\|_c, \T{Cons}\|y,ys\|_p\rangle
\end{align*}
%
The argument to the constructor is a tuple. We apply the rule for translating a
tuple again. Both element of the tuple are variables, so their translated cost
is $0$ and their translated potential is their corresponding variable in the
complexity language.
%
\begin{align*}
  \|\langle y,ys\rangle\| &= \langle \|y\|_c + \|rs\|_c, \langle\|y\|_p,\|rs\|_p\rangle\rangle \\
                         &= \langle \langle 0,y\rangle_c + \langle 0, rs\rangle_c, \langle \langle0,y\rangle_p, \langle 0,rs\rangle\rangle\rangle \\
                         &= \langle 0, \langle y,ys\rangle\rangle
\end{align*}
%
We use this to complete the translation of $\T{Cons}\langle y,ys\rangle$.
%
\begin{align*}
  \|\T{Cons}\langle y,ys\rangle\| &= \langle \|y,ys\|_c, \T{Cons}\|y,ys\|_p\rangle \\
                                  &= \langle 0, \T{Cons}\langle y,ys\rangle\rangle
\end{align*}
%
We use this result to complete the translation of $\langle x,\T{Cons}\langle y,ys\rangle\rangle$.
%
\begin{align*}
  \|\langle x, \T{Cons}\langle y,ys\rangle\rangle\| &= \langle \|x\|_c + \|\T{Cons}\langle y,ys\rangle\|_c, \langle \|x\|_p,\T{Cons}\langle y,ys\rangle\|_p\rangle\rangle \\
                                                    &= \langle 0, \langle x,\T{Cons}\langle y,ys\rangle\rangle\rangle
\end{align*}
%
And finally we use this to complete the translation of $\T{Cons}\langle x,\T{Cons}\langle y,ys\rangle\rangle$.
%
\begin{align*}
  \|\T{Cons}\langle x,\T{Cons}\langle y,ys \rangle\rangle\| &= \langle \|\langle x, \T{Cons} \langle y,ys\rangle\rangle\|_c, \T{Cons}\|\langle x,\T{Cons}\langle y,ys\rangle\rangle\|_p\rangle \\
                                                            &= \langle 0, \T{Cons}\langle x,\T{Cons}\langle y,ys\rangle\rangle\rangle
\end{align*}
%
% END TRUE BRANCH
%
% BEGIN FALSE BRANCH
%
Next we will translate the \T{False} branch.

The translation \T{True} and \T{False} branches are given in figures
\ref{fig:insert_True} and \ref{fig:insert_False} respectively.
%
\begin{align*}
  \|\T{Cons}\langle y,\T{force}(r)\rangle\| &= \langle \|\langle y,\T{force}(r)\rangle\|_c, \T{Cons}\|\langle y,\T{force}(r)\rangle\|_p\rangle
\end{align*}
%
To complete this we must first translate the tuple. The two elements of the
tuple are $y$ and $\T{force}(r)$.  The translation of the variable $y$ is
$\langle 0, y\rangle$. The translation of $\T{force}(r)$ is
$\|r\|_c +_c \|r\|_p$. Like $y$, $r$ is a variable so its translation is
$\langle 0,r\rangle$. So the translation of $\T{force}(r)$ is $0 +_c r$ which
simplifies to $r$.
%
\begin{align*}
  \|\langle y,\T{force}(r)\rangle\| &= \langle \|y\|_c + \|\T{force}(r)\|_c, \langle\|y\|_p,\|\T{force}(r)\|_p\rangle\rangle
                                    &= \langle 0 + r_c, \langle y,r_p\rangle\rangle \\
                                    &= \langle r_c, \langle y,r_p\rangle\rangle
\end{align*}
%
We substitute this into the translation of $\T{Cons}\langle y,\T{force}(r)\rangle$.
%
\begin{align*}
  \|\T{Cons}\langle y,\T{force}(r)\rangle\| &= \langle \|\langle y,\T{force}(r)\rangle\|_c, \T{Cons}\|\langle y,\T{force}(r)\rangle\|_p\rangle \\
                                            &= \langle r_c, \T{Cons}\langle y,r_p\rangle\rangle
\end{align*}
%
% END FALSE BRANCH
%
% BEGIN INNER REC
%
Next we use the translation of \T{f x y} and the \T{True} and \T{False}
branches to construct the translation of the inner \T{rec} construct.
%
\begin{align*}
  \|\T{rec}&(f\ x\ y, \T{True}\mapsto\T{Cons}\langle x,\T{Cons}\langle y,ys\rangle\rangle, \T{False}\mapsto \T{Cons}\langle y,\T{force}(r)\rangle)\| \\
           &= \|f\ x\ y\|_c +_c \T{rec}(\|f\ x\ y\|_p, \T{True}\mapsto 1 +_c \|\T{Cons}\langle x,\T{Cons}\langle y,ys\rangle\rangle\|, \\
           &\quadsix \T{False}\mapsto 1 +_c \|\T{Cons}\langle y,\T{force}(r)\rangle\|) \\
           &= \|f\ x\ y\|_c +_c \T{rec}(\|f\ x\ y\|_p, \T{True}\mapsto 1 +_c \langle 0, \T{Cons}\langle x,\T{Cons}\langle y,ys\rangle\rangle\rangle, \\
           &\quadsix \T{False}\mapsto 1 +_c \langle r_c, \T{Cons}\langle y,r_p\rangle\rangle) \\
           &= (2 + \|f\|_c + \|x\|_c + \|y\|_c + (\|f\|_p \|x\|_p)_c) \\
           &\quadthree +_c \T{rec}((\|f\|_p\ \|x\|_p)_p \|y\|_p, \T{True}\mapsto \langle 1, \T{Cons}\langle x,\T{Cons}\langle y,ys\rangle\rangle\rangle, \\
           &\quadsix \T{False}\mapsto \langle 1 + r_c, \T{Cons}\langle y,r_p\rangle\rangle) \\
\end{align*}
%
% END INNER REC
%
% BEGIN NIL
%
% END NIL
%
% BEGIN CONS
%
% END CONS
%
%
The \T{Nil} and \T{Cons} branches of the outer \T{rec} construct are given in
figures \ref{fig:insert_nil} and \ref{fig:insert_cons}, respectively.
%
\begin{figure}[H]
\caption{Translation of the \T{Nil} branch of the outer \T{rec} in \T{insert}.
The insertion of an element into an empty list results in a singleton list containing only the element.
This branch is also reached when the ordering given by \T{f} dictates \T{x} comes after than everything in the list,
  and should be placed at the back of the list.
}
\label{fig:insert_nil}
\begin{lstlisting}
$\|$Nil$\mapsto$Cons$\langle$x,Nil$\rangle\|$
  $=$ Nil$\mapsto 1 +_c \|$Cons$\langle$x,Nil$\rangle\|$

  $=$ Nil$\mapsto 1 +_c \langle\|\langle$x,Nil$\rangle\|_c$,Cons$\|\langle$x,Nil$\rangle\|_p\rangle$

  $=$ Nil$\mapsto 1 +_c \langle\langle\|$x$\|_c + \|$Nil$\|_c,\langle\|$x$\|_p,\|$Nil$\|_p\rangle\rangle_c$,Cons$\langle\|$x$\|_c + \|$Nil$\|_c,\langle\|$x$\|_p,\|$Nil$\|_p\rangle\rangle_p\rangle$

  $=$ Nil$\mapsto 1 +_c \langle\|$x$\|_c + \|$Nil$\|_c$,Cons$\langle\|$x$\|_p,\|$Nil$\|_p\rangle\rangle$

  $=$ Nil$\mapsto 1 +_c \langle\langle$0,x$\rangle_c + \langle$0,Nil$\rangle_c$,Cons$\langle\langle$0,x$\rangle_p,\langle$0,Nil$\rangle_p\rangle\rangle$

  $=$ Nil$\mapsto 1 +_c \langle0 + 0$,Cons$\langle$x,Nil$\rangle\rangle$

  $=$ Nil$\mapsto \langle1$,Cons$\langle$x,Nil$\rangle\rangle$
\end{lstlisting}
\end{figure}
%
\begin{figure}[H]
\caption{Translation of the \T{Cons} branch of the outer \T{rec} in \T{insert}.
In this branch we recurse on a nonempty list.
We check if \T{x} is comes before the head of the list under the ordering given by \T{f}, in which case we are done, 
  otherwise we recurse on the tail of the list.
}
\label{fig:insert_cons}
\begin{lstlisting}
$\|$Cons$\mapsto \langle$y,$\langle$ys,r$\rangle\rangle$.rec(f x y, True $\mapsto$ Cons$\langle$x,Cons$\langle$y,ys$\rangle\rangle$,
                             False $\mapsto$ Cons$\langle$y,force(r)$\rangle$)$\|$

  $=$ Cons$\mapsto \langle$y,$\langle$ys,r$\rangle\rangle$.$1 +_c \|$rec(f x y, True $\mapsto$ Cons$\langle$x,Cons$\langle$y,ys$\rangle\rangle$,
                             False $\mapsto$ Cons$\langle$y,force(r)$\rangle$)$\|$

  $=$ Cons$\mapsto \langle$y,$\langle$ys,r$\rangle\rangle$.$1 +_c (2 + (($f x$)_p$ y$)_c) +_c$rec($(($f x$)_p$ y$)_p$,
                                              True$\mapsto \langle 1, $Cons$\langle$x,Cons$\langle$y,ys$\rangle\rangle\rangle$
                                              False$\mapsto \langle1 + $r$_c,$Cons$\langle$y$,$r$_p\rangle\rangle$)$)$

  $=$ Cons$\mapsto \langle$y,$\langle$ys,r$\rangle\rangle$.$(3 + (($f x$)_p $ y$)_c) +_c$rec($(($f x$)_p$ y$)_p$,
                                           True$\mapsto \langle 1, $Cons$\langle$x,Cons$\langle$y,ys$\rangle\rangle\rangle$
                                           False$\mapsto \langle1 + $r$_c,$Cons$\langle$y$,$r$_p\rangle\rangle$)$)$
\end{lstlisting}
\end{figure}
%
We put these together to give the translation of \T{insert}.
%
\begin{figure}[H]
\caption{Translation of \T{insert}}
\label{fig:insert}
\begin{lstlisting}
$\|$insert$\|$ = $\|\lambda$f.$\lambda$x.$\lambda$xs.rec(xs, Nil$\mapsto$ Cons$\langle$x,Nil$\rangle$,
                      Cons$\mapsto \langle$y,$\langle$ys,r$\rangle\rangle$.rec(f x y, True $\mapsto$ Cons$\langle$x,Cons$\langle$y,ys$\rangle\rangle$,
                                                 False $\mapsto$ Cons$\langle$y,force(r)$\rangle$))$\|$

  $= \langle$0,$\lambda$f.$\langle$0,$\lambda$x.$\langle0,\lambda$xs.$\|$rec(xs, Nil$\mapsto$ Cons$\langle$x,Nil$\rangle$,
                          Cons$\mapsto \langle$y,$\langle$ys,r$\rangle\rangle$.rec(f x y, True $\mapsto$ Cons$\langle$x,Cons$\langle$y,ys$\rangle\rangle$,
                                                     False $\mapsto$ Cons$\langle$y,force(r)$\rangle$))$\|\rangle\rangle\rangle$

  $= \langle$0,$\lambda$f.$\langle$0,$\lambda$x.$\langle0,\lambda$xs.$\langle$0,xs$\rangle_c +_c $
          rec($\langle$0,xs$\rangle_p$,
              Nil$\mapsto \langle1$,Cons$\langle$x,Nil$\rangle\rangle$
              Cons$\mapsto \langle$y,$\langle$ys,r$\rangle\rangle$.$(3 + ((\langle$0,f$\rangle_p$ x$)_p $ y$)_c) +_c$rec($((\langle$0,f$\rangle_p$ x$)_p $ y$)_p$,
                                                   True$\mapsto \langle 1, $Cons$\langle$x,Cons$\langle$y,ys$\rangle\rangle\rangle$
                                                   False$\mapsto \langle1 + $r$_c,$Cons$\langle$y$,$r$_p\rangle\rangle$)$)$

  $= \langle$0,$\lambda$f.$\langle$0,$\lambda$x.$\langle0,\lambda$xs.
          rec(xs,
              Nil$\mapsto \langle1$,Cons$\langle$x,Nil$\rangle\rangle$
              Cons$\mapsto \langle$y,$\langle$ys,r$\rangle\rangle$.$(3 + (($f x$)_p $ y$)_c) +_c$rec($($f x$)_p $ y$)_p$,
                                                 True$\mapsto \langle 1, $Cons$\langle$x,Cons$\langle$y,ys$\rangle\rangle\rangle$
                                                 False$\mapsto \langle1 + $r$_c,$Cons$\langle$y$,$r$_p\rangle\rangle$)$)$
\end{lstlisting}
\end{figure}
%
Finally we give a translation of \T{insert f x xs} in figure
\ref{fig:insert_applied} because this is the term we will interpret in a
size-based semantics.
%
\begin{figure}[H]
  \caption{The translation of \T{insert f x xs}.
  Unlike before, we do not assume that \T{f}, \T{x}, \T{xs} are variables.
  They may be expressions with non-zero costs.}
  \label{fig:insert_applied}
  \begin{lstlisting}
  $\|$insert f x xs$\| = (1 + \|$insert f x$\|_c + \|$xs$\|_c) +_c \|$insert f x$\|_p \|$xs$\|_p$

     $= (1 + \|$insert f x$\|_c + \|$xs$\|_c) +_c \|$insert f x$\|_p  \|$xs$\|_p$

     $= (2 + \|$insert f$\|_c + \|$x$\|_c + (\|$insert f$\|_p \|$x$\|_p)_c + \|$xs$\|_c) +_c \|$insert f$\|_p \|$x$\|_p \|$xs$\|_p$

     $= (2 + \|$insert f$\|_c + \|$x$\|_c + \|$xs$\|_c) +_c \|$insert f$\|_p \|$x$\|_p \|$xs$\|_p$

     $= (3 + \|$insert$\|_c + \|$f$\|_c+ (\|$insert$\|_p \|$f$\|_p \|$x$\|_p)_c + \|$x$\|_c  + \|$xs$\|_c) +_c \|$insert$\|_p \|$f$\|_p \|$x$\|_p \|$xs$\|_p$

     $= (3 + \|$f$\|_c + \|$x$\|_c + \|$xs$\|_c) +_c \|$insert$\|_p \|$f$\|_p \|$x$\|_p \|$xs$\|_p$

     $= (3 + \|$f$\|_c + \|$x$\|_c + \|$xs$\|_c) +_c $rec($\|$xs$\|_p$,
              Nil$\mapsto \langle1$,Cons$\langle$x,Nil$\rangle\rangle$
              Cons$\mapsto \langle$y,$\langle$ys,r$\rangle\rangle$.$(3 + ((\|$f$\|_p \|$x$\|_p)_p $ y$)_c) +_c$rec($(\|$f$\|_p \|$x$\|_p)_p $ y$)_p$,
                                                 True$\mapsto \langle 1, $Cons$\langle \|$x$\|_p$,Cons$\langle$y,ys$\rangle\rangle\rangle$
                                                 False$\mapsto \langle1 + $r$_c,$Cons$\langle$y$,$r$_p\rangle\rangle$)$)$
  \end{lstlisting}
\end{figure}
%
The result is:
%
\begin{lstlisting}
$\|$insert f x xs$\|= (3 + \|$f$\|_c + \|$x$\|_c + \|$xs$\|_c)$
        $+_c $rec($\|$xs$\|_p$,
              Nil$\mapsto \langle1$,Cons$\langle$x,Nil$\rangle\rangle$
              Cons$\mapsto \langle$y,$\langle$ys,r$\rangle\rangle$.$(3 + ((\|$f$\|_p \|$x$\|_p)_p $ y$)_c) +_c$rec($(\|$f$\|_p \|$x$\|_p)_p $ y$)_p$,
                                                 True$\mapsto \langle 1, $Cons$\langle \|$x$\|_p$,Cons$\langle$y,ys$\rangle\rangle\rangle$
                                                 False$\mapsto \langle1 + $r$_c,$Cons$\langle$y$,$r$_p\rangle\rangle$)$)$
\end{lstlisting}
%


\section{Interpretation}
%
We well use an interpretation of lists as a pair of their greatest element and
their length.  Figure \ref{fig:interp_sizes} formalizes this interpretation.
%
\begin{figure}[H]
  \caption{Interpretation of lists as lengths}
  \label{fig:interp_sizes}
  \begin{align*}
    \llbracket list \rrbracket &= \mathbb{Z} \times \mathbb{N}^\infty \\
    D^{list} &= \{\ast\} + \{\mathbb{Z}\} \times \mathbb{N}^\infty \\
    size_{list} (Nil) &= (-\infty,0) \\
    size_{list} (Cons(i,(j,n))) &= (max\{i,j\},1 + n)
  \end{align*}
\end{figure}
%
We use the mutual ordering on pairs.  That is, $(s,n) \leq (s',n')$ if
$n \leq n'$ and $s < s'$ or $n < n'$ and $s \leq s'$.

First we interpret the \T{rec}, which drives of the cost of \T{insert}.  As in
the translation, we break the interpretation up to make it more manageable.  We
will write $map, \lambda$ and $+_c$ in the semantics, which stand for the
semantic equivalents of the syntactic \T{map}, $\lambda$ and $+_c$.  The
definitions of these semantic functions mirror the definitions of their
syntactic equivalents.  Figures \ref{fig:interp_sizes_inner_rec} and
\ref{fig:interp_sizes_outer_rec} walk through the interpretation.
%
\begin{figure}[H]
  \caption{Interpretation of the inner \T{rec} of \T{insert} with lists abstracted to sizes}
  \label{fig:interp_sizes_inner_rec}
  \begin{lstlisting}
    $\llbracket$rec($($f x$)_p $ y$)_p$,
       True$\mapsto \langle 1, $Cons$\langle$x,Cons$\langle$y,ys$\rangle\rangle\rangle$
       False$\mapsto \langle1 + $r$_c,$Cons$\langle$y$,$r$_p\rangle\rangle$)$)$ $\rrbracket \xi \{$f$ \mapsto f, $x$ \mapsto x, $y$ \mapsto y, $ys$ \mapsto (i,n), $r$ \mapsto r\}$

    $f_{True} (\langle\rangle) = \llbracket \langle$1,Cons$\langle$x,Cons$\langle$y,ys$\rangle\rangle\rangle\rrbracket \xi \{$f$ \mapsto f, $x$ \mapsto x, $y$ \mapsto y, $ys$ \mapsto (i,n), $r$ \mapsto r\}$

           $= (1, (max\{x,y,i\},2 + n))$

    $f_{False} (\langle\rangle) = \llbracket \langle1 + $r$_c,$Cons$\langle$y$,$r$_p\rangle\rangle$)$)$ $\rrbracket \xi \{$f$ \mapsto f, $x$ \mapsto x, $y$ \mapsto y, $ys$ \mapsto (i,n), $r$ \mapsto r\}$

           $= (1 + \pi_0 r, (max\{y,\pi_0\pi_1 r\}, 1 + \pi_1\pi_1 r))$
    
    $= \bigvee_{size(w) \leq \pi_1(\pi_1(f\ x)\ y)} case(w, (f_{True}, f_{False}))$

    $= \bigvee_{size(w) \leq \pi_1(\pi_1(f\ x)\ y)} case(w, (\lambda \langle\rangle.(1, (max\{x,y,i\},2 + n)), \lambda \langle\rangle.(1 + \pi_0 r, (max\{y,\pi_0\pi_1 r\}, 1 + \pi_1\pi_1 r))$

    $= (1, (max\{x,y,i\},2 + n)) \vee (1 + \pi_0 r, (max\{y,\pi_0\pi_1 r\}, 1 + \pi_1\pi_1 r))$
  \end{lstlisting}
\end{figure}
%
\begin{figure}[H]
  \caption{Interpretation of \T{rec} in \T{insert}.}
  \label{fig:interp_sizes_outer_rec}
  \begin{lstlisting}
   $g(i,n) = \llbracket$rec(xs,
        Nil$\mapsto \langle1$,Cons$\langle$x,Nil$\rangle\rangle$
        Cons$\mapsto \langle$y,$\langle$ys,r$\rangle\rangle$.$(3 + (($f x$)_p $ y$)_c) +_c$rec($($f x$)_p $ y$)_p$,
                                         True$\mapsto \langle 1, $Cons$\langle$x,Cons$\langle$y,ys$\rangle\rangle\rangle$
                                         False$\mapsto \langle1 + $r$_c,$Cons$\langle$y$,$r$_p\rangle\rangle$)$)\rrbracket \xi \{$f$\mapsto f, $x$\mapsto x,$xs$\mapsto (i,n)\}$

       $f_{Nil}(\langle\rangle) = \llbracket \langle1$,Cons$\langle$x,Nil$\rangle\rangle \rrbracket \xi \{$f$\mapsto f, $x$\mapsto x,$xs$\mapsto (i,n)\}$

       $f_{Nil}(\langle\rangle) = (1,(x,1))$

       $f_{Cons}((j,(j,m))) = \llbracket (3 + $((f x)$_p$ y)$_c) +_c $rec($\dots$)$ \rrbracket \xi $
            $\{$f$\mapsto f, $x$\mapsto x,$xs$\mapsto (i,n),\langle$y,$\langle$ys,r$\rangle\rangle\mapsto (map^{\mathbb{Z} \times \mathbb{N}^\infty} (\lambda a.(a,\llbracket$rec($w, \dots$)$\rrbracket \xi \{w \mapsto a\}),(j,(j,m))))\}$

           $= \llbracket \dots \rrbracket \xi \{\dots \langle$y,$\langle$ys,r$\rangle\rangle\mapsto (j,map^{\mathbb{N}^\infty} (\lambda a.(a,\llbracket$rec($w, \dots$)$\rrbracket \xi \{w \mapsto a\}),(j,m)))\}$

           $= \llbracket \dots \rrbracket \xi \{\dots \langle$y,$\langle$ys,r$\rangle\rangle\mapsto (j,((j,m),\llbracket$rec($w, \dots$)$\rrbracket \xi \{w \mapsto (j,m)\}))\}$

           $= \llbracket \dots \rrbracket \xi \{\dots \langle$y,$\langle$ys,r$\rangle\rangle\mapsto (j,((j,m),g(j,m)))\}$

           $= (3 + \pi_0(\pi_1(f\ x)\ j)) +_c$
                $((1,(max\{x,j\},2+m))) \vee (1+\pi_0g(j,m),(max\{j,\pi_0\pi_1g(j,m)\},1 + \pi_1\pi_1g(j,m)))$

           $= (3 + \pi_0(\pi_1(f\ x)\ j)) +_c$
                $(1 \vee (1+\pi_0g(j,m)), (max\{x,j,\pi_0\pi_1g(j,m)\}, 2 + m \vee 1 + \pi_1\pi_1g(j,m)))$

           $= (4 + \pi_0(\pi_1(f\ x)\ j) + \pi_0g(j,m), (max\{x,j\pi_0\pi_1g(j,m)\}, 2+m \vee 1 + \pi_1\pi_1g(j,m)))$

       $g(i,n) = \bigvee_{size(z) \leq (i,n)} case(z, (f_{Nil},f_{Cons}))$
  \end{lstlisting}
\end{figure}
%
The initial result is given in equation \ref{eq:insert_initial_recurrence}.
%
\begin{align*}
  &f_{Nil}(\langle\rangle) = (1,(x,1)) \\ 
  &f_{Cons}(j,(j,m)) = (4 + \pi_0(\pi_1(f\ x)\ j) + \pi_0g(j,m), \\
  &\ \ \ \ (max\{x,j,\pi_0\pi_1g(j,m)\}, 2+m \vee 1 + \pi_1\pi_1g(j,m))) \\
  \label{eq:insert_initial_recurrence}
  &g(i,n) = \bigvee_{size(z) \leq (i,n)} case(z, (f_{Nil},f_{Cons})) \numberthis
\end{align*}
%
This recurrence is difficult to work with.  Specifically, we cannot apply
traditional methods of solving it.  We will manipulate it into a more usable
form by eliminating the arbitrary maximum.  We will separate the recurrence
into a recurrence for the cost and a recurrence for the potential, and solve
those independently.
%
\begin{lemma}
  \label{lem:insert_rec_cost}
  $g_c(i,n) \leq (4 + ((f\ x)_p\ i)_c n + 1$
\end{lemma}
%
\begin{proof}
  We prove this by induction on $n$.
  Recall we use the mutual ordering on pairs.
  \begin{description}
    \item[case $n=0$]\hfill \\
      $g_c(i,n) = (1, (x, 1))_c = 1$
    \item[case$n>0$]\hfill \\
      \begin{align*}
        &= \bigvee_{size(z) \leq (i,n)} case(z, (f_{Nil}, f_{Cons})) &&\\
        &= \bigvee_{j < i, m \leq n \text{ or } j \leq i, m < n} case((j, m), (f_{Nil}, f_{Cons})) &&\\
        &= \bigvee_{j < i, m \leq n \text{ or } j \leq i, m < n} 4 + ((f\ x)_p\ j)_c + g_c(j, m')) &&\text{where $m' = m - 1$}\\
        &= \bigvee_{j < i, m \leq n \text{ or } j \leq i, m < n} 4 + ((f\ x)_p\ j)_c + (4 + ((f\ x)_p\ j)_c)m' + 1 &&\text{by the induction hypothesis}\\
        &= \bigvee_{j < i, m \leq n \text{ or } j \leq i, m < n} (4 + ((f\ x)_p\ j)_c) (m' + 1) + 1 &&\\
        &= \bigvee_{j < i, m \leq n \text{ or } j \leq i, m < n} (4 + ((f\ x)_p\ j)_c) m + 1 &&\\
        &\leq \bigvee_{i < j, m \leq n \text{ or } i \leq j, m < n} (4 + ((f\ x)_p\ i)_c) n + 1 &&\\
        &\leq (4 + ((f\ x)_p\ i)_c) n + 1&&
      \end{align*}
  \end{description}
\end{proof}
%
As expected, we find the cost of insert is bounded by the length of the list and the largest element.
%
\begin{lemma}
  \label{lem:insert_rec_potential}
  $g_p(i,n) \leq (max\{x, i\}, n+1)$
\end{lemma}
%
\begin{proof}
  We prove this by induction on $n$.
  \begin{description}
    \item[case $n=0$]\hfill \\
      $g_p(i,n) = (1, (x, 1))_p = (x, 1)$.
    \item[case $n>0$]\hfill \\
      \begin{align*}
        &= \bigvee_{size(z) \leq (i,n)} case(z, (f_{Nil}, f_{Cons}) &&\\
        &= \bigvee_{j < i, m \leq n \text{ or } j \leq i, m < n} (max\{x, j, \pi_0\pi_1g(j, m')\}, 2 + m' \vee 1 + \pi_1\pi_1g(j, m')) && \text{where $m' = m - 1$}\\
        &\leq \bigvee_{j < i, m \leq n \text{ or } j \leq i, m < n} (max\{x, j\}, 2 + m')&&\text{by the induction hypothesis}\\
        &\leq \bigvee_{j < i, m \leq n \text{ or } j \leq i, m < n} (max\{x,i\}, 1 + n)&&\\
        &\leq (max\{x,i\}, 1 + n)&&
      \end{align*}
  \end{description}
\end{proof}
%
We find the length of the potential is bounded by one plus the length of the
input, and the largest element in the output is bounded by the maximum of the
element being inserted and the largest element in the input.  This is somewhat
unsatisfactory, since we would expect the relationship to be equality.  What
happens if we try to prove the equality?
%
\begin{lemma}
  \label{lem:insert_rec_potential_wrong}
  $g_p(i,n) = (max\{x, i\}, n+1)$
\end{lemma}
\begin{proof}
  We attempt to prove this by induction on $n$.
  The first steps proceed similarly to \ref{lem:insert_rec_potential}.
  \begin{description}
    \item[case $n=0$]\hfill \\
      $g_p(i,n) = (1, (x, 1))_p = (x, 1)$.
    \item[case $n>0$]\hfill \\
      \begin{align*}
        &= \bigvee_{size(z) \leq (i,n)} case(z, (f_{Nil}, f_{Cons}) &&\\
        &= \bigvee_{j < i, m \leq n \text{ or } j \leq i, m < n} (max\{x, j, \pi_0\pi_1g(j, m')\}, 2 + m' \vee 1 + \pi_1\pi_1g(j, m')) && \text{where $m' = m - 1$}\\
        &= \bigvee_{j < i, m \leq n \text{ or } j \leq i, m < n} (max\{x, j\}, 2 + m')&&\text{by the induction hypothesis}\\
        &= \bigvee_{j < i, m \leq n} (max\{x, j\}, 1 + m) \vee \bigvee_{j \leq i, m < n} (max\{x, j\}, 1 + m)&&
      \end{align*}
  \end{description}
\end{proof}
%
We see that we get stuck.  Because of the mutual ordering on pairs, our big
maximum is over all $z$ such that $size(z) < (i, n)$.  This includes $(j, m)$
such that $j < i ^ m \leq n$.  We have no way of reasoning about the potential
of $g(i-1, n) \vee g(i, n -1)$.  So we cannot prove equality for
\ref{lem:insert_rec_potential}.  This indicates we may not have the optimal
ordering on pairs.

Using lemmas \ref{lem:insert_rec_cost} and \ref{lem:insert_rec_potential}, we
can express the cost and potential of \T{insert} in terms of its arguments.
%
\begin{equation}
  \label{eq:insert_interp}
  insert\ f\ x\ xs \leq (4 + ((f\ x)_p\ i)_c n + 1, (max\{x, i\}, n+1))
\end{equation}
%

\subsubsection{Sort}
%
\subsubsection{Translation}
The translation of sort is shown in figure \ref{fig:sort}.  The translation of
the \T{Nil} and \T{Cons} branches in the \T{rec} are walked through in figures
\ref{fig:sort_nil} and \ref{fig:sort_cons}, respectively.  The translation of
\T{sort} applied to its arguments is given in figure \ref{fig:sort_applied}.
%
\begin{figure}[H]
  \caption{Translation of \T{Nil} branch of \T{sort}.}
  \label{fig:sort_nil}
  \begin{lstlisting}
  $\|$Nil$\mapsto$Nil$\|$

  $= $Nil$\mapsto 1 +_c \|$Nil$\|$

  $= $Nil$\mapsto 1 +_c \langle$0,Nil$\rangle$

  $= $Nil$\mapsto \langle$1,Nil$\rangle$
  \end{lstlisting}
\end{figure}
%
\begin{figure}[H]
  \caption{Translation of \T{Cons} branch of \T{sort}.}
  \label{fig:sort_cons}
  \begin{lstlisting}
  $\|$Cons$\mapsto\langle$y,$\langle$ys,r$\rangle\rangle$.insert f y force(r)

  $= $Cons$\mapsto 1 +_c \|$insert f y force(r)$\|$

  $= $Cons$\mapsto\langle$y,$\langle$ys,r$\rangle\rangle. 1 +_c (\|$force(r)$\|_c) +_c \|$insert f y$\|_p \|$force(r)$\|_p$

  $= $Cons$\mapsto\langle$y,$\langle$ys,r$\rangle\rangle. 1 +_c ((\|$r$\|_c +_c \|$r$\|_p)_c) +_c \|$insert f y$\|_p (\|$r$\|_c +_c \|$r$\|_p)_p$

  $= $Cons$\mapsto\langle$y,$\langle$ys,r$\rangle\rangle. 1 +_c $r$_c +_c \|$insert f y$\|_p $r$_p$

  $= $Cons$\mapsto\langle$y,$\langle$ys,r$\rangle\rangle. 1 +_c $r$_c +_c 3 +_c \|$insert$\|_p$ f y r$_p$

  $= $Cons$\mapsto\langle$y,$\langle$ys,r$\rangle\rangle. (4 + $r$_c) +_c \|$insert$\|_p$ f y r$_p$

  \end{lstlisting}
\end{figure}
%
\begin{figure}[H]
\caption{Translation of \T{sort}}
\label{fig:sort}
\begin{lstlisting}
$\|$sort$\|$ = $\langle0,\lambda$f.$\langle0,\lambda$xs.$\|$rec(xs, Nil$\mapsto$ Nil,
                        Cons$\mapsto \langle$y,$\langle$ys,r$\rangle\rangle$.insert f y force(r))$\|\rangle\rangle$

      = $\langle0,\lambda$f.$\langle0,\lambda$xs.rec(xs, Nil$\mapsto \langle$1,Nil$\rangle$
                        Cons$\mapsto \langle$y,$\langle$ys,r$\rangle\rangle.4 +_c $r$_c +_c \|$insert$\|_p$ f y r$_p$
\end{lstlisting}
\end{figure}
%
\begin{figure}[H]
  \caption{Translation of \T{sort} applied to variables \T{f} and \T{xs}}
\label{fig:sort_applied}
\begin{lstlisting}
$\|$sort f xs$\| = (1 + \|$sort f$\|_c + \|$xs$\|_c) +_c \|$sort f$\|_p \|$xs$\|_p$

          $ = (1 + (1 + \|$sort$\|_c + \|$f$\|_c + \|$xs$\|_c)) +_c \|$sort$\|_p \|$f$\|_p \|$xs$\|_p$

          $ = (1 + (1 + 0 + 0 + 0)) +_c \|$sort$\|_p \|$f$\|_p \|$xs$\|_p$

          $ = 2 +_c \|$sort$\|_p $f$\|_p \|$xs$\|_p$

          $ = 2 +_c $rec($\|$xs$\|_p$, Nil$\mapsto \langle$1,Nil$\rangle$
                        Cons$\mapsto \langle$y,$\langle$ys,r$\rangle\rangle.(4 + $r$_c) +_c \|$insert$\|_p$ f y r$_p$
\end{lstlisting}
\end{figure}
%
%
\subsubsection{Interpretation}
%
The \T{rec} construct again drives the cost and potential of \T{sort}.  The
walk through of the interpretation of the \T{rec} is given in figure
\ref{fig:sort_rec_interp}.
%
\begin{figure}[H]
  \caption{Interpretation of \T{rec} in \T{sort}.TO DO FIX THIS}
  \label{fig:sort_rec_interp}
  \begin{lstlisting}
  $g(i, n) = \llbracket $rec($\|$xs$\|_p$, Nil$\mapsto \langle$1,Nil$\rangle$
                Cons$\mapsto \langle$y,$\langle$ys,r$\rangle\rangle.(4 + $r$_c) +_c \|$insert$\|_p$ f y r$_p$)$\rrbracket \xi \{ xs \mapsto n\}$

     $ = \llbracket $rec($\|$xs$\|_p$, Nil$\mapsto \langle$1,Nil$\rangle$
                Cons$\mapsto \langle$y,$\langle$ys,r$\rangle\rangle.(4 + $r$_c) +_c \|$insert$\|_p$ f y r$_p$)$\rrbracket \xi \{ xs \mapsto n\}$

     $ = \llbracket $rec($\|$xs$\|_p$, Nil$\mapsto \langle$1,Nil$\rangle$
                Cons$\mapsto \langle$y,$\langle$ys,r$\rangle\rangle.(4 + $r$_c) +_c \|$insert$\|_p$ f y r$_p$)$\rrbracket \xi \{ xs \mapsto n\}$

     $ = \bigvee_{size(z)\leq n} case(z,(f_{Nil},f_{Cons}))$

  $f_{Nil}(\langle\rangle) = \llbracket \langle$1,Nil$\rangle \rrbracket \xi$
     $ = f_{Nil}(\langle\rangle) = (1,(\neg\infty,0))$

  $f_{Cons}((j,m)) = \llbracket \dots \rrbracket \xi \{\langle$y,$\langle$ys,r$\rangle\rangle\mapsto map^{j\times\mathbb{N}^\infty} (\lambda a.(a,\llbracket$rec($w, \dots$)$\rrbracket \xi \{w \mapsto a\}), (j,m)) \}$

  $ = \llbracket \dots \rrbracket \xi \{\langle$y,$\langle$ys,r$\rangle\rangle\mapsto (map^{int} (\lambda a.(\dots), j),map^{\mathbb{N}^\infty} (\lambda a.(a,\llbracket$rec($w, \dots$)$\rrbracket \xi \{w \mapsto a\}), m)) \}$

    $ = \llbracket \dots \rrbracket \xi \{\langle$y,$\langle$ys,r$\rangle\rangle\mapsto (j,(m,\llbracket$rec($w, \dots$)$\rrbracket \xi \{w \mapsto m\})) \}$

    $ = \llbracket (4 + $r$_c) +_c \|$insert$\|_p$ f y r$_p \rrbracket \xi \{\langle$y,$\langle$ys,r$\rangle\rangle\mapsto (j,(m,g(j, m)) \}$

    $ = (4 + \pi_0 g(j, m)) +_c insert\ f\ j\ \pi_1g(j,m)$

  $g(i, n) = \bigvee_{size(z)\leq n} case(z,(\lambda(\langle\rangle).(1,(\neg\infty,0)),\lambda(j,m).(4 + \pi_0 g(j,m)) +_c insert\ f\ j\ \pi_1g(j, m)))$
  \end{lstlisting}
\end{figure}
%
Equation \ref{eq:sort_interp0_init} shows the initial recurrence extracted.
%
\begin{equation}
  \label{eq:sort_interp0_init}
  g(i,n) = \bigvee_{size(z)\leq (i,n)} case(z,(\lambda(\langle\rangle).(1,(\neg\infty,0),\lambda(j,m).4 + \pi_0 g(j,m)) +_c(insert\ f\ j\ \pi_1g(j, m)))
\end{equation}
%
Observe that in equation \ref{eq:sort_interp0_init}, the cost is depends on the
potential of the recursive call.  Therefore we must solve the recurrence for
the potential first.
%
\begin{lemma}
  \label{lem:sort_interp_potential}
  $\pi_1g(n) \leq (j, n)$
\end{lemma}
\begin{proof}
  We prove this by induction on $n$.
  We use equation \ref{eq:insert_interp} to determine the potential of the $insert$ function.
  \begin{description}
    \item[case $n=0$]$\pi_1g(i,n) = (i, 0)$
    \item[case $n>0$]\hfill \\
      \begin{align*}
        \pi_1g(i,n) &= \pi_1 \bigvee_{size(z)\leq n} case(z,(\lambda(\langle\rangle).(1,(\neg\infty,0)),\lambda(j,m).4 + \pi_0 g(j,m)) +_c(insert\ f\ j\ \pi_1g(j, m)))\\
        &= \bigvee_{j \leq i, m < n \text{ or } j < i, m \leq n} \pi_1 (insert\ f\ j\ \pi_1g(j', m'))\ \ \ \ \ j' \leq j, m' = m - 1\\
        &\leq \bigvee_{j \leq i, m < n \text{ or } j < i, m \leq n} \pi_1 (insert\ f\ j\ (j', m'))\\
        &\leq \bigvee_{j \leq i, m < n \text{ or } j < i, m \leq n} (max\{j, j'\}, m' + 1\\
        &\leq \bigvee_{j \leq i, m < n \text{ or } j < i, m \leq n} (j, m)\\
        &\leq \bigvee_{j \leq i, m < n \text{ or } j < i, m \leq n} (i, n)\\
        &\leq (i, n)
      \end{align*}
  \end{description}
\end{proof}
%
As in the interpretation of \T{insert} we are left with a less than
satisfactory bound on the potential of \T{sort}.  It would grievous mistake to
write a sorting function whose output was smaller than its input.  Under the
current interpretation of lists, this would mean either the length of the list
decreased or the size of the largest element in the list decreased.
Unfortunately we are stuck with an upper bound on the size of the output
because or interpretation of \T{insert} only provides an upper bound on the
potential of its output. We may solve the recurrence for the cost of \T{sort}.
%
\begin{lemma}
  \label{lem:sort_interp_cost}
  $\pi_0g(n) \leq (4 + \pi_0(\pi_1(f\ x)\ i)n^2 + 5n + 1$
\end{lemma}
%
\begin{proof}
  We prove this by induction on $n$.
  \begin{description}
    \item[case $n=0$] $\pi_0 g(i,n) = 1$
    \item[case $n>0$] \hfill \\
      \begin{align*}
        \pi_0g(i,n) &= \pi_0 \bigvee_{size(z) \leq (i,n)} case(z, (\lambda(\langle\rangle).(1,(\neg\infty,0)),\lambda(j,m).4 + \pi_0 g(j, m)) +_c (insert\ f\ j\ \pi_1g(j, m)))\\
        &= \bigvee_{j < i, m \leq n \text{ or } j \leq i, m < n} 4 + \pi_0 g(j, m - 1) + \pi_0(insert\ f\ j\ \pi_1g(j, m - 1))\\
        &\leq \bigvee_{j < i, m \leq n \text{ or } j \leq i, m < n} 4 + \pi_0 g(j, m - 1) + \pi_0(insert\ f\ j\ (j, m - 1))\\
        &\leq \bigvee_{j < i, m \leq n \text{ or } j \leq i, m < n} 4 + \pi_0 g(j, m - 1) + (4 + \pi_0(\pi_1(f\ j)\ j))(m - 1) + 1\\
        & \text{let $c_1 = (4 + \pi_0(\pi_1(f\ j)\ j))$}\\
        &\leq \bigvee_{j < i, m \leq n \text{ or } j \leq i, m < n} 4 + c_1(m-1)^2 + 5(m-1) + 1 + c_1(m - 1) + 1\\
        &\leq \bigvee_{j < i, m \leq n \text{ or } j \leq i, m < n} 4 + c_1m^2 - 2c_1m +c_1 + 5m-5 + 1 + c_1m - c_1 + 1\\
        &\leq \bigvee_{j < i, m \leq n \text{ or } j \leq i, m < n} c_1m^2 - c_1m + 5m + 1\\
        &\leq \bigvee_{j < i, m \leq n \text{ or } j \leq i, m < n} (4 + \pi_0(\pi_1(f\ i)\ i))n^2 + 5n + 1\\
        &\leq (4 + \pi_0(\pi_1(f\ i)\ i))n^2 + 5n + 1
      \end{align*}
  \end{description}
\end{proof}
%
As expected the cost of \T{sort} is $\mathcal{O}(n^2)$ where $n$ is the length
of the list.  It is clear from the analysis how the cost of the comparison
function determines the running time of \T{sort}.  We can see that the
comparison function is called order $n^2$ times.
