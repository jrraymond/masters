\section{Parametric Insertion Sort}
%
Parametric insertion sort is a higher order algorithm which sorts a list using
a comparison function which is passed to it as an argument.  The running time
of insertion sort is $\mathcal{O}(n^2)$.  This characterization of the
complexity of parametric insertion sort does not capture role of the comparison
function in the running time.  When sorting a list of integers, where
comparison between any two integers takes constant time, this does not matter.
However, when sorting a list of strings, where the complexity of comparison is
order the length of the string, the length of the strings may influence the
running time more than the length of the list when sorting small lists of large
strings.

We use the familiar \T{list} datatype.
%
\begin{flalign*}
  \T{data list} &= \T{Nil of unit | Cons of int $\times$ list}
\end{flalign*}
%
The function \T{sort} relies on the function \T{insert}. \T{insert} inserts an
element into a sorted list.
%
\begin{flalign*}
  \T{insert} &= \lambda f.\lambda x.\lambda xs.\T{rec}(xs, \T{Nil} \mapsto \T{Cons} \LP x, \T{Nil}\RP,\\
             &\quadeight \T{Cons}\mapsto \LP y, \LP ys,r \RP\RP.\T{rec}(f\ x\ y, \T{True}\mapsto \T{Cons}\LP x,\T{Cons}\LP y,ys \RP\RP, \\
             &\quadten\quadten\quad \T{False}\mapsto \T{Cons}\LP y,\T{force}(r)\RP))
\end{flalign*}
%
The \T{sort} function recurses on the list, using the \T{insert} function to
insert the head of the list into the recursively sorted tail of the list.
%
\begin{flalign*}
  \T{sort} &= \lambda f.\lambda xs.\T{rec}(xs, \T{Nil} \mapsto \T{Nil}, \T{Cons} \mapsto \LP y,\LP ys,r \RP\RP.\T{insert}\ f\ y\ \T{force}(r))
\end{flalign*}
%
\subsection{Translation of \T{insert}}
%
We walk through the translation of \T{insert}. We will translate from the
bottom up. We translate the \T{True} and \T{False} branches of the inner
\T{rec}, then we translate the inner \T{rec}, then we translate the \T{Nil} and
\T{Cons} branches of the outer \T{rec}, and finally complete the translation of
\T{insert}.
%
% BEGIN INSERT TRANSLATION
%
%
% BEGIN TRUE BRANCH
%
The translation of the \T{True} branch of the inner \T{rec} is given below.
The translation of a datatype is the cost of translating its argument, and
complexity language constructor applied to the potential of the translated
argument.
%
\begin{flalign*}
  \|\T{Cons}\LP x,\T{Cons}\LP y, rs \RP\RP\| &= \LP \|\LP x, \T{Cons} \LP y,ys\RP\RP\|_c, \T{Cons}\|\LP x, \T{Cons}\LP y,ys\RP\|_p\RP
\end{flalign*}
%
The argument to the \T{Cons} constructor is a tuple. The cost of the
translation of a tuple is the cost of the translation of each element and the
potential is the tuple of the potentials of the translations of each element.
%
\begin{flalign*}
  \|\LP x, \T{Cons}\LP y,ys\RP\RP\| &= \LP \|x\|_c + \|\T{Cons}\LP y,ys\RP\|_c, \LP \|x\|_p,\T{Cons}\LP y,ys\RP\|_p\RP\RP
\end{flalign*}
%
The first element of the tuple is a variable, but the second element is another
\T{list}. So we translate the second element first. To do so we apply the rule
for translating a datatype.
%
\begin{flalign*}
  \|\T{Cons}\LP y,ys\RP\| &= \LP \|y,ys\|_c, \T{Cons}\|y,ys\|_p\RP
\end{flalign*}
%
The argument to the constructor is a tuple. We apply the rule for translating a
tuple again. Both element of the tuple are variables, so their translated cost
is $0$ and their translated potential is their corresponding variable in the
complexity language.
%
\begin{flalign*}
  \|\LP y,ys\RP\| &= \LP \|y\|_c + \|rs\|_c, \LP\|y\|_p,\|rs\|_p\RP\RP \\
                         &= \LP \LP 0,y\RP_c + \LP 0, rs\RP_c, \LP \LP0,y\RP_p, \LP 0,rs\RP\RP\RP \\
                         &= \LP 0, \LP y,ys\RP\RP
\end{flalign*}
%
We use this to complete the translation of $\T{Cons}\LP y,ys\RP$.
%
\begin{flalign*}
  \|\T{Cons}\LP y,ys\RP\| &= \LP \|y,ys\|_c, \T{Cons}\|y,ys\|_p\RP \\
                                  &= \LP 0, \T{Cons}\LP y,ys\RP\RP
\end{flalign*}
%
We use this result to complete the translation of $\LP x,\T{Cons}\LP y,ys\RP\RP$.
%
\begin{flalign*}
  \|\LP x, \T{Cons}\LP y,ys\RP\RP\| &= \LP \|x\|_c + \|\T{Cons}\LP y,ys\RP\|_c, \LP \|x\|_p,\T{Cons}\LP y,ys\RP\|_p\RP\RP \\
                                                    &= \LP 0, \LP x,\T{Cons}\LP y,ys\RP\RP\RP
\end{flalign*}
%
And finally we use this to complete the translation of $\T{Cons}\LP x,\T{Cons}\LP y,ys\RP\RP$.
%
\begin{flalign*}
  \|\T{Cons}\LP x,\T{Cons}\LP y,ys \RP\RP\| &= \LP \|\LP x, \T{Cons} \LP y,ys\RP\RP\|_c, \T{Cons}\|\LP x,\T{Cons}\LP y,ys\RP\RP\|_p\RP \\
                                                            &= \LP 0, \T{Cons}\LP x,\T{Cons}\LP y,ys\RP\RP\RP
\end{flalign*}
%
% END TRUE BRANCH
%
% BEGIN FALSE BRANCH
%
Next we will translate the \T{False} branch.
%
\begin{flalign*}
  \|\T{Cons}\LP y,\T{force}(r)\RP\| &= \LP \|\LP y,\T{force}(r)\RP\|_c, \T{Cons}\|\LP y,\T{force}(r)\RP\|_p\RP
\end{flalign*}
%
To complete this we must first translate the tuple. The two elements of the
tuple are $y$ and $\T{force}(r)$.  The translation of the variable $y$ is
$\LP 0, y\RP$. The translation of $\T{force}(r)$ is
$\|r\|_c +_c \|r\|_p$. Like $y$, $r$ is a variable so its translation is
$\LP 0,r\RP$. So the translation of $\T{force}(r)$ is $0 +_c r$ which
simplifies to $r$.
%
\begin{flalign*}
  \|\LP y,\T{force}(r)\RP\| &= \LP \|y\|_c + \|\T{force}(r)\|_c, \LP\|y\|_p,\|\T{force}(r)\|_p\RP\RP \\
                                    &= \LP 0 + r_c, \LP y,r_p\RP\RP \\
                                    &= \LP r_c, \LP y,r_p\RP\RP
\end{flalign*}
%
We substitute this into the translation of $\T{Cons}\LP y,\T{force}(r)\RP$.
%
\begin{flalign*}
  \|\T{Cons}\LP y,\T{force}(r)\RP\| &= \LP \|\LP y,\T{force}(r)\RP\|_c, \T{Cons}\|\LP y,\T{force}(r)\RP\|_p\RP \\
                                            &= \LP r_c, \T{Cons}\LP y,r_p\RP\RP
\end{flalign*}
%
% END FALSE BRANCH
%
% BEGIN INNER REC
%
We put together the translations of the \T{True} and \T{False} branches to
translate the inner \T{rec}.  We will need the translation of the comparison
function $f$ applied to $x$ and $y$.  To translate $f\ x\ y$ we apply the
function application rule twice. First we apply the rule to $(f\ x)\ y$. Then
we apply the rule to $f\ x$. Then we expand the $+_c$ macro to simplify the
result.
%
\begin{flalign*}
\numberthis \label{eq:fxy}
\|f\ x\ y\| &= (1 + \|f\ x\|_c + \|y\|_c) +_c \|f\ x\|_p \|y\|_p \\
            &\qquad \|f\ x\| = (1 + \|f\|_c + \|x\|_c) +_c \|f\|_p \|x\|_p \\
            &\quadfour = (1 + \|f\|_c + \|x\|_c + (\|f\|_p \|x\|_p)_c, (\|f\|_p \|x\|_p)_p) \\
            &= (1 + (1 + \|f\|_c + \|x\|_c +_c (\|f\|_p \|x\|_p)_c) + \|y\|_c) +_c (\|f\|_p \|x\|_p)_p \|y\|_p \\
            &= (2 + \|f\|_c + \|x\|_c + \|y\|_c + (\|f\|_p \|x\|_p)_c) +_c (\|f\|_p \|x\|_p)_p \|y\|_p
\end{flalign*}
%
We use the translation of \T{f x y} and the \T{True} and \T{False}
branches to construct the translation of the inner \T{rec} construct.
%
\begin{flalign*}
  \|\T{rec}&(f\ x\ y, \T{True}\mapsto\T{Cons}\LP x,\T{Cons}\LP y,ys\RP\RP, \T{False}\mapsto \T{Cons}\LP y,\T{force}(r)\RP)\| \\
           &= \|f\ x\ y\|_c +_c \T{rec}(\|f\ x\ y\|_p, \T{True}\mapsto 1 +_c \|\T{Cons}\LP x,\T{Cons}\LP y,ys\RP\RP\|, \\
           &\quadsix \T{False}\mapsto 1 +_c \|\T{Cons}\LP y,\T{force}(r)\RP\|) \\
           &= \|f\ x\ y\|_c +_c \T{rec}(\|f\ x\ y\|_p, \T{True}\mapsto 1 +_c \LP 0, \T{Cons}\LP x,\T{Cons}\LP y,ys\RP\RP\RP, \\
           &\quadsix \T{False}\mapsto 1 +_c \LP r_c, \T{Cons}\LP y,r_p\RP\RP) \\
           &= (2 + \|f\|_c + \|x\|_c + \|y\|_c + (\|f\|_p \|x\|_p)_c) \\
           &\quadthree +_c \T{rec}((\|f\|_p\ \|x\|_p)_p \|y\|_p, \T{True}\mapsto \LP 1, \T{Cons}\LP x,\T{Cons}\LP y,ys\RP\RP\RP, \\
           &\quadsix \T{False}\mapsto \LP 1 + r_c, \T{Cons}\LP y,r_p\RP\RP) \\
\end{flalign*}
%
% END INNER REC
%
% BEGIN NIL
%
Next we translate the \T{Nil} and \T{Cons} branches of the outer \T{rec} of
\T{insert}. In this branch we append the element to an empty list.
%
\begin{flalign*}
  \|\T{Cons} \LP x, \T{Nil}\RP\| &= \LP \|\LP x,\T{Nil}\RP\|_c,\T{Cons}\|\LP x,\T{Nil}\RP\|_p\RP \\
                                         &= \LP 0,\T{Cons}\LP x,\T{Nil}\RP\RP
\end{flalign*}
%
% END NIL
%
% BEGIN CONS
%
Translation of the \T{Cons} branch of the outer \T{rec} in \T{insert}.  In this
branch we recurse on a nonempty list.  We check if \T{x} is comes before the
head of the list under the ordering given by \T{f}, in which case we are done,
otherwise we recurse on the tail of the list.
%
\begin{flalign*}
  \|\T{rec}&(f\ x\ y, \T{True}\mapsto \T{Cons}\LP x,\T{Cons}\LP y,ys \RP\RP, \T{False}\mapsto \T{Cons}\LP y,\T{force}(r)\RP)\| \\
           &= \|f\ x\ y\|_c +_c \T{rec}(f\ x\ y, \T{True}\mapsto 1 +_c \|\T{Cons}\LP x,\T{Cons}\LP y,ys \RP\RP\|)  \\
           &= (2 + \|f\|_c + \|x\|_c + \|y\|_c + (\|f\|_p \|x\|_p)_c) \\
           &\quadthree +_c \T{rec}((\|f\|_p\ \|x\|_p)_p \|y\|_p, \T{True}\mapsto \LP 1, \T{Cons}\LP x,\T{Cons}\LP y,ys\RP\RP\RP, \\
           &\quadsix \T{False}\mapsto \LP 1 + r_c, \T{Cons}\LP y,r_p\RP\RP) \\
           &\text{We know that $f, x, $ and $y$ are variables, so their translations have $0$ cost.} \\
           &= (2 + (f\ x)_c) \\
           &\quadthree +_c \T{rec}(((f\ x)_p\ y)_p, \T{True}\mapsto \LP 1, \T{Cons}\LP x,\T{Cons}\LP y,ys\RP\RP\RP, \\
           &\quadsix \T{False}\mapsto \LP 1 + r_c, \T{Cons}\LP y,r_p\RP\RP)
\end{flalign*}
%
% END CONS
%
We complete the translation of the outer \T{rec} using the translated \T{Nil}
and \T{Cons}.
%
\begin{flalign*}
  \|\T{rec}&(xs, \T{Nil} \mapsto \T{Cons} \LP x, \T{Nil}\RP,\\
             &\quad \T{Cons}\mapsto \LP y, \LP ys,r \RP\RP.\T{rec}(f\ x\ y, \T{True}\mapsto \T{Cons}\LP x,\T{Cons}\LP y,ys \RP\RP, \\
             &\quadfour \T{False}\mapsto \T{Cons}\LP y,\T{force}(r)\RP))\| \\
             &= \|xs\|_c +_c \T{rec}(\|xs\|_p, \T{Nil} \mapsto 1 +_c \|\T{Cons} \LP x, \T{Nil}\RP\|,\\
             &\quad \T{Cons}\mapsto \LP y, \LP ys,r \RP\RP. 1 +_c \|\T{rec}(f\ x\ y, \T{True}\mapsto \T{Cons}\LP x,\T{Cons}\LP y,ys \RP\RP, \\
             &\quadfour \T{False}\mapsto \T{Cons}\LP y,\T{force}(r)\RP)\|)\| \\
             %
             &\text{We substitute in our translations of the branches. Also note that $xs$}\\
             &\text{is a variable, so its translation is $\LP 0,xs\RP$.} \\
             &= \T{rec}(xs, \T{Nil} \mapsto 1 +_c \LP 0,\T{Cons}\LP x,\T{Nil}\RP\RP, \\
             &\quad \T{Cons}\mapsto \LP y, \LP ys,r \RP\RP. 1 +_c ((2 + (f x)_c) \\
             &\quadthree +_c \T{rec}(((f\ x)_p\ y)_p, \T{True}\mapsto \LP 1, \T{Cons}\LP x,\T{Cons}\LP y,ys\RP\RP\RP, \\
             &\quadsix \T{False}\mapsto \LP 1 + r_c, \T{Cons}\LP y,r_p\RP\RP))) \\
             &= \T{rec}(xs, \T{Nil} \mapsto \LP 1,\T{Cons}\LP x,\T{Nil}\RP\RP, \\
             &\quad \T{Cons}\mapsto \LP y, \LP ys,r \RP\RP. (3 + (f x)_c) \\
             &\quadthree +_c \T{rec}(((f\ x)_p\ y)_p, \T{True}\mapsto \LP 1, \T{Cons}\LP x,\T{Cons}\LP y,ys\RP\RP\RP, \\
             &\quadsix \T{False}\mapsto \LP 1 + r_c, \T{Cons}\LP y,r_p\RP\RP))) \\
\end{flalign*}
%
% END OUTER REC
%
The translation of \T{insert} is just three applications of the application
rule.
%
\begin{flalign*}
  \T{insert} &= \|\lambda f.\lambda x.\lambda xs.\T{rec}(xs, \T{Nil} \mapsto \T{Cons} \LP x, \T{Nil}\RP,\\
             &\quadeight \T{Cons}\mapsto \LP y, \LP ys,r \RP\RP.\T{rec}(f\ x\ y, \T{True}\mapsto \T{Cons}\LP x,\T{Cons}\LP y,ys \RP\RP, \\
             &\quadten\quadeight \T{False}\mapsto \T{Cons}\LP y,\T{force}(r)\RP))\| \\
             &= \LP 0, \lambda f. \LP 0, \lambda x.\LP 0,\lambda xs.\|\T{rec}(xs, \T{Nil} \mapsto \T{Cons} \LP x, \T{Nil}\RP,\\
             &\quadeight \T{Cons}\mapsto \LP y, \LP ys,r \RP\RP.\T{rec}(f\ x\ y, \T{True}\mapsto \T{Cons}\LP x,\T{Cons}\LP y,ys \RP\RP, \\
             &\quadten\quadeight \T{False}\mapsto \T{Cons}\LP y,\T{force}(r)\RP))\|\RP\RP\RP \\
             &= \LP 0, \lambda f. \LP 0, \lambda x.\LP 0,\lambda xs. \T{rec}(xs, \T{Nil} \mapsto \LP 1,\T{Cons}\LP x,\T{Nil}\RP\RP, \\
             &\quad\quad \T{Cons}\mapsto \LP y, \LP ys,r \RP\RP. (3 + (f x)_c) +_c \T{rec}(((f\ x)_p\ y)_p, \\
             &\quadten\quadten \T{True}\mapsto \LP 1, \T{Cons}\LP x,\T{Cons}\LP y,ys\RP\RP\RP, \\
             &\quadten\quadten \T{False}\mapsto \LP 1 + r_c, \T{Cons}\LP y,r_p\RP\RP)))
\end{flalign*}
%
% END INSERT
%
We are interested in the interpretation of applying \T{insert}. So we will give
a translation of \T{insert f x xs}.
%
\begin{flalign*}
  \|\T{insert}\ f\ x\ xs\| &= (1 + \|\T{insert}\ f\ x\|_c + \|xs\|_c) +_c \|\T{insert}\ f\ x\|_p \|xs\|_p \\
     &= (1 + \|\T{insert}\ f\ x\|_c + \|xs\|_c) +_c \|\T{insert}\ f\ x\|_p  \|xs\|_p \\
     &= (2 + \|\T{insert}\ f\|_c + \|x\|_c + (\|\T{insert}\ f\|_p \|x\|_p)_c + \|xs\|_c) \\
     &\quadfour +_c \|\T{insert}\ f\|_p \|x\|_p \|xs\|_p \\
     &= (2 + (\|\T{insert}\|_p\ f)_c + \|x\|_c + \|xs\|_c) +_c \|\T{insert}\ f\|_p \|x\|_p \|xs\|_p \\
     &= (3 + \|\T{insert}\|_c + \|f\|_c+ (\|\T{insert}\|_p \|f\|_p \|x\|_p)_c + \|x\|_c + \|xs\|_c) \\
     &\quadfour +_c \|\T{insert}\|_p \|f\|_p \|x\|_p \|xs\|_p \\
     &= (3 + \|f\|_c + \|x\|_c + \|xs\|_c) +_c \|\T{insert}\|_p \|f\|_p \|x\|_p \|xs\|_p \\
     &= (3 + \|f\|_c + \|x\|_c + \|xs\|_c) +_c \T{rec}(\|xs\|_p, \\
     &\quadfour \T{Nil}\mapsto \LP 1,\T{Cons}\LP x,\T{Nil}\RP\RP  \\
     &\quadfour \T{Cons}\mapsto \LP y,\LP ys,r\RP\RP.(3 + ((\|f\|_p \|x\|_p)_p  y)_c) \\
     &\quadsix +_c \T{rec}((\|f\|_p \|x\|_p)_p  y)_p, \\
     &\quadeight \T{True}\mapsto \LP 1, \T{Cons}\LP \|x\|_p,\T{Cons}\LP y,ys\RP\RP\RP \\
     &\quadeight \T{False}\mapsto \LP 1 + r_c,\T{Cons}\LP y,r_p\RP\RP))
\end{flalign*}
%
% END INSERT APPLIED
%
% BEGIN SORT TRANSLATION
%
\subsection{Translation of \T{sort}}
%
We walk through the translation of \T{sort}. Recall the definition of \T{sort}.
%
\begin{flalign*}
  \T{sort} &= \lambda f.\lambda xs.\T{rec}(xs, \T{Nil} \mapsto \T{Nil}, \T{Cons} \mapsto \LP y,\LP ys,r \RP\RP.\T{insert}\ f\ y\ \T{force}(r))
\end{flalign*}
%
The translation of \T{sort} begins with two applications of the rule for translating abstractions.
%
\begin{flalign*}
  \|\T{sort}\| &= \| \lambda f.\lambda xs.\T{rec}(xs, \T{Nil} \mapsto \T{Nil}, \T{Cons} \mapsto \LP y,\LP ys,r \RP\RP.\T{insert}\ f\ y\ \T{force}(r))\| \\
               &= \LP 0, \lambda f.\LP 0,\lambda xs.\|\T{rec}(xs, \T{Nil} \mapsto \T{Nil}, \\
               &\quadthree \T{Cons} \mapsto \LP y,\LP ys,r \RP\RP.\T{insert}\ f\ y\ \T{force}(r))\|\RP\RP \\
               &= \LP 0, \lambda f.\LP 0,\lambda xs.\|xs\|_c +_c \T{rec}(\|xs\|_p, \T{Nil} \mapsto 1 +_c \|\T{Nil}\|, \\
               &\quadthree \T{Cons} \mapsto \LP y,\LP ys,r \RP\RP.1 +_c \|\T{insert}\ f\ y\ \T{force}(r)\|)\RP\RP \\
               &\text{As $xs$ is a variable, $\|xs\| = \LP 0,xs \RP$.} \\
               &\text{We have seen before $\|\T{Nil}\| = \LP 0,\T{Nil}\RP$.}\\
               &= \LP 0, \lambda f.\LP 0,\lambda xs.\T{rec}(xs, \T{Nil} \mapsto 1 +_c \LP 0,\T{Nil}\RP, \\
               &\quadthree \T{Cons} \mapsto \LP y,\LP ys,r \RP\RP.1 +_c \|\T{insert}\ f\ y\ \T{force}(r)\|)\RP\RP \\
               &\text{We can use our translation of \T{insert} applied to three arguments.} \\
               &= \LP 0, \lambda f.\LP 0,\lambda xs.\T{rec}(xs, \T{Nil} \mapsto 1 +_c \LP 0,\T{Nil}\RP, \\
               &\quad \T{Cons} \mapsto \LP y,\LP ys,r \RP\RP.(4 + \|f\|_c + \|y\|_c + \|force(r)\|_c) \\
               &\quadthree +_c ((\|\T{insert}\|_p \|f\|_p)_p \|y\|_p)_p \|force(r)\|_p)\RP\RP \\
               &\text{The variables $f$, $y$, and $r$ translate to $\LP 0,f\RP$, $\LP 0,y\RP$, and $\LP 0,r\RP$ respectively.}\\
               &\text{The expression $\T{force}(r)$ translates to $\|r\|_c +_c \|r\|$, which is just $r$.} \\
               &= \LP 0, \lambda f.\LP 0,\lambda xs.\T{rec}(xs, \T{Nil} \mapsto 1 +_c \LP 0,\T{Nil}\RP, \\
               &\quad \T{Cons} \mapsto \LP y,\LP ys,r \RP\RP.(4 + r_c) +_c ((\|\T{insert}\|_p f)_p y)_p r_p)\RP\RP
\end{flalign*}
%
We also give the translation of \T{sort} applied to two arguments.
%
\begin{flalign*}
  \|\T{sort}\ f\ xs\| &= (1 + \|\T{sort}\ f\|_c + \|xs\|_c) +_c (\|\T{sort}\ f\|_p)_p \|xs\|_p \\
                      &= (1 + (1 + \|\T{sort}\|_c + \|f\|_c) + \|xs\|_c) +_c (\|\T{sort}\|_p \|f\|_p)_p \|xs\|_p \\
                      &= (2 + \|f\|_c + \|xs\|_c) +_c (\|\T{sort}\|_p \|f\|_p)_p \|xs\|_p
\end{flalign*}
%
%
\subsection{Interpretation of \T{insert}}
%
We well use an interpretation of lists as a pair of their greatest element and
their length.
%
\begin{align*}
   \LB list \RB &= \mathbb{Z} \times \mathbb{N}^\infty \\
                     D^{list} &= \{\ast\} + \{\mathbb{Z}\} \times \mathbb{N}^\infty \\
            size_{list} (\ast) &= (-\infty,0) \\
  size_{list} ((i,(j,n))) &= (max\{i,j\},1 + n)
\end{align*}
%
We use the mutual ordering on pairs.  That is, $(s,n) \leq (s',n')$ if
$n \leq n'$ and $s < s'$ or $n < n'$ and $s \leq s'$.
%
First we interpret the \T{rec}, which drives of the cost of \T{insert}.  As in
the translation, we break the interpretation up to make it more manageable.  We
will write $map, \lambda$ and $+_c$ in the semantics, which stand for the
semantic equivalents of the syntactic \T{map}, $\lambda$ and $+_c$.  The
definitions of these semantic functions mirror the definitions of their
syntactic equivalents.

First we interpret the inner \T{rec} of $\|\T{insert}\|$.
%
\begin{flalign*}
  \LB \T{rec}&(((f\ x)_p\ y)_p, \T{True} \mapsto \LP 1, \T{Cons}\LP x, \T{Cons}\LP y,ys\RP\RP\RP, \T{False}\mapsto \LP 1 + r_c,\T{Cons}\LP y,r_p\RP\RP)\RB \xi \\
                    &= \bigvee\limits_{size(z) \leq ((f\ x)_p\ y)_p} case(z, f_{True}, f_{False}) \\
                    &\text{where} \\
  \xi &= \{f \mapsto f, x \mapsto x, y\mapsto j, ys\mapsto (j, m), r \mapsto r \} \\
  f_{True}(\ast) &= \LB \LP 1, \T{Cons}\LP x, \T{Cons}\LP y,ys\RP\RP\RP \RB\xi \\
                 &= (1, \LB \T{Cons}\LP x, \T{Cons}\LP y,ys\RP\RP\RB)\xi \\
                 &= (1, (max(x,j), 2 + m)) \\
  f_{False}(\ast) &= \LB \LP 1 + r_c,\T{Cons}\LP y,r_p\RP\RP\RB \xi \\
                  &= (1 + r_c, (max(j,\pi_0 r_p), 1 + \pi_1 r_p))
\end{flalign*}
%
Since there are only two $z$, we can simplify the big maximum to a maximum.
%
\begin{flalign*}
  \bigvee\limits_{size(z) \leq ((f\ x)_p\ y)_p} &case(z, f_{True}, f_{False}) \\
                                   &= (1, (max(x,j), 2 + m)) \vee (1 + r_c, (max(j,\pi_0 r_p), 1 + \pi_1 r_p))
\end{flalign*}
%
Using this, we proceed to interpret the outer recurrence.
%
\begin{flalign*}
  g(i, n) &= \LB\T{rec}(xs, \T{Nil} \mapsto \LP 1,\T{Cons}\LP x,\T{Nil}\RP\RP, \\
            &\quad \T{Cons}\mapsto \LP y, \LP ys,r \RP\RP. (3 + (f x)_c) \\
            &\quadthree +_c \T{rec}(((f\ x)_p\ y)_p, \T{True}\mapsto \LP 1, \T{Cons}\LP x,\T{Cons}\LP y,ys\RP\RP\RP, \\
            &\quadsix \T{False}\mapsto \LP 1 + r_c, \T{Cons}\LP y,r_p\RP\RP))\RB \xi \\
            &\text{where} \\
  \xi &= \{xs \mapsto (i, n), x \mapsto x\} \\
  g(i,n) &= \bigvee\limits_{size(z) \leq (i,n)} case(z, f_{Nil}, f_{Cons}) \\
         &\text{where}\\
  f_{Nil}(\ast) &= \LB \LP 1,\T{Cons}\LP x,\T{Nil}\RP\RP \RB \xi \\
                &= (1, (x, 1)) \\
  f_{Cons}(j,m) &= \LB (4 + (f\ x)_c) +_c \T{rec}(((f\ x)_p\ y)_p, \T{True}\mapsto \LP 1, \T{Cons}\LP x,\T{Cons}\LP y,ys\RP\RP\RP, \\
                &\quadthree \T{False}\mapsto \LP 1 + r_c, \T{Cons}\LP y,r_p\RP\RP)\RB \xi \{ y \mapsto j, ys \mapsto (j, m), r \mapsto g(j, m)\} \\
                &= (4 + (f\ x)_c) +_c \LB\T{rec}(((f\ x)_p\ y)_p, \T{True}\mapsto \LP 1, \T{Cons}\LP x,\T{Cons}\LP y,ys\RP\RP\RP, \\
                &\quadthree \T{False}\mapsto \LP 1 + r_c, \T{Cons}\LP y,r_p\RP\RP)\RB \xi \{ y \mapsto j, ys \mapsto (j, m), r \mapsto g(j, m)\} \\
                &= (4 + (f\ x)_c) +_c ((1, (max(x,j), 2 + m)) \\
                &\quadthree \vee (1 + g_c(j,m), (max(j,\pi_0 g_p(j,m)), 1 + \pi_1 g_p(j,m))))
\end{flalign*}
%
So the interpretation of $\|\T{insert}\|$ is
%
\begin{flalign*}
  \|\T{insert}\| &= (0, \lambda f.(0,\lambda x.(0,\lambda (i,n). g(i,n)))) \\
  \text{where}& \\
  g(i,n) &= \bigvee\limits_{size(z) \leq (i,n)} case(z, f_{Nil}, f_{Cons}) \\
         &\text{where}\\
  f_{Nil}(\ast) &= (1, (x, 1)) \\
  f_{Cons}(j,m) &= (4 + (f\ x)_c) +_c ((1, (max(x,j), 2 + m)) \\
                &\quadthree \vee (1 + g_c(j,m), (max(j,\pi_0 r_p), 1 + \pi_1 g_p(j,m))))
\end{flalign*}
%
To obtain a closed form solution, we will seperate the recurrence into a
recurrence for the cost and a recurrence for the potential, and solve those
independently. We use the substitution method to prove a closed form solution
to the cost of the \T{rec} construct of \T{insert}.
%
\begin{lemma}
  \label{lem:insert_rec_cost}
  $g_c(i,n) \leq (4 + ((f\ x)_p\ i)_c n + 1$
\end{lemma}
%
\begin{proof}
  We prove this by induction on $n$.
  Recall we use the mutual ordering on pairs.
  \begin{description}
    \item[case $n=0$]\hfill \\
      $g_c(i,n) = (1, (x, 1))_c = 1$
    \item[case $n>0$]\hfill \\
      \begin{flalign*}
        g_c(i,n) &= \bigvee_{size(z) \leq (i,n)} case(z, (f_{Nil}, f_{Cons})) \\
                 &= \bigvee\limits_{\substack{j < i, m \leq n\\ \text{ or } j \leq i, m < n}} case((j, m), (f_{Nil}, f_{Cons})) \\
                 &= \bigvee\limits_{\substack{j < i, m \leq n\\ \text{ or } j \leq i, m < n}} 4 + ((f\ x)_p\ j)_c + g_c(j, m-1)) \\
                 &\leq \bigvee\limits_{\substack{j < i, m \leq n\\ \text{ or } j \leq i, m < n}} 4 + ((f\ x)_p\ j)_c + (4 + ((f\ x)_p\ j)_c)(m-1) + 1\\
                 &\leq \bigvee\limits_{\substack{j < i, m \leq n\\ \text{ or } j \leq i, m < n}} (4 + ((f\ x)_p\ j)_c) (m-1 + 1) + 1 \\
                 &\leq \bigvee\limits_{\substack{j < i, m \leq n\\ \text{ or } j \leq i, m < n}} (4 + ((f\ x)_p\ j)_c) m + 1 \\
                 &\leq \bigvee\limits_{\substack{i < j, m \leq n\\ \text{ or } i \leq j, m < n}} (4 + ((f\ x)_p\ i)_c) n + 1 \\
        &\leq (5 + ((f\ x)_p\ i)_c) n
      \end{flalign*}
  \end{description}
\end{proof}
%
As expected, we find the cost of insert is bounded by the length of the list
and the largest element. Now we use the same method to obtain a closed form
solution for the potential of the \T{rec} construct of \T{insert}. The
potential of the resulting list is bounded by the list with maximum element
equal to max of the inserted element an the maximum element of the original
list and one more the length of the original list.
%
\begin{lemma}
  \label{lem:insert_rec_potential}
  $g_p(i,n) \leq (max\{x, i\}, n+1)$
\end{lemma}
%
\begin{proof}
  We prove this by induction on $n$.
  \begin{description}
    \item[case $n=0$]\hfill \\
      $g_p(i,n) = (1, (x, 1))_p = (x, 1)$.
    \item[case $n>0$]\hfill \\
      \begin{flalign*}
        g_p(i,n) &= \bigvee_{size(z) \leq (i,n)} case(z, f_{Nil}, f_{Cons}) \\
                 &= \bigvee\limits_{\substack{j < i, m \leq n \\\text{ or } j \leq i, m < n}} (max\{x, j, \pi_0g_p(j, m-1)\}, 2 + (m-1) \vee 1 + \pi_1g_p(j,m-1))\\
                 &\leq \bigvee\limits_{\substack{j < i, m \leq n\\ \text{ or } j \leq i, m < n}} (max\{x, j\}, 2 + (m-1))\\
                 &\leq \bigvee\limits_{\substack{j < i, m \leq n\\ \text{ or } j \leq i, m < n}} (max\{x,i\}, 1 + n)\qquad\text{Since $m\leq n$.}\\
        &\leq (max\{x,i\}, 1 + n)
      \end{flalign*}
  \end{description}
\end{proof}
%
Using lemmas \ref{lem:insert_rec_cost} and \ref{lem:insert_rec_potential}, we
can express the cost and potential of \T{insert} in terms of its arguments.
%
\begin{equation}
  \label{eq:insert_interp}
  insert\ f\ x\ xs \leq (5 + ((f\ x)_p\ i)_c n, (max\{x, i\}, n+1))
\end{equation}
%
%
\subsection{Interpretation of \T{sort}}
%
We use the same denotational semantics as in the intepretation of \T{insert}.
We interpret \T{list} as a tuple of the largest element in the list and the
number of \T{Cons} constructors. As in \T{insert}, the \T{rec} construct again
drives the cost and potential of \T{sort}. We interpret the \T{rec} construct,
manipulate the recurrence to a closed form, and use the result to interpret the
cost and potential of applying \T{sort} to a comparison function $f$ and a list
$xs$.
%
\begin{flalign*}
  g(i, n) &= \LB \T{rec}(xs, \T{Nil} \mapsto \LP 1,\T{Nil} \RP,  \\
          &\quadthree \T{Cons} \mapsto \LP y, \LP ys,r \RP\RP.(4 + r _c) +_c ((\|\T{insert}\|_p f)_p y)_p r_p) \RB \{ xs \mapsto (i, n), f\mapsto f\} \\
          &= \bigvee\limits_{size(z)\leq n} case(z, f_{Nil}, f_{Cons}) \\
  %
          &\text{where}\\
  f_{Nil} &= \LB \LP 1,\T{Nil}\RP \RB \{ xs \mapsto (i,n)\} \\
          &= (1, (-\infty,0)) \\
f_{Cons} &= \LB (4 + r _c) +_c ((\|\T{insert}\|_p f)_p y)_p r_p) \RB \xi \\
         &\text{where}\quad \xi = \{xs \mapsto (i,n), f \mapsto f, y \mapsto j, ys \mapsto (j,m), r \mapsto g(j,m)\} \\
         &= (4 + g_c(j,m)) +_c ((\LB\|\T{insert}\|_p\RB\xi\ f)_p j)_p g_p(j,m) \\
         &= (5 + g_c(j,m)) +_c (((f\ j)_p\ j)_c g_p(j,m), (max\{j,j\},g_p(j, m)+1)) \\
         &= (5 + g_c(j,m) + (f\ j)_p\ j)_c g_p(j,m), (max\{j,j\},g_p(j,m) + 1))
\end{flalign*}
%
Observe that in equation \ref{eq:sort_interp0_init}, the cost is depends on the
potential of the recursive call. Therefore we must solve the recurrence for
the potential first.
%
\begin{lemma}
  \label{lem:sort_interp_potential}
  $g_p(i, n) \leq (i, n)$
\end{lemma}
\begin{proof}
  We prove this by induction on $n$.
  We use equation \ref{eq:insert_interp} to determine the potential of the $insert$ function.
  \begin{description}
    \item[case $n=0$]\hfill \\
      $g_p(i,n) = (i, 0)$
    \item[case $n>0$]\hfill \\
      \begin{flalign*}
        g_p(i,n) &= (\bigvee_{size(z)\leq n} case(z,f_{Nil}, f_{Cons}))_p \\
                 &= \bigvee\limits_{\substack{j \leq i, m < n\\ \text{ or } j < i, m \leq n}} (case(z,f_{Nil},f_{Cons}))_p \\
                 &= \bigvee\limits_{\substack{j \leq i, m < n\\ \text{ or } j < i, m \leq n}} (case(z,f_{Nil},f_{Cons}))_p \vee \bigvee\limits_{\substack{j \leq i, m < n \\ \text{ or } j < i, m \leq n}} (case(z,f_{Nil},f_{Cons}))_p \\
                 &\leq (i, n) % TODO
      \end{flalign*}
  \end{description}
\end{proof}
%
As in the interpretation of \T{insert} we are left with a less than
satisfactory bound on the potential of \T{sort}. It would grievous mistake to
write a sorting function whose output was smaller than its input.  Under the
current interpretation of lists, this would mean either the length of the list
decreased or the size of the largest element in the list decreased.
Unfortunately we are stuck with an upper bound on the size of the output
because the interpretation of $case$ is a maximum over a set of smaller terms.


Using the solution to the recurrence for the potential, we can solve the
recurrence for the cost of \T{sort}.
%
\begin{lemma}
  \label{lem:sort_interp_cost}
  $g_c(i, n) \leq (3 + ((f\ i)_p\ i)_c n^2 + 5n + 1$
\end{lemma}
%
\begin{proof}
  We prove this by induction on $n$.
  \begin{description}
    \item[case $n=0$] $g_c(i,n) = 1$
    \item[case $n>0$] \hfill \\
      \begin{flalign*}
        g_c(i,n) &= (\bigvee_{size(z) \leq (i,n)} case(z, f_{Nil}, f_{Cons}))_c \\
                 &= \bigvee\limits_{\substack{j < i, m \leq n\\ \text{ or } j \leq i, m < n}} 3 + g_c(j, m - 1) + (insert\ f\ j\ \pi_1g(j, m - 1))_c\\
        &\leq \bigvee\limits_{\substack{j < i, m \leq n\\ \text{ or } j \leq i, m < n}} 3 + g_c(j, m - 1) + (insert\ f\ j\ (j, m - 1))_c\\
        &\leq \bigvee\limits_{\substack{j < i, m \leq n\\ \text{ or } j \leq i, m < n}} 3 + g_c(j, m - 1) + (3 + ((f\ j)\ j))(m - 1) + 1\\
        & \text{let $c_1 = (3 + \pi_0(\pi_1(f\ j)\ j))$}\\
        &\leq \bigvee\limits_{\substack{j < i, m \leq n\\ \text{ or } j \leq i, m < n}} 3 + c_1(m-1)^2 + 5(m-1) + 1 + c_1(m - 1) + 1\\
        &\leq \bigvee\limits_{\substack{j < i, m \leq n\\ \text{ or } j \leq i, m < n}} 3 + c_1m^2 - 2c_1m +c_1 + 5m-5 + 1 + c_1m - c_1 + 1\\
        &\leq \bigvee\limits_{\substack{j < i, m \leq n\\ \text{ or } j \leq i, m < n}} c_1m^2 - c_1m + 5m + 1\\
        &\leq \bigvee\limits_{\substack{j < i, m \leq n\\ \text{ or } j \leq i, m < n}} (3 + \pi_0(\pi_1(f\ i)\ i))n^2 + 5n + 1\\
        &\leq (3 + ((f\ i)\ i)_p)_cn^2 + 5n + 1
      \end{flalign*}
  \end{description}
\end{proof}
%
As expected the cost of \T{sort} is $\mathcal{O}(n^2)$ where $n$ is the length
of the list.  It is clear from the analysis how the cost of the comparison
function determines the running time of \T{sort}.  We can see that the
comparison function is called order $n^2$ times.
