\section{Sequential List Map}
\label{sec:sequential_list_map}

This example is provided for comparison with the parallel list map example
given later in this thesis. We use the familiar \T{list} datatype.
%
\begin{align*}
\T{list } &= \T{Nil of unit | Cons of int $\times$ list}
\end{align*}
%
The \T{map} function recurses on the list, applying its first argument to each
element.
%
\begin{equation*}
  \T{map} = \lambda f. \lambda xs . \T{rec}(xs, \T{Nil} \mapsto \T{Nil}, \T{Cons} \mapsto \LP y \LP ys, y \RP\RP. \T{Cons}\LP f\ y, \T{force}(r)\RP)
\end{equation*}

\subsection{Translation}
The translation
\begin{align*}
  &\text{We apply the rule for translating an abstraction twice.}\\
  \|\T{map}\| &= \|\lambda f. \lambda xs . \T{rec}(xs, \T{Nil} \mapsto \T{Nil}, \T{Cons} \mapsto \LP y \LP ys, y \RP\RP. \T{Cons}\LP f\ y, \T{force}(r)\RP)\| \\
              &= \LP 0, \lambda f. \LP 0, \lambda xs . \T{rec}(xs, \T{Nil} \mapsto \T{Nil}, \T{Cons} \mapsto \LP y \LP ys, y \RP\RP. \T{Cons}\LP f\ y, \T{force}(r)\RP)\RP\RP \\
              &\text{We apply the rule for translating a \T{rec} construct. We use $\|xs\| = \LP 0,xs\RP$.}\\
              &= \LP 0, \lambda f. \LP 0, \lambda xs .\T{rec}(xs, \T{Nil} \mapsto 1 +_c \|\T{Nil}\|, \\
              &\quadten \T{Cons} \mapsto \LP y \LP ys, y \RP\RP.1 +_c \|\T{Cons}\LP f\ y, \T{force}(r)\RP\|)\RP\RP \\
  %
              &\text{The translation of \T{Nil} is $\LP 0, \T{Nil}\RP$.}\\
  %
              &\text{To translate the \T{Cons} branch we translate the subexpressions first.}\\
              &\quadthree\|\T{Cons}\LP f\ x,\T{force}(r)\RP\| = \LP \|\LP f\ x,\T{force}(r)\RP_c, \T{Cons}\|\LP f\ x,\T{force}(r)\RP\|\RP \\
              &\quadfive \|\LP f\ x,\T{force}(r)\RP\| = \LP\|f\ x\|_c + \|\T{force}(r)\|_c, \LP \|f\ x\|_p,\|\T{force}(r)\|_p\RP\RP \\
  %
              &\quadsix \|f\ x\| = (1 + \|f\|_c + \|x\|_c) +_c \|f\|_p\|x\|_p = \LP 1 + (f\ x)_c, (f\ x)_p\RP \\
  %
              &\quadsix \|\T{force}(r)\| = \LP 0,r\RP +_c \LP 0,r\RP_p = r \\
  %
              &\quadfive \|\LP f\ x,\T{force}(r)\RP\| = \LP 1 + (f\ x)_c + r_c, \LP (f\ x)_p,r_p\RP\RP \\
              &\quadthree \|\T{Cons}\LP f\ x,\T{force}(r)\RP\| = \LP 1 + (f\ x)_c + r_c, \T{Cons}\LP (f\ x)_p,r_p\RP\RP \\
              &= \LP 0, \lambda f. \LP 0, \lambda xs . \T{rec}(xs, \T{Nil} \mapsto \LP 1,\T{Nil}\RP, \\
              &\quadten \T{Cons} \mapsto \LP y \LP ys, y \RP\RP.\LP 2 + (f\ x)_c + r_c, \T{Cons}\LP (f\ x)_p,r_p\RP\RP \\
\end{align*}
%
We will translate \T{map} applied to some function \T{f} and a list \T{xs}.
%
\begin{align*}
  \|\T{map f xs}\| &= (1 + \|\T{map}\ f\|_c + \|xs\|_c) +_c \|\T{map}\ f\|_p \|xs\|_p \\
                   &\text{The translation of \T{map} partially applied to map is:}\\
                   &\quadthree \|\T{map}\ f\| = (1 + \|\T{map}\|_c + \|f\|_c) +_c \|\T{map}\|_p \|f\|_p \\
                   &\text{The cost of partially applied \T{map} is 0.} \\
                   &= (2 + \|f\|_c + \|xs\|_c) +_c (\|\T{map}\|_p\ \|f\|_p)_p \|xs\|_p
\end{align*}

\subsection{Interpretation}

We interpret lists as a pair of their largest element and length.
%
\begin{align*}
  \llbracket \T{list} \rrbracket &= \mathbb{Z} \times \mathbb{N} \\
  D^\T{list} &= \{*\} + (\llbracket \mathbb{Z} \rrbracket \times \llbracket \T{list} \rrbracket) \\
  size_\T{list}(*) &= (0, 0) \\
  size_\T{list}((x, (m, n))) &= (max(x, m), 1 + n)
\end{align*}
%
The recursor of \T{map} drives the cost, so we will interpret the recursor
first.
%
\begin{align*}
  g(i, n) &= \llbracket \T{rec}(xs, \T{Nil} \mapsto \LP 1,\T{Nil}\RP, \\
       &\quadfive \T{Cons} \mapsto \LP 2 + (f\ x)_c + r_c,\T{Cons}\LP (f\ x)_p,r_p\RP\RP \rrbracket \{xs \mapsto n, f \mapsto f\}\\
       &= \bigvee\limits_{size(z) \leq n} case(z, f_{Nil}, f_{Cons}) \\
       &\text{where} \\
  f_{Nil}(\ast) &= \llbracket \LP 1,\T{Nil}\RP\rrbracket \\
                &= (1,0) \\
  f_{Cons}(i, m) &= \llbracket \LP 2 + (f\ x)_c + r_c,\T{Cons}\LP (f\ x)_p,r_p\RP\RP\rrbracket \xi\\
                 &\qquad \text{where } \xi = \{xs \mapsto n, f \mapsto f, y \mapsto i, ys \mapsto m, r \mapsto g(i, m) \} \\
                 &= (2 + (f\ i)_c + g_c(i,m), (max((f\ i)_p, \pi_0g_p(i,m)), 1 + \pi_1 g_p(i,m)))
\end{align*}
%
We break the recurrence into the cases of $n=0$ and $n>0$
%
\begin{description}
  \item[case $n=0$]\hfill \\
    $g(i,0) = \bigvee\limits_{size(z) \leq 0} case(z, f_{Nil}, f_{Cons}) = (1,0)$
  \item[case $n>0$]\hfill \\
    \begin{align*}
      g(i,n) &= \bigvee\limits_{size(z) \leq n} case(z, f_{Nil}, f_{Cons}) \\
             &= \bigvee\limits_{size(z) \leq n-1} case(z,f_{Nil},f_{Cons}) \vee \bigvee\limits_{size(z) = n} case(z,f_{Nil}, f_{Cons}) \\
             &= g(i,n-1) \vee \\
             &\qquad (2 + (f\ i)_c + g_c(i,n-1), (max((f\ i)_p,\pi_0g_p(i,n-1)), 1 + \pi_1 g_p(i,n-1))) \\
             &\text{Since $g_c(i,n-1) \leq g_c(i,n-1)$, then $g_c(i,n-1)\leq 2 + (f\ i)_c + g_c(i,n-1)$.} \\
             &\text{Since $g_p(i,n-1) \leq g_p(i,n-1)$, then $\pi_0 g_p(i,n-1) \leq max((f\ i)_p, \pi_0g_p(i,n-1))$,} \\
             &\text{and $\pi_1 g_p(i,n-1) \leq 1 + \pi_1 g_p(i,n-1)$.} \\
             &= (2 + (f\ i)_c + g_c(i,n-1), (max((f\ i)_p,\pi_0g_p(i,n-1)), 1 + \pi_1 g_p(i,n-1))) \\
    \end{align*}
\end{description}
%
We obtain a closed form solution for the cost by using the substitution method.
%
\begin{lemma}
  \label{lem:listmap_g_cost}
  $g_c(i,n) = (2 + (f\ i)_c) n + 1$
\end{lemma}
%
\begin{proof}
  The proof is by induction on $n$.
  \begin{description}
    \item[case $n=0$]\hfill \\
      $g_c(i,0) = (1,0)_c = 1$
    \item[case $n>0$]\hfill \\
      \begin{align*}
        g_c(i,n) &= 2 + (f\ i)_c + g_c(i,n-1)\\
                 &= 2 + (f\ i)_c + (2 + (f\ i)_c)(n-1) + 1\\
                 &= 2 + (f\ i)_c + (2 + (f\ i)_c) n - 2 - (f\ i)_c + 1\\
                 &= (2 + (f\ i)_c) n + 1
      \end{align*}
    \end{description}
\end{proof}
%
We obtain a closed form of the solution for the potential of the recursor.
%
\begin{lemma}
  \label{lem:listmap_g_potential}
  $g_p(i,n) = ((f\ i)_p, n)$
\end{lemma}
%
\begin{proof}
  The proof is by induction on $n$.
  \begin{description}
    \item[case $n=0$]\hfill \\
      $g_p(i,0) = (1,0)_p = 0$
    \item[case $n>0$]\hfill \\
      \begin{align*}
        g_p(i,n) &= (max((f\ i)_p,\pi_0g_p(i,n-1)), 1 + \pi_1 g_p(i,n-1)) \\
                 &= (max((f\ i)_p, (f\ i)_p), 1 + n - 1) \\
                 &= ((f\ i)_p, n)
      \end{align*}
  \end{description}
\end{proof}
%
Using the results from lemmas \ref{lem:listmap_g_cost} and
\ref{lem:listmap_g_potential}, we interpret the translation of \T{map f xs}.
Recall the translation is $(2 + \|f\|_c + \|xs\|_c) +_c (\|\T{map}\|_p\ \|f\|_p)_p \|xs\|_p$.
So the interpretation is $2 \pplus_c (\llbracket\|\T{map}\|\rrbracket f)_p (i,n)$,
where $(i,n)$ is the interpretation of $xs$ and we assume $f$ and $xs$ have $0$
cost.
%
\begin{lemma}
  \label{lem:listmap_cost}
  $\llbracket\|\T{map f xs}\|_c\rrbracket = 3 + (2 + (f\ i)_c)n$ where $f$ is the
  interpretation of the translation of \T{f} and $(i,n)$ is the interpretation
  of the translation of \T{xs}.
\end{lemma}
%
\begin{lemma}
  \label{lem:listmap_potential}
  $\llbracket\|\T{map f xs}\|_p\rrbracket = ((f\ i)_p,n)$ where $f$ is the
  interpretation of the translation of \T{f} and $(i,n)$ is the interpretation
  of the translation of \T{xs}.
\end{lemma}
%
Lemma \ref{lem:listmap_cost} shows that the cost of \T{map f xs} is linear in
the size of the list but also depends on the cost of applying \T{f} to the
elements of \T{xs}. Lemma \ref{lem:listmap_potential} shows the cost of future
uses of \T{map f xs} depends on the length of \T{xs} and the size of the result
of applying \T{f} to the elements of \T{xs}.
