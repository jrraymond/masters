\chapter{Conclusions and Future Work}

\paragraph{}
We have demonstrated the method presented in \citet{Danner2015} can be used to
analyze the complexity of higher-order functional programs. Given a source
language program, we can mechanically apply the rules of the translation
function to obtain a recurrence in for the cost and size of the program in the
complexity language. We can interpret the complexity language recurrence in a
denotational semantics by mechanically applying the rules of the interpretation
function. At this point, a certain amount of cleverness is needed to work the
recurrence into a recognizable form. Once the big maximum is eliminated, a
closed form solution for the recurrence can be obtained using the substitution
method.

\paragraph{}
A drawback of the translation function is the process of applying the
translation to a source language program by hand is tedious and error prone. An
area of future work is to automate this translation. This should not be too
difficult since the translation function is application of a translation rule
to each node in the abstract syntax tree and then recursively translating
subexpression. Similarly we should be able to automate the interpretation
function. The automated interpretation function would be paramaterized by the
interpretations of programmer-defined datatypes.

\paragraph{}
The last step of the process, obtaining closed form solutions to the extracted
recurrences, would be the most difficult to automate. The PURRS project is
working at automatically solving recurrences. The project
is able to solve or approximate many forms of occurences, but not all. For
example, divide and conquer algorithms often produce recurrences of the form
$T(n) = a T(\frac{n}{b}) + S(n)$, called generalized recurrences. The PURRS
project can solve some but not all of these recurrences (\citet{Bagnara2003}).
\citet{Albert2011} automatically obtain closed form upper bounds for cost
recurrences for a program, a system of recurrences that describe the cost a
program with respect to its input, but the system is not complete and the
bounds are not asymptotic. \citet{Albert2013} provides an implementation on
Java bytecode programs to infer upper bounds on the resource usage.

\paragraph{}
The cost semantics can be adapted to different cost models. We changed the cost
semantics from sequential to parallel and proved the complexity language
translation of a program is still an upper bound on the evaluation cost of the
source language program. This demonstrates the flexibility of this framework
with respect to cost models.

\paragraph{}
We also defined a pure potential translation that is a stripped down version of
the complexity translation without the cost. We proved by logical relations the
potential translation is equivalent to the complexity language. Often when
analyzing the recurrences obtained in the complexity language, we break the
complexity recurrence into a pair of recurrences, one for the cost and one for
the potential. This proof demonstrates we well always be able to extract the
potential recurrence from the complexity recurrence because the potential
recurrence does not depend on the cost recurrence. The converse is not always
true. The cost recurrence does sometimes depend on the potential recurrence.
\T{fold} is an example of this.
%
\begin{lstlisting}
fold = $\lambda$f z xs.rec(xs, Nil $\mapsto$ z, Cons $\mapsto \LP x,\LP xs', r \RP\RP$. f x force(r))
\end{lstlisting}
%
The cost of applying a function \T{f} at each
step of the fold depends on the size of the head of the list and the size of
the result of the recursive call to \T{fold}. So in order to analyze the cost
of \T{fold} we must first solve the recurrence for the potential of \T{fold}.
Our pure potential translation and proof demonstrates we will never have to
solve the cost recurrence in order to solve the potential recurrence.


One of the criticisms leveed against traditional complexity analysis is there
is no formal connection between the source language program and the recurrence
for its cost. Although the method from \citet{Danner2015} does provide the
formal connection, the translation from source language to complexity language
and subsequent interpretation is tedious and prone to errors. Automation of the
source to complexity translation and interpretation could easily be automated.
Both transformations are simply forms of cross compilation between different
languages. This leaves only the process of obtaining closed form solutions of
recurrences initially obtained from the interpretation. Solving the recurrences
requires a significant amount of cleverness. For simpler recurrences such as
\T{list map}, the recurrences are straightforward to solve. However for more
complex functions, the recurrences may be very difficult to solve. Consider a
function that calculates the $n$th term of the Fibonacci sequence. Under an
interpretation of natural numbers as the number of successor constructors, the
potential of this function is the $n$th Fibonacci number. The closed form
solution to the potential recurrence is
\[
  \frac{(1 + \sqrt{5})^n + (1 - \sqrt{5})^n}{2^n\sqrt{5}}
\]
Obtaining this solution is not at all obvious.
