\documentclass[12pt,letterpaper]{article}
\usepackage[utf8]{inputenc}
\usepackage[english]{babel}
\usepackage{amsmath}
\usepackage{amsfonts}
\usepackage{amssymb}
\usepackage{amsthm}
\usepackage{float}
\floatstyle{boxed} 
\restylefloat{figure}
\usepackage{framed}
\usepackage[letterpaper]{geometry}
\usepackage{graphicx}
\usepackage{listings}
\usepackage{natbib}
\usepackage[parfill]{parskip}
\usepackage[section]{placeins}
\usepackage{upgreek}
\usepackage{stmaryrd}

\lstset{language=[Objective]Caml}
\newcommand{\T}[1]{\texttt{#1}}


\begin{document}
\begin{flushright}
Potential Translation\\
Justin Raymond\\
\today
\end{flushright}

\section{Introduction}

\section{Potential Translation}

\section{Proof}
\subsection{Logical Relation}
We define the relation between terms in the pure potential language and in the complexity language as follows.
\begin{enumerate}
  \item $e \sim_\tau e'$
    \begin{itemize}
      \item $e \sim_{unit} e'$ is always true.
      \item $\|\langle e_0, e_1\rangle\| \sim_{\tau_0 \times \tau_1} |\langle e_0, e_1 \rangle|$ if and only if $\|e_0\| \sim_{\tau_0} |e_0|$ and $\|e_1\| \sim_{\tau_1} |e_1|$.
      \item $\|\langle e_0, e_1\rangle\| \sim_{\tau_0 \times \tau_1} |\langle e_0, e_1 \rangle|$ if and only if $\|e_0\| \sim_{\tau_0} |e_0|$ and $\|e_1\| \sim_{\tau_1} |e_1|$.
    \end{itemize}
\end{enumerate}



\end{document}
