\chapter{The Tor Network}
\section*{Introduction}
When a message is sent over the Internet, it is transmitted from one computer to another via TCP/IP. In TCP/IP, a message is sent piecemeal as a series of packets, each of which consists of a header containing routing information and a payload containing a fixed size portion of the message. If a user Alice wants to communicate privately with a website bob.com, she may encrypt the payload of these packets to conceal their content from eavesdroppers. While doing so conceals the content of Alice's message, it does not conceal the fact that Alice is communicating with bob.com. By viewing the routing information in Alice's packets, an eavesdropper Eve can easily see that Alice is visiting bob.com. If Alice is a political activist, Eve a totalitarian regime, and bob.com a revolutionary website, this knowledge alone may be enough to have Alice thrown in jail.

To be fully protected, Alice must conceal not just her message to bob.com, but the fact that she is communicating with bob.com at all. For this reason, many \textit{anonymity systems} have been developed which offer this type of protection. Anonymity system typically fall into one of two categories: \textit{low latency system} and high latency systems. High latency systems like Babel, Mixmaster, and Mixminion trade off latency for protection against \textit{correlation attacks}, a class of attack that exploits Alices's packet timing patterns to statistically infer her connection to bob.com. These systems use techniques such as delaying packets and sending messages out of order in order to distort Alice's original timing patterns beyond recognition. The transmission delay introduced by these techniques makes high latency systems suitable only for non-interactive activities like sending e-mail. Low latency systems generally do not attempt to modify Alice's timing patterns. While this leaves such systems vulnerable to correlation attacks, it also offers low enough latency to support interactive activities like web browsing, instant messaging, and video streaming.

Tor is a popular low latency system with an estimated 250,000 daily users \citep{exptorwpaper}. It achieves anonymity by wrapping Alice's entire packet, including the header, in at least three layers of encryption. The packet is then sent through at least three volunteer operated \textit{onion routers} (OR), each of which ``peels off'' a layer of encryption. The last onion router recovers Alice's original packet and forwards it to bob.com. Since Eve cannot decrypt the header of Alice's original packet, she can no longer associate Alice with bob.com.

\section{Onion Encryption}
To route web traffic through the Tor network, the user connects to an \textit{onion proxy} (OP) via SOCKS, a session layer protocol for communicating with proxy servers. The onion proxy is generally run by the user, and it is responsible for connecting to the network, building circuits, and transparently sending and receiving Alice's web traffic.

At startup, the OP first downloads the \textit{consensus document}, a list of all known ORs and their health statistics, from one of Tor's \textit{directory servers}, a set of trusted servers tasked with monitoring the global network. With this list, the OP may then choose a set of routers and build a circuit. All Tor circuits consist of at least three ORs: a \textit{guard router}, one or more intermediate ORs, and an \textit{exit router}. A guard router knows Alice's IP address by virtue of being the first OR in the circuit. As the last OR in the circuit, an exit router sees Alice's unencrypted packets. To onion encrypt the packets, the OP must negotiate symmetric \textit{onion keys} with every router in the circuit. Let $R_1, R_2, \dots R_n$ be these routers, and $k_1, k_2, \dots k_n$ be the corresponding keys. Alice's packet $p$ is encrypted at the OP as $(E_{k_1} \circ E_{k_2} \circ \dots E_{k_n})(p)$. As it moves through the circuit, $R_i$ applies $D_{k_i}$, removing a layer of encryption. When the cell leaves the exit router, all $n$ layers have been removed. A packet $p'$ sent by bob.com to Alice is encrypted in the reverse manner: $R_n$ encrypts with $k_n$, $R_{n-1}$ with $k_{n-1}$, and all routers through $R_1$ until the OP receives $(E_{k_1} \circ E_{k_2} \circ \dots E_{k_n})(p')$. The OP then removes all $n$ layers of encryption and deliver the packet to Alice.

\section{Tor Cells}
All of the OP's communication with the network is done via an application level protocol specified in \textit{tor-spec.txt} \citep{torspec}. The building blocks of this protocol are \textit{Tor cells}, 514 byte messages used to send data and commands to OPs and ORs. The first 3 bytes of a cell contain a circuit id, the 4th byte a command, and the remaining 510 a payload. The circuit id field is needed because Tor can multiplex circuits over one connection; that is, one OP can build multiple circuits through the same OR. The command field contains one of 11 possible Tor commands. Explanations for the four most relevant to our research are given in Figure ~\ref{fig:torcommands}.

\begin{figure}
	\begin{tabular}{| l | p{12cm} |}
		\hline
		\textit{Command} & \textit{Purpose} \\ \hline
		CREATE 	& Open a new circuit. Payload contains onion key challenge. \\	 \hline
		CREATED & Acknowledge successful creation of a circuit. Payload contains  onion key response. \\ \hline
		RELAY 	& Send a relay command and a relay payload to a OR. \\ \hline
		DESTROY & Destroy a circuit. Payload contains reason for destroying.\\
		\hline
	\end{tabular}
	\caption{Four Tor cell commands and their respective purposes.}
	\label{fig:torcommands}
\end{figure}

The payload of a RELAY cell contains additional header fields, a relay subcommand, and a data payload. The data payload contains Alice's onion encrypted TCP cell. Figure ~\ref{fig:relaypayload} shows these fields. The ``Recognized'' and digest fields are used by an OR to determine if the payload is intact and fully decrypted. ``Recognized'' is set to 0 if the payload is fully decrypted. Digest is set to the first four bytes of the running digest of all data either bound for the OR or originating at the OR. It is used to ensure the integrity of the encrypted payload. If the ``Recognized'' field is 0 and the digest has the correct value, then the OP will forward Alice's packet to bob.com. The Length field denotes the length in bytes of the data payload. Payloads shorter than 503 bytes are NUL-padded. The subcommand field may contain one of 15 possible subcommands. The five most relevant ones are listed in Figure ~\ref{fig:relayformat}.

\begin{figure}
	\begin{tabular}{ | c | c | c | c | c | c |}
		\hline
		Relay command & `Recognized' & StreamID & Digest & Length & Data \\
		\hline
		1 byte & 2 bytes & 2 bytes & 4 bytes & 2 bytes & 503 bytes \\
		\hline
	\end{tabular}
\caption{Format of a RELAY cell payload.}
\label{fig:relayformat}
\end{figure}

\begin{figure}
	\begin{tabular}{| l | p{9cm} |}
		\hline
		\textit{Command} & \textit{Purpose} \\ \hline
		RELAY\_BEGIN & Open a new anonymized TCP connection. Payload contains destination address and port. \\ \hline
		RELAY\_DATA & Send data. Payload contains onion encrypted TCP packet. \\ \hline
		RELAY\_CONNECTED & Acknowledgement for RELAY\_BEGIN. Payload contains info about the TCP connection. \\ \hline
		RELAY\_EXTEND & Extend the circuit by adding an OR. Payload unspecified. \\ \hline
		RELAY\_EXTENDED & Acknowledgment for RELAY\_EXTEND. Payload unspecified. \\
		\hline
	\end{tabular}
	\caption{Five RELAY commands and their respective purposes.}
	\label{fig:relaycommands}
\end{figure}

\section{Circuit Creation}
Figure ~\ref{fig:circcreate} demonstrates the process of Alice creating a 3 hop circuit, then anonymously communicating with \texttt{bob.com}. To initiate the circuit, Alice sends a CREATE cell to $R_1$, and $R_1$ acknowledges with CREATED. Alice next sends a relay cell with the RELAY\_EXTEND command, prompting $R_1$ to build the next step of the circuit with $R_2$. Once Alice is notified that her RELAY\_EXTEND was successful with a RELAY\_EXTENDED, she sends a second RELAY\_EXTEND TO $R_3$. When this is acknowledged, Alice sends a RELAY\_BEGIN command to open a TCP connection between $R_3$ and \texttt{bob.com}. With this connection established, Alice may now anonymously send and receive packets through the circuit via her SOCKS connection to the OP.
\begin{figure}
	\centerline{\includegraphics[scale=.7]{figures/circ-create.png}}
	\caption{Circuit creation workflow for a 3-hop circuit.}
	\label{fig:circcreate}
\end{figure}

\section{Contributions of this Thesis}
Tor is widely researched in the academic community. Areas of study include devising new attacks on the network, building defenses against these attacks, and creating new traffic routing algorithms to improve latency and bandwidth. For both ethical and practical reasons, it is often necessary to do this research on a simulation of the extant network. For researching attacks, it is unethical to attempt attacks on real Tor users. For researching routing algorithms, it is not possible to witness the whole--network effects of a change without deploying experimental code to every router on the network, which is both undesirable and likely impossible in practice.

ExperimenTor \citep{exptorwpaper} and Shadow \citep{shadowwpaper} are two  freely available Tor simulation platforms that have been used in academic research. Both of these platforms allows a researcher to run many Tor relays and clients simultaneously within a simulated Internet. They each offer sophisticated models of network factors like packet latency, traffic congestion, and bandwidth limitations. Both also make a thoughtful attempt at ``scaling down'' the global distribution of Tor routers and clients to a computationally manageable size. A weak point of both platforms, however, is their models (or lack thereof) of the traffic generated by network clients. Shadow provides basic, deterministic simulations of World Wide Web traffic, while ExperimenTor puts the onus fully on the researcher.

We believe that the lack of a strong client traffic model is a significant flaw in both of these platforms. Without realistic network traffic, OPs and ORs in the simulated network are not operating under the same conditions that they would be on the extant network. This fact compromises the validity of any conclusions drawn from research done in the simulation. To realize the full benefits of a simulated environment, a better model is needed.

In this thesis, we focus on the problem of modeling the outbound traffic sent from an OP into the network. Using a guard router that we control, we
recorded roughly one day's worth of data representing the timing patterns of client traffic on the extant network. Each time our router received a RELAY cell, we recorded a timestamp, the cells's circuit ID, and an anonymized version of the associated client's IP address. Using this raw data, we constructed a set of time series in which each observation is the number of cells received in a 5 second time window. Our goal is to create a probabilistic model of these time series.

Based on our intuitions about Internet client behavior, we believe that a Markov Model is an appropriate model of these time series. A Markov Model is a finite state machine that emits a symbol each time it visits a state. Transitions between states occur probabilistically, and the symbol emitted can be drawn from an arbitrary set; in our case, the natural numbers. By using a Markov Model, we assume that a Tor client moves between a finite number of states, and that the volume of traffic her OP sends out depends upon her current state.




