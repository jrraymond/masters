\chapter{Analysis of Training Data}
\section{Data Collection}
The training dataset for our model was collected from Feb 21, 2013 at 1:03 am through Feb 22 at 11:49 am, for a total of 22 hours, 46 minutes. To collect the data, we modified of v0.2.35 Tor's OR software to include logging statements for CREATE, RELAY, and DESTROY cells. For each cell, the associated circuit's IP address and circuit id were included in the log statements. IP addresses were anonymized in real time by streaming log output through a Python script maintaining a dictionary mapping real IP addresses to unique, anonymous slugs.

In the collection period, our guard router receieved 104,181 CREATE cells. Of these cells, 18,424 (17.68\%) resulted in \textit{complete} and \textit{valid}  circuits. We consider a circuit to be complete if it contains at least one CREATE, one DESTROY, three inbound and three outbound RELAY cells; at least three RELAY cells in each direction are needed to build a three hop circuit. Furthermore, we consider a circuit to be valid if all of its RELAY cells are logged after its last CREATE cell and before its first DESTROY cell. Since a circuit begins with CREATE and ends with DESTROY, the meaning of RELAY cells outside of this window is undefined. 48,371,742 relay cells were received in total. 95\% of these came from the top 2 most active clients. These same two clients only accounted for 18\% of CREATES, however. This indicates that the majority of the traffic received was concentrated in a relatively small number of circuits. A likely explanation of this behavior is that the two clients in question were uploading one or more very large files (eg. a movie) to other network users.

For each client, a representative time series was constructed by grouping RELAY cells into 5 second, non-overlapping windows. For each 5 second time point, the corresponding observation is the number of RELAY cells received in that window.

\section{Distribution Analysis}
 To examine observation distributions over the whole data set, we plot histograms of the mean, median, minimum, maximum, and standard deviation of each series' observations. We also include a histogram of circuit lengths in seconds, and a histogram of instantaneous (i.e. single window) cell counts. All of these graphs are displayed in Figure ~\ref{fig:datahistograms}. Due to large exponential drop-offs, a log scale is used on the y-axis for every chart. All graphs were produced in Python using Matplotlib \citep{hunter2007}.

\begin{figure}
	\centerline{\includegraphics[scale=0.4]{figures/viz-demos/summary-histograms.png}}
	\caption{Histograms for (left to right, top to bottom) mean cells/window, median cells/window minimum cells/window, maximum cells/window, standard deviation of cells/window, single window cell count, and circuit length.}
	\label{fig:datahistograms}
\end{figure}

 In the means cells/window histogram, the tallest bar is the leftmost one representing means from 0-10 cells/window. This tell us that the majority of clients most of their time sending comparatively little data. The thick tail on the right, however, confirms that there are also many more active clients as well. The two outliers to the right may potentially be attributed to bulk uploaders. Since the y-axis is already on a log scale, the graph's exponential shape indicates a doubly exponential dropoff in the mean cells/window distribution.

 Median cells/window is very similar to mean cells/window. Since the mean and median are both measure of central tendency, this similarity is evidence that both graphs provide an accurate depiction of typical client activity levels. If these graphs were shaped very differently, we would not be able to draw conclusions from either one without further investigation.

The minimum cells/window chart indicates that, even for a series with a very high mean, there are almost always sections with little to no activity. The vast majority of minimums are less than 4 cells/window, suggesting we should expect very low ``valleys'' and flat sections in our time plots.

Maximum cells/window has a similar shape to the mean and median, but it ranges over much higher values. This tells us that the peak data rate of a client may lie well above its average. Due to this, we expect large, distinct spikes on our time plots.

The standard deviation of cells/window show that most of our series have observations well beyond their means. This is consistent with the small minimums and large maximums we have seen. Like the other graphs, it exhibits a doubly exponential dropoff. The highest bar covers the range $\sigma = [0, 5]$. We suspect that this bar represents the same majority of inactive clients that the previous four histograms suggest in their leftmost bar.

Instantaneous cell count yields less of a doubly exponential trend than the previous five graphs. It shows that even at the single--window level, lower activity levels are still the most commonly observed.

The circuit length histogram shows that the bulk of our circuits are less than 33:20 long, but some outliers stay open as long as 18 hours. Exceptionally long circuits can occur if the OP process is left running for an extended period. In this case, the OP leaves previously created circuits open for later reuse indefinitely.

\section{Temporal Visualizations}
To view how our series change over time, we use two different visualizations: a horizon graph, and a color graph. A horizon graph displays multiple time series simultaneously by layering them on top of one another in a nested manner \citep{heer2009}. The darkest portions of a horizon graph represent points in time at which multiple series are observing the same minimum value.

Figure ~\ref{fig:horizon} shows a horizon graph for 1000 randomly selected time series. Its high, jagged peaks are consistent with the high maximums and standard deviations shown in our histograms. The dark peak from near $t=0$ tells us that many of the series in this sample exhibit a peak of $\geq$ 500 cells/second at their beginning. As time progresses forward, the color of the graph becomes progressively lighter. This behavior is consistent with that of the circuit length histogram. The further forward in time the we are, the fewer circuits there are that are still open.

\begin{figure}
	\centerline{\includegraphics[scale=0.4]{figures/viz-demos/horizon-demo.png}}
	\caption{Horizon graph of 1000 randomly selected time series.}
	\label{fig:horizon}
\end{figure}

The horizon graph's greatest virtue is that each time series is the same physical shape as it would be in a single time plot. The principle drawback, though, is that it does not let us easily distinguish individual series from one another. A visualization less susceptible to this problem is the color graph. A color graph displays each time series as a series of colored boxes. Each box's luminance is determined by the series' observation at that point in time. Bright colors correspond to high values, and dark ones to low values.

Figure ~\ref{fig:colorplots} shows a color graph for the same 1000 time series as the horizon graph. For purpose of demonstration, the graph is significantly zoomed in, and hence not all of the series are shown. The series are sorted in ascending order of circuit length along the y-axis. Note that we can now clearly distinguish highly active series, such as the brightly speckled ones toward the top, from inactive ones with nearly uniform color. We can also see how often or sporadically a given series tends to peak by looking for isolated bright points. Series with similar color patterns suggest a natural cluster structure on the dataset.

\begin{figure}
	\centerline{\includegraphics[scale=0.5]{figures/viz-demos/colorplots-demo.png}}
	\caption{Color plot of 1000 randomly selected time series (scaled to highlight patterns - not all series are shown).}
	\label{fig:colorplots}
\end{figure}

The principle weakness of the color graph in comparison to the horizon graph is that the human eye cannot distinguish color as easily as it can physical space. This means that in exchange for greater separation between time series, we sacrifice the resolution of our observations. To gain a thorough understanding of our data, the two visualizations are best in used tandem.

After exploring horizon and color graphs on multiple random samples, we produced 6 times plots representative of common attributes seen in the dataset. These plots are shown in Figure ~\ref{fig:timeplots}. These graphs highlight the following attributes of our data:

\begin{itemize}
	\item A: Demonstrates tendency toward jagged peaks, shows activity in the 40-100 cell/window range
	\item B: Shows possibility of very high peaks followed by periods of little to no activity.
	\item C: Demonstrates a circuit that stays inactive for an extended period, then begins sending data.
	\item D: Shows a higly active circuit potentially characteristic of a bulk file uploader.
	\item E: Shows activity in the 5-25 cell/window range.
	\item F: Shows an inactive circuit. The lone plateau is likely due to RELAY cells during the circuit creation process. No data is sent afterward.
\end{itemize}

\begin{figure}
	\centerline{\includegraphics[scale=1.0]{figures/viz-demos/timeplots-6up.png}}
	\caption{6 time series representative of common patterns seen in the dataset.}
	\label{fig:timeplots}
\end{figure}

\section{Conclusions and Model Justification}
Exploratory analysis revealed that our circuits display a very diverse range of behaviors. In addition, it showed us that despite this diversity, there are common patterns which may lend themselves to a natural cluster structure. Many of the series display strong evidence of a client transitioning between multiple distinct states: building a circuit, sending data, falling inactive, then perhaps sending more data at a later time. Though HMMs are not the only process capable of modeling these series, we believe that they are naturally suited to these types of patterns. Furthermore, due to the diversity  of our time series, it is clear that one HMM is not enough to accurately model client behavior. Training a set of models on clusters is a natural way to capture all of the distinct patterns.
