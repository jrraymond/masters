\documentclass[pdf]{beamer}
\usepackage{amsmath}
\usepackage{amsfonts}
\usepackage{amssymb}
\usepackage{bussproofs}
\usepackage{color}
\usepackage{lmodern}
\usepackage{listings}
\usepackage{natbib}
\usepackage{upgreek}
\usepackage{stmaryrd}

\mode<presentation>{}

\newcommand{\T}[1]{\texttt{#1}}
\newsavebox{\codebox}
\newsavebox{\lengthbox}
\lstset{mathescape=true, basicstyle=\ttfamily}
\newcommand{\LP}{\langle}
\newcommand{\RP}{\rangle}
\newcommand{\LB}{\llbracket}
\newcommand{\RB}{\rrbracket}
\newcommand{\LL}{\langle\!\langle}
\newcommand{\RR}{\rangle\!\rangle}
\newcommand{\quadthree}{\qquad\quad}
\newcommand{\quadfour}{\quadthree\quad}
\newcommand{\quadfive}{\quadfour\quad}
\newcommand{\quadsix}{\quadfive\quad}
\newcommand{\quadseven}{\quadsix\quad}
\newcommand{\quadeight}{\quadseven\quad}
\newcommand{\quadten}{\quadfive\quadfive}

\title{Extracting Cost Recurrences from Sequential and Parallel Functional Programs}

\author{Justin Raymond}
\institute{Professor Norman Danner}

%\usecolortheme{fly}
\setbeameroption{show notes}


\begin{document}

\begin{lrbox}{\codebox}
\begin{lstlisting}
rev = $\lambda$xs.rec(xs,
  Nil $\mapsto$ $\lambda$a.a,
  Cons$\mapsto \LP x,\LP xs,r \RP\RP.\lambda$a.force(r) Cons$\LP$x,a$\RP$))) Nil
\end{lstlisting}
\end{lrbox}

\defverbatim[colored]\lstfold{
  \begin{lstlisting}[language=Caml, xleftmargin=.2\textwidth, xrightmargin=.2\textwidth]
  fold f z xs =
    match xs with
      [] -> z
      x::xs' -> f x (fold f z xs')
  \end{lstlisting}
}

\defverbatim[colored]\lstlength{
  \begin{lstlisting}[language=Caml, xleftmargin=.2\textwidth, xrightmargin=.2\textwidth]
  length xs =
    match xs with
     [] -> 0
     (x::xs) -> 1 + length xs
  \end{lstlisting}
}

\defverbatim[colored]\lstfastreverse{
  \begin{lstlisting}[language=Caml, xleftmargin=.05\textwidth, xrightmargin=.05\textwidth]
  rev ys =
    let go xs =
      match xs with
       [] -> fun ys -> ys
       (x::xs') -> fun a -> (go xs') (x::a)
    in go ys []

  rev xs = foldl (fun xs x -> x::xs) [] xs
  \end{lstlisting}
}

\defverbatim[colored]\lstinsert{
  \begin{lstlisting}[language=Caml, xleftmargin=.05\textwidth, xrightmargin=.05\textwidth]
  insert f y xs =
    match xs with
      [] -> [y]
      x::xs' | f y x -> y::xs
      x::xs' -> x::insert f y xs'
  \end{lstlisting}
}
\defverbatim[colored]\lstsort{
  \begin{lstlisting}[language=Caml, xleftmargin=.05\textwidth, xrightmargin=.05\textwidth]
  sort f xs =
    match xs with
      [] -> []
      x::xs' -> insert f x (sort f xs')
  \end{lstlisting}
}

\begin{frame}
  \titlepage
\end{frame}

\begin{frame}{Overview}
  \begin{itemize}
    \item Complexity analysis aims to predict the resources, most often time and space, which a program requires
    \vfill
    \item Previous work by \citet{Danner2013} and \citet{Danner2015} formalizes the analysis of higher-order function programs
    \vfill
    \item We use the method of \citet{Danner2015} to analyze higher order functional programs
    \vfill
    \item We extend the method to parallel cost semantics
    \vfill
    \item We prove an interesting fact about the recurrences for the cost of programs
  \end{itemize}
\note{\tiny
  Complexity analysis aims to predict the resources, most often time and space,
  which a program requires. Traditiional complexity analysis, you look at the
  program, you write down a recurrence which describes the cost, you solve the
  recurrence, you drop everything but the highest order term. There are
  drawbacks to this approach. The foremost is that there is no formal
  relationship between the source language program and the recurrence. This
  makes it easier to make mistakes. The second drawback is the traditional
  approach is not compositional. To analyze the composition of two functions $g
  \circ f$ we need to know something about the size of the result of $f$. This
  gets more complicated with higher order functions such as \T{fold}.}

\note{\tiny
  My thesis comes in three parts. In the first part I analyze the cost of
  higher-order functional programs. I go through the analysis in complete
  detail.  In the second part I demonstrate the flexibility of the formalism by
  changing the cost model to parallel cost semantics. In the third part I prove
  an interesting theorem about the recurrences extracted from source language
programs.  }
\end{frame}

\begin{frame}{Complexity Analysis}
  \begin{itemize}
    \item[]Source program
      \lstfold
      \vfill
    \item[]Recurrence for cost
      \[ T(n) = \begin{cases}
          &c_0 \text{ if } n = 0 \\
          &c_1 + T(n-1) \text{ otherwise}
        \end{cases}
      \]
      \vfill
    \item[]Closed form solution
      \[T(n) = c_1 n + c_1 = \mathcal{O}(n)\]

  \end{itemize}

  \note{\tiny
    Traditional complexity analysis of a recursive program, we write down a
    recurrence for the cost of the program. Then we solve the recurrence and drop
    the constant factors. There is no formal relation between the program and the
    recurrence.
  }

  \note{\tiny
    Can we make this analysis higher-order? How do we take into account the
    asymptotic complexity of $f$?

    We could try to say that $f$ is $\mathcal{O}(x)$ where $x$ is the size of
    the sum of its arguments. But how big are the sum of the arguments to $f$?

    The size of the inputs to $f$ depend on the size of each element in the
    list and the size of recursive calls to \T{fold}.

    How to address this? Enter the \citet{Danner2015}.
  }
\end{frame}

\begin{frame}{Overview}
  \begin{itemize}
    \item Write programs in a "source language"
    \vfill
    \item Translate the programs to a "complexity language"
    \vfill
    \item The translated programs are recurrences for the complexity of the source language program
    \vfill
    \item \textbf{complexity} = cost $\times$ potential
    \vfill
    \item \textbf{cost}: steps to run a program
    \vfill
    \item \textbf{potential}: size of the result of evaluating program
  \end{itemize}

  \note{ \tiny
    How do anaylsis in a way that we have a formal connection between the
  program and the recurrence, is naturally higher-order, and compositional?

  Previous work by \citet{Danner2013} and \citet{Danner2015} developed a method
  to address this. The overview of the approach is the programmer writes their
  programs in a source language. Then the source language program is translated
  to whats called a complexity language. The complexity language is essentially
  a language for expressing the recurrence for the complexity of the source
  language program. What is a complexity? A complexity is a pair of a cost and
  a potential. The cost is a bound on the steps required to run the program.
  The potential is the size of the result of evaluating the program.
}
\end{frame}

\begin{frame}{Source Language}
  \begin{itemize}
    \item Variant of System-T
      \begin{align*}
        e ::=\ &x\ |\ \LP\RP\ |\ \lambda x.e\ |\ e\ e\ |\ \LP e,e\RP\ |\ \T{split}(e, x.x.e) \\
               &|\ \T{delay}(e)\ |\ \T{force}(e)|\ C^\delta\ e\ |\ \T{rec}^\delta(e, \overline{C \mapsto x.e_C}) \\
               &|\ \T{map}^\phi(x.v, v)\ |\ \T{let}(e, x.e)
      \end{align*}
    \vfill
    \item Programmer defined datatypes
      \begin{itemize}
        \item \T{datatype list = Nil | Cons int$\times$list}
      \end{itemize}
    \vfill
    \item Structural Recursion
      \begin{itemize}
        \item OCaml: \lstlength
        \item $\lambda xs.\T{rec}(xs, \T{Nil}\mapsto 0, \T{Cons}\mapsto\LP x,\LP xs,r\RP\RP.1 + \T{force}(r))$
      \end{itemize}
    \vfill
  \end{itemize}

  \note{\tiny
    So what does the source language look like? It is a variant of System
    T. If you're like me and can never remember the differences between the
    variants on the lambda calculus, System T is the simply typed lambda
    calculus with primitive recursion. The source language as structural
    recursion as well as some additional mechanisms to control the cost of
    programs. So we have \T{let} expressions to avoid recomputations of values
    and suspensions to evaluating expressions we don't actually need.

    The source language also has programmer-defined datatypes. Here is an
    example of how you would define a list. The arguments to the constructor of
    a datatype must be strictly positive.

    As an example of what the source language program looks like. Here is a
    recursive function that computes the length of a list in OCaml and the same
    function in the source language. The \T{rec} construct gives us structural
    recursion. It evaluates an expression to a value, and based on the
    constructor of the value, evaluates to the appropriate branch. Inside the
    branch we are given access to the arguments to the constructor as well as a
    delayed computation representing the result of the recursive call.
  }
\end{frame}


\begin{frame}{Sequential Cost Semantics}
  \begin{itemize}
      \vfill
    \item[]
      \begin{prooftree}
        \AxiomC{$e_0 \downarrow^{n_0} v_0$}
        \AxiomC{$e_1 \downarrow^{n_1} v_1$}
        \BinaryInfC{$\LP e_0, e_1 \RP \downarrow^{n_0 + n_1} \LP v_0, v_1 \RP$}
      \end{prooftree}
      \vfill
    \item[]
      \begin{prooftree}
      \AxiomC{$e_0 \downarrow^{n_0} \lambda x.e_0'$}
      \AxiomC{$e_1 \downarrow^{n_1} v_1$}
      \AxiomC{$e_0'[v_1/x] \downarrow^n v$}
      \TrinaryInfC{$e_0\ e_1 \downarrow^{1 + n_0 + n_1 + n} v$}
      \end{prooftree}
      \vfill
    %\item[]
    %  \tiny
    %  \begin{prooftree}
    %    \AxiomC{$e \downarrow^{n_0} C v_0$}
    %    \AxiomC{$\T{map}^{\phi_C}(y.\LP y, \T{delay}(rec(y, \overline{C \mapsto x.e_C}))\RP, v_0) \downarrow^{n_1} v_1$}
    %    \AxiomC{$e_C[v_1/x] \downarrow^{n_2} v$}
    %    \TrinaryInfC{$rec(e, \overline{C \mapsto x.e_C}) \downarrow^{1 + n_0 + n_1 + n_2} v$}
    %  \end{prooftree}
      \vfill
    \item[]
      \begin{figure}
        \AxiomC{}
        \UnaryInfC{$\T{delay}(e) \downarrow^0 \T{delay}(e)$}
        \DisplayProof
        \qquad
        \AxiomC{$e \downarrow^{n_0} \T{delay}(e_0)$}
        \AxiomC{$e_0 \downarrow^{n_1} v$}
        \BinaryInfC{$\T{force}(e) \downarrow^{n_0 + n_1} v$}
        \DisplayProof
      \end{figure}
      \vfill
  \end{itemize}

  \note{\tiny
    The semantics of the source language are given by operation cost
    semantics. Operational cost semantics are big step operational semantics
    but include a notion of steps to execute the program. The judgments are of
    the form $e$ steps to $v$ in $n$ steps.
  }
\end{frame}

\begin{frame}{Complexity Language}
  \begin{itemize}
    \item Source language without syntactic constructs for controlling costs
    \item Types
      \begin{align*}
        T &::= \textbf{C} \ |\ \T{unit} \ |\ \Delta \ |\ T \times T \ |\ T \rightarrow T \\
        \Phi &::= t \ |\ T \ |\ \Phi \times \Phi \ |\ T \rightarrow \Phi \\
        \textbf{C} &::= 0\ |\ 1\ |\ 2\ |\ ... \\
        \T{datatype}\Delta &= C^\Delta_0 \T{of} \Phi_{C_0}[\Delta] \ |\ ... \ |\ C^\Delta_{n-1} \T{of} \Phi_{C_{n-1}}[\Delta]
      \end{align*}
    \item Expressions
      \begin{align*}
        E &::= x | 0 | 1 | E + E | \LP\RP | \LP E,E \RP | \\
          &\quad \pi_0 E | \pi_1 E | \lambda x.E | E\ E | C^\delta\ E | \text{rec}^\Delta(E, \overline{C \mapsto x.E_C})
      \end{align*}
    \vfill
    \item No longer need mechanisms for controlling cost
  \end{itemize}

  \note{\tiny
    The complexity language is a language for recurrences for the cost and
    potential of a source language program. We add an additional type
    \textbf{C} for costs. The rest of the language is very similar to the
    source language except we no longer need the syntactic constructs for
    controlling the costs. So the complexity language does not have suspensions
    or let expressions.
  }
\end{frame}

\begin{frame}{Translation}
  \begin{itemize}
    \item Translate source language programs of type $\tau$ to complexity language programs of type $\textbf{C}\times \LL \tau \RR$
    \vfill
    \item \textbf{C} bound on the steps to evaluate the program
    \vfill
    \item $\LL\tau\RR$ expression for the size of the value
    \vfill
    \item Types of the translation function $\|\cdot\|$:
      \begin{align*}
        \|\tau\| &= \textbf{C} \times \LL \tau \RR \\
        \LL\T{unit}\RR &= \T{unit} \\
        \LL \sigma \times \tau \RR &= \LL \sigma \RR \times \LL \tau \RR \\
        \LL \sigma \rightarrow \tau \RR &= \LL \sigma \RR \rightarrow \|\tau\| \\
        \LL \T{susp}\ \tau \RR &= \|\tau\| \\
        \LL \delta \RR &= \delta \\
      \end{align*}
  \end{itemize}

  \note{\tiny
    The translation function cross compiles source language programs to
    complexity language programs. So a source language program of type $\tau$
    is translated to a complexity language program of type $\textbf{C}\times
    \LL \tau \RR$. The \textbf{C} component is a bound on the steps to evaluate
    the program and $\LL \tau \RR$ is a complexity language expression for the
    size of the result of executing the program.

    Abstractions are less straightforward. The cost of
    evaluating an abstraction is $0$, since abstractions are in normal form.
    The potential of the translation of an abstraction is a function from
    potentials to complexities. Recall that a potential captures the cost of
    future uses of an expression. The cost of future uses of a function depends
    on the size of inputs you apply it to. So the potential of an abstraction
    is a function whose argument is a potential. Since applying a function has
    both a cost and a result. The codomain of the potential of the abstraction
    translation is a cost and a potential, aka a complexity.
}
\end{frame}

\begin{frame}{Translation}
  \begin{itemize}
    \item[] Some cases of the translation function
      \begin{align*}
        \|x\| &= \LP 0,x \RP \\
        \|\LP e_0,e_1 \RP\| &= \LP \|e_0\|_c + \|e_1\|_c, \LP \|e_0\|_p,\|e_1\|_p\RP\RP \\
        \|\lambda x.e\| &= \LP 0, \lambda x.\|e\|\RP \\
        \|e_0\ e_1\| &= (1 + \|e_0\|_c + \|e_1\|_c) +_c \|e_0\|_p \|e_1\|_p \\
        \|\T{delay}(e)\| &= \LP 0,\|e\| \RP \\
        \|\T{force}(e)\| &= \|e\|_c +_c \|e\|_p
      \end{align*}
  \end{itemize}

  \note{\tiny
    Here are some cases of the translation function to get a better idea of
    what is going on. In the variable case, the result of the translation is
    $\LL 0,x \RR$. The cost of evaluating the variable is $0$ as it is already
    in normal form. The potential is the variable itself. The cost of
    translating a pair is the sum of the costs of translating the elements of
    the tuple. The potential is the pair of the potentials of the translations
    of the expressions.
  }
\end{frame}


\begin{frame}{Fast Reverse - Specification and Implementation}
  \begin{itemize}
    \item
        $\T{datatype list} = \T{Nil of unit}\ |\ \T{Cons of int} \times \T{list}$
    \item
      Specification: \T{rev [$x_0,\dots,x_{n-1}$] = [$x_{n-1},\dots,x_0$]}
    \item
      Implementation:
      \usebox{\codebox}
    \item[]
      \lstfastreverse
      %Specification of auxilary function:\\
      %\T{rec($[x_0,\dots,x_{n-1}],\dots$) [$y_0,\dots,y_{m-1}$] = [$x_{n-1},\dots,x_0,y_0,\dots,y_{m-1}$]}
  \end{itemize}

  \note{\tiny
    I'll go through two examples. The first is a higher-order function. The
    function reverses a list but uses a higher-order function to do so in linear
    time. Here is the implementation in OCaml. You'll notice that this is in fact a
    higher-order fold, and that we can write the function using a fold like so.
}
\end{frame}

\begin{frame}{Fast Reverse - Specification and Implementation}
  \begin{itemize}
    \item
      \T{rev (Cons$\LP$0,Cons$\LP$1, Nil$\RP\RP$)}
    \item
    \T{$\to_\beta$
      rec(Cons$\LP$0,Cons$\LP$1,Nil$\RP\RP$,
          Nil $\mapsto\lambda$a.a
          Cons$\mapsto \LP x,\LP xs,r\RP\RP.\lambda$a.force(r) Cons$\LP$x,a$\RP$) Nil}
    \item
      \T{$\to^*_\beta (\lambda$a0.($\lambda$a1.($\lambda$a2.a2) Cons$\LP$1,a1$\RP$) Cons$\LP$0,a0$\RP$) Nil}
    \item
      $\to_\beta$ ($\lambda$a1.($\lambda$a2.a2) Cons$\LP$1,a1$\RP$) Cons$\LP$0,Nil$\RP$
    \item
      $\to_\beta$ ($\lambda$a2.a2) Cons$\LP$1,Cons$\LP$0,Nil$\RP\RP$
    \item
      $\to_\beta$ Cons$\LP$1,Cons$\LP$0,Nil$\RP\RP$
  \end{itemize}

  \note{\tiny
    To give you an intuition into how this works, we the recursion results
    in nested functions. Each function takes a list of an argument and conses
    and element of to the list. The outermost function conses the first element
    of the original list and the innermost function conses the last element.
    Since the outermost function is evaluated first, the first element gets
    consed onto an empty list first and the last element gets consed on last,
    resulting in a reversed list. (DRAW PICTURE).
  }
\end{frame}

\begin{frame}{Fast Reverse - Translation}
  \small
  \begin{itemize}
    \item $\|\T{rev}\|$
      \begin{align*}
       &\LP 0, \lambda xs. 1 +_c \T{rec}(xs, \T{Nil} \mapsto \LP 1, \lambda a. \LP 0,a \RP\RP \\
      &\quad \T{Cons}\mapsto \LP x, \LP xs', r\RP\RP.\LP 1, \lambda a.(1 + r_c) +_c r_p\ \T{Cons}\LP \pi_1 x, a \RP\RP)\ \T{Nil}\RP\\
      \end{align*}
    \item $\|\T{rev xs}\|$
    \begin{align*}
      &1 +_c (\lambda xs.\T{rec}(xs, \T{Nil} \mapsto \LP 1, \lambda a. \LP 0, a \RP \RP \\
  &\quad \T{Cons}\mapsto \LP x, \LP xs', r\RP\RP. \LP 1, \lambda a.(1 + r_c) +_c r_p\ \T{Cons}\LP x, a \RP \RP)\ \T{Nil})\ xs
    \end{align*}
  \end{itemize}

  \note{\tiny
    This is the result of the translation into the complexity language. The
    first term is the complexity of the function \T{rev} itself. The cost is
    $0$ and the potential is a function from potentials to complexities. We are
    interested in the analysis of applying the function \T{rev} to some list,
    so the translation of \T{rev xs} also shown.
  }
\end{frame}

\begin{frame}{Fast Reverse - Interpretation}
  We need to provide an interpretation for programmer-defined datatypes
  \begin{flalign*}
    \LB \T{list} \RB &= \mathbb{N}\\
    D^{list} &= \{\ast\} + \{1\} \times \mathbb{N}\\
    size_{list}(\ast) &= 1\\
    size_{list}(1,n) &= 1 + n\\
  \end{flalign*}

  \note{\tiny
    The translation of a term to the complexity language does not throw
    away any information. That is, there is no abstractions about the list. To
    do so we interpret the complexity language expression in a denotational
    semantics. Denotational semantics assign meanings to programs by
    interpreting types as sets and terms as elements of the set corrresponding
    to their type. Abstractions of type $\tau \to \sigma$ are interpreted as
    mathematical functions with domain $\tau$ and codomain $\sigma$. We use
    standard denotational semantics. The exception is programmer-defined
    datatypes. There are multiple interpretations of the size of a datatype.
    For example, if our natural numbers are fixed width integers we may want to
    interpret all natural numbers as having the same size. We may also
    interpret natural numbers as an inductively defined datatypes, where the
    interpretation of a natural number is the number itself. Another example is
    trees. We may want to interpret trees as their number of nodes. We may also
    want to interpret trees as their number of nodes and the maximum size of
    their label. Another possible interpretation is the height of the tree.

    Here we interpret lists as their lengths. We need $D$ because we need to be
    able to branch on a datatype in the denotational semantics, so we introduce
    the sum type $D$ representing the unfolding of the constructors.
  }
\end{frame}

\begin{frame}{Fast Reverse - Interpretation}
  Interpretation of the recursor
  \begin{align*}
    g(n) &= \bigvee\limits_{size\ z \leq n} case(z, f_C, f_N) \\
  &\text{where} \\
    f_{Nil}(x) &= (1, \lambda a.(0, a)) \\
   f_{Cons}(b) &= (1, \lambda a. (1 + g_c(\pi_1 b)) +_c g_p(\pi_1 b)\ (a + 1))\\
  \end{align*}

  \note{\tiny
    The interpretation of a structural recursion is a maximum over a $case$
    expression.

    This is a recurrence for the function. We want the recurrence for the
    complexity of the function applied to some value.
  }
\end{frame}

\begin{frame}{Fast Reverse - Interpretation}
  \begin{itemize}
    \item Let $h(n, a) = g_p(n)\ a$.
    \item The recurrence for the cost:
      \begin{equation}
        h_c(n,a) = \begin{cases}
          0 & n = 0 \\
          2 + h_c(n-1,a+1) & n > 0
        \end{cases}
      \end{equation}
    \item$h_c(n,a) = 2n$
    \item The recurrence for the potential:
      \begin{equation}
        h_p(n,a) = \begin{cases}
          a & n = 0 \\
          h_p(n-1,a+1) & n > 0
        \end{cases}
      \end{equation}
    \item $h_p(n,a) = n + a$
  \end{itemize}

  \note{\tiny
    So we let $h$ be the result of applying $g$ to the interpretation of some
    list $a$. In order to make the analysis easier we break the complexity
    recurrence into a recurrence for the cost and a recurrence for the
    potential. Then we can solve the recurrences to get closed form solutions.
    So the conclusion is the cost is linear in the length of the list and the
    potential is the sum of the two lists.
  }
\end{frame}

\begin{frame}{Parametric Insertion Sort - Source Language Insert}
  \small
  \begin{itemize}
    \vfill
    \item[]
      $\T{data list} = \T{Nil of unit | Cons of int $\times$ list}$
    \vfill
    \item[]
      \begin{flalign*}
        \T{insert} &= \lambda f.\lambda x.\lambda xs.\T{rec}(xs, \T{Nil} \mapsto \T{Cons} \LP x, \T{Nil}\RP, &\\
                   &\qquad \T{Cons}\mapsto \LP y, \LP ys,r \RP\RP.\T{rec}(f\ x\ y, \T{True}\mapsto \T{Cons}\LP x,\T{Cons}\LP y,ys \RP\RP, &\\
                   &\qquad\quad \T{False}\mapsto \T{Cons}\LP y,\T{force}(r)\RP)) &
      \end{flalign*}
      \vfill
    \item[] \lstinsert
    \vfill
  \end{itemize}
\end{frame}

\begin{frame}{Parametric Insertion Sort - Source Language Sort}
  \small
  \begin{itemize}
    \vfill
    \item[]
      \begin{flalign*}
        \T{sort} &= \lambda f.\lambda xs.\T{rec}(xs, \T{Nil} \mapsto \T{Nil}, &\\
                 &\qquad \T{Cons} \mapsto \LP y,\LP ys,r \RP\RP.\T{insert}\ f\ y\ \T{force}(r)) &
      \end{flalign*}
    \vfill
    \item[] \lstsort
    \vfill
  \end{itemize}
\end{frame}

\begin{frame}{Parametric Insertion Sort - Complexity Language}
  \vfill
  \begin{align*}
    \|\T{insert}\| &= \LP 0, \lambda f. \LP 0, \lambda x.\LP 0,\lambda xs. \T{rec}(xs, \T{Nil} \mapsto \LP 1,\T{Cons}\LP x,\T{Nil}\RP\RP, \\
             &\quad\quad \T{Cons}\mapsto \LP y, \LP ys,r \RP\RP. (3 + (f x)_c) +_c \T{rec}(((f\ x)_p\ y)_p, \\
             &\quadten\T{True}\mapsto \LP 1, \T{Cons}\LP x,\T{Cons}\LP y,ys\RP\RP\RP, \\
             &\quadten\T{False}\mapsto \LP 1 + r_c, \T{Cons}\LP y,r_p\RP\RP)))
  \end{align*}
  \vfill
  \begin{align*}
    \|\T{sort}\| &= \LP 0, \lambda f.\LP 0,\lambda xs.\T{rec}(xs, \T{Nil} \mapsto 1 +_c \LP 0,\T{Nil}\RP, \\
               &\quad \T{Cons} \mapsto \LP y,\LP ys,r \RP\RP.(4 + r_c) +_c ((\|\T{insert}\|_p f)_p y)_p r_p)\RP\RP
  \end{align*}
  \vfill
\end{frame}

\begin{frame}{Parametric Insertion Sort - Insert Interpretation}
  \begin{itemize}
    \item[]
      \small
      \begin{align*}
      g(i,n) &= \bigvee\limits_{size(z) \leq (i,n)} case(z, f_{Nil}, f_{Cons}) \\
             &\text{where}\\
      f_{Nil}(\ast) &= (1, (i, 1)) \\
      f_{Cons}(j,m) &= (4 + (f\ i)_c) +_c ((1, (max(x,j), 2 + m)) \\
                    &\quadthree \vee (1 + g_c(j,m), (max(j,\pi_0 r_p), 1 + \pi_1 g_p(j,m))))
      \end{align*}
    \item Closed form solution for the cost:
      \[g_c(i,n) \leq (4 + ((f\ x)_p\ i)_c n + 1\]
    \item Closed form solution for the potential:
      \[g_p(i,n) \leq (max\{x, i\}, n+1)\]
  \end{itemize}
\end{frame}

\begin{frame}{Parametric Insertion Sort - Sort Interpretation}
  \begin{itemize}
    \item[]
      \small
      \begin{align*}
      g(i, n) &= \bigvee\limits_{size(z)\leq n} case(z, f_{Nil}, f_{Cons}) \\
      %
              &\text{where}\\
      f_{Nil} &= (1, (-\infty,0)) \\
    f_{Cons} &= (5 + g_c(j,m) + (f\ j)_p\ j)_c g_p(j,m), (max\{j,j\},g_p(j,m) + 1))
      \end{align*}
  %cost depends on potential
    \item Closed form solution for potential:
      \[g_p(i,n) \leq (i, n)\]
    \item Closed form solution for the cost:
      \[g_c(i,n) \leq (3 + ((f\ i)_p\ i)_c n^2 + 5n + 1\]
  \end{itemize}
\end{frame}


\begin{frame}{Parallel Cost Semantics}
  \begin{itemize}
    \item Cost graphs
      \[
        \mathcal{C} ::= 0\ |\  1\ |\  \mathcal{C} \oplus \mathcal{C}\ |\  \mathcal{C} \otimes \mathcal{C}
      \]
    \item[] Evaluation Semantics
      \begin{prooftree}
        \AxiomC{$e_0 \downarrow^{n_0} v_0$}
        \AxiomC{$e_1 \downarrow^{n_1} v_1$}
        \BinaryInfC{$\LP e_0, e_1 \RP \downarrow^{n_0 \otimes n_1} \LP v_0, v_1 \RP$}
        \end{prooftree}
        \begin{prooftree}
        \AxiomC{$e_0 \downarrow^{n_0} \lambda x.e_0'$}
        \AxiomC{$e_1 \downarrow^{n_1} v_1$}
        \AxiomC{$e_0'[v_1/x] \downarrow^n v$}
        \TrinaryInfC{$e_0\ e_1 \downarrow^{(n_0 \otimes n_1) \oplus n \oplus 1} v$}
        \end{prooftree}
  \end{itemize}
\end{frame}

\begin{frame}{Work and Span}
  \begin{itemize}
    \item \textbf{Work} total steps to run program
      \begin{equation*}
        work(c) = \begin{cases}
          0 &\text{if } c = 0 \\
          1 &\text{if } c = 1 \\
          work(c_0) + work(c_1) &\text{if } c = c_0 \otimes c_1 \\
          work(c_0) + work(c_1) &\text{if } c = c_0 \oplus c_1
        \end{cases}
      \end{equation*}
    \item \textbf{Span} critical path of program
      \begin{equation*}
        span(c) = \begin{cases}
          0 &\text{if } c = 0 \\
          1 &\text{if } c = 1 \\
          max(span(c_0), span(c_1)) &\text{if } c = c_0 \otimes c_1 \\
          span(c_0) + span(c_1) &\text{if } c = c_0 \oplus c_1
        \end{cases}
      \end{equation*}
  \end{itemize}
\end{frame}

\begin{frame}{Parallel Complexity Translation}
  \begin{align*}
    \|\LP e_0, e_1 \RP \| &= \LP \|e_0\|_c \otimes \|e_1\|_c, \LP \|e_0\|_p, \|e_1\|_p\RP\RP \\
    \|\lambda x.e\| &= \LP 0, \lambda x.\|e\| \RP \\
    \|e_0\ e_1\| &= 1 \oplus (\|e_0\|_c \otimes \|e_1\|_c) \oplus_c \|e_0\|_p\ \|e_1\|_p \\
    \|delay(e)\| &= \LP 0, \|e\|\RP \\
    \|force(e)\| &= \|e\|_c \oplus_c \|e\|_p
  \end{align*}
  Bounding relation: if $\gamma \vdash e : \tau$, then $e \sqsubseteq_\tau \|e\|$.
  Proof by logical relations.
\end{frame}

\begin{frame}{Mutual Recurrences}
  Pure Potential Translation
    \begin{align*}
      |\LP e_0, e_1 \RP | &= \LP |e_0|, |e_1| \RP                                  \\
      |\lambda x.e | &= \lambda x.|e|                                                              \\
      |e_0\ e_1| &= |e_0|\ |e_1|                                                                   \\
      |delay(e)| &= |e|                                                                            \\
      |force(e)| &= |e|
    \end{align*}
  \begin{theorem}
    For all $\gamma \vdash e : \tau$, $|e| : \LL \tau \RR \sim_\tau \|e\| : \|\tau\|$
  \end{theorem}
  \begin{proof}
    by logical relations
  \end{proof}
\end{frame}


\begin{frame}{Bibliography}
  \bibliography{bibliography}
  \bibliographystyle{plainnat}
\end{frame}

\end{document}
