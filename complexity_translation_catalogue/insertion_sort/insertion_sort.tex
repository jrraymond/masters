\documentclass[12pt,letterpaper]{article}
\usepackage[utf8]{inputenc}
\usepackage[english]{babel}
\usepackage{amsmath}
\usepackage{amsfonts}
\usepackage{amssymb}
\usepackage{amsthm}
\usepackage{float}
\floatstyle{boxed} 
\restylefloat{figure}
\usepackage{framed}
\usepackage[letterpaper]{geometry}
\usepackage{graphicx}
\usepackage{listings}
\usepackage{natbib}
\usepackage[parfill]{parskip}
\usepackage[section]{placeins}
\usepackage{upgreek}
\usepackage{stmaryrd}

\newtheorem{theorem}{Theorem}[section]
\newtheorem{corollary}{Corollary}[theorem]
\newtheorem{lemma}[theorem]{Lemma}

\newcommand{\T}[1]{\texttt{#1}}
\lstset{basicstyle=\scriptsize\ttfamily,mathescape}

\begin{document}
\begin{flushright}
Insertion Sort\\
Justin Raymond\\
\today
\end{flushright}

\section{Insertion Sort}
Insertion sort is a quadratic time sorting algorithm which sorts a list by inserting an element from an unsorted segment of a container into a sorted segment of the container.
Although the asymptotic complexity of insertion sort is less than the optimal $\mathcal{O}(nlog_2n)$, insertion sort does have redeeming attributes. 
Insertion sort has small constant factors, making it more efficient on small datasets. 
The standard Python sorting algorithm, timsort, is a hybrid sorting algorithm that uses mergesort and switches to insertion sort for small datasets (\citet{cpython}).
Insertion sort may be done in-place (\citet{Cormen2001}).
The running time of insertion sort is $\mathcal{O}(n^2)$.

\begin{figure}[H]
\caption{Insertion sort in the source language}
\begin{lstlisting}
data list = Nil of unit | Cons of int $\times$ list

insert = $\lambda$x.$\lambda$xs.rec(xs, Nil$\mapsto$ Cons$\langle$x,Nil$\rangle$,
                      Cons$\mapsto \langle$y,$\langle$ys,r$\rangle\rangle$.rec(x <= y, True $\mapsto$ Cons$\langle$x,Cons$\langle$y,ys$\rangle\rangle$,
                                                 False $\mapsto$ Cons$\langle$y,force(r)$\rangle$))

sort = $\lambda$xs.rec(xs, Nil$\mapsto$ Nil, Cons$\mapsto \langle$y,$\langle$ys,r$\rangle\rangle$.insert y force(r))
\end{lstlisting}
\end{figure}

\subsection{Insert}
\T{sort} relies on the function \T{insert} to insert the head of the list into the result of recursively sorted tail of the list.
We will begin with a translation and interpretation of \T{insert}.

\subsubsection{Translation}
The translation of \T{insert} is broken into chunks to make it more managable.
Figure \ref{fig:fxy} steps through the translation of the comparison function \T{<=} applied to variables \T{x} and \T{y}.

\begin{figure}[H]
\caption{Translation of \T{x <= y}.
\label{fig:fxy}
We assume \T{x} and \T{y} are variables.
Consequently the cost of the translation of both is 0.
We assume \T{<=} is a function with a constant cost of \T{2}.
Since \T{<=} is a function of two arguments, the cost of \T{<=} is 0 unless \T{<=} is fully applied.
}
\begin{lstlisting}
$\|$x <= y$\| = (2 + \|$x$\|_c + \|$y$\|_c) +_c \|$x$\|_p$ <= $\|$y$\|_p$
  $= \langle 2 +_c ($x <= y$)\rangle$
  $= \langle \langle 4, ($x <= y$)_p\rangle$
\end{lstlisting}
\end{figure}

The translation \T{true} and \T{false} branches are given in figures \ref{fig:insert_true} and \ref{fig:insert_false} respectively.

\begin{figure}[H]
\caption{Translation of \T{True$\to$Cons$\langle$x,Cons$\langle$y,ys$\rangle\rangle$} in the inner \T{rec} of \T{insert}.
In this case the element we are inserting into the list comes before the head of the list under the ordering given by \T{f}.
}
\label{fig:insert_true}
\begin{lstlisting}
  $=$ True$\mapsto 1 +_c \|$Cons$\langle$x,Cons$\langle$y,ys$\rangle\rangle\|$

  $=$ True$\mapsto 1 +_c \langle \|\langle$x,Cons$\langle$y,ys$\rangle\rangle\|_c,$Cons$\|\langle$x,Cons$\langle$y,ys$\rangle\rangle\|_p\rangle$

  $=$ True$\mapsto 1 +_c \langle \langle \|$x$\|_c + \|$Cons$\langle$y,ys$\rangle\|_c,\langle\|$x$\|_p,\|$Cons$\langle$y,ys$\rangle\|_p\rangle\rangle_c$,
           Cons$\langle \|$x$\|_c + \|$Cons$\langle$y,ys$\rangle\|_c,\langle\|$x$\|_p,\|$Cons$\langle$y,ys$\rangle\|_p\rangle\rangle_p\rangle$

  $=$ True$\mapsto 1 +_c \langle \|$x$\|_c + \|$Cons$\langle$y,ys$\rangle\|_c,$Cons$\langle\|$x$\|_p,\|$Cons$\langle$y,ys$\rangle\|_p\rangle\rangle$

  $=$ True$\mapsto 1 +_c \langle \langle$0,x$\rangle_c + \langle\|\langle$y,ys$\rangle\|_c,$Cons$\|\langle$y,ys$\rangle\|_p\rangle_c$,
          Cons$\langle\langle$0,x$\rangle_p,\langle\|\langle$y,ys$\rangle\|_c,$Cons$\|\langle$y,ys$\rangle\|_p\rangle_p\rangle\rangle$

  $=$ True$\mapsto 1 +_c \langle 0 + \|\langle$y,ys$\rangle\|_c$,Cons$\langle$x,Cons$\|\langle$y,ys$\rangle\|_p\rangle\rangle$

  $=$ True$\mapsto 1 +_c \langle \langle\|$y$\|_c + \|$ys$\|_c,\langle\|$y$\|_p,\|$ys$\|_p\rangle\rangle_c$,Cons$\langle$x,Cons$\langle\|$y$\|_c + \|$ys$\|_c,\langle\|$y$\|_p,\|$ys$\|_p\rangle\rangle_p\rangle\rangle$

  $=$ True$\mapsto 1 +_c \langle\|$y$\|_c + \|$ys$\|_c$,Cons$\langle$x,Cons$\langle\|$y$\|_p,\|$ys$\|_p\rangle\rangle\rangle$

  $=$ True$\mapsto 1 +_c \langle\langle$0,y$\rangle_c + \langle$0,ys$\rangle_c$,Cons$\langle$x,Cons$\langle\langle$0,y$\rangle_p,\langle$0,ys$\rangle_p\rangle\rangle\rangle$

  $=$ True$\mapsto 1 +_c \langle0,$Cons$\langle$x,Cons$\langle$y,ys$\rangle\rangle\rangle$

  $=$ True$\mapsto \langle 1, $Cons$\langle$x,Cons$\langle$y,ys$\rangle\rangle\rangle$
\end{lstlisting}
\end{figure}

\begin{figure}[H]
\caption{Translation of the \T{False} branch of the inner \T{rec} of \T{insert}.
\label{fig:insert_false}
\T{r} stands for the recursive call, and has type \T{susp list}.
In this case the element we are inserting into the list comes after the head of the list under the ordering given by \T{f}.
}
\begin{lstlisting}
$\|$False$\mapsto $Cons$\langle$y,force(r)$\rangle\|$
  $=$ False$\mapsto 1 +_c \|$Cons$\langle$y,force(r)$\rangle\|$

  $=$ False$\mapsto 1 +_c \langle \|\langle$y,force(r)$\rangle\|_c$,Cons$\|\langle$y,force(r)$\rangle\|_p\rangle$

  $=$ False$\mapsto 1 +_c \langle \langle \|$y$\|_c + \|$force(r)$\|_c,\langle \|$y$\|_p,\|$force(r)$\|_p\rangle\rangle_c$,
            Cons$\langle \|$y$\|_c + \|$force(r)$\|_c,\langle \|$y$\|_p,\|$force(r)$\|_p\rangle\rangle_p\rangle$

  $=$ False$\mapsto 1 +_c  \langle\|$y$\|_c + \|$force(r)$\|_c$,Cons$\langle \|$y$\|_p,\|$force(r)$\|_p\rangle\rangle$

  $=$ False$\mapsto 1 +_c  \langle\langle$0,y$\rangle_c + (\|$r$\|_c +_c \|$r$\|_p)_c,$Cons$\langle \langle$0,y$\rangle_p,(\|$r$\|_c +_c \|$r$\|_p)\rangle\rangle$

  $=$ False$\mapsto 1 +_c  \langle0 + $r$_c,$Cons$\langle$y$,$r$_p\rangle\rangle$

  $=$ False$\mapsto \langle1 + $r$_c,$Cons$\langle$y$,$r$_p\rangle\rangle$
\end{lstlisting}
\end{figure}

Figure \ref{fig:insert_inner_rec} uses the translation of \T{f x y} and the \T{true} and \T{false} branches to construct the translation of the inner \T{rec} construct.

\begin{figure}[H]
\caption{Translation of the inner \T{rec} in \T{insert}.
In the \T{True} case, we have found the place of \T{x} in the list and we so stop.
In the \T{False} case, \T{x} comes after the head of list under the ordering given by \T{f} and we must recurse on the tail of the list.
}
\label{fig:insert_inner_rec}
\begin{lstlisting}
$\|$rec(f x y,True$\mapsto$Cons$\langle$x,Cons$\langle$y,ys$\rangle\rangle$,False$\mapsto$Cons$\langle$y,force(r)$\rangle$)$\|$
  $= \|$f x y$\|_c +_c$rec($\|$f x y$\|_p$,True$\mapsto 1 +_c \|$Cons$\langle$x,Cons$\langle$y,ys$\rangle\rangle\|$,
                          False$\mapsto 1 +_c \|$Cons$\langle$y,force(r)$\rangle\|$)

  $= 2 + ((\|$f$\|_p$ x$)_p $ y$)_c +_c$rec($((\|$f$\|_p$ x$)_p $ y$)_p$,
                        True$\mapsto 1 +_c \|$Cons$\langle$x,Cons$\langle$y,ys$\rangle\rangle\|$,
                        False$\mapsto 1 +_c \|$Cons$\langle$y,force(r)$\rangle\|$)

  $= 2 + ((\|$f$\|_p$ x$)_p $ y$)_c +_c$rec($((\|$f$\|_p$ x$)_p $ y$)_p$,
                        True$\mapsto \langle 1, $Cons$\langle$x,Cons$\langle$y,ys$\rangle\rangle\rangle$
                        False$\mapsto 1 +_c \|$Cons$\langle$y,force(r)$\rangle\|$)

  $= 2 + ((\|$f$\|_p$ x$)_p $ y$)_c +_c$rec($((\|$f$\|_p$ x$)_p $ y$)_p$,
                        True$\mapsto \langle 1, $Cons$\langle$x,Cons$\langle$y,ys$\rangle\rangle\rangle$
                        False$\mapsto \langle1 + $r$_c,$Cons$\langle$y$,$r$_p\rangle\rangle$)

\end{lstlisting}
\end{figure}


The \T{Nil} and \T{Cons} branches of the outer \T{rec} construct are given in figures \ref{fig:insert_nil} and \ref{fig:insert_cons}, respectively.

\begin{figure}[H]
\caption{Translation of the \T{Nil} branch of the outer \T{rec} in \T{insert}.
The insertion of an element into an empty list results in a singleton list containing only the element.
This branch is also reached when the ordering given by \T{f} dictates \T{x} comes after than everything in the list,
  and should be placed at the back of the list.
}
\label{fig:insert_nil}
\begin{lstlisting}
$\|$Nil$\mapsto$Cons$\langle$x,Nil$\rangle\|$
  $=$ Nil$\mapsto 1 +_c \|$Cons$\langle$x,Nil$\rangle\|$

  $=$ Nil$\mapsto 1 +_c \langle\|\langle$x,Nil$\rangle\|_c$,Cons$\|\langle$x,Nil$\rangle\|_p\rangle$

  $=$ Nil$\mapsto 1 +_c \langle\langle\|$x$\|_c + \|$Nil$\|_c,\langle\|$x$\|_p,\|$Nil$\|_p\rangle\rangle_c$,Cons$\langle\|$x$\|_c + \|$Nil$\|_c,\langle\|$x$\|_p,\|$Nil$\|_p\rangle\rangle_p\rangle$

  $=$ Nil$\mapsto 1 +_c \langle\|$x$\|_c + \|$Nil$\|_c$,Cons$\langle\|$x$\|_p,\|$Nil$\|_p\rangle\rangle$

  $=$ Nil$\mapsto 1 +_c \langle\langle$0,x$\rangle_c + \langle$0,Nil$\rangle_c$,Cons$\langle\langle$0,x$\rangle_p,\langle$0,Nil$\rangle_p\rangle\rangle$

  $=$ Nil$\mapsto 1 +_c \langle0 + 0$,Cons$\langle$x,Nil$\rangle\rangle$

  $=$ Nil$\mapsto \langle1$,Cons$\langle$x,Nil$\rangle\rangle$
\end{lstlisting}
\end{figure}

\begin{figure}[H]
\caption{Translation of the \T{Cons} branch of the outer \T{rec} in \T{insert}.
In this branch we are recursing on a nonempty list.
We check if \T{x} is comes before the head of the list under the ordering given by \T{f}, in which case we are done, 
  otherwise we recurse on the tail of the list.
}
\label{fig:insert_cons}
\begin{lstlisting}
$\|$Cons$\mapsto \langle$y,$\langle$ys,r$\rangle\rangle$.rec(f x y, True $\mapsto$ Cons$\langle$x,Cons$\langle$y,ys$\rangle\rangle$,
                             False $\mapsto$ Cons$\langle$y,force(r)$\rangle$)$\|$

  $=$ Cons$\mapsto \langle$y,$\langle$ys,r$\rangle\rangle$.$1 +_c \|$rec(f x y, True $\mapsto$ Cons$\langle$x,Cons$\langle$y,ys$\rangle\rangle$,
                             False $\mapsto$ Cons$\langle$y,force(r)$\rangle$)$\|$

  $=$ Cons$\mapsto \langle$y,$\langle$ys,r$\rangle\rangle$.$1 +_c (2 + (($f x$)_p$ y$)_c) +_c$rec($(($f x$)_p$ y$)_p$,
                                              True$\mapsto \langle 1, $Cons$\langle$x,Cons$\langle$y,ys$\rangle\rangle\rangle$
                                              False$\mapsto \langle1 + $r$_c,$Cons$\langle$y$,$r$_p\rangle\rangle$)$)$

  $=$ Cons$\mapsto \langle$y,$\langle$ys,r$\rangle\rangle$.$(3 + (($f x$)_p $ y$)_c) +_c$rec($(($f x$)_p$ y$)_p$,
                                           True$\mapsto \langle 1, $Cons$\langle$x,Cons$\langle$y,ys$\rangle\rangle\rangle$
                                           False$\mapsto \langle1 + $r$_c,$Cons$\langle$y$,$r$_p\rangle\rangle$)$)$
\end{lstlisting}
\end{figure}


We put these together to give the translation of \T{insert}.

\begin{figure}[H]
\caption{Translation of \T{insert}}
\label{fig:insert}
\begin{lstlisting}
$\|$insert$\|$ = $\|\lambda$f.$\lambda$x.$\lambda$xs.rec(xs, Nil$\mapsto$ Cons$\langle$x,Nil$\rangle$,
                      Cons$\mapsto \langle$y,$\langle$ys,r$\rangle\rangle$.rec(f x y, True $\mapsto$ Cons$\langle$x,Cons$\langle$y,ys$\rangle\rangle$,
                                                 False $\mapsto$ Cons$\langle$y,force(r)$\rangle$))$\|$

  $= \langle$0,$\lambda$f.$\langle$0,$\lambda$x.$\langle0,\lambda$xs.$\|$rec(xs, Nil$\mapsto$ Cons$\langle$x,Nil$\rangle$,
                          Cons$\mapsto \langle$y,$\langle$ys,r$\rangle\rangle$.rec(f x y, True $\mapsto$ Cons$\langle$x,Cons$\langle$y,ys$\rangle\rangle$,
                                                     False $\mapsto$ Cons$\langle$y,force(r)$\rangle$))$\|\rangle\rangle\rangle$

  $= \langle$0,$\lambda$f.$\langle$0,$\lambda$x.$\langle0,\lambda$xs.$\langle$0,xs$\rangle_c +_c $
          rec($\langle$0,xs$\rangle_p$,
              Nil$\mapsto \langle1$,Cons$\langle$x,Nil$\rangle\rangle$
              Cons$\mapsto \langle$y,$\langle$ys,r$\rangle\rangle$.$(3 + ((\langle$0,f$\rangle_p$ x$)_p $ y$)_c) +_c$rec($((\langle$0,f$\rangle_p$ x$)_p $ y$)_p$,
                                                   True$\mapsto \langle 1, $Cons$\langle$x,Cons$\langle$y,ys$\rangle\rangle\rangle$
                                                   False$\mapsto \langle1 + $r$_c,$Cons$\langle$y$,$r$_p\rangle\rangle$)$)$

  $= \langle$0,$\lambda$f.$\langle$0,$\lambda$x.$\langle0,\lambda$xs.
          rec(xs,
              Nil$\mapsto \langle1$,Cons$\langle$x,Nil$\rangle\rangle$
              Cons$\mapsto \langle$y,$\langle$ys,r$\rangle\rangle$.$(3 + (($f x$)_p $ y$)_c) +_c$rec($($f x$)_p $ y$)_p$,
                                                 True$\mapsto \langle 1, $Cons$\langle$x,Cons$\langle$y,ys$\rangle\rangle\rangle$
                                                 False$\mapsto \langle1 + $r$_c,$Cons$\langle$y$,$r$_p\rangle\rangle$)$)$
\end{lstlisting}
\end{figure}


Finally we give a translation of \T{insert f x xs} in figure \ref{fig:insert_applied} because this is the term we will interpret in a size-based semantics.

\begin{figure}[H]
  \caption{The translation of \T{insert f x xs}.
  Unlike before, we do not assume that \T{f}, \T{x}, \T{xs} are variables.
  They may be expressions with non-zero costs.}
  \label{fig:insert_applied}
  \begin{lstlisting}
  $\|$insert f x xs$\| = (1 + \|$insert f x$\|_c + \|$xs$\|_c) +_c \|$insert f x$\|_p \|$xs$\|_p$

     $= (1 + \|$insert f x$\|_c + \|$xs$\|_c) +_c \|$insert f x$\|_p  \|$xs$\|_p$

     $= (2 + \|$insert f$\|_c + \|$x$\|_c + (\|$insert f$\|_p \|$x$\|_p)_c + \|$xs$\|_c) +_c \|$insert f$\|_p \|$x$\|_p \|$xs$\|_p$

     $= (2 + \|$insert f$\|_c + \|$x$\|_c + \|$xs$\|_c) +_c \|$insert f$\|_p \|$x$\|_p \|$xs$\|_p$

     $= (3 + \|$insert$\|_c + \|$f$\|_c+ (\|$insert$\|_p \|$f$\|_p \|$x$\|_p)_c + \|$x$\|_c  + \|$xs$\|_c) +_c \|$insert$\|_p \|$f$\|_p \|$x$\|_p \|$xs$\|_p$

     $= (3 + \|$f$\|_c + \|$x$\|_c + \|$xs$\|_c) +_c \|$insert$\|_p \|$f$\|_p \|$x$\|_p \|$xs$\|_p$

     $= (3 + \|$f$\|_c + \|$x$\|_c + \|$xs$\|_c) +_c $rec($\|$xs$\|_p$,
              Nil$\mapsto \langle1$,Cons$\langle$x,Nil$\rangle\rangle$
              Cons$\mapsto \langle$y,$\langle$ys,r$\rangle\rangle$.$(3 + ((\|$f$\|_p \|$x$\|_p)_p $ y$)_c) +_c$rec($(\|$f$\|_p \|$x$\|_p)_p $ y$)_p$,
                                                 True$\mapsto \langle 1, $Cons$\langle \|$x$\|_p$,Cons$\langle$y,ys$\rangle\rangle\rangle$
                                                 False$\mapsto \langle1 + $r$_c,$Cons$\langle$y$,$r$_p\rangle\rangle$)$)$
  \end{lstlisting}
\end{figure}


The result is:
\begin{lstlisting}
$\|$insert f x xs$\|= (3 + \|$f$\|_c + \|$x$\|_c + \|$xs$\|_c)$
        $+_c $rec($\|$xs$\|_p$,
              Nil$\mapsto \langle1$,Cons$\langle$x,Nil$\rangle\rangle$
              Cons$\mapsto \langle$y,$\langle$ys,r$\rangle\rangle$.$(3 + ((\|$f$\|_p \|$x$\|_p)_p $ y$)_c) +_c$rec($(\|$f$\|_p \|$x$\|_p)_p $ y$)_p$,
                                                 True$\mapsto \langle 1, $Cons$\langle \|$x$\|_p$,Cons$\langle$y,ys$\rangle\rangle\rangle$
                                                 False$\mapsto \langle1 + $r$_c,$Cons$\langle$y$,$r$_p\rangle\rangle$)$)$
\end{lstlisting}



\subsubsection{Interpretation}
We well use an interpretation of lists as their lengths.
Figure \ref{fig:interp_sizes} gives formalizes this interpretation.
\begin{figure}[H]
  \caption{Interpretation of lists as lengths}
  \label{fig:interp_sizes}
  \begin{align*}
    \llbracket list \rrbracket &= \mathbb{N}^\infty \\
    D^{list} &= \{\ast\} + \{1\} \times \mathbb{N}^\infty \\
    size_{list} (Nil) &= 0 \\
    size_{list} (Cons(1,n)) &= 1 + n
  \end{align*}
\end{figure}

First we interpret the \T{rec}, which drives of the cost of \T{insert}.
As in the translation, we break the interpretation up to make it more managable.
We will write $map, \lambda$ and $+_c$ in the semantics, which stand for the semantic equivalents of \T{map}, $\lambda$ and $+_c$ in the syntax.
The definitions of these semantic functions mirror the definitions of their syntactic equivalents.
Figures \ref{fig:interp_sizes_inner_rec} and \ref{fig:interp_sizes_outer_rec} walk through the interpretation.

\begin{figure}[H]
  \caption{Interpretation of the inner \T{rec} of \T{insert} with lists abstracted to sizes}
  \label{fig:interp_sizes_inner_rec}
  \begin{lstlisting}
    $\llbracket$rec($($f x$)_p $ y$)_p$,
       True$\mapsto \langle 1, $Cons$\langle$x,Cons$\langle$y,ys$\rangle\rangle\rangle$
       False$\mapsto \langle1 + $r$_c,$Cons$\langle$y$,$r$_p\rangle\rangle$)$)$ $\rrbracket \xi \{$f$ \mapsto f, $x$ \mapsto 1, $y$ \mapsto 1, $ys$ \mapsto n, $r$ \mapsto r\}$

    $f_{True} (\langle\rangle) = \llbracket \langle$1,Cons$\langle$x,Cons$\langle$y,ys$\rangle\rangle\rangle\rrbracket \xi \{$f$ \mapsto f, $x$ \mapsto 1, $y$ \mapsto 1, $ys$ \mapsto n, $r$ \mapsto r\}$

           $= (1,2 + n)$

    $f_{False} (\langle\rangle) = \llbracket \langle1 + $r$_c,$Cons$\langle$y$,$r$_p\rangle\rangle$)$)$ $\rrbracket \xi \{$f$ \mapsto f, $x$ \mapsto 1, $y$ \mapsto 1, $ys$ \mapsto n, $r$ \mapsto r\}$

           $= (1 + \pi_0 r, 1 + \pi_1 r)$
    
    $= \bigvee_{size(w) \leq \pi_1(\pi_1(f\ 1)\ 1)} case(w, (f_{True}, f_{False}))$

    $= \bigvee_{size(w) \leq \pi_1(\pi_1(f\ 1)\ 1)} case(w, (\lambda \langle\rangle.(1,2+n), \lambda \langle\rangle.(1+\pi_0r,1+\pi_1r))$

    $= (1,2+n) \vee (1+\pi_0r,1+\pi_1r)$
  \end{lstlisting}
\end{figure}

\begin{figure}[H]
  \caption{Interpretation of \T{rec} in \T{insert}.}
  \label{fig:interp_sizes_outer_rec}
  \begin{lstlisting}
   $g(n) = \llbracket$rec(xs,
        Nil$\mapsto \langle1$,Cons$\langle$x,Nil$\rangle\rangle$
        Cons$\mapsto \langle$y,$\langle$ys,r$\rangle\rangle$.$(3 + (($f x$)_p $ y$)_c) +_c$rec($($f x$)_p $ y$)_p$,
                                           True$\mapsto \langle 1, $Cons$\langle$x,Cons$\langle$y,ys$\rangle\rangle\rangle$
                                           False$\mapsto \langle1 + $r$_c,$Cons$\langle$y$,$r$_p\rangle\rangle$)$)\rrbracket \xi \{$f$\mapsto f, $x$\mapsto 1,$xs$\mapsto n\}$

       $f_{Nil}(\langle\rangle) = \llbracket \langle1$,Cons$\langle$x,Nil$\rangle\rangle \rrbracket \xi \{$f$\mapsto f, $x$\mapsto 1,$xs$\mapsto n\}$

       $f_{Nil}(\langle\rangle) = (1,1)$

       $f_{Cons}((1,m)) = \llbracket (3 + $((f x)$_p$ y)$_c) +_c $rec($\dots$)$ \rrbracket \xi \{$f$\mapsto f,  $x$\mapsto 1,$xs$\mapsto n\}$

           $= (3 + \pi_0(\pi_1(f\ 1)\ 1)) +_c$
                $\llbracket $rec($\dots$)$ \rrbracket \xi \{$f$\mapsto f, \dots ,\langle$y,$\langle$ys,r$\rangle\rangle\mapsto map^{1\times\mathbb{N}^\infty} (\lambda a.(a,\llbracket$rec($w, \dots$)$\rrbracket \xi \{w \mapsto a\}), (1,m))\}$

           $= (3 + \pi_0(\pi_1(f\ 1)\ 1)) +_c$
                $\llbracket $rec($\dots$)$ \rrbracket \xi \{$f$\mapsto f, \dots ,\langle$y,$\langle$ys,r$\rangle\rangle\mapsto (1,map^{\mathbb{N}^\infty} (\lambda a.(a,\llbracket$rec($w, \dots$)$\rrbracket \xi \{w \mapsto a\}),m))\}$

           $= (3 + \pi_0(\pi_1(f\ 1)\ 1)) +_c$
                $\llbracket $rec($\dots$)$ \rrbracket \xi \{$f$\mapsto f, \dots ,\langle$y,$\langle$ys,r$\rangle\rangle\mapsto (1,(m,\llbracket$rec($w, \dots$)$\rrbracket \xi \{w \mapsto m\}))\}$

           $= (3 + \pi_0(\pi_1(f\ 1)\ 1)) +_c$
                $\llbracket $rec($\dots$)$ \rrbracket \xi \{$f$\mapsto f, $x$\mapsto 1,$xs$\mapsto n, \langle$y,$\langle$ys,r$\rangle\rangle\mapsto (1,(m,g(m)))\}$

           $= (3 + \pi_0(\pi_1(f\ 1)\ 1)) +_c ((1,2+m) \vee (1+\pi_0g(m)),1+\pi_1g(m)))$

           $= (3 + \pi_0(\pi_1(f\ 1)\ 1) + (1 \vee (1+\pi_0g(m))), (2+m) \vee (1+\pi_1g(m))$

           $= (4 + \pi_0(\pi_1(f\ 1)\ 1) + \pi_0g(m), (2+m) \vee (1+\pi_1g(m)))$

       $g(n) = \bigvee_{size(z) \leq n} case(z, (f_{Nil},f_{Cons}))$
  \end{lstlisting}
\end{figure}

The initial result is given in equation \ref{eq:insert_initial_recurrence}.
\begin{align}
  &f_{Nil}(\langle\rangle) = (1,1) \\ 
  &f_{Cons}((1,m)) = (4 + \pi_0(\pi_1(f\ 1)\ 1) + \pi_0g(m), (2+m) \vee (1+\pi_1g(m))) \\ 
  \label{eq:insert_initial_recurrence}
  &g(n) = \bigvee_{size(z) \leq n} case(z, (f_{Nil},f_{Cons}))
\end{align}

This recurrence is difficult to work with.
Specifically, we cannot apply traditional methods of solving it.
We will manipulate it into a more usable form by eliminating the arbitrary maximum.
Observe that for $n=0$, $g(n) = f_{Nil}(\langle\rangle) = (1,1)$.
For $n>0$, 
\begin{align*}
  g(n) &= \bigvee_{size(z) \leq n} case(z, (f_{Nil},f_{Cons})) &&\\ 
  &= g(n-1) \vee \bigvee_{size(z) = n} case(z, (f_{Nil},f_{Cons})) &&\\
  &= g(n-1) \vee f_{Cons}(n) &&\\
  &= g(n-1) \vee (4 + \pi_0(\pi_1(f\ 1)\ 1) + \pi_0g(n-1), (1+n) \vee (1+\pi_1g(n-1))) &&\text{$m=n-1$}\\ 
  &= (4 + \pi_0(\pi_1(f\ 1)\ 1) + \pi_0g(n-1), (1+n) \vee (1+\pi_1g(n-1))) &&\text{lemma \ref{lem:insert_g_monotonicity}}\\ 
  &= (4 + \pi_0(\pi_1(f\ 1)\ 1) + \pi_0g(n-1), 1+\pi_1g(n-1)) &&\text{lemma \ref{lem:insert_potential_inc}}
\end{align*}
\begin{lemma}
  \label{lem:insert_g_monotonicity}
  $g(n) > g(n-1)$
\end{lemma}
\begin{proof}
  TODO
\end{proof}

\begin{lemma}
\label{lem:insert_potential_inc}
$\pi_1 g(n) > n$
\end{lemma}
\begin{proof}
We prove this by induction on $n$.
\begin{description}
  \item[case $n=0$:] $\pi_1g(0) = 1$
  \item[case $n>0$:]\hfill
    \begin{align*}
      \pi_1g(n) &= \pi_1(g(n-1) \vee (4 + \pi_0(\pi_1(f\ 1)\ 1) + \pi_0g(n-1),(1+n) \vee (1 + \pi_1 g(n-1)))) \\
      &= \pi_1g(n-1) \vee (1+n) \vee (1 + \pi_1 g(n-1)) \\
      &\geq n-1 \vee (1+n) \vee (1 + n - 1) \\
      &\geq 1+n \\
      &> n
    \end{align*}
\end{description}
\end{proof}

Equation \ref{eq:insert_recurrence} shows the extracted recurrence.
Without the arbitrary maximum, it is much more obvious how to find a solution to the recurrence.
The recurrence is from a potential to a complexity, consequently we can extract a recurrence for the cost,
equation \ref{eq:insert_cost}, and a recurrence for the potential, equation \ref{eq:insert_potential}, simply by taking the projections of equation \ref{eq:insert_recurrence}.
The extracted recurrences for the cost and potential can then be solved by the substitution method.
\begin{equation}
  \label{eq:insert_recurrence}
  g(n) = \begin{cases}
    (1,1) & n = 0 \\
    (4 + \pi_0(\pi_1(f\ 1)\ 1) + \pi_0g(n-1), 1+\pi_1g(n-1)) & n > 0
  \end{cases}
\end{equation}

The cost recurrence is given by $\pi_0 \circ g$.
\begin{equation}
\label{eq:insert_cost}
c(n) = \begin{cases}
  1 & n = 0 \\
  4 + \pi_0(\pi_1(f\ 1)\ 1) + c(n-1) & n > 0
\end{cases}
\end{equation}

This recurrence is quite simple to solve.
The solution and proof of the solution are given in theorem \ref{thm:insert_cost}.
\begin{theorem}
\label{thm:insert_cost}
  $c(n) = (4 + \pi_0(\pi_1(f\ 1)\ 1))n + 1$
\end{theorem}
\begin{proof}
  We prove this by induction on $n$.
  \begin{description}
    \item[case $n=0$] $c(0) = \pi_0g(0) = 1$
    \item[case $n>0$]
      \begin{align*}
        c(n) &= 4 + \pi_0(\pi_1(f\ 1)\ 1) + c(n-1) \\
        &= 4 + \pi_0(\pi_1(f\ 1)\ 1) + (4 + \pi_0(\pi_1(f\ 1)\ 1))(n-1) + 1 \\
        &= (4 + \pi_0(\pi_1(f\ 1)\ 1)) n + 1
      \end{align*}
  \end{description}
\end{proof}
The solution tells us the cost of the \T{rec} construct in insert is linear in the size of the list.
The constant factor cannot be determined because we do not know the cost of applying $f$ to its arguments.

The potential recurrence is given by $\pi_1 \circ g$.
\begin{equation}
  \label{eq:insert_potential}
  p(n) = \begin{cases}
    1 & n = 0 \\
    1 + p(n-1) & n > 0
  \end{cases}
\end{equation}

\begin{theorem}
  $p(n) = 1 + n$
\end{theorem}
\begin{proof}
  We prove this by induction on $n$.
  \begin{description}
    \item[case $n=0$] $p(0) = 1$
    \item[case $n>0$] $p(n) = 1 + p(n - 1) = 1 + n$
  \end{description}
\end{proof}
The solution of this recurrence tells us the size of the output in terms of the size of the input.
As one would expect of \texttt{insert}, the size of the output is one larger than the size of the input.


\subsection{Sort}



\subsubsection{Translation}
The translation of sort is shown in figure \ref{fig:sort}.
The translation of the \T{Nil} and \T{Cons} branches in the \T{rec} are walked through in figures \ref{fig:sort_nil} and \ref{fig:sort_cons}, respectively.
The translation of \T{sort} applied to its arguments is given in figure \ref{fig:sort_applied}.


\subsubsection{Interpretation}
The \T{rec} construct again drives the cost and potential of \T{sort}.
The walkthrough of the interpretation of the \T{rec} is given in figure \ref{fig:sort_rec_interp}.
Equation \ref{eq:sort_interp0_init} shows the initial recurrence extracted.
\begin{equation}
  \label{eq:sort_interp0_init}
  g(n) = \bigvee_{size(z)\leq n} case(z,(\lambda(\langle\rangle).(1,0),\lambda(1,m).4 + \pi_0 g(m)) +_c insert\ f\ 1\ \pi_1g(m))
\end{equation}

We work the recurrence into a more recognisable form using some manipulation of the big max operator and some facts about $insert$.
Observe for $n=0$, $g(n) = (1,0)$ and for $n>0$
\begin{align*}
  g(n) &= \bigvee_{size(z)\leq n} case(z,(\lambda(\langle\rangle).(1,0),\lambda(1,m).4 + \pi_0 g(m)) +_c insert\ f\ 1\ \pi_1g(m)) \\
  &= g(n-1) \vee \bigvee_{size(z) = n} case(z,(\lambda(\langle\rangle).(1,0),\lambda(1,m).4 + \pi_0 g(m)) +_c insert\ f\ 1\ \pi_1g(m)) \\
  &= g(n-1) \vee (4 + \pi_0 g(n-1)) +_c insert\ f\ 1\ \pi_1g(n-1) \\
  &= g(n-1) \vee (4 + \pi_0 g(n-1) + \pi_0 (insert\ f\ 1\ \pi_1g(n-1)), \pi_1 (insert\ f\ 1\ \pi_1g(n-1))) \\
  &\text{since $\pi_0 (insert\ f\ 1\ m) > 0$ and $\pi_1 (insert\ f\ 1\ m) = 1 + m$} \\
  &= (4 + \pi_0 g(n-1) + \pi_0 (insert\ f\ 1\ \pi_1g(n-1)), \pi_1 (insert\ f\ 1\ \pi_1g(n-1)))
\end{align*}

So our simplified recurrence is
\begin{equation}
  \label{eq:sort_rec_final}
  g(n) = \begin{cases}
    (1,0) & n=0 \\
    (4 + \pi_0 g(n-1) + \pi_0 (insert\ f\ 1\ \pi_1g(n-1)), \pi_1 (insert\ f\ 1\ \pi_1g(n-1))) & n > 0
  \end{cases}
\end{equation}

From this we can extract recurrences for the cost and the potential simply by taking projections from $g$.
We begin with the potential because we will require the solution to the potential recurrence to solve the cost recurrence.

Let $p = \pi_1 \circ g$.
\begin{equation}
  \label{eq:sort_rec_potential}
  p(n) = \begin{cases}
    0 & n=0 \\
    \pi_1 (insert\ f\ 1\ \pi_1g(n-1))) & n > 0
  \end{cases}
\end{equation}

We prove the size of the potential of the output is same as the size of the input.
In other words, \T{sort} does not change the size of the list.
\begin{theorem}
  \label{thm:sort_potential}
  $p(n) = n$
\end{theorem}
\begin{proof}
  We prove this by straightforward induction on $n$.
  \begin{description}
    \item[case $n=0$] $p(0) = 0$\hfill \\
    \item[case $n>0$] $p(n) = \pi_1(insert\ f\ 1\ pi_g(n-1)) = \pi_1(insert\ f\ 1\ (n-1)) = n$
  \end{description}
\end{proof}

Let $c = \pi_0 \circ g$.
\begin{equation}
  \label{eq:sort_rec_cost}
  c(n) = \begin{cases}
    1 & n=0 \\
    4 + \pi_0 g(n-1) + \pi_0 (insert\ f\ 1\ \pi_1g(n-1) & n > 0
  \end{cases}
\end{equation}

\begin{theorem}
  \label{thm:sort_cost}
\end{theorem}
\begin{proof}
  We prove this by straightforward induction on $n$.
  \begin{description}
    \item[case $n=0$:] $c(0) = 1$
    \item[case $n>0$:]
      \begin{align*}
        c(n) &= 4 + \pi_0 g(n-1) + \pi_0(insert\ f\ 1\ \pi_1g(n-1)) \\
        &= 4 + \pi_0 g(n-1) + \pi_0(insert\ f\ 1\ (n-1) \\
        &= 4 + (4 + \pi_0(\pi_1(f\ 1)\ 1))(n-1) + 1) + \pi_0 g(n-1)
      \end{align*}
      TODO COMPLETE
  \end{description}
\end{proof}

OTHER THINGS TODO:
FIX FREE VARIABLES IN APPLICATION
SMOOTH OUT THE INSERT F X XS SITUATION
THE INTERPRETATION OF INSERT IS NOT QUITE CORRECT.



\begin{figure}[H]
  \caption{Translation of \T{Nil} branch of \T{sort}.}
  \label{fig:sort_nil}
  \begin{lstlisting}
  $\|$Nil$\mapsto$Nil$\|$

  $= $Nil$\mapsto 1 +_c \|$Nil$\|$

  $= $Nil$\mapsto 1 +_c \langle$0,NiL$\rangle$

  $= $Nil$\mapsto \langle$1,NiL$\rangle$
  \end{lstlisting}
\end{figure}

\begin{figure}[H]
  \caption{Translation of \T{Nil} branch of \T{sort}.}
  \label{fig:sort_cons}
  \begin{lstlisting}
  $\|$Cons$\mapsto\langle$y,$\langle$ys,r$\rangle\rangle$.insert f y force(r)

  $= $Cons$\mapsto 1 +_c \|$insert f y force(r)$\|$

  $= $Cons$\mapsto\langle$y,$\langle$ys,r$\rangle\rangle. 1 +_c (\|$force(r)$\|_c) +_c \|$insert f y$\|_p \|$force(r)$\|_p$

  $= $Cons$\mapsto\langle$y,$\langle$ys,r$\rangle\rangle. 1 +_c ((\|$r$\|_c +_c \|$r$\|_p)_c) +_c \|$insert f y$\|_p (\|$r$\|_c +_c \|$r$\|_p)_p$

  $= $Cons$\mapsto\langle$y,$\langle$ys,r$\rangle\rangle. 1 +_c $r$_c +_c \|$insert f y$\|_p $r$_p$

  $= $Cons$\mapsto\langle$y,$\langle$ys,r$\rangle\rangle. 1 +_c $r$_c +_c 3 +_c \|$insert$\|_p$ f y r$_p$

  $= $Cons$\mapsto\langle$y,$\langle$ys,r$\rangle\rangle. 4 +_c $r$_c +_c \|$insert$\|_p$ f y r$_p$

  \end{lstlisting}
\end{figure}

\begin{figure}[H]
\caption{Translation of \T{sort}}
\label{fig:sort}
\begin{lstlisting}
$\|$sort$\|$ = $\langle0,\lambda$f.$\langle0,\lambda$xs.$\|$rec(xs, Nil$\mapsto$ Nil,
                        Cons$\mapsto \langle$y,$\langle$ys,r$\rangle\rangle$.insert f y force(r))$\|\rangle\rangle$

      = $\langle0,\lambda$f.$\langle0,\lambda$xs.rec(xs, Nil$\mapsto \langle$1,Nil$\rangle$
                        Cons$\mapsto \langle$y,$\langle$ys,r$\rangle\rangle.4 +_c $r$_c +_c \|$insert$\|_p$ f y r$_p$
\end{lstlisting}
\end{figure}

\begin{figure}[H]
\caption{Translation of \T{sort}}
\label{fig:sort_applied}
\begin{lstlisting}
$\|$sort f xs$\| = (1 + \|$sort f$\|_c + \|$xs$\|_c) +_c \|$sort f$\|_p \|$xs$\|_p$

          $ = (1 + (1 + \|$sort$\|_c + \|$f$\|_c + \|$xs$\|_c)) +_c \|$sort$\|_p \|$f$\|_p \|$xs$\|_p$

          $ = (1 + (1 + 0 + 0 + 0)) +_c \|$sort$\|_p \|$f$\|_p \|$xs$\|_p$

          $ = 2 +_c \|$sort$\|_p $f$\|_p \|$xs$\|_p$

          $ = 2 +_c $rec($\|$xs$\|_p$, Nil$\mapsto \langle$1,Nil$\rangle$
                        Cons$\mapsto \langle$y,$\langle$ys,r$\rangle\rangle.4 +_c $r$_c +_c \|$insert$\|_p$ f y r$_p$
\end{lstlisting}
\end{figure}


\begin{figure}[H]
  \caption{Interpretation of \T{rec} in \T{sort}.}
  \label{fig:sort_rec_interp}
  \begin{lstlisting}
  $g(n) = \llbracket $rec($\|$xs$\|_p$, Nil$\mapsto \langle$1,Nil$\rangle$
                Cons$\mapsto \langle$y,$\langle$ys,r$\rangle\rangle.4 +_c $r$_c +_c \|$insert$\|_p$ f y r$_p$)$\rrbracket \xi \{ xs \mapsto n\}$

     $ = \llbracket $rec($\|$xs$\|_p$, Nil$\mapsto \langle$1,Nil$\rangle$
                Cons$\mapsto \langle$y,$\langle$ys,r$\rangle\rangle.4 +_c $r$_c +_c \|$insert$\|_p$ f y r$_p$)$\rrbracket \xi \{ xs \mapsto n\}$

     $ = \llbracket $rec($\|$xs$\|_p$, Nil$\mapsto \langle$1,Nil$\rangle$
                Cons$\mapsto \langle$y,$\langle$ys,r$\rangle\rangle.4 +_c $r$_c +_c \|$insert$\|_p$ f y r$_p$)$\rrbracket \xi \{ xs \mapsto n\}$

     $ = \bigvee_{size(z)\leq n} case(z,(f_{Nil},f_{Cons}))$

  $f_{Nil}(\langle\rangle) = \llbracket \langle$1,Nil$\rangle \rrbracket \xi$
     $ = f_{Nil}(\langle\rangle) = (1,0)$

  $f_{Cons}((1,m)) = \llbracket \dots \rrbracket \xi \{\langle$y,$\langle$ys,r$\rangle\rangle\mapsto map^{1\times\mathbb{N}^\infty} (\lambda a.(a,\llbracket$rec($w, \dots$)$\rrbracket \xi \{w \mapsto a\}), (1,m)) \}$

    $ = \llbracket \dots \rrbracket \xi \{\langle$y,$\langle$ys,r$\rangle\rangle\mapsto (map^1 (\lambda a.(\dots), 1),map^{\mathbb{N}^\infty} (\lambda a.(a,\llbracket$rec($w, \dots$)$\rrbracket \xi \{w \mapsto a\}), m)) \}$

    $ = \llbracket \dots \rrbracket \xi \{\langle$y,$\langle$ys,r$\rangle\rangle\mapsto (1,(m,\llbracket$rec($w, \dots$)$\rrbracket \xi \{w \mapsto m\})) \}$

    $ = \llbracket (4 + $r$_c) +_c \|$insert$\|_p$ f y r$_p \rrbracket \xi \{\langle$y,$\langle$ys,r$\rangle\rangle\mapsto (1,(m,g(m)) \}$

    $ = (4 + \pi_0 g(m)) +_c insert\ f\ 1\ \pi_1g(m)$

  $g(n) = \bigvee_{size(z)\leq n} case(z,(\lambda(\langle\rangle).(1,0),\lambda(1,m).4 + \pi_0 g(m)) +_c insert\ f\ 1\ \pi_1g(m)))$
  \end{lstlisting}
\end{figure}

\bibliography{jrrbib}{}
\bibliographystyle{plainnat}
\end{document}
